\tchapter{政治文化真的起作用吗?}

\quo[——托克维尔]{新英格兰的居民依恋他们的乡镇,因为乡镇是强大的和独立的;他们关心自己的乡镇,因为他们参加乡镇的管理;他们热爱自己的乡镇,因为他们不能不珍惜自己的命运。}

\quo[——加布里埃尔 · A.阿尔蒙德与西德尼 · 维巴]{一个稳定的和有效率的民主政府,不光是依靠政府结构和政治结构:它依靠人民所具有的对政治过程的取向——即政治文化。除非政治文化能够支持民主系统,否则,这种系统获得成功的机会将是渺茫的。}

\quo[——罗纳德 · 英格尔哈特]{在发达工业国家里,主流发展方向是从现代化转变到后现代化。这条新的轨迹使得对作为工业国家标志的功能理性的强调出现衰退,而对自我表现和生活质量的强调在增加。……随着后现代化进程的推进,一种新的世界观正在逐步替代工业革命以来一直支配工业化国家的信仰框架。它反映了某个问题上的态度转变,即人们到底渴望从生活中得到什么这一问题。}

\quo[——弗朗西斯 · 福山]{在经济领域,社会资本能够降低交易成本;在政治领域,社会资本能够有助于有限政府和现代民主制的成功。}

\tsection{政治文化与政治社会化}

什么是政治文化?白鲁恂(又译鲁恂 · 派伊)认为:

\quo{政治文化是一组态度、信仰和情感,它赋予政治过程以秩序和含义,并提供一种基本的假设和规则用以规范政治体系中的行为。它包裹着政治观念和政制运行的规则。因此,政治文化是对政治中心理和主观层面的一种集合形式和表述。……简而言之,政治文化之于政治体系犹如文化之于社会。\nauthor{鲁恂 · W.派伊:《政治发展面面观》,任晓、王元译,天津:天津人民出版社 2009 年版,第 124 页。}}

安德鲁 · 海伍德在《政治学》教科书中这样定义 “政治文化” :

\quo{政治学家在更为狭隘的意义上用该词来指人们的心理倾向,政治文化就是针对政党、政府和宪法等政治客体的倾向模式(pattern of orientations),并表现为信仰、符号和价值。政治文化不同于公共舆论,它由长期的价值而非对具体政策、问题或人物的反应塑造而成。\nauthor{安德鲁 · 海伍德:《政治学》(第二版),张立鹏译,北京:中国人民大学出版社 2006 年版,第 242 页。}}

这样的定义听上去比较抽象,但实际上,政治文化涉及的是人们对重要政治议题的态度和倾向。第 3 讲曾提及政治意识形态争论中的若干重要议题,这些议题也与政治文化有关。比如,大家如何看待自由与权威的关系?中国人与欧洲人的态度和倾向是否相似?一般认为,西欧人通常更重视自由或权利,而中国人更重视权威或权力。

比如,如何看待国家?英国的洛克和德国的黑格尔对国家的看法完全不同。洛克把国家视为一种 “必要的恶” ,这是一种自由主义观点。黑格尔则把国家视为一种神圣事物,甚至说国家是 “神在地上行走” 。对普通人而言,当说到国家时,不同社会的人们——从政治精英到普通民众——脑海里浮现的景象是完全不同的。

又比如,如何看待民主政体与威权政体的关系?如果我们做一个 “威权—民主” 的谱系出来,分别标上从 1 到 10 的刻度,1 代表最支持威权主义,10 代表最支持民主。然后,做公民抽样调查,看看每个人都处在从 1 到 10 的哪个位置上。如果做一个上万人或数万人的抽样调查,其结果就有较大的意义。这些人的看法,至少部分地代表了这个国家的民众对于民主与威权的看法。每个人是怎么想的,很大程度上决定着这个国家的政治是怎样的。现有研究揭示,不同国家的民众在这一问题上的基本政治态度和倾向的差异很大。

再进一步说,人们对很多基本政治概念都有着完全不同的理解。比如,中国人听到 “国家” 或 “政府” 这个词,跟英国人听到 “State” 或 “Government” 这个词,脑海中的景象是相似的吗?实际上,两国民众对国家和政府这类基本概念的理解是有差异的。

再比如,政党是最重要的政治概念之一。有的国家天然地把党视为一个领导机构,但有的国家更可能把政党视为一个 “争权夺利” 的组织。政党的英文是 political party,party 就其词源来看本身就包含了部分(part)的含义,而不是全体或整体。所以,英美国家的民众不认为政党能代表全体。这个例子也说明了人们基本政治观念的差异。

讨论政治文化,还需要理解政治社会化问题,这是指对政治文化习得的过程,包括政治认知、态度、价值观与行为的习得。一般认为,12—30 岁是政治社会化的关键时期。对任何一个公民的政治观念养成来说,其从小到大的成长环境非常重要。家庭、学校、教堂、工作场所、社交网络、新闻媒体与互联网以及政府等等都会对塑造他的政治观念产生影响。

家庭是很重要的政治社会化的场所,绝大多数人从小都会在父母那里学到很多东西。比如,父母在餐桌上关于政府或政党的不经意的谈话,会给孩子打下最早的政治烙印。美国的选举调查表明,一个选民投票支持共和党还是民主党很大程度上受其家庭政党支持传统的影响。如果他父亲支持共和党,他很有可能也会支持共和党。学校也是重要的政治社会化场所。如果一个国家的教育系统推行浓厚的意识形态教育,那么学校对国民政治文化的影响会更大。在具有宗教传统的国家,教堂通常也是重要的政治社会化机构。当然,20 世纪以来,欧美社会中教堂和教会的政治影响力总体上在衰落。但是,选举调查显示,那些经常去教堂的选民更有可能投票支持右翼政党,这说明教会会影响一个人的政治观念。

工作场所也会影响人的政治立场与观点。以中国目前的境况为例,民营企业和跨国公司希望政府放松管制,很多国有企业可能希望维持既有的市场管制。当然,这只是一个方面。所以,进入不同的工作场所会影响一个人的政治倾向。同样重要的是,一个成年人在工作场所待的时间通常都比较长,与同事之间往往会形成密切的人际关系。这种频繁的人际互动肯定会影响一个人的政治倾向,这就与社交网络有关。当然,人的社交网络不全是在工作场所建立的,儿时的伙伴、学校的同学与老师、成年以后的其他朋友圈,都可能影响一个人的政治态度。

当然,媒体的影响也很重要,包括传统新闻媒体与网络媒体。很多人每天都要通过报纸、电视或互联网获取信息。无论他以何种方式获得何种信息,都会对其政治观点产生影响。相对来说,如果新闻媒体与互联网受到控制,就更容易选择性地传播信息。这样的结果是,普通公民的政治文化有可能会被有意地塑造。当然,这种意图并不总是能够奏效。此外,并非所有媒体的信息传播特性与方式都是一样的。与传统新闻媒体相比,互联网是一种更加自由开放、解构权威、强化互动的新型信息媒介。互联网的最新趋势则是互动社交媒体的兴起。当一个人身处这种社交媒体网络中时,他的政治观念就会受到影响。

最后,政府也是非常重要的政治社会化机构。比如,政府领导人的讲话、总统的国情咨文及首相的大学演讲等等,都是政府影响其国民政治态度的重要方式。美国前总统克林顿在其自传中提到,他青少年时有机会作为美国童子军的代表受到肯尼迪总统的接见。这对当时的克林顿来说,是一种重要的荣誉。他当时就产生一种强烈的渴望,希望自己有朝一日能够成为美国总统。这一事例也能说明政府在政治社会化过程中的作用。此外,政府还在很大程度上控制着公立教育系统,这也是政府影响政治观念的主要途径。

\tsection{托克维尔论政治文化}

关于政治文化,法国思想家托克维尔并非最早的研究者。但托克维尔在《论美国的民主》中阐述了大量与政治文化有关的内容,他甚至把政治文化视为美国民主政体得以稳固的基本原因。在他的论述中,民情是一个重要概念。与 “民情” 对应的现代政治学术语无疑应该是 “政治文化” 。 “它(民情)不仅指通常所说的心理习惯方面的东西,而且包括人们拥有的各种见解和社会上流行的不同观点,以及人们的生活习惯所遵循的全部思想。” \nauthor{托克维尔:《论美国的民主》(上卷),董国良译,北京:商务印书馆 1989 年版,第 354—362 页。}他还认为,美国新英格兰地区特有的政治文化是 “乡镇精神” 。

1831 年,法国人托克维尔踏上了美利坚这块与众不同的土地。托克维尔此行名义上是要考察美国的监狱制度,但他实际上是要考察和研究美国的民主制度。托克维尔并不认为从希腊罗马的哲学中能够发展出最好的民主理论,他认为最好的民主理论在于对真正的民主社会的系统考察,而他到美国就是为了完成这一使命。他一方面是要发展自己的民主理论,另一方面是为法国找到可资借鉴的经验。回国后,托克维尔于 1835 年出版了他的不朽名著《论美国的民主》。

托克维尔的美国之行是他生命中的奇遇记,美国有利的地形、身份的平等、美国的联邦宪法与强大的司法权都给他留下了难以磨灭的印象。正是对美国政治制度的考察,使托克维尔确立了对民主的无限信心,他后来在书中写道: “身份平等的逐渐发展,是事所必至,天意使然。这种发展具有的主要特征是:它是普遍的和持久的,它每时每刻都能摆脱人力的阻挠,所有的事和所有的人都在帮助它前进。” 

实际上,托克维尔在书中发出了民主进步的宣言。尽管上述内容占了《论美国的民主》的很大篇幅,但托克维尔对美国民主考察的起点却是美国的乡镇组织,对乡镇的考察是他美国之行的第一站,托克维尔之所以首先考察美国的乡镇,一方面是由于乡镇是一种最自然的人类组织状态,只要有人群的地方就有乡镇,而且乡镇的组织构成了一个国家的制度状况的基础;另一方面,按照托克维尔自己的看法,乡镇自由是 “在各种自由中最难实现的” ,因为它 “最容易受到国家政权的侵犯” ,乡镇组织 “绝对斗不过庞大的中央政府” 。所以,由乡镇组织的状况可以想见国家制度的状况,由乡镇人民的自由程度可以想见一国人民的自由程度。

正是这个问题上,托克维尔发现了与他的祖国完全不同的情况,因为他在美国找到了乡镇自由。相比之下, “在欧洲大陆的所有国家中,可以说知道乡镇自由的国家连一个都没有” 。但是,在美国——

\quo{乡镇却是自由人民的力量所在。……乡镇组织将自由带给人民,教导人民安享自由和学会让自由为他们服务。在没有乡镇组织的条件下,一个国家虽然可以建立一个自由的政府,但它没有自由的精神。片刻的激情、暂时的利益或偶然的机会可以创造出独立的外表,但潜伏于社会机体内部的专制迟早会重新冒出表面。}

在当时的美国,乡镇的人口规模大约在两三千人,相当于现今中国一个村的规模。立法与行政工作几乎完全是在被统治者面前完成的,托克维尔时代的美国乡镇实行的是直接民主,没有乡镇议会。在任命行政官员后,由选举团对他们进行全面的领导,工作程序非常简便。而乡镇的行政官员都要按照本镇居民事先通过的规则办事。但是,他们若想对既定的事项作出更改,或希望拟办一项新的事业,那么这些官员就要请示他们权力的授予者。比如,他们打算创办一所新的学校。出现这种情况,几位行政委员就要找一个日子,在事先确定的地方召集全体选民大会。在大会上,由行政委员提出具体的事项,然后由大会对所有问题进行讨论和表决,确定办事规则、地点及经费的筹集等等,然后交由行政委员去执行。因此,托克维尔在美国时,他看到乡镇自治的美妙图景,这使他感到耳目一新。

美国人信奉这样的观念:个人是本身利益的最好的和唯一的裁判者。这一观念不仅对美国人的日常生活有很大的影响,它还直接作用于美国的乡镇制度和其他政治制度。在托克维尔看来, “从对中央政府的关系来说,整个乡镇亦如其他行政区一样,也像是一个个人来行使自己的权利。”  “乡镇在只与其本身有关的一切事物上仍然是独立的。” 托克维尔认为,独立、自治与有权是美国乡镇组织的一般状况,乡镇是自由的,从而乡镇的人民能享有自由。

那么为什么美国人能够安享乡镇自由呢?这与美国的法律和政治制度有很大的关系,但这更离不开美国的民情。用托克维尔的话来说, “在美国,乡镇不仅有自己的制度,而且有支持和鼓励这种制度的乡镇精神。” 所谓乡镇精神,大体是说乡镇居民对本地公共事务的参与、决定以及对乡镇的依恋、热爱。托克维尔说:

\quo{新英格兰的居民依恋他们的乡镇,因为乡镇是强大的和独立的;他们关心自己的乡镇,因为他们参加乡镇的管理;他们热爱自己的乡镇,因为他们不能不珍惜自己的命运。}

这样,乡镇生活可以说每时每刻都与自己休戚相关,每天每日都在通过履行一项义务或行使一项权利而实现。 “这样的乡镇生活,使社会产生了一种勇往直前而又不致打乱社会秩序的稳步运动。” 正是美国居民对乡镇事务的参与和热爱,造就了新英格兰幸福甜美的乡镇生活。正是乡镇精神成了美国民主的灵魂。\nauthor{托克维尔:《论美国的民主》(上卷),第 66—75 页。}

讲到政治,很多国家的民众首先想到的是权力,美国人首先想到的是权利;讲到政府,很多国家的民众首先想到的是政府应该做好事,美国人首先想到的是政府可能做坏事。对于权利与服从的关系,托克维尔认为美国人的政治观念也是独特的——

\quo{使人们能够确定什么是跋扈和暴政,正是权利观念。权利观念明确的人,可以独立地表现出自己的意志而不傲慢,正直地表示服从而不奴颜婢膝。\nauthor{托克维尔:《论美国的民主》(上卷),第 272 页。}}

托克维尔在书中总结道,对于维护美国的民主政体, “自然环境不如法制,而法制又不如民情” 。这里的民情,当然是指政治文化。因此,如果不是乡镇精神,即便有联邦宪法与三权分立,美国人也恐难享有真正的自由。

\tsection{阿尔蒙德与公民文化}

托克维尔作为一位 19 世纪的政治思想家,他研究美国政治文化是基于经验观察。美国学者加布里埃尔 · 阿尔蒙德和西德尼 · 维巴的重要贡献则是在政治文化研究中引入了问卷调查和定量分析的方法。阿尔蒙德和维巴从 1957 年到 1962 年对美国、英国、联邦德国、意大利和墨西哥五个国家的公民政治态度进行了问卷调查,共完成约 5000 份问卷,平均每个国家约 1000 份。他们的基本方法是调查问卷和面谈,然后对调查数据做简单的量化分析,进行跨国比较。熟悉 20 世纪中叶政治学趋势的读者会知道,这项研究受到了行为主义革命的影响,当时的美国政治学者在选举和投票研究中开始使用问卷调查和定量分析方法。在此基础上,他们的研究成果《公民文化》于 1963 年出版,很快获得声誉并成为学术名著。

政治文化之所以重要,在阿尔蒙德和维巴看来,政治文化是微观的政治和宏观的政治之间的连接纽带。通过研究政治文化,特别是从微观层次上观察公民个体的政治行为、信念与倾向,可以发掘出一个国家民主或不民主这一宏观政治现象的成因。他们认为,可以用政治文化来解释从微观的个体行为到宏观的政治现象之间的机制。换句话说,这些个体拥有什么样的政治文化,会影响到民主政体能否维系或实现稳定。

他们首先区分出了政治文化的三种不同类型。第一种是村民文化(parochial culture)。这里的村民概念,强调的是他们的活动范围和视野都局限在一个非常小的范围之内。他们从小生活在自己的家乡,也只关心极小范围内的一些事情。通常来说,这是一种比较原始落后的生活状态所塑造的政治文化。第二种是臣民文化(subject culture)。这种文化的直接表现是政治上比较消极,这些人认为普通民众是无力影响政治的,他们具有更好的服从权威的意识。第三种是参与者文化(participant culture)。这种文化强调公民意识,这些人关心政治,正如美国新英格兰的乡镇居民一般。他们通常是政治积极分子,希望参与政治,希望通过政治参与来改善公共治理。

那么,什么样的政治文化有利于民主政体的稳定呢?他们认为是公民文化。公民文化是参与者文化、臣民文化和村民文化三者的混合。阿尔蒙德认为,公民文化有时明显地包含着互相矛盾的诸种政治态度,但这似乎特别适合于民主政治系统,原因在于民主政治系统也是一种矛盾的混合体。简单地说,身为民主政体下的公民,理想状态应该是:该参与的时候就要参与,该服从的时候就要服从;该积极的时候就要积极,该消极的时候就要消极。当公民文化把这三种东西结合起来时,最有利于民主政体的维系和稳定。阿尔蒙德这样论述具有混合特质的公民文化:

\quo{当这些被保留的较传统的态度和参与者取向相融合的时候,便导致了一种平衡的政治文化,在这种文化中,既存在着政治的积极性、政治卷入和理性,但又为消极性、传统性和对村民价值的责任心所平衡。\nauthor{加布里埃尔 · A.阿尔蒙德、西德尼 · 维巴:《公民文化——五个国家的政治态度和民主制》,徐湘林等译,北京:东方出版社 2008 年版,第 29 页。}}

这意味着民主政体下的公民需要平衡好积极与消极、卷入政治又不高度卷入政治、追求政治影响又懂得服从权威之间的关系。有些民主国家在特定事件中,会出现大量公民热衷政治参与,政治动员程度极高,甚至多数公民在政治上异常亢奋的状态。但这种状态对民主政体的稳定很难说是有利的。

这里有两个相反的例子。在 2000 年美国总统选举中,共和党候选人小布什跟民主党候选人戈尔在佛罗里达州的选票非常接近,有人试图重新清点选票,但最后由法院判决小布什胜出,从而使其赢得总统大选。戈尔本人和民主党选民都深感遗憾,但他们都能平静地接受这一司法判决和选举结果。假如这一政治事件发生在泰国,将会发生什么呢?一种可能的情形是,反对派的政治力量会不断地游行示威,抗议选举结果,要求当选执政党或总理下台,甚至会发展到占领街道、交通枢纽乃至政府机构。过去十年中,这种政治秀在泰国已上演多次,直至最后发生军事政变。所以,阿尔蒙德这样说: “民主政治中的公民要求寻求相对立的矛盾的目标:他必须是积极的,也是消极的;卷入的,也是不太卷入的;有影响力的,也是服从的。” 阿尔蒙德等人的结论是:

\quo{一个稳定的和有效率的民主政府,不光是依靠政府结构和政治结构:它依靠人民所具有的对政治过程的取向——即政治文化。除非政治文化能够支持民主系统,否则,这种系统获得成功的机会将是渺茫的。\nauthor{加布里埃尔 · A.阿尔蒙德、西德尼 · 维巴:《公民文化——五个国家的政治态度和民主制》,徐湘林等译,北京:东方出版社 2008 年版,第 421—449 页。}}

对很多国家来说,这种结论略显悲观。这意味着,除非政治文化倾向于支持民主政体,否则民主获得成功的机会很小。当然,这种观点也充满争议。

在阿尔蒙德与维巴出版《公民文化》之后,罗伯特 · 达尔在其名著《多头政体》中用一章来专门讨论:多头政体能否维系很大程度上取决于政治积极分子的信念。按照他的逻辑,政治信念会影响行动,政治行动会影响到政体维系的机会。换句话说,一个国家政治精英们的政治信念和价值观会影响该国政体的稳定性。\nauthor{罗伯特 · 达尔:《多头政体》,谭君久、刘慧荣译,北京:商务印书馆 2003 年版。}

美国的建国历程可以说明政治精英信念对构建和维系民主政体的重要性。独立战争后期,华盛顿领导的大陆军与英国作战正酣。有一位名叫尼古拉斯的军官给华盛顿写信。信的大意是说:共和政体对战事多有不利,您既然已经领导大陆军,不如做我们的国王。尼古拉斯建议实行君主制,并宣称效忠于华盛顿。华盛顿写了一封回信给尼古拉斯,大意是说:对尼古拉斯上校的这封信表示 “深恶痛绝,斥之为大逆不道” ,并把实行帝制视为 “对国家祸害最烈之事” ;若 “以国家为念” ,就应该 “务请排除这一类谬念,勿再任其流传” 。

在这样一个重要的政治关头,华盛顿领导了一场为期数年的独立战争,他领导着军队,并在整个北美殖民地拥有崇高的威望。他会利用这些有利条件控制军队、控制政治并进而控制整个国家吗?他会利用这些有利条件自立为王并成为独裁者吗?从全球经验来看,这种可能性不能排除。比如,拉丁美洲就不乏这样的案例,拉美独立运动中的很多英雄或领袖人物最后都成了独裁者。但是,华盛顿的回信说明,他是一个君主政体的坚定反对者。所以,特别是在政治转型的关头,一些重要政治领袖或政治精英的价值观,很大程度上左右着一个国家的命运。

\tsection{英格尔哈特:政治文化的集大成者}

最近二三十年中,最有影响的政治文化学者要数美国密歇根大学的罗纳德 · 英格尔哈特教授,他发表了一系列有影响的论文与著作,被视为当代政治文化研究的集大成者。英格尔哈特政治文化研究的涉及面很宽,但他研究的核心问题是:在工业化国家,经济社会变迁和现代化会对政治文化产生何种影响?他研究中所涉及的现代化与后现代化问题、性别平等问题、宗教与世俗问题等等,都受到这一经济社会变迁进程的影响。

他 1988 年的论文《政治文化的复兴》是一项定量研究,在研究方法上对阿尔蒙德等人又有所发展。\nauthor{Ronald Inglehart, “The Renaissance of Political Culture,” \italic{American Political Science Review}, Vol.82, No.4(Dec., 1988), pp.1203-1230.}阿尔蒙德和维巴的定量研究主要是描述性的,英格尔哈特的则是推断性的。他后来还发起了 “世界价值观调查” 机构并成为该机构的主任。目前,世界价值观调查是政治文化领域全球最重要的数据库。

《政治文化的复兴》一文研究了公民的诸种态度及政治倾向与民主稳定性之间的关系。他的第一项内容涉及生活满意度和民主稳定性之间的关系。他对很多欧洲发达国家做了长期跟踪,从 20 世纪 70 年代跟踪到 80 年代,然后观察这些国家人们生活满意度的差异和变化。他注意到,人均 GNP 和生活满意度之间的相关性比较高。总体上,人均 GNP 越高,生活满意度越高。由此看来,经济发展水平是关键变量。当然,亦有例外。在英格尔哈特这项研究中,日本是属于人均 GNP 高但生活满意度低的国家,而爱尔兰是属于人均 GNP 低但生活满意度高的国家。所以,人均 GNP 水平只是一个重要的影响因子,但不是决定性的。在此基础上,他进行了生活满意度和民主稳定性的相关性研究。

第二项内容涉及人际信任和民主稳定性之间的关系。他调查的主要问题是:你相信别人吗?被调查者只能在是与否之间做选择。从他的调查结果来看,有些国家非常高,90\% 以上的人认为别人都是值得信任的;有些国家非常低,比如意大利——特别是意大利南部——就显著低于其他发达国家。很多游历过欧洲的人都有感受。比如,英国的小镇与意大利的小镇就有区别。在英国的小镇上,每家每户门前的栅栏或篱笆通常都比较低,不少只有一米来高,一个成年人可以轻易翻越。低矮的栅栏说明英国人对邻居和外人的防备心理比较低,人与人的信任度比较高。但意大利的小镇就是另外一个样子。比如,罗马附近的一些小镇上,很多独栋住房都有着非常高大的围墙,有厚实的大铁门,围墙上部加有尖锐的防护装置,不少家庭还有护院的大型犬类。这给人的直观印象是他们的人际信任程度要比英国为低。实际上,关于民众价值观的调查统计数据也支持这一结论。

那么,人际信任对政治有无影响呢?按照托克维尔对美国乡镇治理的观察,人际信任会影响一个国家公民的自治能力。越是人际信任度高的国家,越有可能发展出自治的治理方式;越是人际信任度低的国家,越有可能产生威权领导人和政治压制的统治方式。按照英格尔哈特的研究,人际信任程度跟人均 GNP 也有相关性,但这种关系并不那么确定。总的来说,人均 GNP 高的国家,人际信任度也较高。但同时人均 GNP 并不能决定人际信任度,比如美国人均 GNP 高于北欧诸国,但北欧的人际信任度高于美国,法国位于人均 GNP 最高系列的国家但其人际信任度却是较低的,参见图 10.1。

\img{../images/image00322.jpeg}[图 10.1 经济发展与人际信任]

资料来源:Ronald Inglehart, “The Renaissance of Political Culture,” \italic{American Political Science Review}, Vol.82, No.4(Dec., 1988), pp.1203-1230,figure 4。

第三项内容涉及对变革的态度:是支持激进变革,还是支持渐进变革,或是支持维持现状?这一问题考察的是被调查者对目前社会秩序的认可程度。支持激进变革,意味着被调查者对现有社会秩序是不认可的;支持维持现状,则是认可现有社会秩序;支持渐进变革,则是处于中间状态。一国人口在上述三种态度上的不同分布比例,是关乎政治文化的重要信息。

英格尔哈特的这项研究认为,一个国家的公民文化跟三个因素呈现显著的相关性:生活满意度、人际信任度和支持激进变革的程度,相关度分别高达 0.79、0.81 和 0.60,见图 10.2。这种相关度从统计学上讲是非常高的。数据显示,公民文化与民主政体的维系也存在较高的相关性,相关度达 0.74。所以,他的研究展示了不同的政治文化有着重要的政治后果,特别与民主政体的维系密切相关。当然,这项研究同时发现,经济因素对政治是非常重要的,人均 GNP 与公民文化之间的相关度也达到了 0.62。

\img{../images/image00323.jpeg}[图 10.2 稳定民主政体的经济与文化条件]

资料来源:Ronald Inglehart, “The Renaissance of Political Culture,” \italic{American Political Science Review}, Vol.82, No.4(Dec., 1988), pp.1203-1230, figure 6。

上述研究仅是英格尔哈特早期的一篇代表性论文。此后,他又发表了大量跟政治文化有关的论文与著作。比如,在 1990 年出版的《发达工业社会的文化转型》一书中,英格尔哈特以西方 20 多个发达国家的价值观调查为基础,探讨了经济社会变迁如何影响文化转型,以及文化转型又带来了怎样的政治与社会影响。他的核心观点是,随着西方发达工业国家 1973—1988 年间(更早可以追溯至 20 世纪 50 年代)的经济社会发展,大众的价值观念发生了重大的变化,开始从 “物质主义” 价值观转向 “后物质主义” 价值观。英格尔哈特的研究团队用 12 项价值观问题来进行社会调查,并以此来衡量受访者的价值观取向。这 12 个维度分别是:代表物质主义的 6 个维度——抵制物价上涨、经济增长、稳态经济(前面三者代表经济安全),以及维持秩序、打击犯罪、强大的国防力量(前面三者代表人身安全);代表后物质主义的 6 个维度——在政府中有更多话语权、在工作和社区中有更多话语权、人性化社会(前面三者代表归属与自尊),以及自由言论、想法更重要、美丽城市/自然(前面三者代表审美与知识),参见图 10.3。为什么会发生这种从物质主义价值观向后物质主义价值观的巨大变迁?英格尔哈特认为,这是由于二战之后西方发达国家获得了相对持久的和平和前所未有的经济繁荣,这样大众不再把已经实现的经济安全作为首要选项,而开始转向后物质主义的需要和诉求。\nauthor{罗纳德 · 英格尔哈特:《发达工业社会的文化转型》,张秀琴译,北京:社会科学文献出版社 2013 年版。}

\img{../images/image00324.jpeg}[图 10.3 物质主义与后物质主义价值观]

资料来源:罗纳德 · 英格尔哈特:《发达工业社会的文化转型》,张秀琴译,北京:社会科学文献出版社 2013 年版,第 149 页,图 4-3。

英格尔哈特 1997 年出版的《现代化与后现代化:43 个国家的文化、经济与政治变迁》一书,是对《发达工业社会的文化转型》这一研究的延续。他把研究范围扩大至 43 个国家样本。他明确认为,物质主义价值观是把经济和物质安全放在第一位,而后物质主义价值观是将自我表现和生活质量作为优先目标。两者的分野很大程度上与大众对安全的感知有关,一个普遍感知不安全的社会和一个普遍感知安全的社会,会呈现巨大的差异。这种感知的差异也会产生显著的政治、经济与社会后果,参见表 10.1。

\tbl{../images/image00325.jpeg}[表 10.1 安全与不安全:两种相对的价值体系]

资料来源:罗纳德 · 英格尔哈特:《现代化与后现代化》,严挺译,北京:社会科学文献出版社 2013 年版,第 43 页,表 1-1。

在此基础上,基于世界价值观调查数据,英格尔哈特确定两个维度来衡量世界各国的文化差异。一个维度是传统权威(traditional authority)对世俗—理性权威(secular-rational authority)的维度;另一个维度是生存价值观(survival values)对幸福价值观(well-being values)的维度。然后,他进一步绘制了基于这两个维度的价值观结构图,参见图 10.4。按照 1990—1993 年世界价值观调查的数据,他绘制了 43 个主要国家在上述图形中的分布谱系。从这些国家的分布来看,属于不同宗教传统的国家似乎具有某种集群的特征,这证明了宗教作为社会传统的重要性。

\img{../images/image00326.jpeg}[图 10.4 不同类型国家所强调的不同价值观]

资料来源:罗纳德 · 英格尔哈特:《发达工业社会的文化转型》,张秀琴译,北京:社会科学文献出版社 2013 年版,第 90 页,图 3-2。

总的来说,英格尔哈特这样表述他在《现代化与后现代化》这项研究中的发现:

\quo{经济发展、文化转型和政治转型以一种有着内在联系的、大体可预测的模式共同出现,社会经济转变的某些轨迹远远比其他轨迹更明显。

但是转变是非线性的。在发达工业国家里,主流发展方向是从现代化转变到后现代化。这条新的轨迹使得对作为工业国家标志的功能理性的强调出现衰退,而对自我表现和生活质量的强调在增加。随着后现代主义价值观日渐扩散化,从妇女权利平等到民主政治制度的各种社会转变,以及国家社会主义政权的衰落,都变得日益可能。随着后现代化进程的推进,一种新的世界观正在逐步替代工业革命以来一直支配工业化国家的信仰框架。它反映了某个问题上的态度转变,即人们到底渴望从生活中得到什么这一问题。它正在转变那些支配政治、工作、宗教、家庭和性行为的基本规范。\nauthor{罗纳德 · 英格尔哈特:《现代化与后现代化》,严挺译,北京:社会科学文献出版社 2013 年版,第 372 页。}}

在与克里斯蒂娜 · 维尔泽合著的《现代化、文化变迁与民主:人类发展时序》(\italic{Modernization, Cultural Change and Democracy: The Human Development Sequence})一书中,英格尔哈特认为,关于现代化理论中经济发展导致政治民主的单一线性假说是有问题的,但是长期当中,经济发展驱动了文化转型,而后者使得个人自主、性别平等和民主政治更加成为可能。他们认为,现代化过程中的价值观念变迁可以分为两个阶段:一是从传统权威向世俗—理性权威的转变,二是从生存价值观向自我表现价值观的转变。他们这项研究采用的数据是覆盖 80 个国家、全球 85\% 人口的四次世界价值观调查,并借助多元回归模型进行了量化分析,论证了上述结论。\nauthor{Ronald Inglehart and Christian Welzel, \italic{Modernization, Cultural Change, and Democracy: the Human Development Sequence}, Cambridge: Cambridge University Press, 2004.}

上面已经提到,英格尔哈特的一项重要工作是参与了 “世界价值观调查” 。在 “世界价值观调查” 机构的网站,可以下载每一个调查年份的调查问卷。\nauthor{世界价值观网站链接:http://www.worldvaluessurvey.org。}最近的 2010—2012 年度调查问卷由 200 多个问题组成,除了少数问题涉及个人背景信息外,多数调查都是关于一个人的价值观和态度,下面试举几例来简要说明。在调查问卷中,第 4—9 个问题问的是被调查者对于诸种重要价值的排序,第 10—11 个问题问的是被调查者生活快乐程度与健康感知程度,第 12—22 个问题问的是被调查者认为培养孩子哪些品质最为重要,参见表 10.2。

\tbl{../images/image00327.jpeg}[表 10.2 世界价值观调查问卷示例(2010—2012)]

\img{../images/image00328.jpeg}

比如,第 4—9 个问题问的是:下面每一个选项在你生活中有多重要?这六个选项分别是家庭、朋友、闲暇时间、政治、工作和宗教,可供选择的答案有:非常重要、较重要、不太重要或根本不重要。如果认为所有六项都非常重要,当然也可以,但那样填写问卷意义就不是太大。一般人回答这些问题时,应该对重要性有所区分,大致能够区分哪些非常重要或较重要,哪些不太重要或根本不重要。当样本数量足够大时,被调查者对上述诸种价值的排序就具有了意义。这些排序的不同,往往展示了不同的政治文化特质。

第 12—22 个问题问的是:对培养小孩来说,哪些品质尤为重要?问题说明是在总共 11 个选项中最多可以选 5 项。这 11 项分别是:独立性、努力工作、责任感、想象力、宽容和尊重他人、节俭与省钱、决心和毅力、宗教信仰、无私、服从、自我表达与表现。在中国、美国、俄罗斯、南非或者印度做调查,最后得到的结果会有很大的差异。这种价值观差异的背后就是政治文化的差异。

当然,政治文化的研究也遭到很多学者的质疑。本书第 8 讲曾介绍过对政治文化研究的挑战,此处不再赘述。总体上,这些挑战都非常严厉。因此,政治文化领域需要更好的研究范式和研究成果。

\tsection{社会资本理论的兴起}

社会资本理论有时被视为政治文化研究的一部分。过去 20 年中, “社会资本” 成了社会科学领域的一个热门词汇。人们通常把社会资本理论与一部重要著作联系在一起,即罗伯特 · 帕特南 1993 年出版的《使民主运转起来》。提到社会资本时,帕特南这样说:

\quo{社会资本指的是普通公民的民间参与网络,以及体现在这种约定中的互惠和信任的规范。

社会资本是指社会组织的特征,诸如信任、规范以及网络,它们通过促进合作行为来提高社会的效率。\nauthor{罗伯特 · D.帕特南:《使民主运转起来》,王列、赖海榕译,南昌:江西人民出版社 2001 年版,中译本序第 1 页、正文第 195 页。}}

较早讨论社会资本概念的学者应该是科尔曼,他总体上算是一个社会学家,但他有着良好的经济学背景,所以更容易想到 “资本” 这个词。在他看来,社会资本是一种责任与期望、信息渠道以及一套规范与有效的约束,它们能限制或者鼓励某些行为。经合组织(OECD)在其研究报告中把社会资本定义为: “社会资本是一种网络以及共享的规范、价值观念和理解,它们有助于促进群体内部或群体之间的合作。” 福山也被视为研究社会资本的重要学者,他认为社会资本是——

\quo{群体成员之间共享的非正式的价值观念、规范,能够促进他们之间的相互合作。如果全体的成员与其他人将会采取可靠和诚实的行动,那么他们就会逐渐相互信任。信任就像是润滑剂,可以使人和群体或组织更高效的运作。\nauthor{Francis Fukuyama, “Social capital, Civil Society and Development,” \italic{Third World Quarterly}, Vol.22, No 1(Feb., 2001), pp.7-20.}}

按照上述诸种定义,社会资本大概有几个主要特征:首先,它不是正式的制度安排;第二,它存在于人与人之间、群体与群体之间的网络之中;第三,它总体上跟人际互动、互惠机制、合作互助、信任关系这些东西有关。帕特南认为: “在现代的复杂社会里,社会信任能够从这样两个互相联系的方面产生:互惠规范和公民参与网络。” 

那么,社会资本为什么重要?有学者用闯不闯红灯为例来解释。到了十字路口,如果每个人都不闯红灯,对大家都有利;如果都闯红灯,对大家都不利。那么,为什么有的社会众人会闯红灯,而有的社会众人不闯红灯呢?当然,制度规制与惩罚是重要的。但除此之外,社会资本高的社会更能促成众人在红灯问题的合作。詹姆斯 · 科尔曼在他的著作中有一种更为严谨的学理阐述:

\quo{像其他形式的资本一样,社会资本也是生产性的,它使得某些目标的实现成为可能,而在缺乏这些社会资本的情况下,上述目标就无法实现……例如,一个团体,如果其成员是可以信赖的,并且成员之间存在着广泛的互信,那么它将能够比缺乏这些资本的相应团体取得更大的成就……在一个农业共同体中……那里的农民互相帮助捆干草,互相大量出借或借用农具,这样,社会资本就使得每一个农民用更少的物质资本(如农具和设备)干完了自己的农活。\nauthor{参见詹姆斯 · S.科尔曼:《社会理论的基础》,邓方译,北京:社会科学文献出版社 1999 年版,第 351—376 页。}}

那么,社会资本具有何种政治效应呢?这就需要提到让社会资本概念走俏的学术名著《让民主运转起来》。帕特南在书中研究的是 20 世纪 70 到 90 年代意大利的地方分权改革。从意大利 15 个地区政府分权改革的绩效来看,北部明显好于南部。帕特南提出的问题是:同样是地方分权改革,为什么有些地区成功而有些地区不成功呢?可能的解释包括制度设计理论、社会经济条件理论以及社会文化因素理论,等等。帕特南把因变量设定在政策的制定、颁布和实施三个方面,包括 12 项指标。他的研究结论是:经济发展程度与民主改革绩效的相关度是 0.77——这已经很高了;但公民共同体传统与民主改革绩效的相关度高达 0.92,后者的显著性远远超过前者。他认为,在一个具有良好公民共同体传统的社会,自愿合作更容易出现,互相信任更有可能,互惠网络更容易形成——而这些方面意大利北部做得比南部更好。

帕特南的这项研究出版之后,引起了很大的轰动。后来,又有大量学者跟进社会资本的研究。正如上文的科尔曼那样,不少学者都认为社会资本使得某些目标的实现成为可能,而在缺乏这些社会资本的条件下,这些目标则难以实现。比如,如果人际信任度很高,交易成本就会比较低;如果彼此不太信任,整个交易过程需要反复考核对方,那么交易费用就会非常高。实际上,很多发展中国家有大量资源耗费在交易成本上,其生产效率和经济绩效就会受到影响。福山认为: “在经济领域,社会资本能够降低交易成本;在政治领域,社会资本能够有助于有限政府和现代民主制的成功。” 

自从帕特南的专著出版以来,国际学术界已经出版了大量与社会资本有关的专著和论文。按照全球最大网络书店亚马逊网站的数据,全网站的书名、章节名及内容简介中出现 “社会资本” 字样的高达 1 万条次以上。这说明,社会资本已经成为一个热门领域。

尽管如此,社会资本理论也遭到很多批评。其中两个主要的批评是:第一,如何衡量社会资本?这是社会资本研究需要回应的挑战,尽管有人做了很多努力来衡量社会资本,但衡量和测定社会资本的挑战仍然是很大的。第二,社会资本这个概念由于缺少明确所指,容易被拿来作为一个可以蒙混过关的解释变量。究竟什么是社会资本?有学者认为,从帕特南到科尔曼都没有说得很清楚。这样,社会资本这一概念容易成为一个包罗万象的框。这是社会资本理论需要正视的挑战。

\tsectionnonum{推荐阅读书目}

加布里埃尔 · 阿尔蒙德、西德尼 · 维巴:《公民文化——五个国家的政治态度与民主制》,徐湘林等译,北京:东方出版社 2008 年版。

罗纳德 · 英格尔哈特:《现代化与后现代化》,严挺译,北京:社会科学文献出版社 2013 年版,第 372 页。

罗伯特 · 帕特南:《使民主运转起来》,王列、赖海榕译,南昌:江西人民出版社 2001 年版。

塞缪尔 · 亨廷顿、劳伦斯 · 哈里森主编:《文化的重要作用:价值观如何影响人类进步》,程克雄译,新华出版社 2010 年版。
