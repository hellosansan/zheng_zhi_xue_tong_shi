\tchapter{民族主义与族群政治}

\quo[——莫里斯 · 布洛克]{如果要用一个对句来概括我们的民族原则,我们可以说:如果民族原则是用来把散居的群体结合成一个民族,那么它就是合法的;但若是用来分裂既存的国家,就会被视为非法。}

\quo[——本尼迪克特 · 安德森]{它(民族)是一种想象的政治共同体——并且,它是被想象为本质上是有限的,同时也享有主权的共同体。}

\quo[——唐纳德 · 霍洛维茨]{族群冲突是一个世界性的现象。}

\quo[——莫妮卡 · 托夫特]{现在几乎三分之二的武装冲突都包含了族群因素。……族群冲突是武装冲突的最主要形式,在较短时期内、甚至较长时期内大概都不会缓和。……仅在二战之后,就有数百万人因为身为特定族群的一员而丧命。}

\tsection{什么是民族主义?}

在世界范围内,民族主义依然扮演着重要的政治角色。比如,以东北亚的朝鲜和韩国为例,大家会发现两国的共同点并不多。从政治体制、经济发展、生活方式等维度来看,两国差异都是极大的——一个是最落后的发展中国家,一个是经济发达的新兴工业化国家。尽管如此,两国有一个重要的共同点:都是朝鲜族。朝鲜和韩国在历史上属于一个国家,他们彼此认为两国都属于一个更高的共同体的一部分。这个共同体就是统一的朝鲜民族。从这个案例看,民族有时甚至可以超越国家。尽管两国已分裂六十多年,而且互相承认对方为独立国家,但两国的政治家仍然宣称实现朝鲜半岛的统一是其政治目标。

又比如,捷克斯洛伐克曾经是一个欧洲国家,但今天已和平分裂为两个国家:捷克和斯洛伐克。该国分裂的主要原因也是民族问题,过去捷克斯洛伐克版图的西侧主要是捷克族的聚居区,东侧主要是斯洛伐克族的聚居区,两族人口占东区和西区人口的比例都超过 80\% 。鉴于这种地理与人口结构,两个民族认为他们没有必要成为一个国家,可以和平分手。1993 年 1 月 1 日,捷克斯洛伐克正式和平分裂,史称 “天鹅绒分离” 。

前南斯拉夫则是一个较为悲惨的案例。前南斯拉夫今天已分裂为七个国家,其分裂跟该国历史上各个族群的仇怨有关。客观地说,历史关系复杂的多族群国家,如果启动民主转型,国家分裂是一种可能的风险。启动民主转型之后——特别是族群人口聚居的结构下,不同族群的选民都会投自己族群候选人的票,组建本族群的政党。有人甚至会提出来,既然我们这块地方绝大多数都是斯洛文尼亚人,可不可以独立呢?当马其顿、克罗地亚、斯洛文尼亚、塞尔维亚、黑山等族群都这样想的时候,南斯拉夫最后就分裂了。

从韩国与朝鲜,到捷克和斯洛伐克,再到南斯拉夫,大家会发现,上述政治现象都跟民族主义与族群政治有关。

什么叫民族主义?伦敦政治经济学院安东尼 · 史密斯教授在《民族主义:理论、意识形态、历史》中认为民族主义是——

\quo{一种为某个群体争取和维护自治、统一和认同的一种意识形态运动。该群体的部分成员认为有必要组成一个事实上的或潜在的一种 “民族” 。\nauthor{安东尼 · 史密斯:《民族主义:理论、意识形态、历史》(第二版),叶江译,上海:上海人民出版社 2011 年版,第 9 页。}}

按照他的观点, “民族主义的基本目标有三个:民族自治、民族统一和民族认同” 。民族自治是指每个民族在政治上应该是自治和自决的,能够决定自己的命运。这样的民族不需要依赖于别的民族,也不允许别的民族来左右自己的命运。民族统一是指同一个民族应该成为一个统一的政治共同体。在现代世界,统一的政治共同体通常是一个国家。民族认同是指属于同一民族的成员能够发展出一种强烈的成员意识,以及归属于同一民族的团体感和自豪感。所以, “民族主义是将民族作为关注的焦点,并力求促进民族利益的一种意识形态” ,旨在推动民族自治、民族统一和民族认同。

那么,到底什么是民族?理解民族有两种主要路径:客观路径与主观路径。过去苏联教科书中对民族的定义是客观路径,这一观点认为:民族是人们在历史上形成的一个有共同语言、共同地域、共同经济生活以及表现于共同文化上的共同心理素质的稳定的共同体。而主观路径是把民族视为一个 “想象的共同体” ,这是本尼迪克特 · 安德森的著名观点。安德森认为,民族 “是一种想象的政治共同体——并且,它是被想象为本质上是有限的(limited),同时也享有主权的共同体” 。在他看来,这些 “想象的共同体” 的缘起主要取决于宗教信仰的领土化、古典王朝家族的衰微、时间观念的改变以及资本主义与印刷术之间的交互作用等。特别是,他提到——

\quo{也许没有什么东西比印刷资本主义更能加快这个追寻的脚步,并且使之获得更丰硕的成果了,因为,印刷资本主义使得迅速增加的越来越多的人得以用深刻的新方式对他们自身进行思考,并将他们自身与他人关联起来。\nauthor{本尼迪克特 · 安德森:《想象的共同体:民族主义的起源与分布》,吴叡人译,上海:上海人民出版社 2008 年版,第 33 页。}}

而安东尼 · 史密斯试图融合民族的客观定义与主观定义,他把民族定义为: “具有名称在感知到的故土上居住,拥有共同的神话、共享的历史和与众不同的共同文化,所有成员拥有共同的法律与习惯的人类共同体。” \nauthor{安东尼 · 史密斯:《民族主义:理论、意识形态、历史》(第二版),叶江译,第 13 页。}

在政治思潮的演进中,民族主义如今已经成为一种重要的意识形态。那么,民族主义有哪些基本主张呢?安东尼 · 史密斯认为,民族主义有个六个基本主张:\nauthor{安东尼 · 史密斯:《民族主义:理论、意识形态、历史》(第二版),叶江译,第 25 页。

第一,世界由不同的民族所组成,每个民族都有自己的特定、历史和认同;

第二,民族是政治权力的唯一源泉;

第三,还有对于民族的忠诚超出对其他所有的忠诚;

第四,为了赢得自由,每个个人必须从属于某个民族;

第五,每个民族都需要完全的自决与自治;

第六,全世界的和平和正义需要一个各民族自治的世界。}

在民族主义意识形态的范围中,上述六个主张都是基本常识。但是,其他意识形态可能有着完全不同的观点。比如,自由主义者会认为,世界主要是由不同的个人组成的,民族未必有那么重要。举例来说,一个人 20 岁以前生活在中国,20 岁以后去了英国读书,在英国读了 6 年书又工作了 10 年,后来他又移民并加入了加拿大籍。那么,该如何界定这个人的民族身份呢?美国已经出过不止一位华裔政府部长,其中一位叫赵小兰。有一次,赵小兰到中国访问,有位记者前去采访,其中一个提问大致是:身为中国人,您对某事有何看法?结果,赵小兰首先声明的是: “对不起,我是美国人。” 当然,这位记者的真实意思应该是赵小兰作为华人——记者在此混淆了族裔身份与国籍——是如何看待某个问题的。但无论如何,这一案例都显示,换一种视角,大家对世界、国家、民族与个人关系见解会非常不同。

民族主义认为,民族是政治权力的来源。但有人认为, “人权是高于主权” 的。比如,捷克前总统哈维尔就认为,如果在一个民族政治共同体中发生了大规模的践踏人权事件,国际社会是有权而且应该干预的。这种干预的合法性就在于 “人权高于主权” 的主张。比如,在德意志第三帝国时期,希特勒大规模地屠杀德国的犹太人。假如希特勒没有把他的战车驶出德国边界,他只是在德国境内进行族群屠杀,国际组织有权进行政治或军事干预吗?当然,这是一种较为严酷的情形。不同的意识形态对此会有不同的看法。

尽管民族主义具有诸多的共有特征,但在其内部,不同类型的民族主义差异很大。历史地看,存在过几种不同类型的民族主义,比如自由主义的民族主义——寻求民族自决的美国独立战争呈现的是这样一种特征;保守主义的民族主义——欧盟内部的反移民运动同时兼具保守主义和民族主义两种特质,扩张主义的民族主义——纳粹德国当年追求扩张与侵略的做法即是一例,反殖民主义的民族主义——印度谋求国家独立的民族主义运动展现的基本诉求就是反对殖民主义。这些都是民族主义意识形态内部的不同类型。\nauthor{相关内容,参考安德鲁 · 海伍德:《政治学》(第二版),张立鹏译,北京:中国人民大学出版社,第 138—149 页。}

需要注意的是,正如安东尼 · 史密斯所指出的, “民族不是国家也不是族群” 。举例来说,苏联和南斯拉夫都是一个国家,但他们都不是一个民族。这两个国家内部,都有大量的不同民族。又比如,英国是一个国家,但英国包括了英格兰、苏格兰、威尔士和北爱尔兰等四个主要地区,这些地区生活着不同的民族。苏格兰人与英格兰人生活在不同的地方,他们说不同的语言,具有不同的历史文化传统。当我们说英国人的时候,是指国家意义上的。而当我们说日耳曼人、塞尔维亚人、苏格兰人时,是指某个特定的民族。所以,民族不同于国家。

那么,民族与族群区别何在呢?安东尼 · 史密斯认为:

\quo{民族不是族群,因为尽管两者有某种重合并都属于同一类现象(拥有集体文化认同)。但是族群通常没有政治目标,并且在很多情况下没有公共文化;且由于族群并不一定要有形地拥有其历史疆域,因此它甚至没有疆域空间。而民族至少要在相当的一个时期必须在其自己所谓的祖国中定居,以将自己构建成民族;而且为了立志成为民族并被承认,它需要发展某种公共文化,以及追求相当程度的自决。另一方面,就如我们所见到的,民族并不一定要拥有一个自己的主权国家,但需要在对自己故乡有形地占有的同时,立志争取自治。\nauthor{安东尼 · 史密斯:《民族主义:理论、意识形态、历史》(第二版),叶江译,第 12 页。}}

尽管如此,但民族与族群很多时候还是难以区分。史密斯将政治目标、公共文化、疆域空间以及是否寻求自决作为区分两者的主要标准。但这种区分可能仍然是语焉不详的。笔者的基本判断是,如果一个族群或族群内部的多数人口发展出一种普遍的政治诉求——本族群应该成为一个相对自治、甚至独立的政治共同体——时,这个族群基本上已上升为一个民族。当然,如果由于某些原因,这种政治诉求不能兑现,长期处于被压制状态,甚至慢慢地被压服了,直至不再提出要求自治或独立的主张了,这个群体可能又演变为一个族群,成为一个民族共同体的一部分。关于族群政治,后面还会详细介绍。

\tsection{民族主义的起源与理论}

前面简要介绍了民族主义的概念与特征,政治学者们更关心作为一种意识形态的民族主义为什么会兴起?如何从理论上解释民族主义?多数学者把民族主义视为一种现代现象,因而民族主义的兴起是与人类向现代社会的转型有关的。循着这一视角,安东尼 · 史密斯总结了解释民族主义的几种主要理论:\nauthor{关于民族主义的几种主要理论,参见安东尼 · 史密斯:《民族主义:理论、意识形态、历史》(第二版),叶江译,第 52—53 页。}

一是社会经济的解释。 “在这一视角中,各种民族主义和民族源自新型的经济社会因素,如工业资本主义、区域不平等和阶级冲突。” 史密斯较为重视现代化过程中地区不平等在民族情感激发和民族理想塑造所具有的作用。还有一种角度来自于法国社会学家涂尔干的启发,他认为社会团结有两种不同形式:一种是机械团结,一种是有机团结。机械团结,如同马克思在论述亚细亚生产方式时讲的 “一袋马铃薯” 一般;而有机团结指社会中不同的人扮演不同的角色,然后彼此融合交织在一起,形成互赖的社会组织方式。总的来说,随着经济社会的现代化,一个社会中人与人的交易频率大大提高,彼此的融合与结合程度大大提高,整个社会的互赖程度也大大提高了。这就促成了共同体意识的成长,在此过程中民族主义逐渐形成。

二是社会文化的解释。 “根据欧内斯特 · 盖尔纳的观点,各种民族主义和民族都是在 ‘现代化’ 转型过程中产生的,是现代的、工业化时代的必然社会现象。” 这里强调的是,民族和民族主义是一种高级文化现象,而正是现代化实现对社会文化的改造,作为现代现象的知识增长和教育普及是其基础。另一种角度则强调地理上的接近和经济上的相似会催生出相似的文化,然后形成共同的传统。在现代化过程中,这种文化上的相似性会逐渐上升到民族主义。

三是政治的解释。 “在这里,民族和民族主义是由现代专业化国家,或直接地,或在对抗特定的(帝国的/殖民的)国家中所造就的。” 换句话说,民族和民族主义是现代国家兴起的伴随物。一个例子是国际竞争与群体生存需要促成民族主义的兴起。比如,德国民族主义的兴起是拿破仑战争刺激的产物。拿破仑战争使得德意志这块土地上的人们产生了强烈的焦虑感。从地缘政治格局来说,德意志东有俄罗斯,西有法兰西。德意志——特别是普鲁士——的精英阶层认为,他们要是不以某种方式联合起来,形成一个强大的民族和国家,德意志的生存都会成问题。马克斯 · 韦伯在《民族国家与经济政策》的演讲中也表达了这样的焦虑感。所以,对当时的德意志来说,可以理解为民族主义是由国际政治竞争和地缘政治局势引发的。

四是意识形态的解释。 “这种视角强调民族主义意识形态的欧洲本源及其现代性;强调民族主义类似宗教的力量,以及它在分裂帝国和在没有出现民族的地方创立民族所起的作用。” 这种视角认为启蒙运动在动员民族主义方面发挥了重要作用,所以民族主义本身是现代意识形态兴起的产物。

五是建构主义的解释。 “这种视角与其他的现代主义形式有相当的不同,尽管它也采纳民族和民族主义是完全现代的观点,但是却强调它们的社会建构特征。” 本尼迪克特 · 安德森所谓的民族是 “想象的共同体” ,就是建构主义的路径。这种观点把民族主义的出现跟工业资本主义引发的印刷品和阅读的普及关联起来。总的来说,民族是通过想象来构建的。所以,在民族主义兴起过程中,印刷品和媒体扮演着重要角色。当一个国家的精英阶层和普通大众都开始阅读类似的报纸与杂志时,共同体意识就开始被构建起来了。

总之,民族主义作为一种重要的意识形态和政治现象,可以从不同的理论视角来解读。

\tsection{民族国家与族群政治}

跟民族有关的一个重要概念是民族国家(nation-state 或 nation state)。民族国家被视为一种政治组织形式和政治理想,是指民族和国家的重叠状态。所以,严格意义上的民族国家是指,一个国家就是一个民族,一个民族成为一个国家,亦即意大利政治家马志尼所言的 “一国一族,一族一国” 。但是,如今人们所习惯使用的民族国家通常很少是由一个民族构成的。今天看来,一般意义上的国家——只要内部没有太明显的民族或族群裂痕的话——通常都被视为民族国家。所以,民族国家这个概念现在有了很多约定俗成的用法,而未必是在其最初的严格定义上使用的。

如今一个显著的世界性现象是,大量国家都是多族群国家。既然那么多国家都是多族群社会,族群政治就是值得关注的一个现象。特别是在非洲、南亚与东南亚、东欧等地区,族群政治都是重要的政治问题。比如,像尼日利亚和印度这样的国家,族群政治甚至是最重要的政治议题。然而,国内过去对族群政治的研究和介绍比较少。随着中国边疆省区暴力恐怖事件的抬头,这一问题正在引起国内学界的重视。

前面已经对族群的概念做过简略讨论,但实际上要想精确定义族群非常困难。现有的主流研究认为,族群被视为基于血缘或世系而互相认同的一个群体,这个群体拥有共同的语言文化、宗教习俗及身体特征。马丁 · 麦格认为,族群具有如下主要特征:

\quo{独特的文化特征基本上,族群是指在一个较大的社会里有一套自己独特的文化特质的群体。……族群是亚文化群体,它们保持的特定的行为特征,在某种程度上使它们区别于社会主流文化或典型文化。

社群意识除了一套共同的文化特性,族群成员之间展现出一种社群意识,也就是一种亲切的感觉或紧密联系的意识;更简洁地说,就是在族群成员之间存在着一个 “我们” 。

族群中心主义/优越感族群成员的 “我们” 意识通常都会自然而然地导致族群中心主义(ethnocentrism),即倾向于用某个族群的标准和价值评价其他的群体。这将必然导致该族群认为自己优越于其他族群。

与生俱来的成员资格族群的成员资格通常是出生时获得的。也就是说,一个人的族群性是出生时获得的一种特征,并且不容易发生根本性变化。

领地族群常常在一个较大的社会中占据一个独立的区域。\nauthor{马丁 · N.麦格:《族群社会学》,祖力亚提 · 司马义译,北京:华夏出版社 2007 年版,第 9—13 页。}}

既然多数国家都是多族群国家,族群间的政治关系就非常重要。多族群国家的国内族群关系主要有两种类型:一种是以竞争与冲突为主,一类是以合作与融合为主。当然,有些国家族群之间的关系是既冲突又融合的中间模式。无疑,对一个国家来说,合作与融合为主的族群关系是较为理想的类型。然而,目前仍然有大量国家的族群关系以竞争与冲突为主。所以,作为政治问题的族群冲突及其后果是目前国际学界的热门议题。

今天,欧洲国家的族群问题总体比较缓和,但欧洲历史上也出现过比较严重的族群政治问题。米歇尔 · 韦耶维欧卡认为,对二战以前的多数西欧国家来说,族群和族群政治仍然是一个非常重要的政治问题,但后来族群问题慢慢趋于缓和。这位学者指出,西欧国家主要通过三个途径来实现了民族整合和族群融合:\nauthor{Michel Wieviorka, “Racism in Europe: Unity and Diversity,” in Montserrat Guibernau and John Rex, eds., \italic{The Ethnicity Reader}, Cambridge: Polity, pp.291-302.}

第一个途径是现代化,即充分发展工业化和工业社会。总的来说,现代化程度越高,原先基于原始身份的认同就会降低。经济因素对族群身份认同改变的影响非常大。比如,中国某省区最近几年发生多次涉及族群问题的恐怖主义暴力事件。如果只提一条政策建议的话,笔者认为,那就是要尽快地让该区的所有人口——特别是少数族群人口——融入全国性的生产、分工与交易系统,融入全国性的市场网络与经济系统。这种融入的速度越快,越有利于族群关系的稳定。如果少数族群每天都要生产和交易,他们一方面要把本地的牧业产品、资源产品及特色产品销售给内地的企业和消费者;另一方面要不断地购买和消费内地生产的工业产品。这样,双方就产生了市场层面的互赖,由此不同族群之间的融合速度就会加快。所以,现代化、工业化和统一的市场网络非常重要,现代化长期当中会促进族群融合。

第二个路径是建立一个平等主义的国家。什么样的国家最容易导致内部的族群冲突呢?查尔斯 · 蒂利称之为 “种类不平等” ,在族群问题上即不同族群之间的政治与经济不平等。很多国家可能事实上都会存在种类不平等的问题,但有的国家在立国原则上崇尚平等主义。比如,在美国,白人群体的平均收入比黑人高,这是一种事实上的种类不平等。但是,在法律原则和基本政策方面,黑人已在政治、经济及社会权利上获得了跟白人同等的地位。所以,美国奉行的就是平等主义原则。如今,美国黑人族裔都能当选总统。但是,在很多别的国家,不同族群之间的情况可能大不一样。按照欧洲的经验,建立一个平等主义的国家是非常重要的做法。这里的平等主义主要是指形式的平等和资格的平等,即所有人都有同等的政治、经济与社会资格。由于公民法律身份的平等,公民的族群身份和族群认同就大大淡化了。当然,一个社会若能在法律平等的基础上,推进不同族群之间的实质性平等,将更有利于缓和族群关系。相反,一个国家如果使公民因户籍、城乡、体制内外等因素发生严重的隔离,必然不利于塑造平等主义的国家,也不利于强化族群融合。

第三个路径是塑造民族认同。这里民族认同是超越族群认同之上的、基于民族国家身份的认同感。对一个苏格兰人来说,他首先认为自己是英国人还是苏格兰人,这是一个重要问题,关系到这个公民的民族认同。拿法国来说,考虑到移民问题可能会弱化法国公民的民族认同,法国专门设立了移民部负责归化事务,其目的都是为了加强法国国内的族群融合和民族认同。但是,有些国家在身份制度上强调公民的族群身份,这种做法与塑造民族认同、强化族群融合的原则背道而驰。诸如此类的政策问题,都需要系统的检讨与反思。

二战前的多数欧洲国家,族群问题是一个远比今天普遍的现象。此后,这些国家借助上述三个途径进行了成功的民族整合。这些政策弱化了公民的族群身份,强化了公民身份。这样,欧洲国家的族群政治问题就趋于缓和了。总之,欧洲国家应对族群问题的这种政治经验是值得其他国家借鉴的。当然,今天欧洲的少数地方仍存在族群政治问题,比如西班牙的巴斯克地区和英国的苏格兰地区。

\tsection{族群政治与政治发展}

对今天的世界来说,族群政治与族群冲突是一个重要问题。美国族群政治学者唐纳德 · 霍洛维茨认为: “族群冲突是一个世界性的现象。” \nauthor{关于族群冲突,参见这部高质量的族群政治名著:Donald L. Horowitz, \italic{Ethnic Groups in Conflict}, Berkeley: University of California Press, 1985。}莫妮卡 · 托夫特则认为:

\quo{现在几乎三分之二的武装冲突都包含了族群因素。……族群冲突是武装冲突的最主要形式,在较短时期内、甚至较长时期内大概都不会缓和。……仅在二战之后,就有数百万人因为身为特定族群的一员而丧命。\nauthor{Monica D. Toft, \italic{The Geography of Ethnic Violence: Identity, Interests, and the Indivisibility of Territory}, Princeton: Princeton University Press, 2003, p.3.}}

在历史上,印度尼西亚这样的东南亚国家、印度这样的南亚国家、尼日利亚这样的非洲国家、前南斯拉夫这样的欧洲国家,出现过与族群政治有关的大量的暴力事件。族群冲突还是 20 世纪中叶之后全球范围内国内武装冲突和内战的最主要诱因。现有研究认为,二战以后 50\% —70\% 的国内武装冲突与内战都跟族群问题有关。从现有的趋势看,族群冲突在相当长时间内仍然是很多发展中国家国内政治的主要挑战。

国内学界过去关于族群政治与族群冲突的研究比较少,但国际新闻中族群暴力报道的增加及国内陆续出现与少数族群有关的暴力恐怖事件,使得国内学界开始重视这一领域的研究。从国际学界来看,族群政治是最近一二十年最受关注的比较政治研究领域之一。

国际上有研究关注一个国家的族群结构跟该国族群冲突之间的关系。图 11.3 用描述性定量统计分析了族群分化程度(纵轴)与族群极化程度(横轴)之间的关系。这里的族群分化程度衡量的是一个国家内部的族群结构,族群极化程度衡量的是一个国家内部的族群冲突程度。图 11.1 揭示的相关性是:在族群分化程度较低的阶段,随着族群分化程度的提高,族群冲突的程度会增加;但族群分化高到一定程度之后,随着族群分化程度的继续提高,族群冲突反而会趋于缓和。

\img{../Images/image00329.jpeg}[图 11.1 族群分化程度与族群冲突程度的相关性]

资料来源:José G. Montalvo and Marta Reynal-Querol, “Ethnic Polarization, Potential Conflict, and Civil Wars,” \italic{The American Economic Review}, Vol.95, No.3(Jun., 2005), pp.796-816, Figure 1。

举例来说,第一种情况是某国 90\% 人口属于一个族群,即该国族群分化程度很低。这种情况下,发生族群冲突的可能性较低。第二种情况是该国 30\% 人口属于 A 族群,25\% 人口属于 B 族群,20\% 人口属于 C 族群,剩下 25\% 人口属于其他多个族群。与上一种情况比较,该国的族群分化程度大大提高了,相应地该国族群冲突程度也大幅提高。从已有经验来看,如果一个国家存在两个或几个主要族群,它们彼此竞争又势均力敌,族群冲突可能会非常严重。第三种情况是该国最大单个族群所占人口的比例不过 10\% ,后面几个较大族群人口比例依次仅为 8\% 、6\% 、5\% 和 4\% ,其余人口属于规模更小但数量众多的不同族群。与第二种情况相比,这个国家的族群分化程度更高,但族群冲突程度反而会降低。理由在于,由于整个国家的人口像一个族群万花筒,不同族群之间反而不易形成对抗关系,族群冲突因此趋于缓和。当然,上述相关性是基于跨国的大样本数据得出的,具体到每一个国家,族群政治的实际情形往往千差万别。

族群政治通常还跟发展中国家的政治发展和民主转型有关。在发展中世界,很多国家的族群冲突正在成为威胁新兴民主政体稳定的主要问题。从已有研究来看,族群冲突与民主冲突是互相影响的。一方面,民主转型有可能影响一国族群冲突的程度;另一方面,族群冲突会影响、甚至决定一国民主转型的成败。有学者研究发现,对多族群国家来说,民主转型前期有可能加剧族群冲突;随着民主政体维系时间的延长,族群冲突会趋于缓和。\nauthor{比如,下面这项研究认为,民主化与族群冲突两者之间是一条 “倒 U 曲线” 的关系:Demet Yalcin Mousseau, “Democratizing with Ethnic Divisions: A Source of Conflict?” , \italic{Journal of Peace Research}, Vol.38, No.5, 2001, pp.547-567。}在图 11.2 中,纵轴是族群冲突的程度,横轴是民主政体的维系时间。该图显示,对于一个族群多样化程度很高的威权国家来说,随着民主转型的启动,该国的族群冲突水平可能出现先上升后下降的过程。换句话说,在一个多族群国家,初始的民主政体往往比此前的威权政体具有更高的族群冲突水平;而巩固的民主政体往往比此前的威权政体具有更低的族群冲突水平。

\img{../Images/image00330.jpeg}[图 11.2 民主政体维系时间与族群冲突程度]

那么,如何解释民主转型与族群冲突之间前高后低的现象呢?在威权政体之下,尽管国内存在不同族群并且这些族群有着历史上的恩怨关系,但由于缺少充分的政治参与,不同族群集团的政治诉求被压制了,这样就表现为直接的族群冲突程度较低。但是,随着民主转型的启动,所有族群集团都拥有政治参与和政治竞争的权利,都可以通过公开方式表达政治诉求,甚至都组建政党参与政治竞争。这样,特别是对落后国家来说,基于原始身份认同的族群诉求会快速上升。在这样的国家,族群身份会成为主要的政治动员手段。所有这些,都可能会推动族群冲突的快速上升。而当民主转型时间较长、民主政体趋于巩固时,不同族群集团开始学会用和平而非暴力、制度化而非冲突方式来表达政治诉求和处理族群关系,这样族群冲突程度会逐步降低。因此,长期当中,民主政体维系时间越长,族群冲突水平就会越低。

上述分析意味着,当多族群国家启动民主转型时面临着一个严峻的挑战——该国民主政体能否在急剧上升的族群冲突中生存下来?如果一个国家族群冲突严重,能否完成民主转型就成了一个问题。对一个多族群的大国来说,即便只有一小块地方因为严重的族群冲突而不得不采用武力解决的话,对整个新兴民主政体都会构成一种实质性的威胁。国际学术界的共识是,族群冲突通常不利于民主转型与巩固。

所以,对发展中世界的多族群国家来说,如何有效地控制族群冲突就是一个重大议题,也是实现民主转型和巩固的关键。马丁 · 麦格较为重视一国国内的族群结构。他认为,族群关系主要有三种模式:一是同化,二是平等多元主义,三是不平等多元主义。通常,不平等多元主义最容易引发政治问题。当弱势族群认识到这种不平等并开始抗争时,往往就会导致族群冲突。麦格借用斯蒂芬 · 斯坦伯格的话说: “如果说存在族群问题的铁律,那就是当各个族群处在权力、财富和地位的不同等级时,冲突不可避免。” 这意味着,不平等多元主义的族群关系模式最有可能导致族群冲突。\nauthor{马丁 · N.麦格:《族群社会学》,祖力亚提 · 司马义译,北京:华夏出版社 2007 年版,第 91—119、501—524 页。}总的来看,这种理论视角更注重族群关系的社会结构。

另一理论视角则偏向于制度主义。关于如何在多族群社会或高度分裂的社会通过控制族群冲突来维系新兴民主政体的稳定,学术界有很多讨论。目前,国际学术界基于制度主义形成了两种主要的理论主张,本书第 6 讲对此已有介绍。一种主张被归入权力分享学派,以阿伦 · 利普哈特为主要代表。他们认为,应该通过比例代表制、联邦制、赋予少数族群充分自治权等制度安排来促成不同族群之间的权力分享,从而弥合族群分裂,实现民主政体的稳定。利普哈特的观点又被称为协和主义民主理论或共识民主理论。\nauthor{阿伦 · 利普哈特:《民主的模式》,陈崎译,北京:北京大学出版社 2006 年版。}另一张主张被归入政治整合学派,以罗纳德 · 霍洛维茨为主要代表。他认为,利普哈特的共识民主方案不仅不能弥合族群分裂,反而可能激化族群冲突。他倡导采用偏好性投票制度来弱化族群冲突。他认为,对于高度分裂的社会来说,首先要解决有效政治整合的问题。\nauthor{Donald L. Horowitz, “Constitutional Design: Proposals versus Processes,” in Andrew Reynolds, ed., \italic{The Architecture of Democracy: Constitutional Design, Conflict Management, and Democracy}, Oxford: Oxford University Press, 2002, pp.19-25.}笔者认为,对于呈现高度族群分裂的社会来说,离心型制度安排很容易导致民主政体的不稳定。特别是,中央与地方高度分权的地区主义安排——可以被视为一种高度分权的联邦制——容易引发族群冲突和民主崩溃。要想在高度分裂的社会实现民主政体稳定,关键是要通过有效的宪法设计和制度安排为政治精英提供跨族群的政治激励。\nauthor{参见包刚升:《民主崩溃的政治学》,北京:商务印书馆 2014 年版;包刚升:《民主转型中的宪法工程学:一个理论框架》,载于《开放时代》2014 年第 5 期,第 111—128 页。}

\tsectionnonum{推荐阅读书目}

安东尼 · 史密斯:《民族主义:理论、意识形态、历史》,叶江译,上海:上海人民出版社 2011 年版。

本尼迪克特 · 安德森:《想象的共同体:民族主义的起源与散布》,吴叡人译,上海:上海人民出版社 2005 年版。

埃里克 · 霍布斯鲍姆:《民族与民族主义》,李金梅译,上海:上海人民出版社 2000 年版。

马丁 · N.麦格:《族群社会学》,祖力亚提 · 司马义译,北京:华夏出版社 2007 年版。
