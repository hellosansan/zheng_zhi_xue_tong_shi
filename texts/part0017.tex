\tintro{后记}

撰写一部政治学普及作品,在很多学者看来是一件吃力不讨好的事情。一方面,一部普及作品远不如一部学术专著能提升作者的学术声誉;另一方面,一部高质量的普及作品同样需要作者付出巨大的时间与精力——要做到知识准确、深入浅出、通俗易懂绝非易事。

那么,为何还要撰写这部作品呢?这与作者的学术经历有关。我的第一个学位是经济学学士,因而习惯于经济学知识广泛普及的现象。比如,天下没有免费的午餐、搭便车、机会成本、公地悲剧、交易成本、委托代理、道德风险、囚徒困境等经济学概念时常出现在大众媒体的版面上。从企业家、公司管理者到新闻记者、政府官员,很多人已经习惯在日常生活中频繁使用经济学概念,甚至在日常沟通中开始依赖这些概念。

然而,当我开始学习和研究政治学以后,我发现政治学不仅不如经济学那样深入人心,而且还容易遭到公众的误解。复旦大学一位非常资深的政治学教授在一次内部会议上说,他年轻时在政治学专业读书,甚至不得不经常告诉别人自己学的是国际政治,以免引起不必要的误解。如今,尽管二十多年过去了,这种情况并未得到根本的改观。在很多从未接触过这个学科的人看来,政治学不是等同于宫廷政治或做官的技艺,就是等同于意识形态宣教,最直观的就是不少公共政治课的刻板印象。当然,更直接的问题是,政治学的基本概念和主要理论并没有像经济学概念和理论一样广为人知。这意味着中国社会中政治学常识的相对匮乏。

按照卡尔 · 波普尔的看法,一个社会发达与否取决于知识的有效积累和持续增长。通常,这里的知识容易仅仅被视为自然科学和工程技术知识,但其实社会科学知识也同样重要。经济增长和繁荣不仅需要自然科学和工程技术知识,而且需要关于制度、市场、企业和组织的有效知识。没有后者的有效积累,一个社会通常无法实现持久的增长与繁荣。同样,公共部门善治的实现,很大程度上也取决于这个社会累积的关于国家、政体、制度、治理与公民权利有关的知识。这些知识通常是由政治学这一学科来提供的。如果说知识构成了一个社会发展的限度,那么政治学常识的匮乏无疑将构成中国社会进一步发展的制约因素。

正是在这样的背景下,我自从首次在复旦大学开设政治学基础课程以来,就注意授课讲义的整理,以期能逐步完成一部政治学普及作品。如今,这一任务总算完成了!

这部作品能够完成得益于很多人的帮助与关怀。首先要感谢复旦大学国际关系与公共事务学院林尚立、陈明明与陈周旺三位老师领衔的《政治学原理》课程的教学团队。正是他们的安排,为我提供了讲授政治学基础课程的机会。这门课程是复旦大学校级精品课程,既有统一的授课要求与标准,又做到了每个教师各具特色和自成一体。正是在讲授这门课程的过程中,我逐步积累了写作本书的素材。从 2013 年春季学期到 2015 年春季学期,复旦大学各个社会科学专业有 400 余位学生修读我的课程,本书的出版亦有这些学生们的贡献。

其次要感谢北京大学政府管理学院李强、朱天飚与张健三位老师领衔的 “政治学概论” 教学团队。我攻读博士学位期间,曾连续两年担任该课程的助教,参与了三位老师设计课程和编定教学大纲的整个过程,同时兼任讨论课教师。我在此过程中的收获,要远远超过自己的付出。此外,需要说明的是,我在社会科学方法与政治经济学两个方面受到了两位导师傅军教授和朱天飚教授的很大影响,本书第 13 讲 “经济增长与国家治理的政治学” 和第 14 讲 “如何做政治科学研究?” 的一些想法最初来自于跟他们的互动;我在政治哲学和思想史方面受到了李强教授的很大影响,本书第 2 讲 “政治学:智者如何思考?” 与第 3 讲 “意识形态大论战” 的一些想法最初来自于跟他的互动。

再次要感谢两家有影响力的媒体《东方早报》与《南风窗》——特别是《东方早报 · 上海经济评论》主编张云坡先生、编辑吴英燕女士以及《南风窗》常务执行副主编赵义先生,他们的约稿和督促推动我撰写了不少通俗易懂的评论文章,其中的一些文章成为这部作品的一部分,包括:

\quo{《岛屿的寓言:谁之统治?何种秩序?》,载于《东方早报》2013 年 9 月 24 日;

《蛋糕政治定律》,载于《东方早报 · 上海经济评论》2014 年 1 月 21 日;

《激励结构与国家治理》,载于《东方早报 · 上海经济评论》2013 年 1 月 8 日;

《被误解的民主》,载于《东方早报》2014 年 3 月 18 日;

《民主转型僵局》,载于《南风窗》2014 年第 7 期。}

我过去撰写的《政治发展的社会维度》(宋磊、朱天飚主编:《发展与战略》第十四章,北京大学出版社 2013 年版),构成了本书第 8 讲的底稿。感谢北京大学出版社社会科学编辑部主任耿协峰博士对这部作品的重视,使得我的第二部学术作品能在母校出版社出版;他专业的编辑工作也保证了本书出版环节的高质量。还要感谢我的同事张骥博士在我写作本书过程中给予的勉励与支持。

最后要感谢我的太太杨小静,她在从事文学创作之余承担了绝大部分家庭事务,使得我有更多的时间用于研究、教学和写作;包如伊同学则给我们的生活增添了无限的欢乐。

很多学者的体会是,研究愈是深入,愈是不敢发声和落笔。学术世界的博大精深和学术边疆的不断拓展,使得任何一部著作都可能招来批评。所以,我的希望仅仅是,读者们会说,有这样一部作品要好过没有这样一部作品。这就实现了对世界的微小改进。

当然,这部作品可能还存在错误或瑕疵,文责自负。如果读者对本书有任何建议,请跟我联系,电子邮箱是:baogangsheng AT hotmail DOT com\neditor{这里 AT 指的是 “@” ,DOT 指的是 “.” 。—— TeX 编者注}。

\closing{包刚升}{2015 年 8 月于复旦大学文科楼}
