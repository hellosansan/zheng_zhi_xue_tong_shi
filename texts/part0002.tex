\tchapter{为什么政治很重要?(代序)}

亲爱的读者朋友,很荣幸您能打开这本书!

本书是作者在复旦大学政治学课程讲义的基础上修改润色而成的,力求成为一部通俗易懂、深入浅出的政治学普及作品,旨在让您在较短时间内掌握政治学的常识,加深对中国与世界政治的认知,逐步养成系统的政治思考能力。

为什么政治很重要?先从一个小故事说起——

美国独立战争期间,后来出任第二任总统的约翰·亚当斯非常忙碌,他远离故乡和家人,整日忙于政治事务。1780年,亚当斯太太的来信对此多有抱怨。收到夫人的信件以后,亚当斯回了一封信,信中这样说:“为了我们的孩子们能够自由地研究数学与哲学,我必须研究政治与战争。”亚当斯用这句后来很出名的话强调了政治的重要性。在他看来,解决好政治问题是解决其他问题的前提。实际上,约翰·亚当斯阐述的道理不仅对两百多年前的北美殖民地适用,而且对今天很多尚未完成现代转型的国家也同样适用。

在亚当斯看来,政治的重要性再怎么强调都不为过。那么,今天是否依然如此呢?

大家先来看几则新闻——

在中纪委十八届二次全会上,习近平说:“要从严治党,惩治这手绝不能放松,要坚持老虎、苍蝇一起打。要加强对权力的制约和监督,把权力关进制度的笼子里。”这是1949年以后党的最高领导人首次明确表示要“把权力关进制度的笼子里”。那么,怎样理解这句话呢?“把权力关进制度的笼子里”意指政治权力必须要受到制约,这种制约应该是有效的制度约束。为什么要提出这个观点呢?一个实际的考虑是目前腐败现象还比较严重,控制腐败的主要办法应该是用制度约束政治权力。还可以进一步追问:什么是制度的笼子?如何才能把权力关进制度的笼子里?这些问题都离不开政治学。

第二个新闻事件与美国政府有关。巴拉克·奥巴马总统上任以来,就面临着美国政府债务上限是否要上调的问题。最近几年,国际媒体上频繁出现美国政府面临“财政悬崖”(fiscal cliff)的说法。2013年10月,奥巴马还不得不临时关闭了部分联邦政府机构。面对这一现象,很多重要报纸和媒体都刊发了评论,但一些评论对美国财政与公债问题缺少深入理解,因此就难以挖掘政治现象背后的真正问题。按照经合组织公布的数据,美国政府公债占GDP的比例已超过100\%,日本的比例已超过200\%,有的欧洲国家的比例高达120\%~160\%。这些都是惊人的数字!那么,发达国家的公债危机是如何造成的?政府公债固然是一个经济问题,但更是一个政治问题。要深入了解这些问题,就离不开政治学的思考。

有读者还注意到,在奥巴马政府临时关闭事件中,总统、参议院和众议院是主要的当事人。总统、参议院和众议院是何种政治机构?三者是何种关系?为什么美国总统奥巴马的提案经常遭到国会的否决?这个新闻似乎还揭示出,美国是世界上最富有的国家,但他们却有一个最缺钱的政府;奥巴马总统是世界上最具权势的政治家,但他在国内却处处受到掣肘。相比之下,英国就很少发生这样的事情。大家很少听到英国首相跟议会就提案或法案问题“打架”。这又是为什么呢?学过政治学,就会知道美国是总统制,英国是议会制,法国是半总统制。总统制是什么?议会制是什么?半总统制又是什么?不同政府形式的政治逻辑有何不同?这些问题都需要政治学来回答。如果一个人不了解这些,即便他整天看国际新闻,甚至本人到了华盛顿、伦敦或巴黎,很多事件也未必能看得清楚透彻。

再看两则环境新闻——

2013年初,北京《新京报》公布了一张照片,读者可以在照片近处看到国旗,但不远处的天安门就只是隐约可见了。有网友给这张照片留下的评论是:“您站在天安门广场,却看不见毛主席。”这说明当天北京雾霾极其严重,估计PM2.5已超过500。2012年华北地区的重度雾霾,标志着中国工业化发展到一定阶段后引发的严重环境问题。更糟糕的是,2013年下半年开始,中国更大范围内频现重度雾霾,不少中西部城市都面临雾霾的困扰。

那么,如何治理雾霾呢?环境专家容易认为,这是一个技术问题。PM2.5的主要成分是什么?是从哪里来的?这两个问题搞清后,就不难拿出治理方案。具体怎么治理呢?技术上讲很简单,只要把构成PM2.5的几个主要成分通通干掉,把几个主要来源通通降下去,空气质量就能变好。的确,治理雾霾的技术原理是这样的,环境专家的说法并没有错。但是,环境专家的这种技术解决方案能否成为一种有效的公共政策呢?这就关系到政治问题。技术解决方案能否成为一种公共政策,是政策背后的政治决定的。比如,有些措施从技术层面来看是可行的,但是从政治层面来看却难以实施。这样,此种技术解决方案就会被否决掉。因此,治理雾霾貌似是一个技术问题,其实也是一个政治问题。

另一个环境问题是地下水污染。最近两年,中国地下水污染一度成为媒体关注的焦点。那么,如何治理地下水污染呢?比如,假定全国人大常委会制定一部新的立法,旨在治理地下水污染。政治学关心的一个问题是,这部新立法在实施过程中的实际效果怎样?比方说,某省某县有一家年销售额达百亿规模的大型造纸企业,它是该县的GDP大户与纳税大户,这家企业的董事长可能还是全国人大代表或省政协委员。过去,这家企业为了降低污水排放成本,打了一根很深的污水排放管道,通过高压办法把污水排放到1000米深的地底下去了。但是,按照全国人大的新立法,这种污水排放措施属于重点整治和清查之列。这个企业经过计算发现,改造污水排放系统的一次性投资就需要一两亿,此后每年污水处理的成本还会增加数千万。显然,这样做会大幅增加企业的经营成本。

那么,这家造纸企业会怎样做呢?一种可能的方案是逃避监管。改造和维持排污设施的成本是多少?逃避监管的成本又是多少?该企业会比较两种成本。经过权衡,企业可能会跟当地政府沟通以达成某种“交易”。通过这种“交易”,企业得到的好处是可以降低运营成本和提高经营绩效,地方政府得到的好处是更高的GDP指标、更多的税收和就业机会。当然,可能还会出现企业与政府官员——特别是直接监管机构官员——之间的权钱交易。这样的话,地方政府就更难阻止企业进行违法污水排放了,新立法在地方层面落实也就容易成为一句空话。这个假想的案例揭示,治理地下水污染不只是一个法律问题,更是一个政治问题,背后涉及一整套的政治体制与制度安排。

此外,媒体报道披露的大量新闻事件都跟政治有关。从奶粉质量、食品安全到医疗问题、养老保障,从农民工子女上学、异地高考到房价与土地财政、钓鱼岛与中日邦交,等等,都跟政治有关。实际上,每个人从摇篮到坟墓都离不开政治。无论是否喜欢,政治总在影响着每一个人的生活:出生的时候,生育当中就有政治;上学的时候,教科书当中就有政治;工作的时候,就业当中就有政治;落户的时候,户口当中就有政治;上网的时候,网络当中就有政治;就医的时候,医疗当中就有政治;投票的时候,选票本身就是政治;最后离开人世时,墓地可能也关乎政治。从摇篮到坟墓,政治对每个人不离不弃,更说明了政治的重要性。

既然政治如此重要,理解政治应该成为每个公民的必修课。政治学就是理解和研究政治现象的学科。政治学感兴趣的是在政治领域发生了什么、如何发生的以及为什么发生。前两个问题是描述性的。要了解政治现象,我们既可以通过报纸、电视和互联网去观察,也可以在现实政治生活中去体验。通过这样做,我们就能了解到政治领域发生了什么以及如何发生的问题。有人不仅对中国政治感兴趣,也对外国政治感兴趣。除了借助媒体与网络,现在中国人有越来越多的机会到国外去读书、访问、旅行、工作或定居,大家可以借助这些机会去观察外国的政治。

第三个问题是解释性的。“为什么发生”会更深奥一些,更多地涉及政治学理论与政治学研究。比如,为什么各国政治体制会如此不同、各国治理水平的差距如此之大?有人注意到,这个世界上有些国家被称为民主国家,有些国家被称为不民主国家,这是为什么?有人说民主能解决腐败问题,但现在仍然有不少民主国家没有很好地治愈腐败,那又是为什么?还可以在政治领域找出很多类似的问题去问“为什么”。试图解释一些重大的政治现象何以发生,是政治学研究的基本任务。

当然,本书作为一部政治学普及作品,受限于篇幅和难度,无法做到面面俱到。具体来说,本书试图介绍和探讨政治学领域的基本概念、主要理论与重大议题。比如:

古希腊人如何理解政治?孔子与韩非的政治观有何不同?

身处荒岛的众人如何构建合理的政治秩序?从哲学思辨到博弈论,政治学走过怎样的路?

自由主义、保守主义和社会主义分别主张何种政治观点?论战焦点何在?

如何理解现代国家?如何解读国家构建和国家能力等时髦概念?

全球范围内有哪些主要的政体类型?不同政体的政治逻辑是什么?

如何理解民主政治模式的多样性?为什么政治制度很重要?

国王可以强拆吗?宪法和法律在政治生活中扮演何种角色?

为什么有的国家实现了成功的民主转型,而有的国家则没有?

选民根据什么来投票?为什么非暴力不合作运动能奏效?

为什么不同国家的政治文化差异如此之大?民主可以跨文化移植吗?

什么是民族主义?怎样解读族群政治与族群冲突?

如何理解政治生活中的暴力现象?内战何以发生?

私人部门与公共部门如何实现有效治理?什么是经济增长的政治学?

如何理解政治科学研究的常见误区?什么是真正的政治科学研究?

总之,这部书正是为那些对政治和公共事务感兴趣的读者朋友们而写的。现在,让我们一起开始现代政治学常识的探索之旅吧!
