\tchapter{经济增长与国家治理的政治学}

\quo[——弗里德里希·哈耶克]{在自由主义的基本原则中没有什么东西能使它成为一个静止的教条,也不存在一成不变的一劳永逸的原则。在安排我们的事务时,应该尽可能多地运用自发的社会力量,而尽可能少得借助于强制,这个基本原则能够做千变万化的应用。}

\quo[——米尔顿·弗里德曼]{我们如何能防止我们建立的政府成为一个无法控制的怪物——它会破坏自由而我们建立政府的本来目的是保护自由?……为了保护我们的自由,政府是必要的,通过政府这一工具我们可以践行自由;然而,由于权力集中在当权者手中,它也是自由的威胁。}

\quo[——曼瑟·奥尔森]{当存在激励因素促使人们去攫取而不是创造,也就是从掠夺中而不是从生产或者互为有利的行为中获得更多收益的时候,那么社会就会陷入低谷。}

\quo[——詹姆斯·布坎南]{既然政治和政治过程最终是在交易范式中加以构造,那么简单的和直接的观察就可以使人们联想到,政治家和官僚是内在组成部分。这些人的行为同经济学家研究的其他人的行为(意指经济人的行为)没有任何不同。}

\tsection{13.1 蛋糕政治定律}

大\nauthor{本节主要内容曾以《蛋糕政治定律》为题刊载于《东方早报·上海经济评论》2014年1月21日。}家经常用“做蛋糕”和“切蛋糕”来比喻一国经济活动中的生产和分配。“做大蛋糕”意味着经济增长和总量扩张,“切好蛋糕”意味着合理分配和规则公平。基于对人性的认知,经济学家普遍认为,只有切好蛋糕才能做大蛋糕。切好蛋糕是塑造正确的激励结构,鼓励那些扩大生产、改进效率和推动创新的经济行为,奖励那些对经济增长做出贡献的个人与组织。这样,人们才有动力去做大蛋糕,并使整个社会受益。所以,做大蛋糕很大程度上是切好蛋糕的经济结果。上面这段话,可以被理解为最简明的“蛋糕经济定律”。

然而,“蛋糕经济定律”所忽略的是政治在“做蛋糕”和“切蛋糕”过程中扮演何种角色?实际上,比经济规则更强硬的是政治规则,经济领域的规则最终可能是政治领域的规则决定的。国家作为一种垄断强制力的机构,完全可能与市场一起——甚至取代市场——成为资源配置的主要机制。所以,恰当地理解国家与“蛋糕”的关系,是打开经济增长黑箱的另一把钥匙。

那么,国家与“蛋糕”是什么关系?回顾人类的政治经济史,可以总结出三条简单的法则:第一,没有国家时的主要规则是抢蛋糕;第二,绝对主义国家的主要规则是分蛋糕;第三,立宪主义国家的主要规则是做蛋糕。本书称之为“蛋糕政治定律”。

第一,没有国家时的主要规则是抢蛋糕。如果没有国家,财产就得不到保护。自己做了蛋糕,但未必能吃到蛋糕。这种情况下,比较“聪明”的人会发现,要想吃到蛋糕,最重要的不是做蛋糕的能力,而是抢蛋糕的能力。所以,获取暴力手段成为生存的关键。这样,训练体能,提高作战技艺,改进武器装备,甚至是建立控制暴力的组织,成为一个人的核心能力。凭借这些,有人可以抢到更多蛋糕。因此,如果没有国家,抢蛋糕是常态,是社会的主要游戏规则。而所谓有能有为,是指一个人拥有抢蛋糕的高超技艺。

从理性角度看,这样的社会没有人会去选择搞生产。当然,实际上会有人搞生产。否则,所有人都活不下去,或者只能依赖于采集食物、捕鱼和狩猎。但是,至少相对比较“强”的人不会选择做蛋糕,他们的主要工作更可能是抢蛋糕。这是一种理性选择。别的人辛辛苦苦做蛋糕,但如果他有一支AK-47冲锋枪,蛋糕就是他的了。这里描述的状态就是英国哲学家霍布斯所说的自然状态,或者说是“人与人的战争状态”。没有国家,人类的生活可能就是如此。这也接近于美国经济学家奥尔森所说的“流寇统治”——“流寇统治”之下,经济增长绝无可能,贫穷、饥饿和随时会到来的死亡成为常态。

第二,绝对主义国家的主要规则是分蛋糕。在这样的国家,分蛋糕的规则很随意,掌握分蛋糕权力的人基本上想怎么分就怎么分。比如,司马迁《史记》中记载的一个事例是,秦始皇公元前221年统一中国后,“徙天下豪富于咸阳十二万户”,就是命令全国各地的十二万户贵族和富户举家搬迁到咸阳。如果这段史料是可信的,那么这正好为那句话做了注脚——“普天之下,莫非王土;率土之滨,莫非王臣。”另一个事例是,公元前119年,汉武帝由于征伐匈奴、国库空虚,决定向全国商人和富户征收额外的财产与收入税,史称“算缗”。这实际上接近于政府单方面的财产征收。汉武帝说,我要打匈奴但没有钱怎么办,收点新税吧?我说一个数字或比例,符合条件的就照此上缴。大家知道,这里并没有什么规则可言,国家说怎样就是怎样。后来,为了有效征收算缗,汉武帝还鼓励国民互相揭发,不按规则做的商人或富户就有坐牢甚至灭门的危险。当然,此类事件并非古代中国所独有。诺贝尔经济学奖得主诺思在其著作中提到,欧洲近代绝对主义国家兴起之时,法国亦曾有过类似事例。

在这样的国家,社会精英的理性选择不会是进入做蛋糕的部门,而是进入分蛋糕的部门。只有进入分蛋糕部门的人,才可能真正拥有和控制财富,才可能拥有一些安全感。否则,拥有再多财产,随时都可能处于“人为刀俎,我为鱼肉”的境地。所以,这类国家的一个重要特征是,社会精英和上层社会家庭的子女竞相谋求进入分蛋糕部门。政治权力部门职位一旦出现空缺,就有无数人即使挤破头也要去考试或竞争。与做蛋糕的部门相比,分蛋糕的部门拥有控制利益分配和资源配置的更大权力。分蛋糕的手会随时伸向做蛋糕的部门,这是绝对主义国家的现象。

第三,立宪主义国家的主要规则是做蛋糕。为什么主要规则是做蛋糕呢?因为分蛋糕的规则是由做蛋糕的人一起制定的。在上面讨论的绝对主义国家,分蛋糕的规则是分蛋糕的人自己制定。如果同一批人既决定分蛋糕的规则,又实际负责操刀分蛋糕,结果就乱套了。财产的重新分配和税率的随意更改就会成为常态。但是,在立宪主义国家,财产受到确定无疑的保护,税率的更改需要互相协商和经过复杂的程序。实际上,做蛋糕的人通过一整套复杂的制度和程序控制了分蛋糕的部门,使得后者的权力被限定在有限范围之内,而不能随意裁量和恣意妄为。

在这样的国家,社会精英更愿意进入做蛋糕的部门,而不是分蛋糕的部门。由于分蛋糕部门的权力受到了有效制约,政治权力不再是致富的捷径。相反,进入做蛋糕部门成为渴望致富、雄心勃勃的年轻人的不二法门。这样,大量社会精英选择进入做蛋糕的部门,并受到做大蛋糕的有效激励。立宪主义国家的蛋糕就更容易做大,持久的增长、创新与繁荣成为可能。这里的逻辑也符合诺思的经验研究结论:从13世纪到17世纪英国政治体系的变革和立宪政体的确立,塑造了当时世界上最有效率的产权制度,从而成就了后来的工业革命。

所以,“蛋糕政治定律”可以简明地表述为:“没有国家时的主要规则是抢蛋糕;绝对主义国家的主要规则是分蛋糕;立宪主义国家的主要规则是做蛋糕。”比较三种国家状态,就会发现,没有国家时经济增长几无可能,人类只能在谋求生存和自我防卫的低水平上徘徊。与绝对主义国家相比,立宪主义国家的长期经济前景无疑会更好,因为多数社会精英都选择去努力做蛋糕,成为财富的创造者。相反,绝对主义国家的社会精英们最感兴趣的是如何分蛋糕,成为财富的分配者——甚至是财富的掠夺者。两种类型国家的高下一目了然。奥尔森曾经这样说:“当存在激励因素促使人们去攫取而不是创造,也就是从掠夺中而不是从生产或者互为有利的行为中获得更多收益的时候,那么社会就会陷入低谷。”这是一位经济学家对不同类型国家长期经济绩效的判断。所以,国家与“蛋糕”其实是一个严肃的政治议题。

\tsection{作为微观基础的经济人假设}

众所周知,经济增长并不只是一个经济学问题,也是一个政治学问题。这一讲的主题是经济增长的政治学,关注的是政治经济学领域。关于政治经济学的概念辨析及研究领域,参见朱天飚:《比较政治经济学》,北京:北京大学出版社2006年版。政治经济学需要借鉴经济学的分析范式,而经济学通常以人性的基本假设作为分析起点。历史上,关于性本善还是性本恶的争论由来已久,更多人倾向于认为人性是复杂的。但是,如果人性是复杂的,那么如何确定人类行为的分析起点呢?这就会变得很棘手。

经济学的重大贡献在于它提出了人性的基本假设,即经济人假设,并以此作为分析人类行为的起点。很多经济学家认同人性是复杂的,但为了分析方便,先可以假定人是经济人。在主流经济学框架中,经济人假设构成了分析的微观基础。

那么,什么是经济人假设呢?经济人假设主要包含三层意思。首先,人是自利的(self-interested)。“Self-interested”的本义是对自我感兴趣,这是一个中性词,现在通译为“自利的”。这里的“自利”不应该被理解为自私或损人利己。人对自我感兴趣,到底是什么意思呢?比如,午餐时间一到,我首先会感觉自己肚子饿了,不会感觉到上课的同学们肚子饿了。这是一种很天然的感觉,也是一种生物本能——任何人首先都会对自己感兴趣。

其次,人是理性的(rational)。古典经济学倾向于假设人是完全理性的,人做决策时知道所有的相关信息,然后他会据此作出最优选择。但是,完全信息假设后来经常遭到质疑。比如,很多大学生毕业前面临几种不同的选择:去企业工作,去政府部门工作,国内读研,海外读研,可能还有创业。那么,哪种选择最优呢?实际上,没有人拥有完全的信息。后来,美国管理学家赫伯特·西蒙提出人其实是在“有界理性”条件下做决策的。这一观点得到了普遍认可,所以,人的理性更多是指有界理性。

再次,人追求效用最大化(utility maximizing)。既然人是自利的和理性计算的,那么他希望实现何种均衡呢?答案是效用最大化,效用最大化在经济分析中经常被简化为财富最大化或收益最大化。但实际上,经济学定义的效用是主观的。比如,政治家可能追求政治权力最大化,或追求名垂青史;学者可能追求学术贡献最好化,或影响力最大化;而企业家追求的可能是财富最大化。这是他们对效用的不同定义。由于个人偏好不同,不同的人对效用的定义是不一样的。

上面讨论了经济人假设的三层含义,那么经济人假设是否被普遍接受呢?实际上,自经济人假设提出以来,就不断地遭到质疑和挑战。比如,有人提出来社会人、道德人、政治人或宗教人等人性假设。经济学家当然了解这些质疑和挑战,但他们经常这样问:如果不接受经济人假设,还有更好的替代性假设吗?经济人假设尽管并不完美,却是一个较为恰当的分析起点。

\tsection{私人部门治理}

如果暂且接受经济人假设,那么私人部门治理应该基于何种规则呢?首先,既然人是自利的、理性计算的和追求效用最大的,所以先要明确一条基本规则:每个人的财产都应该得到保护。由人的自利性可以推导出:财产权利构成了人类社会的基本激励结构。如果财产不受保护,这种激励结构就会遭到破坏,社会的基本规则就会乱套。因此,私人部门治理的第一个条件是界定和保护财产权利。如果没有明确的产权规则,就会导致非常严重的社会后果。

其次,由人是理性的可以推导出人应该是自我利益的最好判断者,所以经济自由就非常重要。经济自由至少有两个好处:第一,市场主体可以自主选择,他可以决定做什么和不做什么;第二,不同市场主体之间可以自由竞争,而竞争是效率改善和创新的主要动力。放眼全球,所有发达国家的私人部门都符合这两个条件——即市场主体的自由选择和自由竞争。一个经济体的效率提高、创新出现及财富增长都跟这两个条件有关。

再次,私人部门治理除了需要保护产权和经济自由,还需要什么条件呢?比如,甲和乙签订了一份大额采购合同,甲向乙支付了不菲的定金后,乙说由于原材料涨价不打算给甲供货了,但也不准备退还定金。那么,甲应该怎么办呢?所以,私人部门治理要想有效还需要第三个条件,即契约的强制执行。如果没有一个机构来强制执行契约,市场机制可能会垮掉。此外,有人不想生产和交易,而是想通过欺诈、偷盗乃至抢劫获取财富,这样做可以吗?当然不可以。由此可见,好的市场经济必须是法治经济。如果没有法治,交易环节的造假欺诈、食品药品的安全问题,以及其他经济犯罪活动等就会层出不穷。对私人部门来说,国家的强制力是必需的,法治是市场经济的必要条件。

由此可见,实现私人部门有效治理需要三个基本条件:一是产权的界定和保护,二是交易与经济自由,三是契约的强制执行与法治。没有这些条件,私人部门就难以实现有效治理,也就难以实现持久的经济繁荣。

相反,如果产权得不到有效保护,多数人就会失去工作的动力与激励。曼瑟·奥尔森在《独裁、民主与发展》一文中业已阐明,流寇统治的最大问题是破坏产权制度。一旦破坏了产权制度,也就破坏了整个社会的激励结构。所以,当产权得不到有效保护时,人们的经济行为就发生了变化,整个社会的激励结构就会被破坏掉。

如果没有经济自由,资源就很难实现有效配置。大家每天早餐吃什么?午餐吃什么?晚餐吃什么?大家希望自己选择,还是由某个机构决定每个人每顿吃什么,然后分配给大家?有人说免费分配当然好,但实际上没有什么东西是免费的。某个机构分配食物,不过是改变了资源配置和成本分摊方式。与自己选择吃什么相比,只能吃某个机构分配的食物,显然满意程度会低很多。没有经济自由,资源配置的效率就会很低,人的很多需要通常也无法满足。

问题的另一面是:没有经济自由,就没有市场竞争。人们今天享受的一切美好物质成果,几乎可以说都是竞争的结果。19世纪末,马车本来是优质的交通工具。为什么后来马车消失了?因为有人发明了汽车。汽车刚发明时,其生产方式是定制的,一辆一辆地分别生产,价格也很贵,性能稳定性也比较低。后来,亨利·福特引入了大规模生产流水线,生产效率提高了数十倍。这样,汽车就逐渐成了马车的完美替代物。20世纪10、20年代,福特的T型车已经够好了,既性能稳定,又价格实惠。但是,阿尔弗雷德·斯隆认为还可以生产更好的汽车,他领导通用汽车开发设计出外形更美观、性能更优良的汽车,结果又超越了福特公司。这种市场竞争的结果是,消费者能以更低的价格买到更好的汽车。此外,由于这种竞争,汽车技术也在不断进步。所以,市场竞争的意义是重大的。很多人的生活经验是:凡是满意度比较高的地方,基本上都有比较充分的竞争;凡是满意度低的地方,基本上都是竞争缺乏和管制过度。

假定在一个保护产权和自由竞争的环境里,有人想致富,他应该做什么呢?20世纪80年代初,有一个叫斯蒂文·乔布斯的年轻人想挣钱,他和他的合伙人开发出了世界上第一台个人电脑。这是非常了不起的事情。当他们能以合理的价格为市场提供性能优越的创新产品时,就有了挣大钱的机会。但后来,苹果公司有段时间经营不善,乔布斯甚至被迫离开了苹果公司。随后,他在别的商业项目上再次获得成功后,又设法回到苹果公司。在他回归苹果之后,Ipod、Iphone和Ipad这一个个神奇的创新产品就在他的手上诞生。乔布斯因癌症离世时,苹果公司已成为世界上最成功的公司,乔布斯也成了21世纪初美国最富有和最伟大的企业家之一。实际上,乔布斯和苹果公司的成就基于他们能比竞争对手提供更好的产品。这一过程也是他们推动产品创新与技术进步的过程。

当然,市场并非总是如此美妙。实际上,市场也有可能带来很糟糕的东西。比如,有人会销售假货,有人会恶意欺诈,有人会破坏环境,有人甚至想巧取豪夺。这些都会带来很严重的问题。所以,如果只有市场而没有政府,市场将失去保存其自身的手段,甚至整个社会都会陷入霍布斯所说的“人与人的战争状态”。因此,国家和政府是必需的,公共部门是必需的,法律和法治是必需的。没有国家与政府,规则良好的市场将不复存在。

基于新古典政治经济学的视角来理解私人部门和市场,20世纪后半叶已经出现很多有趣的著作。比如,弗里德里希·哈耶克在《个人主义与经济秩序》和《通往奴役之路》等著作中批评了计划经济模式,认为只有自由市场经济才能既带来效率又带来自由;曼瑟·奥尔森则在《权力与繁荣》中强调了正确的激励结构的重要性,特别是要防止权力蜕化为掠夺之手;道格拉斯·诺思在《经济史上的结构与变革》中认为,宪政规则与产权保护是经济繁荣的前提条件,而只有有效的产权制度安排和市场制度才能带来更好的经济绩效;米尔顿·弗里德曼则在《资本主义与自由》一书中认为,经济自由是政治自由的前提,经济自由也是经济繁荣的前提,限制政府权力是经济自由的基础条件。\nauthor{相关研究,参见弗里德里希·哈耶克:《个人主义与经济秩序》,邓正来译,北京:读书·生活·新知三联书店,2003年;弗里德里希·冯·哈耶克:《通往奴役之路》修订版,王明毅、冯兴元等译,北京:中国社会科学出版社2013年版;曼瑟·奥尔森:《权力与繁荣》,苏长和、嵇飞译,上海:上海人民出版社2005年版;道格拉斯·诺思:《经济史上的结构与变革》,厉以平译,北京:商务印书馆2013年版;米尔顿·弗里德曼:《资本主义与自由》,张瑞玉译,北京:商务印书馆2004年版。}这些杰出的经济学家与政治经济学家以不同方式论证了私人部门或市场部门的有效治理需要遵循何种原则。

\tsection{公共部门治理}

与私人部门相比,公共部门治理至少同等重要。如果暂且接受经济人假设,那么公共部门的治理应该基于何种规则呢?在现代政治中,公共部门存在着三种主要角色:政治家、官员和选民。政治家和官员被视为公共服务的供应者,而选民被视为公共服务的需求者。按照经济人假设,从政治家、官员(两者可以统称为官员)到选民都被视为经济人,都是自利的、理性计算的和追求效用最大化的。这里隐含的一个假设是:一个人不会因为成为政治家或公务员就变得高尚起来,而过去国内一般认为政治家或公务员更高尚一些,至少经济学家们会认为这是一种错觉。

基于这种微观基础,公共部门治理的基本规则应该是什么呢?很多人都听说过“公地悲剧”。经济学家们假设,只要存在公地,就容易出现公地的悲剧。比如,在一个草原上,如果每个牧场都是私人所有,每个牧场主大概都会考虑:自己牧场的规模有多大,长草量如何,适合饲养多少头牲畜。如果养得过少,牧场就没有充分利用;如果养得过多,牧场就会因为过度放牧而出现退化。所以,这一决策是他经过理性计算做出的。但是,如果众多私人牧场中间还有一大片公共牧场,情况就不是这样了。公共牧场意味着任何私人牧场主都可以在上面自由放牧。这样,每个私人牧场主都会优先在公共牧场放牧,结果是公共牧场的草很快就被吃得干干净净。由于过度放牧,公共牧场很快就会发生退化,第二年的牧草量可能也会减少。这就是“公地悲剧”的一例。再比如,现在太平洋的大型鱼类被捕捞过度了。为什么会这样?人类目前只有国别政府,没有全球政府,而太平洋大部分都是公海。这又是“公地悲剧”的一例。尽管现在有一些国际性的海洋组织和渔业组织在居间协调,但成效并不显著。

那么,如何让公共部门治理更为有效呢?首先,如果政治家和官员都是经济人,公共部门的资源和权力就不应该由少数人控制,而应该由多数人控制。倘若少数人可以决定公共部门的资源和权力,考虑到他们都是经济人,在没有明确规则的约束下,他们都倾向于追求自身利益的最大化,可能置整个政治共同体利益于不顾。要知道,政治家和官员不会天然地服务于公众的利益。一个政治家是否服务于政治共同体的利益,取决于能否设计出一整套官员与公民之间“激励相容”(incentive compability)的制度安排。从人类已有的经验来看,选举就是这样一种制度安排。当多数人掌握选票并能决定少数处于公共政治职位上的人去留时,政治家就不得不考虑多数人的政治偏好与利益诉求。在这种制度安排下,政治家只有为公众服务、追求政治共同体利益的最大化、迎合多数选民的政治偏好,他才有机会赢得选票和职位。如果没有选举制度和投票机制,少数人控制公共部门资源和权力之后,几乎注定不会考虑多数人的利益。所以,公共部门的资源和权力应该控制在多数人手中,基本制度安排应该是让多数人通过政治参与、以投票方式来决定应该由哪些人来掌管公共部门。

其次,如果政治家和官员都是经济人,公共部门的职位是否要有竞争呢?如果说市场竞争对改善经济绩效是有利的,那么政治竞争的好处也是显而易见的。倘若没有政治竞争,某些特定的少数人就会一直执掌政治权力与控制公共资源,他们可能根本不会顾及多数人的利益和诉求。凡是缺乏政治竞争的地方,某些特定的少数人组成的权力集团就会控制政治权力和公共资源,多数人的政治偏好可能会被搁置。在不少国家,至今仍然能看到这种情形。政治竞争还关系到公共政策优化。不同的政治家可以提供不同的、互相竞争的政策方案。多数人可以基于理性判断这些方案的优劣,这样就更有机会得到一个较好的政策方案。当然,政治竞争离不开自由表达与信息传播。所以,言论自由和媒体自由也是公共部门治理有效性的重要方面。

除了上述两条规则,公共部门的有效治理还需要何种条件呢?众所周知,任何社会都存在贫富分化,所有社会都是富人相对较少而穷人相对较多。这样,如果有选举竞争,不排除有人会提出重新分配财产的政治纲领。比如,有候选人提出这样的政策主张:对最富有的10\%的人征收高额财产税和所得税,然后把这部分收入用于补贴较贫穷的70\%选民。由于这位候选人的施政纲领迎合了多数普通选民的利益诉求,他当选的机会还比较大。但是,如果他当选后真的要这样干的话,可能会激起最富有的10\%选民的剧烈反抗。这部分人通常是一个社会的精英阶层,他们通常不会认同这一政策方案。他们甚至不惜与同情工商阶层的军人联手,颠覆了这个对他们意味着重大财产风险的民选政体。实际上,这是很多发展中国家历史上曾经发生过的事情。当穷人要求重新分配财产时,富人不惜以发动军事政变来应对。

上面的问题就涉及民主决策的范围和边界。现代政治的一项常识是:即便是民主决策,亦须限定其范围和边界。换句话说,不管你拥有多少比例的选民支持,有些事情是民主决策所不能做的,这样才能保证公共部门治理的有效性。实际上,这就涉及立宪政治或宪政原则。在这个问题上,宪政原则意味着,民主决策的范围和边界不能无限扩展,有些基本规则是民主决策亦不能逾越的。在现代民主政体下,宪政有两个重要功能:一方面,宪政是对政治权力和政府的一种约束,另一方面,宪政是对民主本身的一种约束。所以,立宪主义或宪政原则也是公共部门实现有效治理的必要条件。

总之,公共部门治理的有效性取决于三条基本规则:一是多数人通过政治参与和投票控制公共部门的权力和资源;二是建立政治竞争的规则和机制;三是要确立政治权力与民主决策的范围与边界,亦即实施宪政原则。凡是公共治理有效的国家,几乎都践行着上述三条基本规则。

从政治经济学视角理解公共部门治理,20世纪后半叶以来产生了不少经典的文献。前面已经提到,安东尼·唐斯是这一领域的开创者之一,他在《民主的经济理论》中把所有政治参与者都视为经济人,把政治视为一个交易过程。在政治市场上,政治家和政党通过提供不同的公共政策来竞争选民的选票,而选民通过投出选票来决定购买什么样的公共政策,由此形成一种政治市场的均衡。\nauthor{安东尼·唐斯:《民主的经济理论》,姚洋等译,上海:上海人民出版社2005年版。}詹姆斯·布坎南亦借鉴这种理论视角,系统地发展了在政治学研究中产生重大影响的公共选择理论,并因此而获得诺贝尔经济学奖。\nauthor{关于公共选择理论,参见丹尼斯·C.勒纳:《公共选择理论》(第三版),韩旭、杨春学等译,北京:中国社会科学出版社2010年版。}总之,政治经济学为理解公共部门治理和政府行为提供了新的理论视角。

\tsection{激励结构与经济增长}

从蛋糕政治定律到私人与公共部门治理,都跟经济增长的政治条件有关。在美国经济学家、诺贝尔经济学奖得主罗伯特·卢卡斯看来,经济学最重要的问题就是“为什么有的国家富而有的国家穷”?这个问题试图探究的是经济增长的一般原因。

农业社会的一项重要见解是:“土地是财富之父,劳动是财富之母。”这种一古老的经济增长理论认为,决定增长的主要是两个变量:土地和劳动。后来,随着工商业的兴起,资本的因素变得越来越重要。资本积累和储蓄被视为经济增长的主要源头。著名的哈罗德-多玛模型主张的就是这种理论:经济增长率取决于资本积累率。罗斯托也认为,经济起飞需要以一定的资本积累为前提。工业革命之后,土地不再被视为惟一重要的资源,煤矿、油田和金属矿产资源的重要性日益提高。这样,资源也被视为经济增长的重要原因。当然,相反的理论也存在,著名的资源诅咒(resource curse)理论即主张丰富的资源储量长期当中反而不利于经济增长。到了20世纪70年代,西奥多·W.舒尔茨在研究农业经济时发现,在资本投入不增加的情况下,农业生产率得到了有效的提高,于是他提出了一个新概念:人力资本,并把人力资本增加视为解释经济增长的主要变量。后来,以罗伯特·索洛为代表的学者通过对美国战后经济增长研究发现,技术和技术进步才是经济增长的源泉。由此,技术创新经济学成为解释经济增长一个重要流派。此外,很多人还听过一个著名观点,即强调文化因素对经济增长的重要性。马克斯·韦伯认为,对西欧和北美来说,新教伦理成为推动这些地区的经济增长和资本主义兴起的重要因素。强调文化因素的流派中后来又加入了社会资本的理论。从詹姆斯S.科尔曼到弗朗西斯·福山,无不强调社会资本对经济领域的生产性作用。新制度主义经济学则更重要制度在经济增长扮演的角色。道格拉斯·诺思把英国宪政革命视为工业革命的前提,因为宪政革命确立了稳定的产权规则。这一流派一般认为,有效的制度可以降低交易成本,从而有利于经济增长。循着新制度主义经济学的基本思路,北京大学傅军教授提出了解释经济增长的BMW模型,即财富(Wealth)增长取决于市场制度(Market)与官僚制度(Bureaucracy)。\nauthor{关于经济增长理论的一般介绍,参见菲利普·阿英格、彼得·豪伊特:《增长经济学》,杨斌译,北京:中国人民大学出版社2011年版。关于傅军教授的研究,参见傅军:《国富之道》,北京:北京大学出版社2014年版。}

那么,如何从政治经济学的视角去理解经济增长呢?这里可以再提供一个新的视角。一个国家的经济总量等于该国的人均产出与人口的乘积。在人口既定的条件下,一个国家的经济增长取决于每个人人均产出的增长。实际上,人均产出是真正衡量经济发展水平的标准。从政治经济学视角来看,每个人都身处一定的激励结构之中。正是激励结构决定了一个人行为的激励与约束机制。如果激励结构鼓励一个人提高产出,他大概倾向于提高产出;如果激励结构不鼓励提高产出,他大概不会去努力提高产出。

比如,有两位智力、素质和能力相当的同学大学毕业后进入两家机构工作,甲同学进入A机构工作,乙同学进入B机构工作。甲同学待了三个月后发现,在该机构,所有重要事情都是机构一把手说了算,他每个季度都会调整管理层与员工的职位及薪水。甲同学的进一步观察发现,这位机构老大主要根据自己的主观印象和个人喜好来调整管理层与员工的职位及薪水。换句话说,只要他对谁印象好,谁就会在这个机构里得到更多的机会和资源。这样,如果甲同学足够“理性”,他会怎么做?他会把主要精力用于讨好这位老大。对甲同学来说,只要他不辞职,A机构的这种正式制度和非正式制度就构成了他的激励结构。相反,乙同学到B机构工作三个月后发现,这个机构的一把手尽管脾气不算很好,但是这位老大有一个明确的规则:只要谁的销量大,谁的职位及薪水就会往上走。如果你销售业绩表现突出,他就会不断地提拔你和给你加薪。至于其他方面的表现,则在其次。这样,如果乙同学足够“聪明”,他会怎么做?他会拼命去做销售,拿最好的销售业绩出来。

不同的激励结构一旦确立以后,久而久之还会塑造整个机构的文化。上面的A机构会形成何种文化呢?大家都努力讨好上头的人,上下级慢慢也会变成这样一种关系。B机构会形成什么文化呢?很清楚,应该是一种讲求绩效的文化,大家都努力工作和拿业绩说话。有人认为组织文化很重要,但这个案例说明组织文化很有可能由激励结构塑造的。如果碰巧这两个公司从事的是相同行业,有一天两家公司开始竞争,估计谁更有可能胜出呢?显然,A公司是无法与B公司竞争的,因为A公司的激励结构就决定它不是一家有竞争力的公司。两者比较,也可以看出激励结构对一个组织兴衰的重要性。

还可以继续问:如果一个组织要追求更好的绩效,应该给组织成员设定何种激励结构呢?不难发现,一种有效激励结构的关键特征是:当组织成员为组织创造更高绩效时,该成员本身也可以从这种高绩效行为中获得应有的回报。用效用函数来表示:

\[Ui=F(Pi)\]

其中U是效用(utility),P是绩效(performance),Ui是员工i的效用函数,Pi是员工i为组织创造的绩效,两者应该是正相关关系。既然员工是经济人,当他为组织创造更多绩效时,他自己的效用也会增加,这样的激励结构就是有效的。所以,这应该是一种激励相容的结构,即员工效用函数与组织绩效函数正相关。对整个社会来说,各个企业、政府机构和其他组织中能否建立起一套有效的激励结构,直接决定了这个社会是否会有较高的绩效。基于这个视角,一个国家经济增长的关键在于能否普遍建立起有效的激励结构。这种激励结构的核心是个人与组织之间的激励相容。

这里以企业家、医生和政府官员三种职业为例,进一步说明激励结构的重要性。\nauthor{这里的部分文字参见包刚升:《激励结构与国家治理》,载于《东方早报·上海经济评论》,2013年1月8日。}首先来看企业家。有的经济学家把企业家视为一种高级的经济动物,比如法国经济学家萨伊和美国经济学家熊彼特都认为,企业家能够发掘市场机会,通过创新来推动经济增长。萨伊认为,企业家能够把资源从产出低的地方转移到产出高的地方。熊彼特认为,经济发展的关键在于创新,创新的关键在于企业家和企业家精神。但是,美国经济学家卡尔·刘易斯认为,企业家并非天生就是创新的推动者,而且创新本身的风险就很大。企业家的主要动力在于牟利,当创新能够牟利时他就努力创新;当不创新也能牟利时,他就会通过很多与创新无关的方法来牟利。如果不借助创新就能牟利的机会很多,企业家就不会成为创新的推动者。\nauthor{阿瑟·刘易斯:《经济增长理论》,周师铭等译,北京:商务印书馆2005年版,第20—62页。}

如果是一个政府干预很少的市场中,市场本身就处在不断寻找均衡的过程中,这样的话,企业家不创新就很难牟利。比如,以手机市场为例,在苹果公司进入手机市场之前,诺基亚、摩托罗拉和三星三家公司生产的手机是很受欢迎的,消费者总体的满意度也比较高。这种条件下,苹果公司如何能赢得消费者呢?答案只有一个:创新,就是要生产顾客价值更高的手机。苹果公司发明新一代智能手机Iphone以后,他们才在手机市场获得巨大的成功。同时,消费者的满意度提高了,手机技术进步了,新的GDP被创造出来了,政府税收也增加了。总之,在市场化程度很高的经济体系中,一个企业家要想获得成功,他必须要为社会提供更好的产品或服务。如果只有通过创新才能牟利,企业家就不得不去创新。这种制度条件才使得企业家成了一种具有创新精神的经济动物。

但是,企业家的基本激励并非创新本身,而是牟利。某些条件下,不创新也能成功地牟利。比如,通过占有资源、获得牌照、与政府合作等等——一句话,企业家通过与政治权力的结合或参与寻租活动——就能获取丰厚利润时,他们根本没有动力去从事创新。判断一个国家的企业家是如何致富的,最简单的办法是观察该国的富豪排行榜。对一个经济真正繁荣的国家来说,哪里会产生更多富豪呢?往往是创新部门。对一个经济结构扭曲的国家,哪里会产生更多富豪呢?往往是垄断部门、资源部门和寻租部门。对后者来说,社会中被视为最有经济天赋和创造力的那部分人从事的往往是跟创新关系不大的活动,这样的国家要产生高质量的经济增长和实质性的技术进步就非常难。

对企业家而言,好的激励结构应该是:如果他想成为成功的企业家,他必须为消费者提供更好的产品或服务。在这个过程中,他给社会的直接贡献是创造了更高的顾客价值和满意度,社会拥有了性能更优、成本更低和价值更高的产品或服务;间接贡献是他同时在不断地发展新技术和提升管理,推动一个国家的技术进步、效率提升和管理优化。但是,如果一个企业家不需要通过提供更好的产品或服务、不需要通过技术创新和改进管理、不需要通过提高顾客价值和满意度,就能获得财务上的巨大收益和事业上的“成功”,那么这就是一种坏的激励结构。在这种激励结构下,一个国家也会形成一个富人阶层,但这样的社会无法实现真正的繁荣。富人阶层不仅更难获得其他阶层心悦诚服的尊重,而且往往为其他人树立了“坏榜样”。总之,企业家的激励结构决定了他们的所作所为,而这种所作所为会影响乃至决定整个国家的经济绩效。

再来看医生。医生,大概是目前中国非常敏感的一个职业。这些年关于医生、医院和医患关系的新闻也非常多。一些医生感觉目前的处境比较尴尬,而社会也给医生两方面的不同评价。一方面,很多人认为医生普遍接受过良好教育,从事医治病患、救死扶伤的工作,是一份体面的工作;另一方面,不少人认为一些医生没有尽到自己的责任,收受红包回扣,做了很多不应做的事情。如果我们看到一个医生的行为出现问题,可以说是个别医生的问题,但如果整个社会中从事同一种职业的人有一定比例出现问题,这就意味着整个行业的激励结构出了问题。

那么,如果我们从头给医生设计一种激励结构,应该设计一种什么样的激励结构呢?对医生来说,一个好的激励结构应该是:如果他想获取自己的回报,成为一个成功而富有的医生,那么他应当成功地治疗或治愈更多的患者。在这个过程中,他给社会的直接益处是提供了有效的医疗服务——使更多患者受益,帮助他们改善健康和延长生命;间接益处是他发展和提高了医疗技术,并塑造了职业医生的形象和正直医生的人格。但是,倘若一个医生这样做并不能获得合理的回报,而一个并不诚实的医生却获得了更多回报——比如,通过选用更劣质或更昂贵的药物、增加不必要的检查和治疗、甚至人为延长治疗周期和次数,等等,那么这样的激励结构就有问题。如果医生的收益和这些关系不大,主要取决于他卖出了多少药品,那你会发现整个激励机制就扭曲了。在这种激励结构下,会有更多的医生违背作为一个医生本来应该服务的目的。这样,患者有时就无法得到有效医治,药物和医疗技术也难以有效进步,医生作为一种职业也无法赢得患者与社会的真正尊敬。

医生作为经济人,社会应该给他们塑造一种有效的激励结构:只要他努力去做一个优秀的医生,去救死扶伤和精研医术,他就能获得良好的回报。如果医疗或医患关系出现了较大问题,首先需要检讨的是医生和医院背后的整个激励结构。现在的激励结构,不仅对医生是不利的,而且对患者变成了一件风险巨大的事情。总之,整个社会都没有好处,每个人都处在这样的风险里面。

最后再来看政治家与政府官员。对政治家或者官员而言,给他们建立一种什么样的激励结构才会对整个社会有利呢?最近看到一个新闻,美国总统奥巴马到美国某个州“视察”,但是该州州长并没有时间陪同奥巴马。为什么州长可以不陪同总统?道理很简单,美国的州长能不能做州长,不取决于总统满意不满意,取决于本州的选民满意不满意。所以,一个珍惜时间的州长首先会把时间花在他认为能够更好地履行州长职责、能够让本州选民更满意的事务上。如果换一种制度,比如由总统来任命州长,那估计州长会鞍前马后地全程陪同到访本州的总统先生。由此可见,激励结构不同,州长的政治行为就会不同。

对政治家或政府官员而言,一个有效的激励结构应该是:如果他想赢得更高的职位、更大的政治权力和傲人的职业成就,他必须真正服务于公众利益,为社会提供更好的公共服务。比如,衡量食品药品监管部门官员的标准应该是,该国公民是否享受到更有效的药品和更安全的食品?衡量教育部门官员的标准应该是,该国的教育质量是否得到不断提升及是否培养出越来越多的杰出人才?衡量工商管理部门官员的标准应该是,该国公民创办经营企业是否更便捷、市场交易规则是否得到尊重?衡量整个政府官员的标准应该是,该国公民是否以较低的成本享受到了高质量的公共产品与服务?当这些绩效成为相应官员的激励标准,能够决定他们的去留和升降,这才是一个有效的官员激励结构。

但是,如果一个政府官员无须为社会公共服务贡献真正的价值,他就能获得可观的回报——比如,他只要通过搞好跟上级的关系以及做好一些表面文章,就能获得晋升,或者他只要通过与工商业主的共谋与交易,就能获得巨额的财务收益且风险较低——那么这就是一种坏的激励结构。如果这种激励结构成为普遍的游戏规则,政治与行政的公共性就会沦为一个冠冕堂皇的口号。

当然,不只是医生、企业家和政府官员身处于特定的激励结构之中,职业经理、工程师、科学家、学者、中小学教师、建筑工人、清洁工等各行各业的从业者莫不如此。如果总体上说人是理性的话,那么正是他所面对的激励结构塑造着他的行为——好的激励结构塑造好的行为,坏的激励结构塑造坏的行为。这种各行各业从业者的行为加总,就是我们所看到的整个社会的生态。当盼望一个更好的社会时,我们盼望的其实是更好的官员、更好的企业家、更好的医生、更好的教师和更好的清洁工,而他们都是更好的激励结构的产物。

如果要从激励结构角度回答如何实现一个国家有效治理的问题,其实现有的经济学和政治学理论已经揭示了一条最简单的规则:一个国家要实现有效治理和经济增长,就需要在两个领域——私人部门和公共部门——塑造好的激励结构。私人部门的激励规则应该是:只有那些为他们的顾客、用户、委托人或社会创造真正价值的人,才能获得自身的回报与成功。而公共部门需要类似的激励规则:只有那些为大众与社会提供有效公共产品和服务的人,才能获得自身的回报与成功。从制度技术的层面说,有效治理国家不过是要把这个简单规则在法律、制度和程序上落到实处。如何让一个医生成为更好的医生?如何让一个企业家成为更好的企业家?如何让一个官员成为更好的官员?最直接的做法是改变他们面对的激励结构,这样才能实现有效治理和持久繁荣。

以人类现有的知识来说,善治与繁荣并无多少秘密可言。只需观察这个社会中的多数人是否处在正确的激励结构当中。一个简单标准是,各行各业的人们是否处在这样的激励结构中——他们在寻求自我利益的过程中,是否必须在很大程度上促进他人的利益以及整个社会的利益?如果符合这一标准,就是一个好的激励结构;不符合这一标准,就是一个坏的激励结构。

实际上,曼瑟·奥尔森和德隆·阿西莫格鲁等人已在他们的著作中阐明了激励结构的重要性。比如,曼瑟·奥尔森认为,当一个国家的激励结构出现问题时,其长期经济绩效就不会太好。他说:

\quo{……当存在清晰的激励生产的措施时,通过专业化和贸易的社会合作,社会更有可能获得繁荣的增长。如果一个社会要获得可能的更高的收入,那么激励措施必须是清晰的、明确的,同时还必须促使经济生活中的个人和公司在一种社会最有效的途径中互动。

……当存在激励因素促使人们去攫取而不是创造,也就是从掠夺中而不是从生产或者互为有利的行为中获得更多收益的时候,那么社会就会陷入低谷。\nauthor{曼瑟·奥尔森:《权力与繁荣》,苏长和、嵇飞译,上海:上海人民出版社2005年版,第1页。}}

德隆·阿西莫格鲁及其合作者在《国家为什么会失败》中也论证了激励结构对于一国经济增长的重要性。他们认为,正是一个国家的政治制度和经济制度决定了这个国家的经济绩效。他们把不同类型的政治经济模式分为两种:一种是攫取性的(extractive),一种是包容性的(inclusive)。在攫取性政治经济制度下,一部分人扮演着掠夺者的角色,从而破坏了一个社会较为合理的激励结构,长期当中就无法实现经济增长和繁荣。\nauthor{Daron Acemoglu and James A. Robinson, \italic{Why Nations Fail: The Origins of Power, Prosperity, and Poverty}, London: Profile Books, 2012.}

\tsection{腐败的政治经济学}

腐败是很多国家的政治问题,分析腐败问题也可以借鉴政治经济学的视角。大家经常在媒体上见识各种各样的腐败案件,中国最近两年落马的省部级以上官员之多,确属罕见,这在很大程度上反映出最高层的反腐力度在增加。这里的问题是:在纷繁复杂的腐败现象背后,有一个一致的逻辑吗?腐败的案例不胜枚举,理解腐败的关键是要理解腐败的基本逻辑。\nauthor{关于腐败研究,参见苏珊·罗斯·艾克曼:《腐败与政府》,王江、程文浩译,北京:新华出版社2000年版。}

如何定义腐败?腐败的最简单定义是钱权交易。更一般地说,腐败是一种用公共权力谋取私人利益的行为。腐败最常见的当然是权钱交易,但从已经披露的案件来看,权色交易也很常见。腐败可能还包括各种形形色色的与公共权力有关的交易活动。

由于政府官员也是经济人,他随时都存在谋取合法的或非法的私人利益的冲动。尽管如此,腐败的出现仍然需要特定的条件。第一,政治权力应该包含了大量的资源。只有当政治权力掌握相当资源时,权力才有可能被拿来做交易。比如,有的官员职位也很高,但这个职位并没有掌握什么重要资源,这种权力就很难拿去做权钱交易。但如果某个职位的政治权力中包含了大量资源——即便其职位不是很高——它就有可能被拿去做交易。这还可以推导出,政治权力包含的资源越多,腐败的可能性越大,腐败规模可能越大。第二,即便政治权力包含着很多资源,但如果行使政治权力的过程时时有人监督,权力行使过程是完全透明的,这个职位上的人还敢或还能腐败吗?腐败的可能性就小了很多。所以,权力制衡机制也同样重要。

基于上述简要分析,这里提出一种关于腐败的简单理论,用如下公式来表示:\nauthor{这里关于腐败的理论思考,得益与北京大学傅军教授讨论的启发。}

\[C=F(Pr, C\&B)\]

C表示腐败corruption,Pr表示权力控制的资源(power-resource),C\&B表示分权制衡(checks and balances),F表示函数关系。上述函数代表的理论假说是:腐败程度取决于权力控制的资源多少和分权制衡程度。更具体地说,权力控制的资源越多,分权制衡程度越低,则越腐败;权力控制的资源越少,分权制衡程度越高,则越不腐败。根据上述理论,可以总结出一个关于腐败的四象限表格,如表13.1。

\tbl{../Images/image00336.jpeg}[表13.1 权力资源、分权制衡与腐败]

在表13.1中,如果政治权力控制资源很多,分权制衡机制很低,就会导致最严重的腐败;如果政治权力控制资源很少,分权制衡机制很高,腐败将会减少到最低程度;此外,政治权力控制资源多但分权制衡机制很高,或政治权力控制资源很少但分权制衡机制很低的两种类型,腐败程度处于中间状态。当然,这里仅限于逻辑分析,还没有到经验世界中去检验。最后一讲会涉及如何检验这一理论假说。

如果上述理论假说得以证实,应该如何反腐败呢?理论一旦有了,政策就是一个自然的结果。从腐败函数公式来看,反腐败主要有两种策略:第一,要降低政治权力控制的资源数量;第二,要强化政治权力的分权制衡机制。这才是反腐败的有效方式。

当然,具体政策和措施可以有很多。比如,从政治权力控制资源数量的角度来说,国有企业比例越高,行政审批越多,市场管制越多,政府控制土地、矿产与能源资源越多,甚至政府控制教育指标和户口指标越多,总体上该国就会越腐败。如果要真正反腐败,就需要在这些方面降低政治权力控制资源的数量和比例。

反腐败的另一个维度是分权制衡。那么,这是否意味着越民主,腐败程度就越低呢?应该说,分权制衡并不必然意味着民主政体。但是,在现代世界,民主政体与分权制衡的制度安排是密切相关的。所以,这里的分权制衡,第一层含义可以理解为通过民主方法让选民监督政治权力和政府官员。这是一个最根本的办法。除了民主方法,分权制衡的第二个主要机制是政府内部不同权力之间的分立与制衡。无论是两权、三权还是五权,政府体系内部需要权力制衡的机制。此外,分权制衡还可以借助新闻媒体和公众舆论等途径。如果一个地方的媒体自由度很高,公民言论自由度很大,政府官员就越不可能腐败。借助这一理论框架,反腐败的政策建议是一目了然的。

有人说,如果不这样做,能否有效反腐败呢?短期中当然是有可能的,比如揪出几个贪腐高官,会起到一定的震慑作用。但长期中,只有降低政治权力控制的资源和强化分权制衡机制两种办法,其他的反腐败措施很可能是治标不治本。

\tsection{推荐阅读书目}

曼瑟·奥尔森:《权力与繁荣》,苏长和、嵇飞译,上海:上海人民出版社2005年版。

丹尼尔·耶金、约瑟夫·斯坦尼斯罗:《制高点:重建现代世界的政府与市场之争》,段宏等译,北京:外文出版社2000年版。

丹尼斯·C.勒纳:《公共选择理论》(第三版),韩旭、杨春学等译,北京:中国社会科学出版社2010年版。

朱天飚:《比较政治经济学》,北京:北京大学出版社2006年版。
