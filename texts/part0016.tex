\tchapter{如何做政治科学研究?}

\quo[——卡尔·波普尔]{我建议应当把理论系统的可反驳性或可证伪性作为分界标准。按照我仍然坚持的这个观点,一个系统只有作出可能与观察相冲突的论断,才可以看作是科学的;实际上通过设法造成这样的冲突,也即通过设法驳倒它,一个系统才受到检验。}

\quo[——托马斯·库恩]{范式是一个成熟的科学共同体在某段时间内所接纳的研究方法、问题领域和解题标准的源头活水。因此,接受新范式,常常需要重新定义相应的科学。……以前不存在的或认为无足轻重的问题,随着新范式的出现,可能会成为能导致重大科学成就的基本问题。当问题改变后,分辨科学答案、形而上学臆测、文字游戏或数字游戏的标准经常也会改变。}

\quo[——艾尔·巴比]{大体而言,一个论点必须有逻辑和实证两方面的支持:必须言之成理,必须符合人们对世界的观察。}

\quo[——劳伦斯·纽曼]{我们要警惕那些以一副真正的自然或社会科学面目出现的伪科学。公众始终受着通过电视、杂志、电影、报纸、专门研讨班或工作单位以及类似方式传播的种种伪科学的蒙蔽。}

\tsection{你凭什么相信?}

前面每一讲都跟政治学领域的某种专门知识有关。这一讲要介绍的是“如何获取知识的知识”,即政治学的研究方法问题。我们先从一个问题切入:你凭什么相信?大家每天听到看到各种不同的观点。比如,在课堂上听到很多老师的不同观点,在媒体上知道很多知识分子与公众人物的不同观点,在很多专著上了解思想家与学者的不同观点。然而,其中很多观点是互相冲突的。那么,你更相信哪些观点?你又凭什么相信?

这个问题涉及社会科学方法,这也是本讲的主题。这里有另一个小岛的故事。\nauthor{这是作者过去读到过的一则案例,但写作过程中已无法找到原始出处,作者凭记忆做了改编。

在18、19世纪,有一个离大陆数百公里的小岛上经常发生疫情。这个岛上居住着一个土著部落,每到冬天时,很多人就会生病,少数人病情严重、甚至会病死。到了春天,这种疫情又会慢慢消失。这样,疫情每年终而复始,岛民们百思不得其解:为什么岛上每年都在特定时候出现疫情呢?后来,有人发现,每当冬季到来的时候,经常有一些来自大陆的在海上迷路的船只在他们的码头停留——可以想象,当年的导航系统和动力系统都不像今天这样好。这些迷路的船只通常还会上岛交换一些生活和航海必需品,比如淡水、食物和燃料等等。后来,就有人猜测,是不是大陆船员通过这种方式把疾病传给他们,从而导致了岛上的疫情?当时,医学不如现在发达,岛上居民还不知道细菌或病毒这些概念,他们只是猜测大陆船员身上可能有某种不洁净的东西,正是船员们引起了岛上的疫情。这种说法流传开以后,小岛的居民们开始对那些迷失航向的船只感到恐惧,开始排斥来自大陆的船员。这也造成了小岛居民与大陆船员之间的紧张关系。尽管如此,疫情到了每年冬天还是照样爆发。

后来,到了20世纪有生物学家去研究:为什么这个岛上到了每年特定时间就会出现疫情?经过系统的调查研究后,他们发现,每当冬季到来,岛上的风向就变了。其他季节,小岛上的风是从更远的大洋深处刮来的;而到了冬季,风是从临近的大陆刮来的,风力还比较大。从大陆方向刮来的大风导致了两个结果:一是大陆上的空气携带的病菌随风到了小岛上,加上小岛土著居民缺乏对大陆病菌的抵抗力——正如贾雷德·戴蒙德在《钢铁、病菌与枪炮》中所讨论的,从而引发了小岛疫情;二是风向的改变使得那些在海上迷失航向的船只更有可能顺风漂流到小岛上。20世纪生物学家们的研究表明,并不是大陆船员把病菌带到了岛上,而是大风把大陆病菌带到了岛上。}

这尽管是一个医学和疾病研究的案例,但完全可以用社会科学语言来分析。这个案例涉及三个变量关系:一是小岛的疫情(epidemic,简称E),二是迷失航向后在小岛停留的大陆船只(boats,简称B),三是季节变化导致的风向改变(wind,简称W)。最初岛上居民认为,停留在小岛的大陆船只(B)以及船员身上携带的病菌导致了小岛的疫情(E),B是因,E是果。而20世纪的研究表明,E和B的出现,有一个共同的原因——风向改变W。风向改变(W)同时导致两个结果:一方面风把大陆细菌携带到了岛上;另一方面是迷失航向的船只更容易漂流到这个小岛上。

过去,小岛居民发现了疫情(E)和大陆船只(B)之间的相关性,这种相关性被误以为是因果关系。但实际上,相关性并不等于因果关系。后来的研究证明,这两个因素只是恰好同时出现而已,它们背后有一个共同的原因,就是风向改变。这样,这项研究才找到了真正的因果关系,参见图14.1。从这个案例看出,在科学研究中,相关性与因果关系不是一回事。

\img{../Images/image00337.jpeg}[图14.1 小岛疫情的相关性与因果关系]

\tsection{新闻报道中的事实与观点}

很多人每天都要看新闻节目。新闻报道通常由两部分构成:一是事实,二是观点。新闻报道通常以事实为主,但记者或主持人最后往往还会陈述自己的观点。既然新闻需要报道事实,那么到底什么是事实呢?在每天的生活中,你观察到事实了吗?你到底是在观察事实,还是在想象事实?比如,如果给人装上鹰的眼睛与视觉系统,让人站在摩天大楼的楼顶,若天气和空气质量尚好,一个人应该能看清几公里以外的东西,而且整个世界的颜色不再是人平时看到的样子。这个小例子告诉我们:到底什么是事实,其实不那么容易回答。很多时候我们感知的并非事实本身,而是我们对事实的想象。

对于一篇新闻报道来说,存在两个“事实”:一个是事实本身,另一个可以称之为陈述的事实。普通读者读到的新闻报道不过是记者陈述的事实。比如,某重大突发事件之后,相关机构召开新闻发布会,各路新闻记者到达现场。新闻发布会结束后,不同的记者发回去的新闻报道差异很大,不仅各人报道的重点不一致,而且一些重要信息上甚至存在互相矛盾之处。所以,很多时候,新闻报道的内容并不是那么可靠。如果新闻报道尚且如此,社会科学学者做研究时真的能基于事实吗?如同新闻报道一样,这个世界上被陈述出来的事实,往往并非事实本身,而是陈述者对于事实的想象。事实应该是唯一的,但对事实的陈述却是五花八门。这对从事经验研究的学者无疑是一个巨大的挑战。

很多新闻报道还会同时配发评论。新闻报道的常见模式是:一个重大新闻事件讲完后,各种各样的问题都冒出来了,很多记者或主持人在评论部分经常会说:“希望有关部门出来管一管。”比如,某地出现了一起食品安全事故,记者深入现场采访,写了一篇两千字的报道,最后两三百字往往是事件的评论。评论说,目前食品安全形势非常严峻,有关部门工作没有到位,应该加强食品安全立法与监管,严格食品企业的准入门槛,强化食品安全生产许可证的颁发流程,等等。总之,食品安全有问题,“希望有关部门出来管一管”。

\img{../Images/image00338.jpeg}[图14.2 新闻报道中的事实与观点]

但是,比较较真的人看完新闻报道后开始提问:“你提出的思路与政策建议真的能奏效吗?”甚至有人提出相反的观点:目前食品安全问题严重,恰恰是食品行业管制过度与市场自由度不足导致的。政府只要做少量必要的监管并严格执行已有法律,放松食品行业的管制并鼓励食品企业自由竞争,食品安全问题就能得到改善。这里,大家听到了两种不同的观点。

面对同样的新闻,假如有人表述了相反的观点。那么,一名记者会有信心说自己的观点更合理吗?实际上,当记者报道一个新闻并陈述一个观点时,背后还有一个东西他没有明确地说出来。那就是导致问题的原因是什么?这个重大新闻事件背后的因果关系是什么?上述事例中,要问为什么食品安全问题如此严重?这一问题背后的因果关系是什么?只有基于对新闻背后因果关系的分析,新闻评论中的观点才会更有说服力,参见图14.2。而这又跟社会科学方法论有关。

\tsection{社会科学研究的常见谬误}

美国社会学家艾尔·巴比在《社会研究方法》中认为,社会科学研究会产生几种常见的谬误,特别是:

\quo{不确切的观察。……譬如,你们第一次见到方法论老师时,他穿的鞋是什么颜色?如果你们靠猜测,就表示我们的日常观察都很随意而且漫不经心。这就是为什么大多数的日常观察不同于实际情形的原因。

过度概化。当我们探讨周围事物的模式时,通常会把一些类似的事件当做某种普遍模式的证据。也就是说,我们在有限观察的基础上,作了过度的概括。

选择性观察。……一旦你们认为存在某种特别形态,且获得了对该形态的一般性理解,就很可能只注意符合这种形态的事物或现象,而忽视其他不符合的状况。

非逻辑推理。……统计学家所说的赌徒谬误是日常生活中常见的又一个不合逻辑的例子。风水轮流转,一晚上手气不好的赌徒,总认为再过几把之后幸运就会降临。很多赌徒舍不得离开赌桌的原因就在于此。\nauthor{艾尔·巴比:《社会研究方法》(第十一版),邱泽奇译,北京:华夏出版社2009年版,第8—9页。}}

在上述谬误类型中,选择性观察或选择性偏差是一个最为常见的现象。这里再试举两例。一个例子是,一些国外学者首次访问中国前后,对中国的印象往往截然不同。比如,位于上海的某著名大学要召开一次高端国际学术研讨会,会场安排在学校旁边的某国际连锁五星级酒店。其中一位参会者是美国某大学的著名学者。会务组安排他乘坐从美国飞往上海浦东国际机场的航班——由于这位教授是国际顶级学者加上年岁已高,会务组就负担了他头等舱的机票(通常高校不能负担这个费用,大概要基金会赞助的会议才可以)。这位大牌教授坐头等舱坐到浦东机场以后,会务组派一辆中高级商务车去机场接他,安排他入住五星级酒店。第二天,他就在酒店里参加学术会议。会议期间,会务组考虑到很多国外教授首次来上海,就安排大家利用晚上时间观赏外滩和陆家嘴的夜景。会议结束后再把他送到机场,让他坐头等舱回去。

如果一个国外学者首次到中国走的是这样的路线和行程,他可能会惊叹于中国的发展水平!他所到之处,享受的硬件设施都是一流的。比如,浦东国际机场的硬件设施在国际上也是首屈一指的,整个欧洲几乎没有哪个机场的硬件设施比浦东机场更好。他坐中高级商务车过来,一路上都是高等级的高速公路或高架环线,又在五星级酒店下榻和开会——这种跨国连锁的五星级酒店在全球的设施都是一样的。晚上,他参观和游览了上海外滩这一中国城市景观最奢华的地方,丝毫感觉不到这是一个发展中国家的城市夜景!但是,这位美国学者倘若凭这些印象来判断中国的整体发展水平,就陷入了“选择性偏差”的误区。实际上,上海、北京和深圳这样的城市代表的是中国最高的发展水平。要完整地理解整个中国的发展,还必须要去观察中国更多的普通城市和乡村地区。

再来看一个案例。大家知道,抽样调查与统计分析是目前社会科学研究的常用方法,但抽样调查和统计分析有时也会欺骗人的眼睛。二战前后,美国陆续兴起了很多选举调查机构,这些机构通常会在大选年分阶段公布不同总统候选人的支持率,并预测谁将胜出。这种选举调查通常样本很大,分析技术也比较可靠,所以预测的准确率通常较高。

但是,在1948年美国总统选举中,盖洛普调查机构却出现了总统选举预测的乌龙事件。在那年选举中,民主党总统候选人是时任总统的哈里·杜鲁门,共和党总统候选人是托马斯·杜威。在正式选举之前,盖洛普选民调查都表明,杜威将以较大优势战胜杜鲁门,当选下一任美国总统。在这种选民调查的支撑下,亲共和党的《芝加哥论坛报》甚至在选举结果尚未揭晓的情况下打出了《杜威战胜杜鲁门》的封面报道。但是,实际选举结果恰恰相反,杜鲁门反而以4.5\%左右的选票优势、114张选举人票优势继任美国总统。盖洛普调查机构非常沮丧,他们花了很多钱,雇了不少人,在美国不同地方做抽样,有专业人士做顾问,为什么结果是错误的?

通过回溯整个调查过程,有人找到了关键问题。盖洛普调查机构当时主要通过电话进行抽样调查。但对于1948年的美国来说,电话尚未普及。有钱人首先装电话,然后是中产阶级,然后是低收入群体。所以,用电话进行选民调查,最大的问题是相对有钱的家庭被调查到的可能性更大。尽管被调查的电话号码本身是随机抽样产生的,但电话在所有美国家庭中的分布不是随机的。结果是,这一调查就产生了严重的选择性偏差。这也是一个非常经典的案例。

\tsection{什么是科学与科学方法?}

社会科学研究是科学研究的重要组成部分。在中国的语境中,科学有两个含义:第一个含义是指“体系化的知识”,比如政治科学可以被理解是关于政治的体系化的知识;第二个含义是指“基于对真实世界的观察,藉由提出假说及提供检验方式得到的可靠知识”,这是更加严格的科学定义。根据这一定义,首先,科学要基于对真实世界或经验世界的观察,这是基础。科学既不能以想象或假想为基础,也不是哲学式思辨的探索,这些都不能称之为科学。其次,科学理论研究的起点是提出假说(hypothesis)。理论假说通常是一种因果关系的表述,即何种原因导致何种结果,其简化形式就是原因A导致结果B。国内也有学者喜欢把“hypothesis”翻译成“假设”。最后,科学理论能成立的条件是理论假说在经验世界中经受检验并得以证实。这是验证假说的过程。借用胡适先生的说法,就是“大胆假设、小心求证”的过程。经过证实之后,理论假说就成为一种相对可靠的科学知识。

比如,进化论就是一种重要的科学理论。进化论关注的是地球上不同物种之间的关系。为了解释这一问题,进化论的主要创立者查尔斯·达尔文首先提出一个假说:不同的物种之间是进化的,物种进化的机制主要是两种:遗传和变异,但决定进化的条件是竞争与选择。应该说,这一理论非常简洁。一个理论假说,主要用两种机制,试图把整个生物界的演化都解释了。那么,作为一项科学假说,大家一定会问:这种假说经得起经验证据的检验吗?达尔文以及后世的科学家在古生物化石、物种的全球分布、物种形态比较、物种生长发育过程以及分子生物学等等方面,找到了支持进化论假说的大量系统证据。所以,进化论尽管远非完美,但对于世界上物种演化的解释却是迄今为止最有力的理论。通过进化论也可以看出,科学首先是对真实世界的观察,其次是提出解释真实世界的理论假说,再次是对该假说进行系统的验证,最后得到相对可靠的知识——这才可以被称为科学或科学理论。\nauthor{参见达尔文:《物种起源》,周建人、叶笃庄、方宗熙译,北京:商务印书馆1995年版。}

谈到科学研究,就离不开科学方法。那么,什么是科学方法呢?按照《韦伯斯特词典》的解释,科学方法被视为“有系统地寻求或者获取知识的一种程序”,主要有三个步骤:一是“问题的认知和表述”,就是说明要解释的问题是什么;二是实验数据的收集,就是要在真实的经验世界里收集数据;三是假说的构成与检验,就是先提出假说然后用经验证据去证实假说。由此可见,科学方法既不同于文学表述,又不同于哲学思辨。

\tsection{社会科学需要探索因果关系}

那么,什么是社会科学呢?社会科学是以科学方法研究人类行为和人类社会现象的学科。与自然科学最大的不同是,社会科学以人类行为和人类社会现象作为研究对象。应该承认,研究对象的差异给社会科学研究的科学性带来了极大的挑战。

首先,在社会科学研究中,人的思想和行为本身被包含在研究过程中。人的思想和行为本身又会影响人类社会的诸种现象。这就给社会科学研究的科学性增加了难度。比如,面临同样的结构和制度,不同的人可能会做出不同的选择。举例来说,某个组织给所有人都设计了同样的激励制度,但有的人选择更努力地工作,以获得更多收入;有的人选择只付出适当的努力,以获得更多闲暇。用经济学的话来说,每个人对效用函数的定义是不同的,每个人的偏好并不相同。同样的逻辑可能还会出现在非常棘手的情境中。比如,两国关系紧张时,各种信息与数据都表明那个相对较弱的国家不会首先开战。但是,弱国的政治领导人恰好是一位超常强悍的政治家。即便在不利条件下他仍然决定开战,由此政治家的个性可能就改变了该国政治趋势和地区政治格局。所以,社会科学研究包含着人的思想和行为,能否科学化是一个问题。

其次,社会科学研究中,某些事实与数据的客观性程度是较低的。比如,一项关于美国选民对现政府支持率的研究中,学者提出这样一个理论假说:凡是生活满意度高的选民,更倾向于支持现政府和执政党;凡生活满意度低的选民,倾向于反对现政府和支持反对党。然后,这个学者组建一个团队在全美国做抽样调查。调查问卷上主要设计了两个问题:第一,您对过去四年的生活满意吗?选项包括:A. 非常满意;B. 比较满意;C. 一般;D. 不太满意;E. 很不满意。第二,您在最近的总统选举中投票给谁?选项包括:A. 现任执政党;B. 主要反对党。第二个问题的回答更加客观。如果被调查者没有记错且不故意造假,一般都会给出准确的回答。但是,第一个问题的回答就具有很强的主观性。比如,那些总体满意的选民,会在A、B选哪一项呢?那些总体不满意的选民会在D、E选哪一项呢?还有那些中间摇摆的选民,会在B、C、D选哪一项?很多人有这样的生活经历——比如,你刚上大学或刚工作时,也许你的真实满意程度并不高,然后父母这样问你:“你的大学生活(或工作)怎样?”很多人会习惯性地这样回答:“挺好!”当你说“挺好”的时候,你父母就在你这里得到了一个调查数据。但如果一个学者根据这样的数据去做研究,最后可能得不到多少有意义的结论。而社会科学研究中充斥着此类问题。

再次,社会科学研究中,还存在难以进行可控实验的问题。政治学经常会关注一些重大问题,比如社会革命、政变、内战与国家间战争等。诸如此类的研究完全无法进行可控实验。因此,关于社会革命、政变、内战与国家间战争的理论研究做得再好,也很难有效评估这一理论是否具有预测性。这也是社会科学不同于自然科学或工程科学的地方。

当然,问题的另一面是,社会科学在研究的核心逻辑上与自然科学并无实质区别。社会科学研究的核心仍然是发现现象背后的因果关系。那么,如何确定因果关系呢?在实际研究中,要确定因果关系并非易事。以本讲开头的小岛疫情为例,如果没有后来20世纪科学家的研究,大部分岛上的居民都倾向于认为疫情是大陆船员带来的某种“邪恶因素”。理由也很充分,每当大陆船员来到岛上时,疫情就会出现和流行。但是,实际情况并非如此。

那么,如何确定因果关系呢?确定因果关系至少需要满足三个条件:一是原因与结果之间要有相关性(correlation);二是原因与结果之间要有时间上的先后次序(time),即因在先、果在后;三是原因与结果之间存在引发机制(mechanism),即原因通过某种确定的机制导致结果的产生。当然,最难确定的就是因果机制。

比如,有人发现,某海滨城市冰淇淋销量和溺水死亡事故数据高度相关,而且从时间顺序上看先是冰淇淋的销量逐渐提高,然后是溺水死亡事故数据也逐渐提高。那么,能否据此认为“冰淇淋销售导致溺水事故”呢?当然,这纯属无稽之谈。这个简单的案例揭示,相关性和时间先后不代表因果关系,确定因果关系必须要找到两者之间的引发机制。在这个案例中,冰淇淋销量与溺水事故数量都跟一个共同原因有关:季节转换与气温上升。气温上升,夏季来临,一方面使得冰淇淋销量上升,另一方面使得更多人下海游泳,从而导致溺水事故的上升。

当然,这是一个非常简单的案例。“冰淇淋销量上升导致溺水事故上升”这一假说的荒谬性显而易见,另一个竞争性假说“气温上升既导致冰淇淋销量上升、又导致溺水事故上升”的有效性也很清晰。但是,人类行为和人类社会现象往往要比这个案例复杂得多,因此,因果关系通常不会这样简单清晰。比如,如果研究的问题是一个国家的经济增长或政体转型,往往较难确定背后的因果关系。到底是何种因素决定经济增长或政体转型?这一因素与经济增长或政体转型之间存在确定的相关性、时间的先后次序及明确的引发机制吗?这些问题经常不容易回答,特别是因果机制往往最难确定。

\tsection{社会科学与变量语言}

社会科学研究经常采用一种被称为“变量”(variable)的学术语言。理解什么是变量,首先要理解什么是概念。对所有社会科学研究而言,概念是构建理论的基石,涉及“是什么”的问题。概念通常有两类:实体概念和非实体概念。前者比较具体,而后者比较抽象。比如,桌子、道路、菠萝、电脑等都是实体概念,这些概念具体而容易描述。但是,非实体概念就不是这样,比如,阶级、权力、平等、支配等概念就不那么简单明了。政治科学涉及的很多都是抽象问题。比如,一种理论认为“不平等会导致政治冲突的增加”。这里涉及两个概念:不平等和政治冲突。什么是不平等?如何界定?如何衡量?什么是政治冲突?如何界定?如何衡量?要做这样的研究,首当其冲是要解决概念界定及衡量问题。

北京大学袁方教授认为,在界定概念时要把“概念名词”“抽象定义”和“经验现象或事物”三者统一起来,请参见图14.3。概念直接表现为名词,比如社会主义就是一个名词。接着可以问:社会主义的抽象定义是什么?这个必须要界定清楚。最后还要问:经验世界中实际观察到的社会主义是什么?只有社会主义这一名词、社会主义的抽象定义、经验世界中观察到的社会主义三者相一致时,这个概念才是可以用于社会科学研究的概念。如果这三者不统一,实际上是没有办法做研究的。要么这个概念的边界是模糊的,是随时可以扩张与收缩的;要么名词、抽象定义、经验事实三者之间存在矛盾。当然,政客们可能喜欢玩这种游戏。但对学者来说,如果这三者不统一的话,就没有办法做研究了,因为你甚至无法说清在研究什么。所以,社会科学研究的首要准则是概念要有明确和清晰的界定。

\img{../Images/image00339.jpeg}[图14.3 概念名词、抽象定义、经验现象或事物三者的统一]

资料来源:袁方主编:《社会研究方法教程》,北京:北京大学出版社1997年版(2013年重排本),第56页。

现代社会科学一般用变量语言来阐述理论,因果关系就是两个或一组变量之间的关系。学者们通常把结果称为因变量(dependent variable),把导致结果的原因称为自变量(independent variable)。这样,社会科学理论可以简约表述为“某个或某些自变量如何以及为何引发某个因变量”的形式。社会科学研究中的变量关系,还可以借用数学函数来表示,比如:

\[y = f(x) 或 y = f(x_1,x_2,……x_n)\]

这里的y就是因变量,x就是自变量。$y = f(x)$ 代表的是单因素的因果关系,即自变量x导致因变量y的发生,或者说自变量x的变化引起因变量y的变化。$y = f(x_1,x_2,……x_n)$ 代表的是多因素的因果关系,即自变量 $x_1,x_2,……x_n$ 的共同作用导致因变量y的发生,或者说自变量 $x_1,x_2,……x_n$ 的组合变化引起因变量y的变化。当然,如果自变量过多,一项研究就不能满足简洁性的要求。现有的基于多因素因果关系的社会科学研究,通常把自变量控制在几个以内,常见的研究是控制在4—5个以内。

比如,上一讲曾提出过一个解释腐败的理论假说,用函数公式表示就是:

\[C = F(Pr, C\&B)\]

C表示腐败(corruption),Pr表示权力控制的资源(power-resource),C\&B表示分权制衡(checks and balances),F表示函数关系。上述函数公式代表的理论假说是:腐败程度取决于权力控制的资源多少和分权制衡程度。更具体地说,权力包含的资源越多,分权制衡程度越低,则越腐败;权力包含的资源越少,分权制衡程度越高,则越不腐败。这个例子也说明,用变量语言来表述社会科学研究是一种更为简洁的方式,通常比普通语言的描述更加清晰、明确和简洁,其形式化和模型化程度也更高。

\tsection{比较研究的主要方法}

社会科学研究经常借助比较研究方法来确定因果关系。实际上,在自然科学研究中,像药物疗效等通常也是借助分组对比实验的方法。约翰·斯图亚特·密尔在《逻辑体系》中探讨过两种基本的比较方法:求同法和求异法。这仍然是目前确定因果关系的重要方法,当然每种方法都有其优劣。如图14.4左侧图所示,当研究结果X何以出现时,四个案例中均发现了变量A。尽管其他变量差异很大,但只要有X结果的地方都有A。这样,就可以有把握地说,A是引起X的原因。求同法寻求的是不同案例中的共性,通过共性来揭示因果关系。

\img{../Images/image00340.jpeg}[图14.4 确定因果关系的三种研究方法:求同法、求异法与共变法]

什么是求异法?如图14.6中间图所示,在两组非常相似的案例中,B、C、D三个变量都存在,主要差异在于案例一有A,结果有X;案例二没有A,结果没有X。这样,就可以有把握地说A是导致X的原因。在社会科学研究中,选取两个非常相似的案例,就接近于自然科学中的实验条件。比如,研究人类行为时用双胞胎来做实验就是一个典型方法。双胞胎的基因非常接近,基因这个重要变量就被控制住了。然后,再研究具体培养环境或其他条件不同对双胞胎中两个不同个体成长的影响。这样,通过求异法就能确定因果关系。在国别研究中,两个极其相似的国家往往是一组非常难得的比较案例。比如,有学者在研究东德和西德或朝鲜和韩国时,就能控制很多其他变量,包括人种、历史、语言和文化等。如果有人对东德和西德或朝鲜与韩国的发展差距感兴趣,那么大概可以通过求异法找到这种发展差距的主要原因。

还有一个重要的方法是共变法。这种方法用两条曲线很能说明问题,曲线A代表变量A,曲线X代表变量X。如图14.4右侧图,在一个特定的时间范围里,如果你发现曲线A和曲线X的变化趋势是相同——A变的时候X也变,A上升时X上升(或下降),A下降时X下降(或上升),就能判断两者很可能存在因果关系。当然,要把这种相关关系明确为因果关系,还需要对引发机制进行分析。比如,有学者认为经济危机会导致政治不稳定,证据是同一个国家的不同时期中,第一阶段经济平稳发展时,政治就稳定;第二阶段经济波动剧烈时,政治就不稳定;第三阶段经济平稳发展时,政治又变得稳定。这位学者采用的论证方法就是共变法。共变法也是社会科学研究常用的方法之一。

尽管求同法、求异法和共变法是主要的比较研究方法,但上述方法都存在不同的缺陷。比如,上文业已指出,简单采用共变法只能确定相关关系,而非因果关系。至于求同法和求异法,也存在明显的缺陷。求同法的最大问题是可能存在被现有分析框架所忽略的共同因素,比如在图14.6左侧图中,普遍认为A是引发X的原因,但可能四个案例均存在另一个共同原因S——但S这一变量在现有研究中被忽略了。从技术上讲,由于相关的重要因素难以穷尽,所以求同法常面临这方面的质疑。求异法的问题是类似的。尽管两个案例非常接近,但仍可能存在被现有分析框架忽略的重要变量。比如在图14.6中,当研究者说案例一有A而案例二无A的时候,可能还忽略了案例一有S而案例二无S。如果确实如此,借助此种研究方法得到的因果关系能否成立就是一个问题。

鉴于上述三种方法的问题,有学者提出来,通过比较研究来确定因果关系,最好要找到“最大相异案例中的最大相似性”。亚当·普沃斯基和亨利·托伊恩注意到,即使在高度相似的两个案例中,仍然可以发现许多具有潜在相关性的重要差异,而在论证因果关系的过程中很难有效排除这些差异性因素。所以,他们主张,选择案例时最好符合最大相异的原则,这样能保证把被忽略的共有变量的可能性降到最低;判断因果关系时,则需要符合最大相似的原则,这样其中的因果关系与因果机制就比较可信。\nauthor{Adam Przeworski and Henry Teune, \italic{The Logic of Comparative Social Inquiry}, New York: Wiley-Interscience, 1970, pp.31-46.}举例来说,要论证自变量X导致因变量Y,找案例时要尽可能选择差异较大的案例——如果是做国别研究,最好选择那些地区、发展水平、种族、文化与宗教因素差异很大的一组国家。但是,在这一组差异很大的国家样本中,如果都能找到自变量X导致因变量Y的类似经验证据,那么这项研究的可信度就比较高。

现在用一个非常简单的例子来说明上述几种比较研究方法。抽烟是一种常见的个人行为与社会现象。当然,抽烟有害健康,并不值得提倡。比如,有人对抽烟行为提出这样的假说:一个人抽何种档次的香烟是由他的收入水平决定的。如何通过比较研究验证这个假说呢?首先,可以用求同法来做。随机找到100个抽比较贵的、价格在50元一包香烟的人,用问卷或口头访谈方式进行调查,结果显示:这些人尽管职业、籍贯、学历、个性各异,但绝大多数人都有一个共性,那就是高收入。这样,这个假说就得到了初步的验证。也许这个理论假说不能解释全部,但绝大部分还是可以解释的。这就是求同法。

其次,可以用求异法来做。求异法要求找比较相似的案例,比如研究者找到一组三兄弟的案例。其中一个兄弟是高收入,平时抽比较贵的香烟;另两个兄弟收入较低,抽比较便宜的、价格10元一包的香烟。由于三人是兄弟,基因、家庭成长环境、个性等因素相对接近。这样,就能验证收入是影响香烟购买行为的主要变量。如果研究者随机找出数百对这样的兄弟,发现有较高比例符合——高收入者基本抽高价香烟而低收入者基本抽低价香烟——的现象,这项解释的可信度就比较高。

再次,可以用共变法来做。比如,研究者发现了很多这样的案例,被调查对象刚毕业时收入较低,抽的是价格低的香烟;后来收入增加、成为高收入者后开始抽价格高的香烟;再后来经营没搞好、个人境遇下降,又开始抽价格低的香烟。这样的案例中,研究者观察到收入高低与香烟档次发生“共变”:收入低抽价格低的香烟,收入高抽价格高的香烟。如果研究者能在随机样本中找到很多此类案例,这项研究也会很有说服力。

最后,还可以用“最大相异案例中的最大相似”方法来做。研究者尽可能让被调查对象的地区、行业、职业、教育、个性差异越大越好,然后都发现:不同地区、不同行业、不同职业、不同教育程度、不同个性差异的人群,凡是高收入者多数都抽高价烟,凡是低收入者多数都抽低价烟。这项研究的可信度就比较大。

当然,这项研究可能会发现,除了收入以外其他变量也会影响人的抽烟行为。比如,职业可能会对抽烟行为有影响——高校工作的人即使收入高也未必会抽高价烟,而是抽低价烟;各地习俗也会影响抽烟习惯——比如与华东相比,华北地区更少注重抽烟的价格和档次,北京很多高收入人群习惯抽价格低廉的香烟。这就需要研究者在验证“收入水平—抽烟行为”这一理论假说的过程中,评估其他重要变量对抽烟行为的影响程度。这样的研究就更复杂一些。

\tsection{社会科学研究的不同类型}

社会科学研究有很多不同的类型。从自变量数量看,可分为单因素因果关系研究和多因素因果关系研究。比如,决定学生学习成绩好坏的因素是什么?有人说关键是智商。用社会科学语言来说,自变量智商决定因变量学习成绩,这就是单因素的因果关系。如果进行大样本的统计分析,智商与学习成绩两个变量之间的检验效果非常显著,那么这个单因素的因果关系就是成立的。

但是,有人提出来,并非所有智商高的同学成绩都那么好,智商相当的同学成绩差异也很大。所以,研究者又提出一种新的理论假说,认为学习成绩的高低取决于四个变量:智商高低、勤奋程度、家庭环境及教师教学水平。当这四个因素叠加到一起,其解释力就非常强了。智商单一因素没准只能解释全部样本的70\%或75\%,但这四个因素分析框架就能解释90\%或95\%。

从这个简单的案例,也可以看出单因素解释和多因素解释各自的优劣。单因素解释的最大问题是它不够全面,无法兼顾到实际情形的复杂性,但是其优点是简洁性,主要论点直接明了,更容易把握主要问题。多因素框架的解释力当然更强,但其缺点是理论不够简洁。如果一个因果关系的分析框架中包含自变量过多——比如5个以上,那么这种理论就更接近于描述了——而非对关键因果关系和因果机制的揭示。固然,从现在的研究趋势看,多因素因果关系的解释框架越来越流行。需要注意的是,自变量的数量不能无限地增加。一般认为,用两三个自变量来解释一个因变量,结构上还是较优美的;用四五个自变量来解释一个因变量,就有些略多了;用六个或六个以上的自变量来解释一个因变量,通常算不上很好的社会科学理论。

比如,前面提到过的迈克尔·曼是美国著名社会学者,他的《社会权力的来源》是被广为引用的学术著作。\nauthor{迈克尔·曼:《社会权力的来源》(第一卷),刘北成、李少军译,上海:上海人民出版社2007年版;迈克尔·曼:《社会权力的来源》(第二卷),陈海宏等译,上海:上海人民出版社2007年版。}然而,这一著作也遭到不少批评,原因之一就是解释框架中变量过多。他试图用政治、经济、军事和意识形态四个变量来解释社会权力的来源及其演进,并借助宏大叙事的方式论证自己的观点。有人担心,由于涉及变量太多,所以核心的因果关系反而被弱化了。

社会科学研究还可区分为基础研究和应用研究。在中国,应用研究似乎更受追捧,这一点可以从各类基金的课题指南及学术论文类型中看出来。很多人觉得应用研究比较实用一些。当然,务实是需要的,但对社会科学研究来说,过份务实有时不一定是好事。从两者特点来说,基础研究更多着眼于理论解释,试图回答为什么的问题;应用研究更多着眼于应用,试图回答怎么办的问题。

医学上病理学家和临床医生的关系,就像是社会科学研究中基础研究和应用研究的关系。比如,病理学家关心癌症为什么发生?到目前为止,人类医学尚不能很好地解释癌症为什么发生。但是,病理学家研究出了癌细胞生长和扩散的条件:癌细胞的生长过程需要合成蛋白质,如果能阻断蛋白质的合成,癌细胞就不能生长。所以,用化疗方法来治疗癌症,原理就是要破坏癌细胞生长过程中蛋白质的合成。如果蛋白质的合成机制遭到破坏,癌细胞就难以生长。当然,病理学未来有可能会找到癌症为什么发生的根本原因与机制。如果是这样,治愈癌症就不是太难的事情。

那么,临床医生主要做什么?对于癌症患者,他需要就患者的具体病情做诊断,判断是否要动手术及如何动手术,到了化疗阶段他还需要判断哪种化疗药物是最佳的?他关心的是怎么办的问题。他不会关心癌症为什么发生,不会关心是否能找到治疗癌症的新化疗药物。在中国,临床医生往往比病理学研究人员具有更高的社会声望。但对整个人类社会而言,与临床医生面向患者的个案治疗相比,病理学上的医学突破意义更为重大。如果有人在癌症研究病理学上有突破,因为这种发现拿了诺贝尔生物医学奖之后,全世界的癌症患者都会从中受益。

拿城市犯罪问题来说,社会学家关心的是犯罪为什么发生?为什么有的城市犯罪率高而有的城市犯罪率低?他会提出一种基于因果关系的解释,设计一个分析框架,然后去收集数据并做数据分析,最后他得出结论,若干主要变量在起作用,但其中两个变量可能最为关键。这就是社会科学学者对犯罪问题所做的基础研究或理论研究。他做完这项基础研究后,提出何种政策建议是不言自明的。政策建议无非就是要设法改变分析框架中的两个或几个变量。当然,实际政策制定和实施过程会更复杂一些。

但是,同样面对城市犯罪问题,市长和警察局长关心的是怎么办的问题。如何防止犯罪的发生?如何降低本市的犯罪率?他们更关心制定和实施哪些具体政策能够有效地防止犯罪和降低本市犯罪率。如果需要就这个研究项目做招标的话,他们设计的课题应该是这类“有用”的应用研究。

所以,基础研究和应用研究差异很大。从上面的例子可以看出,一流大学大体上主要以基础研究为主。一流大学的主要使命是追求真理,其长远目标是追求知识创新。而知识创新主要依赖于基础研究。实际上,诺贝尔科学奖得主无一例外都是“知识创新家”。人类的已有研究做到哪里了?能否往前有所突破?一个学者在人类已有的知识边界上创造新知识,这是他有机会赢得诺贝尔科学奖的前提条件。

根据经验证据的类型,社会科学研究还可以分为质性研究和量化研究两类,一般被称为定性研究和定量研究。质性研究和量化研究的主要差别在于事实和资料的类型:一个是数据化的,一个主要是非数据化的。但是,质性和量化不是可以截然分开的,甚至是相通的。比如,关于学生的聪明程度与学习成绩的关系,既可以做质性研究,又可以做量化研究。当根据“聪明”和“不聪明”的标准对学生进行分类时,就是一项质性研究。当根据智商分值的标准对学生进行排序时,这项研究就变成了一项量化研究——确定智商分值后,收集学生的学习成绩,然后进行简单的统计分析。这是定量研究的一个最简单例子。

那么,定性研究与定量研究各自的特点是什么呢?定性研究的优势是对机制和过程的描述更加深入,论述上可以更加透彻;其劣势是样本数量不足。定量研究的优势是能够在一个较大的数量基础上论证一个观点,大样本当然比少数几个案例更具说服力;其劣势则是数量方法只能证明相关性而非因果性,对于因果机制的论证过程难以深入。

国际关系著作《大战的起源》就是一项典型的定性研究。\nauthor{戴尔·科普兰:《大战的起源》,黄福武译,北京:北京大学出版社2008年版。}戴尔·科普兰在这项研究中提出的问题是:历次大战为何发生?尽管此前有大量解释主要战争的理论文献,但科普兰认为已有研究尚有缺陷,他在前人研究的基础上提出了一个新的理论假说,称之为“动态差异理论”。“这一理论表明,大战主要是由那些处于优势地位却害怕明显衰退的军事大国发动的。”换句话说,如果一个大国是领导者,但是它感到自己在国际竞争中可能会衰退,同时又有一个挑战国正在崛起,此时那个即将衰退的大国最有可能发动大战。作者的逻辑请参见图14.5。

\img{../Images/image00341.jpeg}[图14.5 大战的起源:动态差异理论(多极体系)]

资料来源:戴尔·科普兰:《大战的起源》,黄福武译,北京:北京大学出版社2008年版,第18页,图1。

在经验研究部分,科普兰采用的是定性研究方法,他用三个章节分别探讨了三个主要案例:第一次世界大战、第二次世界大战和冷战中的古巴导弹危机,他还用一章分析了从伯罗奔尼撒战争到拿破仑战争为止的一系列主要战争,以此来论证自己的观点。实际上,《大战的起源》构成了一项结构优美的定性研究。当然,大部分的定性研究都达不到这样的优美程度。

这里再介绍一项定量研究的研究设计。上一讲曾提出过一个解释腐败的理论假说,那么怎样用定量方法来做这项研究呢?关于腐败的理论假说被表述为下列函数:

\[C=F(Pr, C\&B)\]

上文已经介绍,这一函数公式表示的是腐败程度取决于权力控制的资源多少和分权制衡程度。那么,怎样做量化研究来证实这一假说呢?因变量腐败可以用腐败指数(Corruption Percepation Index,简称CPI)来度量。\nauthor{用腐败感知指数来衡量腐败有一个问题,腐败和对腐败的感知可能是两回事。现有的数据库都是对腐败感知的衡量,但你会发现,有的国家对腐败的容忍程度很低,有的国家对腐败的容忍程度很高。但是,现在可得到的数据就只有腐败感知指数。这方面的知识得益于跟我复旦大学的同事李辉副教授的讨论。}这一指数实际上不是腐败指数,而是腐败感知指数。但这是目前为止惟一可用的关于腐败的大规模跨国数据库。自变量一“权力控制的资源”可以用下面几项主要指标来加总衡量:国有企业比重、国有资源比重以及市场管制或市场自由化程度。这方面的数据可以采用样本国家的统计数据、世界银行数据、国际货币基金组织数据、联合国数据以及有关评级机构公布的各主要国家市场自由化指数。自变量二“分权制衡”可以用民主指数、法治指数或政治制度分权指数等来加总衡量,这方面国际上有不少数据库。收集这些数据、进行编码之后,可以把自变量一和自变量二数据跟因变量数据做统计分析。这样,就能大致看到两个自变量与因变量之间的相关关系。

当然,一项有效的量化研究还需要控制其他变量,比如,经济发展水平、宗教与政治文化变量、族群分裂程度、不平等程度等等是否会影响一个社会的腐败程度呢?这些都可以找到相应的量化数据库。从研究设计上看,控制其他变量是非常重要的。如果不控制其他变量,即便你发现自变量一和自变量二跟因变量的相关性非常显著,也无法有把握地认为这一理论假说已得到了验证。比如,如果把宗教因素加进来,而宗教因素与因变量的相关性更显著,上述理论就要大打折扣。但如果控制其他变量后,还能证明自变量一和自变量二跟因变量的相关性仍然是显著的,那么这个论证就更为可信。

\tsection{“研究九问”与“洋八股”}

接下来,我们可以总结一下:一项完整的社会科学或政治科学研究应该怎么做呢?应该遵循何种准则又需要哪些步骤呢?笔者在不同场合讲过与政治科学研究方法有关的“研究九问”,这里做一介绍。\nauthor{“研究九问”得益于跟朱天飚教授和宋磊副教授的讨论,前者激发了我的思考,后者邀请我在研究设计课程首次讲解“研究九问”。}“研究九问”既适合作为做社会科学博士、硕士论文的研究指导,又适用于作为一般的研究性学术论文的写作指南,参见图14.6。

\img{../Images/image00342.jpeg}[图14.6 研究九问]

“兴趣是最好的老师。”同样,兴趣是最好的研究引导者。出色的社会科学研究首先需要唤起人的热诚和激情,而不只是把它作为一项工作任务来对待。所以,第一问是:“我有一个感兴趣的研究领域吗?”做政治科学研究,应该有一个感兴趣的研究领域。如果没有,说明一个人的志趣不在学术领域。当然,这丝毫不成为一个问题,因为多数人不会以学术为业。国际上比较卓越的学者通常从年轻时候起就专注于一两个重要的领域,始终会有一个或几个非常重要的问题纠缠着他。比如,美国著名经济学家罗伯特·卢卡斯曾说,他很早就开始专注于一个基本的经济学问题:到底什么原因导致了经济增长?为什么一些国家富而另一些国家穷?事实上,正是持续的研究兴趣激发了伟大的研究。

第二问是:“我有一个好的问题吗?”提出一个好问题,是一项好研究的关键。有时,提出一个正确的问题甚至比回答一个问题本身更重要。诺贝尔经济学奖得主科斯提出的一个问题是:既然市场是有效率的,为什么还会出现企业?正是对这一问题的思考,科斯发现了交易成本这一奠定新制度主义经济学基础的概念。还有学者提出这样的问题:既然人是自利的,为什么美德还会流行?这类问题都是非常基础的问题,而且通常没有引起足够的重视。过去,笔者遇到的一个学生在华北某省调研计划生育问题,发现其中一个乡镇的超生率特别高,可能达到200\%左右;而周围乡镇的超生率要低很多,大概是120—130\%左右。这个学生提出的问题是:为什么该乡镇的超生率显著高于周围的乡镇?这个问题需要探索的是:何种因素决定人们的生育行为?这一问题的结构也非常好,对这个乡镇与周围乡镇进行比较研究,很多变量已经被控制住了,特别是经济发展水平和地域文化因素。所以,这项研究应该致力于发掘这一异常超生现象背后的因果关系。大家可以发现,案例之间的比较本身就是提出问题的一种有效方式。

第三问是:“前人是如何解释这个问题的?”提出一个具体问题之后,接下来要问前人是如何解释这个问题的?实际上,很少有什么问题是从未被研究过的。所以,做学术研究必须要读文献。为什么文献很重要?简单地说,你既然要解释一个问题,首先你至少应该知道别人是怎么解释的。如果不读文献,而你的解释又跟先行研究相似,甚至不如先行研究的解释更有说服力,你的研究就失去了意义。所以,这一步骤要求研究者充分地阅读已有文献,并找到现有研究的不足。

第四问是:“我是如何解释这个问题的?”研究者需要提出一个理论假说。作为一项社会科学研究,这个假说应该关于某种因果关系和因果机制的表述。社会科学研究是为了探索人类行为和社会现象背后的因果关系,所以一项理论假说的表述形式通常是:何种特定的原因(通过某种机制)导致何种特定的结果?这也说明,社会科学研究是为了回答为什么的问题。

接下来的几个问题是要对这一理论假说进行自我检验。第五问是:“我的解释内在逻辑自洽吗?”第六问是:“我的解释是新的吗?”第七问是:“我的解释比以前的解释更好吗?”第一个检验是理论假说的逻辑自洽性检验。很多理论假说内在逻辑是不自洽的,这方面的例子有很多。一项经得起推敲的理论,首先应该经得起自洽性检验。第二个检验是理论假说的原创性或新颖性检验。社会科学研究的目的是创造新知识,而不是复述旧知识。所以,社会科学研究需要提出新理论。无论是一篇博士论文还是一篇投向一流学术刊物的论文,理论解释的原创性与新颖性都是一个必要条件。第三个检验需要比较不同的竞争性假说。有的社会科学研究提出的解释也是新的,但并不比现有的理论解释更好。这种研究也许可以发表学术论文,但意义不是很大。如果新的研究找到的解释变量不比现有的解释更好,这种研究的价值就不会很高。

第八问开始进入经验世界:“经验证据支持我的解释和逻辑吗?”社会科学研究必须要基于事实,而非基于哲学思辨或想象,社会科学研究的口号应该是“拿证据来”。研究者必须拿证据出来说话,研究应该要有事实或数据的支撑——无论是质性证据还是量化证据。研究者提出的理论假说能否成立,核心是经验证据是否支持这一理论假说与因果逻辑。更精细地说,这里还涉及不同的研究方法,但无论怎样,这一步骤的核心是用经验证据去验证理论假说。

第九问是:“我的研究得出什么结论和政策含义?”现在有很多社会科学研究用大量篇幅来讨论政策问题或具体政策建议。但是,符合国际主流学术规范的社会科学基础研究,通常只花很少篇幅来讨论政策问题。如果前面的理论研究做得好,理论假说得到充分验证,政策含义自然就出来了。至于说具体政策应该怎样制定,还必须考虑案例所处的情境,理论研究已经为政策指明了大方向。总之,“研究九问”总结的是国际主流社会科学研究的基本范式。

关于社会科学研究的学术规范,彭玉生教授所著《“洋八股”与社会科学规范》一文论述得非常清楚。\nauthor{彭玉生:《“洋八股”与社会科学规范》,载于《社会学研究》2010年第2期,第180—210页。}他关心的基本问题是:社会科学经验研究怎么做?社会科学学术论文怎么写?借鉴中国古代科举考试中八股文的说法,彭玉生教授用了一个“洋八股”的比喻。事实上,目前国际上主流的社会科学实证研究几乎都是按照“洋八股”的套路来做的。不符合这个套路的经验研究,论文要想在一流国际学术杂志刊发也是不可能的。

那么,什么是“洋八股”呢?彭玉生教授认为,做一项社会科学研究,首先要从提出“问题”(一股)开始。比如,战争为什么发生?这是研究的起点。然后,要进行“文献分析”(二股)。为什么要做文献分析呢?上文已经分析过,只有通过文献分析才能把握此前的理论。做完文献分析,发现已有的研究还不够好,接下来研究者可以提出一个新的理论解释,即提出假说或假设(三股)。提出假说后,研究者不能像政治哲学家那样只做哲学思辨式的思考,而是要到经验世界里去寻找证据。经验证据涉及三个问题:一是找事实,特别是找数据(四股),二是衡量事实或衡量数据(五股),三是使用有效的研究方法(六股)。在此基础上,再进行数据分析(七股),即系统地检验经验证据是否支持前面提出的理论假说。经过数据分析,可以明确证实还是证伪最初的理论假说。最后,通过上述研究过程得出结论(八股)。这一研究结论也是对已有文献的回应。“洋八股”也是国际主流社会科学规范的一种简明表述,参见图14.7。

\img{../Images/image00343.jpeg}[图14.7 洋八股与经验研究的基本结构]

在政治科学研究中,好的理论首先要符合社会科学规范。在国内学界和大众媒体上,不少人对社会科学研究存在着误解。比如,有人把总结实践经验视为社会科学研究,有人认为讲故事是社会科学研究,有人喜欢频频提出新概念或语出惊人的观点,有人主要以文学表述方式提出观点和进行论证,有人则专注于就实际问题提供政策建议——所有这些都不是社会科学研究,或者至少不是严肃的社会科学研究。

\tsection{推荐阅读书目}

艾尔·巴比:《社会研究方法》(第十一版),邱泽奇译,北京:华夏出版社2009年版。

大卫·马什和格里·斯托克编:《政治科学的理论与方法(第2版)》,景跃进等译,北京:中国人民大学出版社2006年版。

托马斯·库恩:《科学革命的结构》,金吾伦译,北京:北京大学出版社2003年版。
