\tchapter{暴力、革命与内战}

\quo[——塞缪尔 · 亨廷顿]{首要的问题不是自由,而是建立一个合法的公共秩序。人当然可以有秩序而无自由,但不能有自由而无秩序。必须先存在权威,而后才谈得上限制权威。}

\quo[——西达 · 斯考切波]{在国内阶级结构和国际紧急事件的交叉压力之下,专制者及其中央集权的行政机构和军队走向了分崩离析,从而为以下层反叛为先锋的社会革命转型开辟了道路。}

\quo[——米涅]{但是,迄今为止,各民族的编年史中还没有过这样的先例:在牵涉到牺牲切身利益时还能保持明智的态度。应当做出牺牲的人总是不肯牺牲,要别人做出牺牲的总要强迫人家做出牺牲。好事和坏事一样,也是要通过篡夺的方法和暴力才能完成。除去暴力之外,还未曾有过其他有效的手段。}

\quo[——卢梭]{即使是最强者,也决不会强大到主人永远做主人,除非他把自己的强力转化为权利,把服从转化为责任。}

\tsection{政治的两幅图像}

政治有时呈现出一幅和平的图像,有时呈现出一幅暴力的图像。和平的政治通常是,即使存在政治分歧,也可以通过自由讨论与互相妥协的方式来解决。比如,美国在 2000 年总统大选时面临了一场政治危机。美国民主党总统候选人、副总统阿尔 · 戈尔和共和党总统候选人乔治 · W.布什在选票上非常接近。到最后关头,除了佛罗里达州以外,其他所有州的选票已清点结束。由于两人的选票非常接近,所以佛罗里达州的总统选举人票投给谁,谁就能当选下一任总统。\nauthor{按照美国的总统选举制度,每个州根据人口规模获得若干张选举人票,无论谁在该州选举中胜出,该州的所有选举人票将全部投给该候选人。}佛罗里达州的选票统计结果将决定谁将成为美国新一任总统。当时的情况是,他们在佛罗里达州的选票也非常接近,小布什在首次选票统计中领先总共约 2000 票左右。在这种情况下,美国民主党人认为选票统计存在瑕疵,要求佛罗里达州重新统计选票。中间经历了州务卿拒绝重新清点选票,而州法院同意在部分郡重新用人工办法清点选票。最后,小布什将此案诉讼至美国最高法院,最高法院以 5 票对 4 票的微弱优势通过禁止佛罗里达州重新在部分郡人工清点选票的司法决定。这样,在最高法院的司法干预下,小布什最终顺利当选美国总统。\nauthor{关于本案的法律细节,参见王希:《2000 年美国总统大选述评》,《美国研究》2001 年第 1 期,第 7—39 页。}

大家不要小看这个政治事件。如果这一事件发生在一个新兴民主国家,最后很可能酿成大规模的政治骚乱,甚至会导致局部武装冲突。当时小布什的兄弟杰布 · 布什恰好是佛罗里达州的州长。如果换了一个国家,很多人马上会联想到暗箱操作。但是,在 2000 年的美国,无论是候选人还是选民,最终都尊重美国最高法院作出的司法判决。戈尔在败选演说中说,他对以这样一种方式落选感到非常遗憾,但他尊重游戏规则,尊重美国的传统,尊重法院的决定。

从 2000 年美国总统选举可以看出,美国对最高政治权力的争夺早已远离了暴力。不管是胜利者还是落选者,大家都在以和平方式从事政治活动。选民们和平地进行政治参与,政治家们和平地从事政治竞争,最后以和平方式解决政治争端。从这个意义上说,美国 2000 年总统大选也再次证明了:现代政治文明的主要特征之一就是对政治权力的争夺已实现去暴力化。

此外,美国总统候选人通过电视辩论方式争取选民的支持,英国国会议员通过讨论和投票来做出重要的政治决定,日本选民四年一次投票选举国会议员,这都是以和平方式参与政治活动的情形。在上述这些国家,政治生活中也难免出现政治冲突,但多数情况下都能以和平方式加以解决。

但是,这种和平的政治图景并不总是存在,政治还有另一幅可能的图景。比如,即便在政治文明程度很高的英国,政治也并非总是以和平方式呈现出来。2011 年 8 月 6 日至 10 日,英国伦敦就发生了小规模的骚乱,伦敦城北部一些街道和商店遭到示威人群的焚烧,后来骚乱还扩散到其他城市。这意味着英国这样的国家都没有完全杜绝政治暴力。2001 年发生在美国的 “9 · 11” 事件,更是一次大规模政治暴力的展示。基地组织劫持和控制的数架飞机直接撞击美国纽约世贸中心双子座大楼,最后把整个大楼摧毁。当然,这一政治事件并非只是美国国内的政治暴力,其源头是活跃于其他国家的一个恐怖主义组织——基地组织。但无论怎样,这一与国际政治有关的政治暴力事件,是人类政治暴力的一部分。

至于发达国家之外的政治暴力现象,那就更为普遍了。比如,2010 年 3 月上旬,非洲人口最多的国家尼日利亚发生教派屠杀事件。据英国 BBC 新闻网的报道,大约有不少于 500 人死于教派屠杀。在该国,有人信仰基督教,有人信仰伊斯兰教。这些不同的宗教团体经常因为某个事件发生冲突。在 2010 年底开始启动的中东北非国家政治转型中,政治暴力事件更是层出不穷。比如,利比亚就经历了内战,叙利亚出现了严重的暴力冲突,两国至今仍不平静。自美国军队从伊拉克撤走之后,该国并没有从武装冲突的威胁中走出来。总之,从世界范围来看,政治既可能是和平的,又可能是暴力的。

\tsection{政治暴力与常见的暴力现象}

按照第 4 讲的定义,国家是一个合法垄断暴力的机构,这是国家的本质属性。从这一定义出发,站在国家角度看,垄断暴力是政治的基本问题。如果一个社会充斥着暴力,那意味着国家不能有效地垄断暴力。当国家不能有效垄断暴力时,一个社会的政治秩序就失去了控制。比如,2011 年利比亚爆发内战,就是一种政治秩序失去有效控制的情形。2013 年 5 月 15 日,尼日利亚总统宣布在北部三个州实行紧急状态,这意味着这些州面临着秩序崩溃的可能。上述政治现象都跟政治暴力或政治秩序失控有关。

塞缪尔 · 亨廷顿早年在其名著《变化社会中的政治秩序》中探讨的,就是发展中国家的政治秩序问题。为什么政治秩序会失去控制?什么情况下政治秩序会彻底崩溃?如何实现政治秩序的稳定?这些都是非常重要的问题。亨廷顿在该书中有一个著名论断:

\quo{首要的问题不是自由,而是建立一个合法的公共秩序。人当然可以有秩序而无自由,但不能有自由而无秩序。必须先存在权威,而后才谈得上限制权威。\nauthor{塞缪尔 · 亨廷顿:《变化社会中的政治秩序》,王冠华、刘为译,上海:上海人民出版社 2008 年版,第 6 页。当然,从另一种视角看,自由和秩序不是冲突的,反而是互补的。塞缪尔 · 亨廷顿在其后续作品《第三波:20 世纪后期的民主化浪潮》一书中也对自己过于偏爱政治稳定和政治秩序的观点做了修正。}}

通常,政治秩序的混乱都跟暴力的失控与泛滥有关。什么是暴力呢?简单地说,暴力是针对个人或群体的一种武力攻击现象。普通的暴力行为,可能是单个人对单个人的,但这种暴力的规模通常不是很大。与这种一对一的、个别的暴力行为相比,政治学者们更感兴趣的是集体暴力(collective violence)。美国学者查尔斯 · 蒂利认为,集体暴力具有三个特点: “对个人立即造成肉体上的伤害;至少有两个作恶者;集体暴力至少是部分地来源于施暴者的相互协作。” \nauthor{查尔斯 · 蒂利:《集体暴力的政治》,谢岳译,上海:上海人民出版社 2006 年版,第 4 页。}按照蒂利的定义,符合这三个条件的可以称为集体暴力。比如,在某地乡村,因为水资源问题,水源上下游的张姓村庄和李姓村庄发生集体斗殴,最后甚至还导致数十人伤亡,这样的事件就是一种典型的集体暴力事件。

那么,什么是政治暴力呢?政治暴力是跟政治有关的集体暴力,是由政治动机引发的、包含明确政治目标或意图的集体暴力。两个村庄如果为了水资源,发生集体斗殴,最后导致数十人伤亡,这个事件一般不称为政治暴力事件。因为总体上,这一事件没有实质的政治动机,也没有明确的政治意图——当然,客观上可能会产生一些政治后果。所以,这种类型的暴力就算不上政治暴力,而是一种普通的集体暴力事件。

在今天的民主国家,一种非常常见的政治暴力是街头骚乱。英国、法国、印度等国都出现过规模不等的街头骚乱。有人认为,街头是一个非常重要的政治场域,街头政治是一种重要的政治类型。在 20 世纪 30 年代初的魏玛共和国,希特勒的纳粹党和党卫军在政治斗争中采取的一个重要手段就是占领街头。纳粹党的一种政治斗争哲学就是:谁能占领街道,谁就能控制政治权力。街头政治固然很多时候表现为和平方式,但有时也表现为暴力或骚乱的方式。除了军队,其他政治力量通常就难以应付这种暴力化、组织化的街头政治。比如,20 世纪 20 年代初,墨索里尼在意大利就是这么搞的。除了墨索里尼领导的法西斯主义政党以外,其他党派都是较为松散的政治组织。墨索里尼建立了组织化程度非常高的法西斯组织,甚至还拥有自己的民兵等准军事组织。借助这种政治组织,通过占领街头,包括在街头政治中展示潜在的暴力,墨索里尼获得了巨大的政治影响力,直至最后成为意大利的总理。

很长时间以来,泰国的街头政治都非常活跃。泰国政治的一大顽症是街头骚乱频发。该国两派政治力量——红衫军和黄衫军——都曾经占领过街道和政府机构。这种做法肯定包含了暴力成分——当然,泰国街头政治的好处是游行示威队伍直接进行人体攻击的案例比较少见。然而,他们占领街道、政府机构甚至是机场,或多或少要使用暴力相威胁。比如,完全不以暴力相威胁,示威队伍是无法占领机场的。当然,实际上机场方面一般不会与示威者直接对抗。这样,实际的暴力现象就不会那么严重。

有人把政治大罢工也视为准政治暴力行为。比如,一个经典案例出现在 1970—1973 年的智利。阿连德当选总统后,开始推行 “社会主义革命” ,当时征收了很多私人产业,对银行和矿产实施国有化。后来,智利的一个重要产业——卡车运输业的业主们担心,阿连德政府要对卡车运输业实施国有化。所以,他们决定发起一场全国大罢工,来抵制这种可能性。当时,智利全国大约有 6 万个卡车运输协会的业主参加了大罢工。他们不仅罢运,而且还把卡车停在了主要道路上。当时,进出智利首都圣地亚哥的主要道路上就停满了罢运的卡车。这种政治大罢工尽管没有直接进行人体攻击,但其后果是非常严重的。大城市的生活用品和垃圾难以运入或运出。结果,圣地亚哥很快就出现了食品和生活品的紧缺。阿连德政府试图调动军队来解决这个问题,但难度很大,因为主要道路上停满了罢运的卡车。因此,某些政治大罢工在效果上跟政治暴力是非常相似的,甚至比小规模的政治暴力事件更令人恐慌。

还有一种典型的政治暴力现象是政治暗杀。中国近现代历史上就有过一次非常著名的政治暗杀。民国初年,著名政党运动领袖宋教仁 1913 年 3 月 20 日在上海火车站遭人枪击,不治身亡。根据当时的调查,宋教仁案刺客的幕后主使与袁世凯任命的国务总理赵秉钧有关,但真相究竟如何,不得而知。印度独立至今,有数位主要领导人死于政治暗杀。圣雄甘地就死于政治暗杀,而后尼赫鲁的女儿、印度和国大党的政治领导人英迪拉 · 甘地死于政治暗杀,英迪拉 · 甘地的儿子、印度和国大党的政治领导人拉吉夫 · 甘地也死于政治暗杀。德国魏玛共和国期间也曾发生过多起重要政治家被暗杀的事件,智利 1970—1973 年政治混乱时期甚至出现过军方重要将领死于暗杀的政治事件。

政治暗杀针对的一般不是普通人,而是比较有影响力的政治家与政治活动家。上文提到的英迪拉 · 甘地是被自己手下信奉锡克教的卫兵打死的。在当时的印度,锡克教与印度教的教派冲突比较厉害。英迪拉 · 甘地身边就有幕僚和安全人员提醒她,要求她把身边锡克教卫兵全部撤换,以免发生意外。但是,英迪拉 · 甘地并不听这种劝阻,她认为印度是一个世俗的多宗教国家,如果把身边信仰其他宗教的卫兵撤换,这会传递出一个怎样的政治信号呢?这至少意味着她作为政治领袖对锡克教卫兵和该宗教信仰的人群是不信任的。但是,意外还是发生了。即便这样,英迪拉 · 甘地的政治遗言强调不要过分追究此事。她临死前非常担心自己的遇刺事件最后会演变为一场全国性的针对锡克教的大规模暴力运动。所以,政治暗杀事件的后续影响往往是极其严重的。

有些国家还发生过这样的现象:A 派政治领导人被 B 派暗杀了,然后 A 派再去暗杀 B 派的政治领导人。如果是这样,政治竞争就脱离了和平的轨道。所以,政治暗杀通常传递的是非常糟糕的政治信号。德国魏玛共和国晚期政治暗杀的频发,也是促成希特勒上台的一个因素。在一些国家,政治暗杀的频发既是政治秩序失控的结果,又会成为政治混乱加剧的原因。

最近二三十年,引人注目、罪大恶极的政治暴力现象是恐怖主义袭击,2001 年美国的 “9 · 11” 事件是恐怖主义袭击的标志性事件。这几年一个比较著名的美剧是《反恐 24 小时》,这个片子的主题与反恐有关。在剧情设计中,有恐怖主义组织试图在美国大型商场里投放化学毒气,或者把核武器偷运至美国并试图在中心城区引爆,等等。这些故事当然是虚构的,但类似的恐怖袭击并非没有可能发生。最近几年,中国边疆省份也开始出现暴力恐怖主义袭击事件,而且正在呈现上升的势头。

当然,典型的政治暴力现象还有军事政变。中国历史上有大量的军事政变,比如著名的玄武门之变。20 世纪的非洲和拉丁美洲出现过大量的军事政变。军事政变是一种以暴力手段取得政权的常见现象。一想到军事政变,很多人脑海里会浮现出大规模的军队调动和武装冲突的场景。但有些国家的军事政变规模较小,伤亡人数也较低。

20 世纪以来另一个严重的政治暴力现象是族群屠杀或种族屠杀,甚至包括族群清洗或种族清洗。20 世纪三四十年代,希特勒就发动了针对德国和欧洲犹太人的族群大清洗。著名电影《辛德勒名单》就是以这一事件为历史背景的。20 世纪 90 年代初,非洲国家卢旺达也曾发生过大规模的族群屠杀。在胡图族和图西族的族群冲突中,胡图族把图西族的多数人口都屠杀殆尽,图西族被清洗的人口总量高达数十万。著名影片《卢旺达饭店》则以电影方式呈现了当时令人震惊的族群清洗事件。与族群屠杀或种族屠杀相类似的,还有不同宗教之间的教派屠杀。这大概也是人类政治暴力中最残忍的一种类型。

除此之外,今天在拉美、非洲和东南亚一些国家,还存在与政府军进行武装斗争的游击队。这种武装组织拥有数百、数千乃至上万规模的士兵。他们集聚在一小块地方,跟政府军打游击战,而中央政府通常又无力彻底清剿。无疑,游击战或局部武装冲突也是一种政治暴力现象。当然,与之相比,更为严重的政治暴力现象是大规模的内战,本讲后面还会专门讨论。

\tsection{政治暴力的类型与逻辑}

上文已经介绍了常见的政治暴力现象。那么,如何从学理上对政治暴力进行类型划分呢?迈克尔 · 罗斯金在其流行的《政治科学》教科书中借鉴弗莱德 · 梅登的研究,把政治暴力分为五种基本类型:

\quo{原生型的(primordial) 原生型暴力产生于基本的社会群体冲突——种族的、民族的或宗教的——这些都是人们与生俱来的。

分裂型(separatist) 分裂型暴力——有时是原生型暴力冲突的产物——目标是要实现相关群体的独立。

革命型(revolutionary) 革命型暴力旨在推翻或取代现政权,例如伊斯兰教主义者想要接管穆斯林国家并把他们变成信奉正统派的人。

政变型(coups) 政变通常是为了反对革命、腐败和混乱。一般来说,政变几乎总是军事性的,尽管军队通常与关键的文官集团有联系并从他们那里获得支持,就像 1964 年巴西的政变那样。

问题型(issues) 一些暴力不适合这些类型中的任何一种。由某一特定问题所引发的暴力是一种兼容的类型,并且常常不像其他类型的暴力那样具有致命性。\nauthor{迈克尔 · 罗斯金:《政治科学》(第九版),林震等译,北京:中国人民大学出版社 2009 年版,第 403—408 页。}}

查尔斯 · 蒂利则用两个维度对人际暴力类型进行了区分:一是暴力伤害的严重程度,二是暴力行为者之间的协同程度,参见图 12.1。从类型学的角度说,类型划分最好符合不重复、不遗漏原则。蒂利的这一分类框架尽管不符合该原则,但这一分类把各种集体暴力纳入了一个粗略的框架。\nauthor{查尔斯 · 蒂利:《集体暴力的政治》,谢岳译,上海:上海人民出版社 2006 年版,第 12—19 页。}

\img{../images/image00331.jpeg}[图 12.1 蒂利:人际暴力的类型]

蒂利还认为,不同政体类型与国家类型也与政治暴力的严重程度有关,参见图 12.2。\nauthor{查尔斯 · 蒂利:《集体暴力的政治》,谢岳译,上海:上海人民出版社 2006 年版,第 42—49 页。}蒂利在过去的研究中,曾根据政体类型的维度——即民主国家还是威权国家——和国家能力的维度——即国家能力高还是国家能力低——区分过四类国家:高能力的民主国家、低能力的民主国家、高能力的非民主国家和低能力的非民主国家。在他看来,政治暴力的严重程度直接受到国家政体类型的影响。在图 12.2 中,蒂利认为,高能力的民主政体对应的是低强度的政治暴力,低能力的非民主政体对应的是高强度的政治暴力,高能力的非民主政体和低能力的民主政体对应的则是中等强度的政治暴力。

那么,导致这些差异的原因是什么呢?对政治暴力来说,政体这个维度代表的是有多少政治诉求能够在现有体制框架中得到表达和满足。一般而言,在民主政体下,政治诉求更容易通过现有体制框架得以满足。在威权体制下,政治诉求更不容易通过现有体制框架得以满足。如果政治诉求能够通过现有体制得以满足,当事人通常没有动力从事或参与政治暴力活动;反之,就存在从事政治暴力行为的动机。国家能力这个维度代表的是国家或政府有效控制政治暴力的能力。国家能力强,对政治暴力的控制就比较有效;国家能力弱,对政治暴力的控制就不那么有效。也就是说,从这个维度上看,国家能力越强,就越能控制政治暴力。

\img{../images/image00332.jpeg}[图 12.2 国家类型与政治暴力]

资料来源:查尔斯 · 蒂利:《集体暴力的政治》,第 45 页。

对高能力的民主国家来说,从政治诉求上讲,只有较低比例的人愿意或需要用政治暴力来表达政治诉求,大量的政治诉求都能在现有民主体制框架内表达;从国家能力上讲,有效的国家能力使得政治暴力不容易发生和蔓延。所以,两者的结合导致的是低强度的政治暴力。一种相反的糟糕情形是:如果是威权政体,就意味着有大量的政治诉求没有办法通过现有体制进行表达,这使得不少人有从事政治暴力行为的动机;但同时由于国家能力比较低,国家对这种可能出现的政治暴力的控制能力比较弱。所以,两者的结合导致的是高强度的政治暴力。其他两种国家类型属于中间状态,其政治逻辑也是相似的,不再赘述。

基于这种逻辑,可以预见,一个国家的政体和国家能力产生变化后,该国的政治暴力程度亦可能发生变化。比如,一个高能力威权国家在政体维度上变得更民主时,政治暴力通常会降低。比如,一个高能力威权国家的国家能力下降时,政治暴力可能会倍增。一种更复杂的情形是,如果一个国家转型之前是高能力威权国家,但在转型过程中,一方面固然是从威权政体过渡到民主政体,另一方面却由于政治冲突、制度垮塌及政治领导人等原因,国家能力大大下降了——这种情况下,该国政治暴力会发生何种变化呢?这就难以判断。从威权到民主,政治诉求获得新的释放通道;但国家能力下降则意味着国家控制政治暴力能力的减弱,这意味着政治暴力的变化趋势是不确定的。

当然,还有学者从另一个角度来探讨政体与政治暴力之间的关系,即政治竞争本身会导致更多的政治暴力。一种观点认为,一个国家的民主转型前期不排除政治暴力上升的可能。有学者在研究非洲国家民主转型后发现,从启动民主转型到完成民主转型,政治暴力会经历先稳步上升、后逐渐下降的过程。\nauthor{比如,其中的一项研究是:Jacqueline M.Klopp and Elke Zuern, “The Politics of Violence in Democratization: Lessons from Kenya and South Africa” , \italic{Comparative Politics}, Vol.39, No.2(Jan., 2007), pp.127-146。}如果民主转型不逆转的话,随着民主政体维系时间的增加,政治暴力会逐渐地降低。但令人忧虑的是,如果转型初期的政治暴力过于剧烈,民主政体最后有可能被搞垮,重新蜕变为威权政体。所以,民主转型对控制一个国家的政治暴力未必有着立竿见影的效果。相反,其中可能还蕴藏着较大的政治风险。如何避免民主转型过程中政治暴力的加剧,这看来是一个重要的学术与实践问题。

\tsection{国家与社会革命}

什么是革命?革命通常是指一种快速剧烈的系统性变革。《易经》中有 “汤武革命,顺乎天而应乎人” 的说法。在政治上,革命是对旧体制或旧制度的一种颠覆。所以,政治革命主要是指对旧政体和旧政权的革命。与政治革命相比,社会革命的含义有所不同。按照哈佛大学教授西达 · 斯考切波的观点——

\quo{社会革命是一个社会的国家政权与阶级结构都发生快速而根本转变的过程;与革命相伴随,并部分地实施革命的是自下而上的阶级反抗。社会革命之所以不同于其他类型的冲突和转型过程,首先在于它是两个同时的组合:社会结构变迁与阶级突变同时进行;政治转型与社会转型同时展开。……政治革命所改造的是政权结构而非社会结构,而且并不必然要经由阶级冲突来实现。\nauthor{西达 · 斯考切波:《国家与社会革命》,何俊志、王学东译,上海:上海人民出版社 2007 年版,第 4—5 页。}}

斯考切波提到的三场典型的社会革命是 1789 年的法国大革命、1917 年的俄国革命以及 1921—1949 年的中国革命。这三场革命不是一般意义上的政治革命,同时也是社会革命——它们不仅是对旧政权或旧政体的颠覆,而且还伴随着大规模的社会动员与底层反抗。跟上层集团的宫廷斗争或军事政变不同,这三个国家都发生了大规模的政治动员与底层反抗,几乎整个社会都被动员和参与进来了。

社会革命通常可以分为几个阶段。第一个阶段是旧制度的衰朽。社会革命首先不是由于革命者或革命力量的推动,而是由于旧制度本身的衰朽。第二个阶段是能量的集聚和革命的发动,这是社会革命的启动阶段。旧制度衰朽以后,社会需要一个能量集聚的过程,直至革命成为可能与现实。第三个阶段是旧制度的垮台,然后整个政治和社会发生急剧变革,中间还伴随着大规模的社会动员与底层反抗。第四个阶段是经过急剧的变动,出现了政治力量的重组和新制度的诞生,最终达到一种新的政治均衡。这是从破坏到重建的关键阶段,新制度诞生后还需要一个适应的阶段。

与历史学家擅长还原社会革命的过程和细节相比,政治学家们更关注这样的理论问题:社会革命为何发生?斯考切波认为,在她的《国家与社会革命》出版之前,解释社会革命有几个主要的理论流派。

首先是马克思主义的解释。社会革命可以归结为生产关系与生产力的冲突与断裂,直接表现为剧烈的阶级斗争和阶级冲突,下层阶级的反抗直接导致了社会革命的发生,最后表现为一个阶级用暴力方式推翻另外一个阶级。这一理论路径采用的是阶级分析方法。

其次是革命的群体心理理论。这种心理学理论更关心作为参与群体行动的个人何时会卷入政治暴力、何种条件下会卷入大规模的政治冲突。古斯塔夫 · 勒庞认为,群众性的集体情感曲线在政治运动过程中,经常会经历一个先是缓慢上升,而后是急速攀升,接下来是直线下降的过程。革命,就是这种群众性集体情感急速攀升过程中爆发的。

再次是系统/共识价值理论,这种理论强调的是整个社会体系和系统中的严重失衡。这种理论借鉴了系统论的方法,这与 20 世纪 70 年代流行的戴维 · 伊斯顿倡导政治系统理论有关,它强调的是对政治过程中输入因素与输出因素的分析,从政治系统的角度来理解革命何以发生。

最后是政治冲突理论,这种理论认为社会革命是源自不同社会集团对政治权力的争夺,这种对政治权力的争夺会导致剧烈的政治冲突。当这种政治冲突使得某一集团开始借助底层动员方式进行政治斗争时,就容易引发社会革命。\nauthor{西达 · 斯考切波:《国家与社会革命》,何俊志、王学东译,上海:上海人民出版社 2007 年版,第 5—15 页。}

在上述革命理论之后,到了 20 世纪 70 年代末,该领域产生了一部重要著作——西达 · 斯考切波所著的《国家与社会革命》。在这部备受关注的学术作品中,斯考切波提出了新的解释社会革命的理论。总体上说,她考察了两个变量对于社会革命的影响,一个变量是整体性的危机和旧制度的崩溃,另一个变量是下层阶级的反抗所引发的政治冲突,两者的结合导致社会革命的发生。这一理论的逻辑结构,参见图 12.3。她这样说:

\quo{在国内阶级结构和国际紧急事件的交叉压力之下,专制者及其中央集权的行政机构和军队走向了分崩离析,从而为以下层反叛为先锋的社会革命转型开辟了道路。\nauthor{西达 · 斯考切波:《国家与社会革命》,何俊志、王学东译,上海:上海人民出版社 2007 年版,第 59 页。}}

\img{../images/image00333.jpeg}[图 12.3 斯考切波:社会革命的解释框架]

在斯考切波的分析框架中,对第一个变量的考察落实在两个因素上:一是原有专制体系中行政和军事系统的能力下降,二是国际结构中政治压力和国家间竞争因素的上升。她认为,无论是法国大革命、俄国革命还是中国的共产主义革命,都有着这样的背景条件。革命什么时候会发生?正如列宁指出的,当统治阶级自己都没有办法再统治下去的时候,革命就会发生。这样,专制体系能力的下降和国际竞争压力的上升,最终导致了整体性的危机和旧制度的崩溃。但是,斯考切波认为,这一条件本身并不足以引发社会革命。社会革命还需要第二个条件,即大规模的底层反抗。这就要求原有传统社会中支配阶级和下层阶级的冲突比较激烈,最后引发严重的政治对抗。这样,一方面旧制度随时面临垮塌的可能,另一方面是下层阶级会形成大规模的政治反抗,两者的结合就导致了社会革命。

这本书在 1979 年出版后,迅速引起轰动,赢得了很多学术荣誉。这本著作的主要贡献体现在两个方面:一是提出并论证一种关于社会革命的新理论,成为革命研究领域的一部重要著作;二是作者展示了比较历史分析作为一种重要研究方法的学术魅力,该书已经成为比较历史分析经典。当然,《国家与社会革命》并非一部完美的著作。相反,该书出版后一直遭到学术界的批评。比如,一种典型的批评意见认为该书在研究方法上存在瑕疵,一个缺陷是斯考切波没有选用相反的案例。

跟革命有关的一个热门话题是:为什么那么多旧制度都不能通过改革获得新生,最后竟为革命所覆灭?这里试图用图 12.4 的简要逻辑来阐述改革与革命的关系,并以中国的 1840 年到 1912 年的历史作为例证。对清王朝这一旧制度来说,其核心力量是清政府、清皇室、官僚集团以及其所控制的军队,这些是清王朝的关键力量和统治资源,也是清朝旧制度的核心。根据旧制度内部对改革态度的不同,可以区分两种主要政治力量:改革派与顽固派。前者主张对旧制度进行改革,而且他们认为惟有改革,才能使旧制度通过自我更新的方式存续下去;后者反对对旧制度进行改革,他们认为应该抵制改革派和压制反对力量。

众所周知,旧制度嵌入在整个社会之中。所以,既有统治体系之外的是整个国内社会。在统治体系之外,固然有既有统治秩序的支持者,但引发旧制度变迁的却主要是旧制度的反对者。反对者分为两个阵营:温和反对派和激进反对派。前者主张推动旧制度进行改革与转型,认为主要应该采取温和手段,他们还试图与旧制度的改革派合作;后者主张颠覆旧制度,认为只有采取激进手段才能达成最终目标。换言之,温和反对派主张的是改革,激进反对派主张的是革命。

\img{../images/image00334.jpeg}[图 12.4 改革与革命的逻辑]

在旧制度晚期,体制内的改革派和顽固派通常会互相竞争。顽固派的主张是守旧,他们的首选项是不变,若迫不得已一定要改革,就装模作样开始改革,但他们实质上是反对改革的。在历史的重大关头,这些顽固派为了自己的生存,可能也会向社会发出改革的号召,比如试行立宪等,但他们并不是真的想要改革。体制内的改革派则真心想要一场重大的变革,他们认为旧制度存在问题,并更愿意接纳新事物。从晚清历史来看,从早期的洋务派到后来的立宪派——以开明官僚集团为核心——基本上都持这种主张。从 19 世纪晚期到 1911 年,在晚清朝廷内部,顽固派与改革派一直在进行政治竞争,他们的政治力量此消彼长。通常的规律是,清王朝危机深重的时候,改革派就比较强一点;然而,一旦迫在眉睫的危机解除,顽固派的力量往往就会反弹。

正是这种背景下,从 1898 年到 1911 年,晚清王朝进行了一场虚情假意的改革。晚清朝廷做了很多事情,包括派大臣出洋考察、预备立宪、建立各省咨议局等等,尽管确有局部的实质性推进,但同时不断地传出与立宪改革相悖的信号。比如,佐证晚清王朝改革态度的一个重要事实是最高层职位中满汉官员的比例。在晚清最后一次重要改革中,顽固派实际上借改革的名义在最高层把汉族大臣都清除出去了。所以,晚清的改革实际上是无法真正推进的。中国有俗语说 “不见棺材不落泪”  “不到黄河心不死” 。当整个局势这样发展时,改革实际上已经沦为一个口号式的标签,而非真正的政治态度与主张。所以,最终清王朝的各种危机叠加,演变为一个整体的旧制度危机。另一方面,由于体制内的政治力量无力推行真正的改革,体制外的温和反对派不断地失去市场,激进反对派逐步崛起并成为主要政治力量。1911 年 10 月 10 日,武昌起义爆发,清王朝只用了几个月的时间就彻底垮台了。

实际上,托克维尔在《旧制度与大革命》中的观点与米涅在《法国革命史》中陈述的思想非常相似,即依附于旧制度的既得利益集团通常都不会放弃自己的既得利益,所以要发动一场成功的改革相当困难。米涅这样说:

\quo{假如人们能互相谅解,假如一些人肯于把过多的东西让给别人,另一些人则虽然匮乏而能知足,那么人们就会是非常幸福的。……但是,迄今为止,各民族的编年史中还没有过这样的先例:在牵涉到牺牲切身利益时还能保持明智的态度。应当做出牺牲的人总是不肯牺牲,要别人做出牺牲的总要强迫人家做出牺牲。好事和坏事一样,也是要通过篡夺的方法和暴力才能完成。除去暴力之外,还未曾有过其他有效的手段。\nauthor{米涅:《法国革命史》,北京编译社译,北京:商务印书馆 1997 年版,第 4 页。}}

\tsection{内战的理论解释}

内战是大规模政治暴力的主要形式之一。国内关于内战的社会科学理论研究非常少,有的主要是关于美国内战、英国内战、国共内战等历史著述。但内战其实是一个重要的政治学理论问题。内战通常是指一个国家或社会内部爆发的战争,或者更严格地说,内战是在一个国家内部不同的组织化武力集团为控制或推翻政权而引发的持续暴力冲突。斯坦福大学教授詹姆斯 · 费隆把内战定义为—— “一个国家内部发生的组织化的集团之间的暴力冲突,这些集团都旨在控制中央或地区的政治权力或改变政府政策。” \nauthor{James D. Fearon and David D. Laitin, “Ethnicity, Insurgency, and Civil Wars,” \italic{The American Political Science Review}, Vol.97, No.1(Feb., 2003), pp.75-90.}

按照学术界的一般看法,暴力冲突能被称为内战,而不是局部武装冲突,需要符合几个基本条件:

第一,内战发生在一个国家或社会的内部,而非国家与国家之间;

第二,通常存在不同的——两个或两个以上的组织化武力集团,内战双方或几方均需一定的实力,占据一定的地理空间,甚至占据国土面积的相当比例,反抗人数需超过该国人口的一定比例,反抗的军队有属于自己的作战标识和特定的政治口号,甚至会成立自己的政府;

第三,内战通常以控制或推翻国内政权为基本目标,有的内战以局部领土的分裂为主要目标,有的内战目标则是要求政府改变某种基本政策;

第四,内战是一种持续的暴力冲突,通常要持续较长的时间;

第五,内战通常是当前的合法政府依赖国家正规军队与反抗的军队进行作战的过程;

第六,内战中造成的伤亡要达到一定规模,现在国际上的常用标准是至少造成 1000 人死亡(死亡人数低于 1000 人一般被视为局部武装冲突)。\nauthor{关于内战的标准,参见 James D.Fearon and David D.Laitin, “Ethnicity, Insurgency, and Civil Wars,” \italic{The American Political Science Review}, Vol.97, No.1(Feb., 2003), pp.75-90。奥斯陆和平研究所(Peace Research Institute Oslo)有大量关于冲突与内战的研究资源,参见网站:http://www.prio.org/。}

众所周知,人类近现代史上有过很多著名的内战。比如,1641—1645 年英国内战、1861—1865 年美国内战、1931—1936 年西班牙内战、1945—1949 年中国内战,等等。20 世纪八九十年代以来,较著名的内战包括 80 年代爆发的第二次苏丹内战——该国如今已分裂为苏丹和南苏丹,1991 年持续至 2000 年的南斯拉夫内战——该国目前已分裂为 7 个政治体,2011—12 年的利比亚内战,等等。

世界银行 2001 年关于内战的研究项目统计了 1960—1999 年全球内战的次数与频率,制成了一个全球各国内战频率的趋势图,参见图 12.5。横轴表示年份,纵轴表示全球该年发生内战的次数。从 20 世纪 60 年代到 90 年代,全球国家经历了一个内战数量持续攀升的过程。80 年代中期到 90 年代中期,全球每年陷入内战的国家超过 20 个——当然,其中一些内战持续了较长时间。如果按国家总数来计算,陷入内战国家的比例并不低。20 世纪 90 年代初,该数字突破了 25 个,说明内战数量的急剧上升。好在 20 世纪 90 年代中期以后,全球内战数量经历了一个快速下降的过程。

政治学家们关心的一个理论问题是:内战为什么发生?目前,对内战的理论解释有几种不同的路径。第一种理论主张,内战主要起源于身份认同的危机。身份认同跟人的归属感有关,这涉及 “我是谁”  “你是谁” 的问题。你是谁?你是苏丹人,还是南苏丹人?你自己认为是南苏丹人,但有人认为你是苏丹人,身份认同就会出现问题。如果双方由此引发冲突,最后可能演变为内战。再比如,有人信仰基督教,有人信仰伊斯兰教,还有人信仰印度教或佛教。如果一个国家伊斯兰教人口较多,有政治家或宗教领袖说,应该把整个国家建成伊斯兰教国家,以《古兰经》作为主要立法准则。如果是这样,该国 15\% 的基督徒很可能会抗争,这就可能引发严重的暴力冲突,甚至最后不得不通过内战来解决分歧。所以,这种因为身份认同引发的内战是一种典型情形。

\img{../images/image00335.jpeg}[图 12.5 全球内战的频率:1960—1999 年]

资料来源:Paul Collier, Anke Hoeffler and Nicholas Sambanis, “The Collier-Hoeffler Model of Civil War Onset and the Case Study Project Research Design” , in Paul Collier, Nicholas Sambanis(eds.), \italic{Understanding Civil War: Evidence and Analysis}, New York: World Bank, 2005, Vol., 1, pp.1-34.

第二种理论强调的是对资源的争夺。一国内部不同群体的实际境遇可能差异很大。境遇更差的群体往往心存不满,甚至会产生强烈的怨恨心理,当他们条件具备时就倾向于反抗。比如,突然在少数族群占据的东部地区发现了大量的石油资源,中央政府又希望直接控制这一石油资源,东部的少数族群就可能跟主导的多数族群发生冲突,一种可能的结果就是内战,因为双方都希望占有这一重要的经济资源。在这种情况下,对资源的争夺构成内战的直接原因。

第三种解释主要着眼于政治权力。这种理论认为,政治就是一个争夺政治权力的过程。不同的人和集团都想控制一个国家的最高政治权力。在此过程中,不同的集团为了获取中央政府的政治权力,彼此之间就有可能发生内战。中国历史上有大量的内战与此有关。比如,古代中国每个王朝末年都会发生大规模内战,因为各个军事政治集团都想获得最高政治权力。再比如,近代中国 1916 年到 1928 年的军阀割据时期,不同军阀之间本身并没有多少仇恨,但是大军阀们都想控制中央政府,统一中国;小军阀们则想控制自己的地盘,割据一方,结果是彼此之间的军事冲突。所以,争夺政治权力也被视为内战的重要驱动力量。

第四种解释来自于革命理论。简单地说,一个国家内部,有的社会集团或政治力量要推翻现有政治秩序,要发动政治革命或社会革命。在此过程中,有人反对革命,希望维持现有政治秩序。这样,两者之间就有可能发生内战。在 1917 年的俄罗斯,列宁发动十月革命之后,沙皇原有军事力量并不接受这种政治结果,双方之间就爆发了内战。这种内战受到革命因素的驱动,是支持革命力量与反对革命力量之间的武装冲突。\nauthor{关于内战的理论解释与实证研究的扼要介绍,参见查尔斯 · H.安德顿、约翰 · R.卡特:《冲突经济学原理》,郝朝艳、陈波主译,北京:经济科学出版社,第 97—117 页。}

除了上述几种解释内战的主要理论之外,两位学者保罗 · 科利尔和安科 · 霍夫勒 2001 年牵头完成的研究报告,提出了一个新的解释内战为何发生的理论模型,后来该模型被称为科利尔-霍夫勒模型,简称 C-H 模型。\nauthor{Paul Collier, Anke Hoeffler and Nicholas Sambanis, “The Collier-Hoeffler Model of Civil War Onset and the Case Study Project Research Design” , in Paul Collier, Nicholas Sambanis(eds.), \italic{Understanding Civil War: Evidence and Analysis}, New York: World Bank, 2005, Vol., 1, pp.1-34.}这是一个世界银行关于内战的研究项目,他们采用的是定量研究方法,基于 1960 年到 1999 年全球不同国家的内战数据,提炼出了六个主要变量进行检验。他们的研究结论是:一个国家未来 5 年中是否会爆发内战会受到六个因素的显著影响。

第一个因素是财务资源获取的容易程度。什么情况下一个国家某一政治集团更容易获得财务资源呢?一是要看该国有没有可供出口的石油资源。这意味着,如果一个国家在某个地区有大规模石油资源的话,就更容易爆发内战。为什么呢?一方面控制石油资源,就能获得购买军火、组建军队所需的资源;另一方面石油资源本身是一种值得争夺的巨大利益,石油的发现意味着 “赌注” 加大了。所以,这更容易导致内战的爆发。当然,有些矿产资源跟石油的效应很相似,比如钻石或金矿。非洲国家塞拉利昂拥有巨大的钻石储量,一小兜钻石就能换回一卡车的 AK-47 冲锋枪。这种国家更有可能发生内战。著名电影《血钻》就是以塞拉利昂的钻石交易与武装冲突为背景的。

二是该国某些政治集团是否能获取国外侨民的财务援助。比如,在撒哈拉以南非洲地区,某族群人口分布在彼此相邻的两国。该族群在 A 国是主导族群,控制大量政治经济资源,在 B 国却是少数族群。当该族群感觉在 B 国受到歧视和不公时,他们就有可能从 A 国同一族群人口中获取经济和军事资源,然后发动一场谋求分裂或独立的内战。所以,在这种情况下,拥有足够财务资源的国外侨民可能成为引发内战的催化条件。无论是石油还是外部援助——财务资源获取的容易程度,是判断内战是否可能爆发的一个重要指标。

第二个因素是反叛机会成本的高低。在一个富裕国家,反叛的机会成本通常比较高,内战发生以后原有的稳定生活就打乱了,金融市场和工商业会大幅萎缩。但是,在一个贫穷社会——特别是拥有大量低于绝对贫困线人口的国家,反抗的机会成本是很低的。有海外学者研究中国 20 世纪早期的军阀混战的原因,其中一个解释就是:当时中国过于贫穷,很多人吃不上饭,当兵算是一个很好的出路,所以军阀的兵源得到源源不断的供给。这是一个有趣的视角。世界银行的研究团队用入学率、人均收入与经济增长这几个指标来衡量机会成本。入学率越低,人均收入越低,经济增长率越低,则机会成本越低——这种条件更容易爆发内战;反之,就不容易爆发内战。

除了上述两大因素,还有若干影响内战爆发的可能因素。第三个因素是基于人口和地理因素的军事优势。如果一个国家的人口很分散,地理和地形比较复杂,就为内战爆发创造了有利的地理条件;如果一个国家的人口很集中,地理和地形非常简单,内战就不太容易会发生。第四个因素是怨恨。一个国家内部部分人口的怨恨主要来自经济不平等、政治权利受压制以及一般意义上的族群和宗教分裂因素。这些因素都可以通过统计方法来衡量和评估。第五个因素是人口规模。总的来说,人口规模愈大,愈有可能发生内战。这一因素可能跟上面讨论的地理面积等有关。第六个因素是时间。研究发现,距上次内战的时间越短,越有可能发生内战。换句话说,如果一个国家 200 年没有发生过内战,接下来再发生内战的可能性很小;但如果一个国家刚刚结束内战不到 10 年,那么下一次内战也很容易发生。总之,世界银行关于内战的这项研究值得借鉴。

内战通常都会导致非常严重的政治后果。内战意味着打破了国家对暴力的垄断。当国家不再能垄断暴力时,国家就容易趋于解体。所以,内战不仅意味着大规模的军事冲突和大量的人员死伤,而且还意味着政治秩序的混乱和局部的无政府状态。这种状态下,可预期的秩序、稳定的生活和经济的繁荣都不太可能。

内战的重要性还在于其解决方式会对此后的政治均衡产生重要影响。大致来说,世界上很多国家现有的政体的起源都与内战有关。英国今天的政体起源贵族与国王之间的战争,美国政体的维系与南北战争有关,中国目前的政体起源于国共内战。大家会发现,大量的政体都有其内战起源。如何应付内战,往往影响甚至决定了一个国家政治变迁的路径。或者说,内战的解决方式有可能决定着下一个政治均衡点。

最后,这一讲需要提醒的是,从政治暗杀到恐怖主义袭击,从社会革命到内战,展示的都是赤裸裸的暴力。众所周知,政治离不开暴力。诚如韦伯所言,国家是合法垄断暴力的机构。但是,区分政治文明程度的一个重要标准就是该国日常政治的暴力使用程度。很多国家到目前为止还没有很好地解决暴力问题,一方面政府在统治过程中频繁地借助暴力或以暴力相威胁,另一方面该国随时都有可能爆发暴力事件、恐怖袭击、局部冲突,甚至是内战。如何从暴力的政治走向非暴力的政治?如何实现日常政治的去暴力化?这都是塑造现代政治文明的关键。卢梭曾经这样说: “即使是最强者,也决不会强大到主人永远做主人,除非他把自己的强力转化为权利,把服从转化为责任。” 实际上,现代政治文明的一个主要特征是把暴力的政治转变为非暴力的政治。这在很大程度上关系到一个国家未来的政治命运。

\tsectionnonum{推荐阅读书目}

米涅:《法国革命史》,北京编译社译,北京:商务印书馆 1997 年版。

希达 · 斯考切波:《国家与社会革命——对法国、俄国和中国的比较分析》,何俊志、王学东译,上海:上海人民出版社 2007 年版。

托克维尔:《旧制度与大革命》,冯棠译,北京:商务印书馆 2012 年版。

米格代尔:《农民、政治与革命:第三世界政治与社会变革的压力》,李玉琪、袁宁译,北京:中央编译出版社 1996 年版。
