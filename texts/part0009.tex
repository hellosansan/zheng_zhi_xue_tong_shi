\tchapter{法治与公民权利}

\quo[——爱德华 · 柯克]{因此,我要说,陛下应当受制于法律;而认可陛下的要求,则是叛国;对于我所说的话,布拉克顿曾经这样说过:国王在万人之上,但却在上帝和法律之下。}

\quo[——约翰 · 马歇尔]{每一个人受到侵害时,都有权要求法律的保护。政府的一个首要责任,就是提供这种保护。合众国政府被宣称为法治的政府,而非人治的政府。如果它的法律对侵犯所赋予的法律权利不能提供救济,它当然就不值得这高尚的称号了。}

\quo[——查尔斯 · 麦克尔文]{在所有相继的用法中,立宪主义都有一个根本的性质:它是对政府的法律制约……真正的立宪主义的本质中最固定和最持久的东西仍然与其肇始时几乎一模一样,即通过法律限制政府。}

\quo[——《人权与公民权宣言》]{自由是指能从事一切无害于他人的行为;因此,每一个人行使其自然权利,只以保证社会上其他成员能享有相同的权利为限制。此等限制只能以法律决定之。}

\tsection{国王可以强拆吗?}

这一讲先从一个不那么确切的故事开始。大家知道,现在拆迁在中国是一个社会热点话题。这个小故事就跟拆迁有关,故事的大意是这样的——

\quo{19 世纪,普鲁士国王威廉一世曾在波茨坦建立了一座行宫。1866 年,有一次他住在行宫里,想要登高远眺波茨坦市的全景,但他的视线却被一座磨坊挡住了。皇帝大为扫兴。这座磨坊 “有碍观瞻” 。他派人与磨坊主去协商,打算买下这座磨坊,以便拆除。不想,磨坊主坚决不卖,理由很简单:这是我祖上世代传下来的,不能败在我手里,无论多少钱都不卖!皇帝大怒,派出卫队,强行将磨坊拆了。

倔强的磨坊主向法院提起了诉讼。让人惊讶的是,法院不仅受理了这一案件,而且还判皇帝败诉。法院要求皇帝在原地按原貌重建这座磨坊,并赔偿磨坊主的经济损失。皇帝服从法院的判决,重建了这座磨坊。

数十年后,威廉一世与磨坊主都相继去世。磨坊主的儿子因经营不善而濒临破产。他写信给当时的皇帝威廉二世,自愿将磨坊出卖给他。威廉二世接到这封信后,感慨万千。他认为磨坊之事关系到国家的司法独立和审判公正的形象。它是一座丰碑,成为德国司法独立和裁判公正的象征,应当永远保留。所以,威廉二世亲笔回信,劝其保留这座磨坊以传子孙,并赠给了他 6000 马克,供其偿还所欠债务。小磨坊主收到回信后,十分感动,决定不再出售这座磨坊,以铭记这段往事。}

有人曾质疑过这个故事的真实性,也有严肃的法学者对此进行过考证。\nauthor{现在这个故事有不同的版本,著名期刊《读者》曾刊载这个故事。按照获德国法兰克福大学博士学位的法学者袁治杰的考证,这个故事是真实的,参见袁治杰:《磨坊主阿诺德案考论》,载《比较法研究》2011 年第 2 期,第 128—142 页。}但无论怎样,这个故事的内容大体上与从普鲁士到德意志的法律传统是一致的。这个故事主要想表达的是:即便贵为国王,亦不能随意剥夺别人的财产。此外,这个小故事中还有一个重要信息:法官竟然能够判决国王败诉!法官不仅判决国王败诉,而且要求国王执行法院的判决。这意味着司法权能够制约政治权力或行政权。

但是,如果脱离了欧洲的法律传统,这件事情听上去就有些匪夷所思了。这个案子完全有别的可能性。比如,第一种可能性是法官连这个案子都不敢接。 “你想告国王?对不起,我们无法接这个案子——国王是得罪不起的。” 第二种可能性是法官根本不敢判决国王败诉。 “你起诉的是国王?对不起,我们总不能让国王输了案子。” 第三种可能性是法官判决国王败诉,但国王拒不执行。如果出现这种情形,法官能派出法警强制国王执行吗?要知道,警察和军队本身就是听命于国王的。所以,这个故事包含了很多有价值的信息。

很多人见过一张比较有名的网络图片:一个巨大的建筑物凹进去一块,中间有一个小平房。为什么会这样呢?当年建造这个大型建筑物的地产商在开发这片土地时,这个小平房就在那里。地产商跟这个小平房的业主反复协商,讨价还价,但人家就是不卖这块地。理由可能很简单,比如人家从小居住于此,这个房子是他一辈子的记忆。后来,地产商没有办法,就建造了这样一个凹进去一块的大楼。周围的地都是地产商已经买下的,惟独这块地不是地产商的。这张图片反映出什么信息?在这样的法律环境中,房屋与土地的产权得到了有效的保护。

这两个案例会促使我们去思考:法律到底是什么?政府的责任是什么?法律是用来干什么的?这也是政治学应该关心的重要问题。

\tsection{政府有权捕杀禽类吗?}

再来看一个案例。自 2012 年开始,中国局部地区出现了 H7N9 禽流感病毒。大约 2012 年底,一些疫情比较严重的省市出现了政府集中捕杀禽类的新闻。那么,在当时的情况下,政府是否有权捕杀禽类呢?就在当时,江苏南京等地还为此闹过纠纷。南京有一个市民在自己小区一楼的私人小花园内养了几只鸡。当地居委会跟城管发现这些鸡之后,想要捕杀掉。但是,这些鸡的主人不同意,由此引发了冲突。那么,你如何看待这个案例?大家对这个问题肯定会有不同的观点。\nauthor{这里的不少观点来自于学生的课堂讨论。

有人认为,政府无权捕杀禽类。在南京某小区的这一案例中,业主在自己小花园养殖的禽类是他的私有财产。在这些禽类并未与其他禽类接触的条件下,感染 H7N9 禽流感病毒的几率是微乎其微的。如果政府随意捕杀,就构成对公民财产权利的侵犯。几只鸡不是关键问题,但政府的这种施政原则推而广之的话,就会产生极大的负面影响。

有人则认为,当时全国大部分地区并没有暴发禽流感,所以没有必要在全国范围内实施这样的政策——就是政府统一捕杀禽类的做法。然而,如果某个区域禽流感疫情非常严重,地方政府统一捕杀禽类就是一种必要的做法,但政府应该对此进行补偿。这种观点强调的是,禽流感是否已经显著地危害到了公共安全?如果没有危害公共安全,政府采取捕杀禽类的措施是不妥当的;但如果已经危害到公共安全,政府就有权捕杀禽类。

有人则明确主张,政府有权力捕杀禽类。当出现禽流感疫情时,政府需要履行公共管理的基本职能。如果小范围内出现禽流感疫情而政府没有及时捕杀禽类,就有可能导致疫情的快速蔓延。所以,捕杀禽类是政府防患于未然的一项积极措施。这种做法尽管一定程度上侵犯了公民的财产权,但在这种特殊情形下还是具有合理性和正当性的。这里强调的特殊情况,在法律上可以视为紧急状态。当出现某种紧急状态时,政府有权这样做。

有人说,政府权力固然是人民让渡的,而人民让渡权力的首要目的是为了寻求自我保护。有人强调私有财产权不可侵犯,但是,人民从政府那里寻求的保护还包括安全。实际上,安全是人民寻求的首要保护。上述案例中,政府大规模捕杀禽类,是基于禽流感病毒已开始传播并威胁了公共安全这一事实。在这种情况下,政府处置私人财产是一种不得已的做法。这一观点强调的是,政府作为一个公共管理机构首先要保证共同体的安全。大家的观点见仁见智。如果接受现在流行的以个人权利为基础的法律体系,就应该承认所有人对禽类的财产权应该受到确定无疑的保护。从原初意义上说,这种权利是绝对的。所以,如果没有特殊情形或紧急状态,政府征用或捕杀禽类是毫无道理的。但是,现在出现了一种特殊情况——在部分禽类中发现了 H7N9 病毒。如果这种病毒广泛传播,就会对公共安全造成巨大的危害。这时,问题出来了:政府在此种情形下是否有权捕杀禽类?这就从一个绝对的财产权利问题变成另外一个问题,即需要在财产权和公共安全之间寻求一种平衡。所以,这里更需要运用比例原则。大家需要评估这种做法对私人财产权的侵害以及 H7N9 禽流感蔓延的潜在风险,然后对两者进行比较,按照两害相权取其轻的原则做出取舍。}

这样看来,正反两方的观点都有些道理。反对政府捕杀禽类的观点更强调财产权神圣不可侵犯,应该得到保护;另外,当时还无法证明这种病毒已经蔓延。所以,这种情况下政府对禽类实施大规模的捕杀,不仅意味着对财产权的侵害,而且会导致社会财富的损失。当时某报的说法是,政府捕杀禽类造成的损失已超百亿。但是,还有可能出现另外一种情形。如果现在有可靠信息显示,若不捕杀禽类,病毒将会出现大范围的快速蔓延。这个时候,恐怕会有非常多的人支持政府采取此类政策。所以,这个问题的逻辑其实是清晰的。

进一步说,即便现在已出现某种较严重的紧急状态,政府在捕杀禽类之前仍需要回答两个程序性问题。第一,一个社会凭什么来判断现在已经出现了某种紧急状态,谁有权决定和宣布这种紧急状态?比如,拿某市来说,是该市的卫生部门,还是市政府,还是市人大?究竟应该由谁来决定这种紧急状态?这既是一个程序问题,又涉及此种紧急状态的 “合法性” 问题。这种紧急状态,在美国由谁来决定?在印度由谁来决定?在韩国由谁来决定?那么,在中国呢?这种决定和宣布某种紧急状态的权力及其程序,直接关系到公共治理法治化的程度。

第二,政府捕杀禽类所造成的损失,应该给予合理补偿。当出现紧急状态时,很多人考虑较多的是公共安全,而对政府捕杀禽类导致的某些群体的经济损失考虑较少。比如,华东某市一个以贩卖禽类为生的商贩,他的全部经营性财产是 50 万元钱。此时,他正好从山东某大型禽类养殖企业购买了数车活鸡和活鸭。这样,他所有的财产就在这几个货车上,刚刚运到上海。由于紧急状态的出现,政府下令全部捕杀处理。按照当时该市有关部门的规定,活禽给予 50\% 的经济补偿。在很多其他地区,经济补偿的比例可能还要低一些。这样,对于这个兢兢业业的商贩来说,他的半数财富瞬间就灭失了。这大概是他过去多年的辛苦经营所得。所以,对于此种紧急状态的处置,必须要考虑经济补偿的问题,这是一个基本公共政策问题,也可以反映出一个社会中个人财产受保护的程度。

总之,上述讨论并没有标准答案,但与具体主张相比,弄清楚这个问题背后的法理逻辑更为重要。

\tsection{宪政与宪法的基本问题}

上面讨论的案例都跟基本公民权利的保护有关,这就涉及宪法、宪政与法治的问题。早在 19 世纪初,美国联邦大法官约翰 · 马歇尔就认为,政府的首要职责是为每个人提供法律保护。政府的基本职责与宪法的基本目标都应该是保护公民的自由和权利。因此,一个国家宪法与法治的实施程度是跟该国公民权利和基本自由高度相关的。

宪法是一个国家的基本大法,它规定了国家正式的政治制度结构,明确了个人所享有的权利与自由。宪法的两个主要内容,一是跟国家的正式政治制度结构有关的,二是跟公民自由和基本权利有关的。从类型上讲,宪法可以分为两种:一种是成文宪法,一种是不成文宪法。对于拥有不成文宪法的国家,通常有一些类似于宪法的基础性法案,构成了宪法的实际组成部分。不成文宪法主要出现在英国,而美国是一个最早制定成文宪法的国家。在 1787 年之前,世界上并没有完整意义上的成文宪法,所以美国 1787 年《宪法》是人类历史上第一部完整意义上的宪法。德国最早的成文宪法出现在 1848 年,是 1848 年欧洲革命过程中颁布的宪法,也算是非常早的成文宪法,但这部宪法并没有施行。后来,很多国家都陆续制定了宪法。到了 20 世纪,脱离殖民统治的发展中国家实现独立之后,往往都要制定一部宪法。所以,现在基本上所有国家都有成文宪法。

与宪法密切相关的一个概念是宪政,又译立宪主义。宪政是国内法学界和政治学界热烈争论的一个概念。那么,什么是宪政或立宪主义?宪政一般是指基于宪法与法律来实施统治,或者说是国家的强制性权力受到宪法与法律普遍约束的观念和制度。理解宪政,主要可以从三个方面入手:

第一,宪法与法律限制政府活动和政治权力的范围。这与有限政府原则是一致的。换句话说,政府不是什么事情都能做,宪法和法律允许政府做的事情是有限的,政府只能在这一限定范围内活动,而不是想做什么就做什么。西方启蒙运动以来的政治哲学传统把国家视为一种 “必要的恶” ,意思是国家当然是必不可少的,但国家又可能对这个社会带来侵害,因为政治权力可能会肆无忌惮地扩张。所以,国家与政治权力应该受到强有力的约束。按照这一原则,如果政府活动范围是无限的,政治权力没有受到制约,这样的国家就不符合宪政原则。有人说,宪政就是 “限政” ,这个说法不那么完整,但总体上是恰当的——宪政包含着限制政府或限制政治权力的意思。

第二,宪法与法律应明确及保障公民平等的自由和权利。宪政意味着每个公民的自由与权利受到明确的保护。以美国宪法为例,尽管 1787 年《宪法》没有权利法案的条款,但随后于 1791 年制定的 10 个修正案都涉及公民的基本权利与自由,这些修正案被称为美国的《权利法案》。这 10 个修正案是:

\quo{第一条修正案国会不得制定关于下列事项的法律:确立国教或禁止宗教活动自由;限制言论自由或出版自由;或剥夺人民和平集会和向政府请愿申冤的权利。

第二条修正案纪律严明的民兵是保障自由州的安全所必需的,因此人民持有和携带武器的权利不得侵犯。

第三条修正案在和平时期,未经房主同意,士兵不得在民房驻扎;除依法律规定的方式,战时也不允许如此。

第四条修正案人民的人身、住宅、文件和财产不受无理搜查和扣押的权利,不得侵犯。除依照合理根据,以宣誓或代誓宣言保证,并具体说明搜查地点和扣押的人或物,不得发出搜查和扣押状。

第五条修正案无论何人,除非根据大陪审团的报告或起诉,不得受判处死罪或其他不名誉罪行之审判,惟发生在陆、海军中或发生在战时或出现公共危险时服现役的民兵中的案件,不在此限。任何人不得因同一罪行而两次遭受生命或身体的危害;不得在任何刑事案件中被迫自证其罪;不经正当法律程序,不得被剥夺生命、自由或财产。不给予公平赔偿,私有财产不得充作公用。

第六条修正案在一切刑事诉讼中,被告享有下列权利:由犯罪行为发生地的州和地区的公正陪审团予以迅速而公开的审判,该地区应事先已由法律确定;得知被控告的性质和理由;同原告证人对质;以强制程序取得对其有利的证人;取得律师帮助为其辩护。

第七条修正案在普通法的诉讼中,其争执价值超过 20 元,由陪审团审判的权利应受到保护。由陪审团裁决的事实,合众国的任何法院除非按照普通法规则,不得重新审查。

第八条修正案不得要求过多的保释金,不得处以过重的罚金,不得施加残酷和非常的惩罚。

第九条修正案本宪法对某些权利的列举,不得被解释为否定或忽视由人民保留的其他权利。

第十条修正案本宪法未授予合众国、也未禁止各州行使的权力,保留给各州行使,或保留给人民行使之。}

第三,宪法与法律创造政府越权时给予救济的手段。如果政府做了违反宪法与法律的事情,又该怎么办?如果个人的自由和权利遭到政府的侵害,又该怎么办?这时,宪法与法律应该要创造一种政府越权时给予救济的手段。举例来说,如果政府通过强制手段把你的房子拆了,有没有一种途径能够纠正这种不当做法呢?如果只有前面两条原则,当政府本身违法或侵权时,这两条原则有可能落空。所以,宪政还意味着当政府越权、公共权力被滥用时,普通公民拥有一些能提供救济的手段。

与宪政密切相关是法律和法治的概念。法律是指一整套运用于政治共同体的公开的、具有强制力的规则,或者说是具有强制约束力的规则。在国内,法治的概念则容易跟法制的概念相混淆。国内学界的一般看法是,法治对应的是 “rule of law” ,法制对应的是 “rule by law” 。法治是指法律的统治,而法制则是用法律统治。对于前者,法律具有至高无上的地位;而对于后者,法律不过是一种统治或治理的工具。就后者而言,如果法律只是一种统治或治理的工具,这意味着法律本身并没有超越政治权力。这样,法律只是政治权力用于实现统治与治理目标的手段。

从全球范围来看,英格兰是最早形成法治传统的国家之一。对近代英格兰的法治史来说,大法官爱德华 · 柯克是一位重要人物。1608 年,英国国王詹姆士一世表示希望亲自对某个重要的案件进行司法审判。国王还威胁法官们,如果不给予他司法审判的权力,他将对法官们控以叛国罪。面对国王的此种要求和威胁,时任大法官的柯克爵士这样应答:

\quo{确实,上帝赋予了陛下以卓越的技巧和高超的天赋;但陛下对于英格兰国土上的法律并没有研究,而涉及陛下之臣民的生命或遗产、或货物、或财富的案件,不应当由自然的理性,而应当依据技艺性理性和法律的判断来决定,而法律是需要长时间地学习和历练的技艺,只有在此之后,一个人才能对它有所把握;法律就是用于审理臣民的案件的金铸的标杆(量杆)和标准;它保障陛下处于安全与和平之中;正是靠它,国王获得了完美的保护,因此,我要说,陛下应当受制于法律;而认可陛下的要求,则是叛国;对于我所说的话,布拉克顿曾经这样说过:国王在万人之上,但却在上帝和法律之下。\nauthor{小詹姆斯 · R.斯托纳:《普通法与自由主义理论:柯克、霍布斯及美国宪政主义之诸源头》,姚中秋译,北京:北京大学出版社 2005 年版,第 48 页。此处对译文略有调整。}}

柯克爵士提到的 “国王在万人之上,但却在上帝和法律之下” 这句名言早在 13 世纪就出自另一位英格兰大法官亨利 · 布拉克顿之口。所以,这种传统把法治视为 “法律的统治” ,而不是说法律是 “统治的工具” 。从这个意义上说,法治与宪政这两个概念是高度相关的。两者的差异仅在于强调的重点不同:宪政更强调是用宪法约束政府和政治权力的概念,认为宪法对政府权力的一种制约和抗衡;法治更强调一般的法律作为社会的基本规则,统治和治理不应该基于人治,而应该基于法律。

除了立宪政治的原则,宪法之所以重要,还因为宪法在一国政治生活中扮演着重要角色。宪法的第一种功能是确立合法性。自启蒙运动以来,公民们的政治意识开始觉醒。在人类历史上,君主的统治总的来说是一件自然而然的事情。但是,近代启蒙运动以来,统治不再是一件自然而然的事情。这种统治要基于一套说法:为什么有人可以统治,为什么其他人需要服从?总之,如今的统治需要有一套言之成理的说法,而宪法正是提供了这样一套说法,其首要功能是确立政府合法性和赋予政治权力。

宪法的第二个功能是确立基本的政治制度结构。美国宪法是一部三权分立宪法,德国宪法是一部议会制宪法。这两部宪法确定了行政权与立法权之间的不同关系,也确定了司法权在政治生活中的不同位置。此外,宪法还确定了中央政府和地方政府之间的关系。这样,宪法实际上规定了一个国家横向和纵向的分权结构。所以,一部宪法实际上就是关于一个国家政治制度结构的说明书。

宪法的第三个功能是明确公民的自由与权利。绝大多数国家的宪法或宪法性法律文件中都会说明,国家应该保护和尊重公民的生命权、财产权与自由权,还有不少宪法规定了需要保护公民的受教育权、工作权与基本保障权,等等。总之,多数宪法都对公民的诸种自由与权利有详细的规定。

宪法的第四个功能是限制政府活动和政治权力。符合宪政原则的宪法通常还规定,什么事情是政府不能做的。比如,美国宪法的第一条修正案就明确规定: “国会不得制定关于下列事项的法律:确立国教或禁止宗教活动自由;限制言论自由或出版自由;或剥夺人民和平集会和向政府请愿申冤的权利。” 这条修正案规定的是国会不能立法禁止什么。实际上,一部宪法还应该有条款规定政府不能做什么。符合宪政原则的宪法,不仅应该规定政府能够做什么,而且应该规定了政府不能做什么。

宪法的第五个功能是提供关键政治争端的解决方法。比如,第 6 讲曾分析,总统制之下总统与议会可能会产生严重的政治冲突,那该怎么办呢?比如,美国总统奥巴马提交的要求提高政府债务上限的法案,参议院否决了。只要总统提交,国会就否决,那会出现什么情况?如果这样的冲突不是发生在民主历史悠久、两大主要政党拥有政治默契的美国,而是发生在另外一个新兴民主国家,就有可能导致政治僵局。为了解决这种政治僵局,智利在宪法中就规定了解决这种政治僵局的相应条款。必要时,总统可以通过付诸全民公决的办法来解决行政机构与立法机构的冲突,即总统与国会的冲突。这是宪法试图解决关键政治争端的一个例子。很多发展中民主国家发生军事政变,一个原因就是已有宪法没有提供解决关键政治争端的机制。对一个发展中国家来说,如果政治僵局持续较久,军队就很可能会出场干预政治。所以,宪法解决政治争端的功能也是非常重要的。

各国宪法的文本尽管篇幅不一,但宪法的文本结构往往是相似的。多数成文宪法均有四个部分构成:第一部分是宪法的序言,往往近似一个煽情的宣言,强调这部宪法和现行政府的合法性;第二部分对政治系统和制度安排的规定,设立哪些不同形式的政府机构,确立它们彼此间的政治关系等;第三部分是权利法案,即保护公民个人权利与自由的条款,可能还包括对救济机制的说明;最后一部分会涉及修改宪法的规则与程序等,通常修改宪法要比一般立法更为严格。总之,大多数宪法都包括序言、政治制度和政府结构、公民权利与自由以及修宪程序等四方面内容。

这里以美国宪法和印度宪法为例加以说明。比如,美国 1787 年《宪法》在序言中说: “我们合众国人民为了建立一个更完善的联邦,树立正义,确保国内安宁,完备共同防御,振兴公共福利,并保证我们子孙后代永享自由的幸福,特制订美利坚合众国宪法。” 这一简洁的序言阐明了制定宪法的目的,旨在论证合法性。

第二部分内容是关于美国政府的机构设置。大致上说,宪法规定了立法权赋予国会参议院和众议院,并规定了选举参议员和众议员的办法;行政权赋予总统,明确了总统职权,并规定了选举总统的办法;司法权赋予法院,并对法院以及行政、立法之间的关系做了界定;宪法还明确了联邦与各州之间的政治关系,即中央政府和地方政府之间的关系。这一部分确立了美国政府的政治机构与制度安排。

1787 年《宪法》正文没有包含权利法案的内容,但这部宪法颁布以后不久的 1791 年,美国国会通过的十条修正案实际上构成了美国的权利法案。这十条修正案规定了美国公民享有的诸种公民权利与自由,以及美国政府不得干涉或侵犯这些权利与自由的若干规定。

最后一个方面涉及宪法修订的程序。通过美国宪法修正案有两种途径:要么是国会参众两院三分之二多数通过;要么是三分之二的州提出,并且得到四分之三州的批准。从这些条款看出来,要通过美国宪法修正案难度是较大的。只有参众两院两大政党均有共识的事项,或是极高比例州均有共识的事项,才有可能通过宪法修正案。这也体现了修改宪法的审慎性原则。

再来看印度宪法。印度于 1947 年独立,但宪法却在 1949 年才颁布,从独立到制宪用了两年多时间。印度宪法起草委员会的主任是伦敦政治经济学院毕业的安贝德卡尔博士,这个人深受英国法律和政治传统的熏陶。按照现在的评价,安贝德卡尔对印度宪法和民主的贡献是很大的。印度宪法也包括四个方面的内容。它的序言这样写道: “我们印度人民已庄严决定,将印度建成为主权的社会主义的非宗教性的民主共和国,并确保一切公民:在社会、经济与政治方面享有公正;思想、表达、信念信仰与崇拜的自由;在地位与机会方面的平等;在人民中间提倡友爱以维护个人尊严和国家的统一和领土完整;鉴此,我们制宪会议于 1949 年 11 月 26 日通过,制定了本宪法,并将其公布于众。” 序言阐明了印度宪法的基本目的。

第二部分是关于印度政治制度的基本规定。宪法规定了联邦和各邦之间的关系,第一条就规定 “印度是一个联邦制国家” 。当然,实际上印度建国初期联邦政府的权力是相当大的,这跟印度实行了具有浓厚计划色彩的经济模式有关。一般认为,印度的尼赫鲁时代只能算是一个中央集权程度很高的准联邦制国家。后来,由于计划经济的改革和市场化,地方权力大幅增加以后,印度的联邦制色彩愈发浓厚。关于政府各主要机构的关系,印度宪法的规定是比较微妙的。从宪法文本上看, “联邦行政权属于总统” ,总统不仅是国家元首还是军队统帅。宪法同时设置了总理和部长会议,而且貌似总理和部长会议是在总统之下行使行政权的。但宪法又有这样的规定: “总统在行使其职权时根据部长会议的建议行事” 。这意味着只有在总理及部长会议认可的情况下,总统才能行使职权。这样一来,决定实际上是总理及部长会议做出的。宪法规定,总理及部长会议由人民院选举产生。这意味着印度是一个标准的议会制国家。宪法还规定,印度国会由联邦院和人民院组成,联邦院由各邦代表组成,人民院根据人口比例由全国各选区选举产生。另外,法院系统享有司法权。

第三部分规定了印度公民的自由权、反剥削权及文化教育权。印度宪法是世界上最长的宪法之一,关于这些权利的规定也很细。具体来说,印度宪法规定了所有印度公民 “法律上平等” 的原则, “禁止宗教、种族、种姓、性别、出生地的歧视” 原则, “公职受聘机会相等” 原则, “言论和表达自由” 原则,结社及迁徙自由原则,个人财产受保护原则, “反剥削权” 原则,享受 “文化教育权” 原则,等等。

第四部分内容则涉及宪法修订的基本规则。相比于美国宪法,印度宪法的修订条款较为宽松。其基本规则是,国会任何一院提出宪法修正案,然后由本院三分之二议员出席,半数通过即可通过宪法修正案。但是,少数涉及联邦与各邦之间关系的宪法条款需要半数邦议会通过方能生效,这是为了防止国会单方面改变印度联邦(中央)与各邦(地方)之间的关系。尽管印度宪法的修订较为容易,但整部宪法从 1949 年制定至今还是非常稳定的。在发展中国家里,印度是为数不多几个独立以后一直沿用一部宪法的国家。

\tsection{宪政与司法审查}

当然,宪法与宪政是两回事。如果宪法不被执行的话,它只不过是几张纸而已。那么,有没有什么办法能够让宪法真正起作用呢?由于宪法既不能自我制定、又不能自我实施,所以宪法必须依赖于机构和人才能起作用。在一些发达国家,实践宪政的一个重要方面就是司法审查或违宪审查制度。司法审查一般是指最高法院或宪法法院对行政机构或立法机构的法律与决定进行合宪性审查的机制。换句话说,如果议会通过的立法或政府作出的决定违反宪法的话,最高法院或宪法法院可以判决此类立法或决定违宪,从而宣布取缔这样的法律或决定。

具体来说,司法审查或违宪审查通常涉及三项内容:一是裁决具体的法律或决定是否符合宪法;二是解决国家和公民关于基本自由权的冲突;三是解决不同政府机构或不同层级政府之间的冲突。那么,由何种机构来实施司法审查或违宪审查呢?全球范围内主要是两种制度安排:一个是像美国那样由联邦法院即最高法院来负责实施,一个是像德国那样由专门的宪法法院来负责实施。

法律或决定是否违宪是一个重要问题。实际上,与此相关的判决直接关系到美国司法审查权的起源,相关案件可以追溯到美国 1803 年马伯里诉麦迪逊案\nauthor{这一案例参考了如下两种资料:任东来、陈伟、白雪峰等:《美国宪政历程:影响美国的 25 个司法大案》,北京:中国法制出版社 2005 年版,第 22—39 页;迈克尔 · C.道夫主编:《宪法故事》(第二版),李志强、牟效波译,张千帆审校,北京:中国人民大学出版社 2012 年版,第 10—23 页。本节的相关引文来自以上两种资料,一些译文根据相关司法资料的原文做了修订。}。美国联邦法院首席大法官约翰 · 马歇尔对此案的判决,深刻地影响了美国的司法与政治体系。事情大致是这样的:前总统已签署威廉 · 马伯里的地方治安官任命状,但国务卿由于事务繁忙而没有把该任命状签发出去。结果,总统和国务卿卸任以后,已经签署的、放在抽屉里的任命状被新任总统和新任国务卿截留了。新任国务卿决定不再颁发这一任命状。被任命的马伯里是一个商人,地方治安官实际上是不大的官职,但马伯里对此并没有逆来顺受、忍气吞声,而是将此案诉至最高法院,要求最高法院对政府下达强制令,请时任国务卿的詹姆斯 · 麦迪逊发出已签署的任命状。首席大法官马歇尔牵头审理此案。那么,最高法院如何审理呢?

这个案件涉及两个基本问题:第一,联邦法院对这一案件本身的看法,即新任国务卿是否应该发出任命状?第二,联邦法院代表的司法机构如何处理与总统代表的行政机构之间的关系?联邦法院有权干预总统与国务卿的决定吗?这的确是一个非常棘手的事情。从 1783 年美国立国、1787 年美国制宪到 1903 年本案,总共不过 20 年时间,美国作为一个新国家的根基还不稳固。尽管美国已经有了宪法和法律,但美国的很多重要制度安排和惯例都还在形成之中。事后来看,马伯里诉麦迪逊案之所以重要,是因为马歇尔通过对此案的判决塑造了美国新的司法与政治传统。

接手这个案件之后,马歇尔大法官的逻辑非常清楚。他认为主要有三个问题:

\quo{1. 申诉人马伯里是否有权利得到其委任状?

2. 若申诉人有这个权利且受到侵犯,政府是否应该为其提供救济?

3. 如果政府提供救济,是否应该由最高法院来下达强制令,要求国务卿颁发委任状?}

马歇尔与联邦法院其他法官经过审理后认为,前面两个问题是非常清楚的。既然前任总统与国务卿已经签发马伯里的任命状,该任命状已经生效——无论是否颁发至当事人手中。这样,拒发马伯里的委任状,侵犯了他作为一个公民的法律权利。所以,马伯里有权得到其任命状。

马歇尔对第二个问题的回答也是肯定的,即政府应该为马伯里提供救济。他这样说:

\quo{每一个人受到侵害时,都有权要求法律的保护。政府的一个首要责任,就是提供这种保护。合众国政府被宣称为法治的政府,而非人治的政府。如果它的法律对侵犯所赋予的法律权利不能提供救济,它当然就不值得这高尚的称号了。}

最为棘手的是第三个问题:最高法院是否应该向国务卿下达强制令呢?要知道,从 1787 年《宪法》算起,美国的国家制度刚刚形成十多年时间。联邦法院应该做什么?联邦法院与总统、与国会是何种关系?所有这些尽管有宪法文本作准则,但彼此的边界都在摸索过程之中。

马歇尔大法官在审理过程中,注意到一个重要的细节,即马伯里直接在联邦法院起诉麦迪逊,援引的是美国国会通过的《1789 年司法条例》第 13 条。这一司法条例的条款规定了类似情形,即马伯里可以将此类案例直接起诉到联邦最高法院,以联邦最高法院作为一审法院。但是,马歇尔发现,美国《宪法》第三条第 2 款规定,涉及大使、公使、领事以及一方以州为当事人的案件,最高法院具有一审管辖权,即这类案件可以直接起诉到最高法院。而马伯里准备接任的地方治安官不在该名单中,即该案件既非涉及公使、大使或领事,又非以一州作为当事人。

根据这种情形,又鉴于美国当时政治上的形势,马歇尔做了一个巧妙的回应,形成两个判决。判决一:由于管辖权问题,马伯里诉麦迪逊案予以驳回。因为按照美国宪法的规定,联邦最高法院不是此类案件的一审法院。判决二:美国国会通过的《1789 年司法条例》第 13 条的规定违宪,联邦最高法院决定撤销这部法律中的第 13 条。也就是说,马歇尔通过判决的方式宣布《1789 年司法条例》第 13 条作废,不再具有法律效力。这意味着联邦法院获得了判定美国国会通过的法律是否违宪的权力。所以,此案成为美国司法审查权的起源。

马歇尔法官在本案判决中充分阐述了宪法与普通法律的关系,他这样说:

\quo{一部普通的法律和宪法之间只有两种关系:第一种关系是平行关系,即普通法律与宪法的效力是相当的;第二种关系是普通法律在宪法之下,即宪法的效力要高于普通法律。……宪法构成国家的根本法律和最高的法律,违反宪法的法律是无效的。而断定什么法律是违宪,显然是司法部门的职权和责任。}

所以,今天美国流传着这样一种说法:美国宪法是什么呢?联邦大法官们说是什么,它就是什么。这句话当然有调侃的成分,但通过这句话大家也可以看出,美国司法部门拥有巨大的权力。自马歇尔大法官判决马伯里诉麦迪逊案开始,违宪审查权或司法审查权就成为美国法治传统的惯例。由此,马歇尔大法官也塑造了联邦法院与总统、国会之间的政治关系。

从这个案例,大家还可以看出,美国今天的政治框架固然是 1787 年《宪法》规定的,但行政权、立法权与司法权的实质性关系也是由不同的政治人物在后来的实践中不断塑造的。对美国政治来说,宪法框架固然重要,但一些重要人物在关键时刻的做法和实践也非常重要。新的政治传统,往往是由那些既有历史担当又有政治智慧的人物们开创的。

司法审查的重要性,还在于它可以解决国家和公民关于基本自由权的冲突。比如,在 20 世纪 60 年代初的美国,一位白人在洛杉矶街头开了一家餐厅。这位餐厅老板对白人与黑人顾客本身没有不同的偏好。他作为一个精明的生意人,主要考虑的是如何挣钱,他只想经营好自己的餐厅。所以,照理说,任何顾客来吃饭他都应该表示欢迎——只要他们愿意掏钱。但是,他发现附近的社区主要是白人,他的主要顾客是白人。这样,按照当时的社会气氛,如果这位餐厅老板允许黑人进入餐厅就餐,白人顾客可能就不会来了。所以,尽管他本身对黑人没有偏见,但出于商业利益的考虑,他贴一个非常礼貌的告示,意思是本餐厅只对白人开放。

但问题就来了。他作为一个餐厅老板,有权作出这样的决定吗?这是一个法理问题。真实的情形可能是,这位老板贴出告示后,若干年中没有人提出任何异议。直到有一天,一位类似马丁 · 路德 · 金的黑人出现在他的餐厅门口。这个人就是要进来吃饭,他进入餐厅,坐在这里不走,掏出了美金,说要点菜。那么,这位餐厅老板可以拒绝给他提供服务吗?或者有权把他轰走吗?万一这位黑人顾客不离开餐厅,他可以叫警察吗?如果警察来了,他还是不离开餐厅,警察可以强行驱离吗?再比如,如果两个警察强行架起这位黑人顾客,把他扔到大街上,并阻止他再次进入餐厅,这位可能受了点轻伤的黑人可以起诉这家餐厅,甚至可以起诉洛杉矶警察局吗?或者,他可以选择起诉加利福尼亚州政府吗?这位黑人顾客可以把这个官司从普通法院打到上诉法院,甚至一直上诉到联邦法院吗?

如果这个案子最终被诉至联邦法院,联邦大法官们的重要性就凸显出来了。面对这样的案子,联邦最高大法官们会怎么判呢?实际上,他们不只是在决定这个案子本身,而是在决定一个国家的基本自由权利及其具体政策。联邦法院最后的判决是所有餐馆以及所有的私人和公共机构必须无差别地对所有公民开放,无论他的肤色是什么,否则就是违宪。后来,借助立法,这一准则又成为美国基本的法律原则。在这一案例中,大家看到了司法权的强大力量,而且这种司法权的影响甚至超越了民主的多数规则。比如,如果要就这个事情进行全民投票,结果则可能完全不同。如果黑人人口比例只有 15\% ,其他有色人种人口比例只有 5\% ,而白人人口比例占 80\% 的话,全民投票更有可能反对联邦法院的判决,而非支持这一判决。实际上,与上面假想的这一案例相似的事件大致在美国历史上就发生过,当然细节的差异是很大的。

此外,司法审查还可以解决不同政府机构或不同层级政府之间的冲突。比如,18 世纪末,美国某个州制定一部法律,由于本州税源不足,要对来往本州的不同州货物征收 3\% 的过境税。那么,州政府有权这样做吗?假使美国联邦政府说,你不能这么干。但州政府说,我现在只能这么干,因为本州财政出现了严重问题。要知道,联邦制下美国联邦政府并非州政府的上级,两者之间没有直接的隶属关系。所以,美国总统不能把州长撤了。那么,这个事情怎么办呢?难道要靠武力解决吗?似乎并不妥当。政治不成熟的国家,出现关键争端以后通常需要用武力来解决,而政治成熟的国家可以借助政治或法律手段来解决。联邦法院大法官们判决,任何州不得制定任何对过境货物征税的法律,否则就是违宪。这样,该州只好老老实实地把这个法案废了,这个政治争端就解决了。在这一案例中,违宪审查通过司法判决,解决了不同层级政府之间的冲突,从而增进了民主政体的稳定性。通常,关于宪政的另一个问题是宪法真的能起作用吗?这是一个极重要的政治问题。《控制国家》一书作者斯科特 · 戈登认为,宪政是国家的强制性权力受到约束的观念。他在书中引用了查尔斯 · 麦克尔文的观点:

\quo{在所有相继的用法中,立宪主义都有一个根本的性质:它是对政府的法律制约……真正的立宪主义的本质中最固定和最持久的东西仍然与其肇始时几乎一模一样,即通过法律限制政府。\nauthor{转引自斯科特 · 戈登:《控制国家——西方宪政的历史》,应奇等译,南京:江苏人民出版社 2001 年版,第 5 页。}}

由此可见,现代国家的宪政观念是指通过宪法和法律约束政府与政治权力。这样,宪政的一边是宪法和法律,另一边是政府和政治权力。但问题是,在政治实践中,宪法和法律可能是死的,政府和政治权力是活的。如果是这样,宪法和法律如何能限制住政府和政治权力呢?这实在是一个非常困难的事情。另一个悖论是,谁是宪法或法律的捍卫者和执行者呢?一个通常的回答是政府(广义的政府)。那么,问题就来了,一方面政府是法律的执行者,另一方面却又要用法律来约束政府,这又如何可能呢?因此,对很多国家来说,实践宪政并非易事。

戈登认为,宪政在政治实践中表现为政治权力分立与制衡的原则,就像美国那样,实现立法权、行政权和司法权的分立并使之互相制衡。他还认为,宪政原则包含着对抗性的元素,即政治权力内部的对抗性。如果有一个最高权力能统辖其他所有的权力,就不符合宪政原则。按照戈登的观点,权力分立与制衡是宪政的核心。

与之相关的问题是,宪法真的能运转起来吗?从历史经验来看,宪法能否运转起来,依赖于很多因素。首先,政治文化就是一个重要因素。在宪政问题上,最重要的文化是信仰规则。规则一旦制定,就应该遵守规则,这是基本的规则信仰。如果有人觉得规则不合适,又该怎么办?有规则信仰的群体不是去违反规则,而是应该先谋求改变规则,等新规则生效以后再实施新的做法。但是,有些国家的文化缺少这种规则信仰,出现问题首先考虑不是如何调整规则,而是考虑如何破坏规则。

其次,统治者、主要政治集团或主要政治力量是否尊重宪法也是一个重要问题。如果这一点没有保证,宪法就难以运转起来。宪法要能够运转起来,要么是居于支配地位的政治集团尊重宪法,信仰宪政;要么几个不同的政治集团在宪法规则基础上达到了政治均衡——尽管几个政治集团并不信仰宪政但他们实力相当,最后不得不遵守既有的主要规则。第一种情形当然是比较有利的。当年印度建国之初,国大党作为一个支配性的政治集团是信奉宪政、尊重宪法的,这对印度民主宪政的维系比较有利。第二种情形下也有可能建成宪政,但只是主要政治集团在现有宪法规则基础上实现政治均衡罢了。一个著名例子是 1919—1933 年的德国魏玛共和国。1918 年德意志帝国倒塌、1919 年制定魏玛宪法时,不同政党和政治力量互相竞争,而只有宪法规则是不同的政治力量都能接受的。但实际上,当时的很多政治家都不是民主与宪政的真诚信仰者,魏玛共和国由此被戏称为 “没有民主主义者的民主国,没有共和主义者的共和国” 。当然,这种均衡可能是不稳定的。

再次,是宪法本身的挑战,即宪法有没有适应变化的能力。有些宪法由于制度设计的问题,本身存在重大缺陷。这种缺陷主要表现在两个方面:一是宪法条款由于设计不当而导致政治僵局;二是宪法条款本身弹性较低,不能适应快速变化的政治局势。这种情况下,宪法也难以运转起来。

上述讨论涉及的是宪法的政治实践。与其说宪法或法律独立于政治,毋宁说宪法或法律就是政治的成果。因而,对宪法与宪政进行政治分析就非常必要。通常,法学界比较重视宪法的文本及宪法条款的法学价值,但比较忽视对宪法的政治分析。政治学思考宪法或宪政问题,更重视政治分析。图 7.1 简要解释了宪政与政治的一般关系。一方面,宪政固然是约束政治的规则。从理想上说,宪政是用来约束和规范政治的,是用来规范政府和主要政治集团行为的。但另一方面,现实的宪政首先是某个政治过程的结果,宪政实现与否是政治过程本身塑造和决定的,宪政可以被视为一种政治均衡状态。所以,如果不进行政治分析,理解宪法和宪政问题上往往会失之偏颇。

\img{../images/image00317.jpeg}[图 7.1 宪政与政治的关系]

\tsection{法律体系与司法系统}

从欧洲文明的演进历史看,其法治传统的塑造离不开自然法或高级法的学说。\nauthor{关于法律原理与不同法律体系的更多内容,请参见博登海默:《法理学:法律哲学与法律方法》,邓正来译,北京:中国政法大学出版社 2004 年版;米健:《比较法导论》,北京:商务印书馆 2013 年版,特别是第五、六章。}这一学说认为,自然法或高级法在逻辑上优先于人类制定法,即制定法或实在法。按照这种法律思维,立法必须要考虑自然法或高级法的法律原则。而自然法学说强调的是自然正义原则,重视人的自然权利,认同天赋人权,认为自然权利不证自明并具有普遍性,而成文法应该遵循自然法的基本原则。一句话,法律的目的是保障所有人无差别的自由权利。从经验来看,如果没有对自然法原则的信仰,要想建设法治社会就会有相当的难度。尤其是,如果一个国家的精英阶层不信仰这些理念,就难以建成一个法治国家。

法律制定以后,执行则有赖于法院系统与司法体系。通常,司法体系由不同层级的法院构成,一般包括基层法院、上诉法院和最高法院。由于各国司法体系的不同,不同层次法院之间的关系也不同。就一般法治原则而言,尽管不同的法院与法官有层级差异,但不应该有行政上的命令与等级关系。换言之,上级法院法官并非下级法院法官的行政上司。所以,尽管法官所在法院的层级不同,但每个法官都应该是独立的审判者,他们只根据法律与良心判案。法治原则要求每一位法官不应受任何其他利益与行政命令的左右。

当然,在不同的法律体系下,法官的角色和法院系统的内部治理是不同的。西方世界有两种主要的法律体系:英美法和大陆法,又称普通法与法典法。从起源上说,普通法起源于英格兰,盛行于美国和英联邦国家;而大陆法起源于古罗马,完善于法国,盛行于欧洲大陆以及其他多数国家。从形式上说,普通法是判例法。所谓判例法,就是基于法院的判决而形成的具有法律效力的判定,这种判定对以后的判决具有法律规范效力,能够作为法院判案的法律依据。大陆法是成文法,先要立法,即制定规则,包括主要条款及细则——法律条文应该讲清楚什么可以做和什么不能做。倘若有人违法,对照法律,即可判定是否违法或犯罪,处罚细则亦有相应规定。英美法依赖的是判例,整个法律体系是过去的判例累积构成的。当然,通常情况下,所有判例都基于某些确定的法律原则。所以,大陆法操作起来相对容易,但灵活性比较低;英美法操作起来比较复杂,但灵活性比较高。

在两种法律体系下,法官的角色也有很大不同。英美法系中的法官是比较消极的,在法庭上更多地扮演一个召集人的角色。照理说,他既不应该偏向原告,又不应该偏向被告,他应该是一个公允的中间人。至于被告罪名是否成立,主要取决于普通公民代表组成的陪审团的决定,陪审团有权决定有罪还是无罪。比如,美国著名的辛普森案中,他被控谋杀前妻,证据似乎也很多,但最后陪审团判决辛普森无罪,谋杀罪名不成立。此案到今天仍然还有很多不同的说法。但无论怎样,英美法系中法官的角色相对消极,而陪审团发挥很大作用。大陆法系的法官更为积极,除了出庭审判的法官,还包括一些起调查作用的法官。在庭审过程中,法官代表了法律,法官最终决定被告有罪还是无罪。

两种法律体系下被告的法律地位也略有差异。普通法坚持被告无罪推定的原则,即在证明并判决为有罪之前,被告或嫌疑人均被视为无罪。比如,在尚未开庭审判之前,就公开传播关于犯罪嫌疑人如何作案的确切信息,这在普通法系下是极不妥当的。任何人在被证实并判决有罪前,都应当被视为无罪。这是普通法的一条重要原则。大家要注意的是,犯罪嫌疑人和最终被定罪的罪犯是完全不同的概念。大陆法系一般要求被告自证清白,当然不同国家的程序是不同的。比如,有人被指控上周五晚上谋杀了本地一位富豪。在普通法系下,被告即便一言不发,也照样无碍。起诉方则需设法证明此事就是他干的。而陪审团的态度通常是,除非控方能够提供足够有力的证据,证明此事确是被告所为,否则就予以驳回,无罪释放。但是,在大陆法系下,被告最好能提供证据,说明自己与多位证人在参加宴会或其他活动,以证明自己无法同时出现在犯罪现场。所以,是否坚持无罪推定原则,也是两大法律体系的差异。

两种法律体系下司法与行政的关系也有差异。普通法系的司法更多地独立于行政,大陆法系的司法跟行政有更为密切的关系。所以,完全意义上的司法独立概念更多适用于普通法系,而非大陆法系。当然,在法治完善的国家,即便是大陆法系,司法尽管并非完全独立于行政,但法官和法院的独立判案之权通常不会受到政治或行政的干扰。

最后,两种法律体系的司法审查权或违宪审查权差异也很大。通常,普通法系下的司法审查权会比大陆法系下的司法审查权更大,但不同国家的具体情形又不一样。司法审查权较为突出的典型国家是美国,但同为普通法国家的英国就很难讲有独立的司法审查权,因为英国更强调议会主权的原则。相比而言,大陆法系的司法审查权通常比较有限,但德国的宪法法院却又比较活跃。因此,两种法律体系下的司法审查权总体上很不一样,但就个别国家而论实际情形又较为复杂。

\tsection{公民权利与《世界人权宣言》}

宪政与法治的主要目的是为了保护公民的自由与权利,或者说保障基本人权。一般认为,人类历史上第一部跟人权或公民权利有关的法案是英国 1689 年《权利法案》。1787 年美国《宪法》尽管没有公民权利条款,但随后通过的十条修正案构成了美国的权利法案。法国在大革命期间的 1789 年颁布了《人权与公民权宣言》,这部宣言受到了启蒙思想与自然法学说的重大影响。这部宣言包含了如下重要条款:

\quo{在权利方面,人人与生俱来而且始终自由与平等,非基于公共福祉不得建立社会差异。

一切政治结合均旨在维护人类自然的和不受时效约束的权利。这些权利是自由、财产、安全与反抗压迫。

整个主权的本原根本上乃存在于国民。任何团体或任何个人皆不得行使国民所未明白授予的权力。

自由是指能从事一切无害于他人的行为;因此,每一个人行使其自然权利,只以保证社会上其他成员能享有相同的权利为限制。此等限制只能以法律决定之。

法律仅有权禁止有害于社会的行为。凡未经法律禁止的行为即不得受到妨碍,而且任何人都不得被强制去从事法律所未要求的行为。}

到了 20 世纪——特别是第二次世界大战以后,与公民自由权利或基本人权有关的国际公约开始出现。比如,联合国 1948 年通过的《世界人权宣言》就是一例。\nauthor{ “人的安全网络” 组织编写:《人权教育手册》,李保东译,北京:生活 · 读书 · 新知三联书店 2005 年版,第 495—500 页。}在此基础上,联合国又于 1966 年通过了《公民权利和政治权利国际公约》,进一步声张和明确了人类的自由权利与基本人权。

本讲此处仅以《世界人权宣言》为例来大致介绍国际上公认的公民权利的基本内容。第一,宣言强调人的生命权与自由权不受侵犯,并且所有人在法律上一律平等。比如,宣言制定了这样的条款:

\quo{第一条人人生而自由,在尊严和权利上一律平等。他们赋有理性和良心,并应以兄弟关系的精神相对待。

第二条人人有资格享受本宣言所载的一切权利和自由,不分种族、肤色、性别、语言、宗教、政治或其他见解、国籍或社会出身、财产、出生或其他身份等任何区别。…………

第三条人人有权享有生命、自由和人身安全。}

第二,宣言保护人享受不受非法拘禁或逮捕、不遭受酷刑以及在合法判定为有罪之前被视为无罪的权利。比如,宣言包含了如下条款:

\quo{第五条任何人不得加以酷刑,或施以残忍的、不人道的或侮辱性的待遇或刑罚。

…………

第九条任何人不得加以任意逮捕、拘禁或放逐。

…………

第十一条 (一)凡受刑事控告者,在未经获得辩护上所需的一切保证的公开审判而依法证实有罪以前,有权被视为无罪。……}

第三,宣言认为人的隐私权、迁徙自由、财产权利、言论与思想自由、集会与结社自由都应该受到法律的保护。比如,宣言有这些条款:

\quo{第十二条任何人的私生活、家庭、住宅和通信不得任意干涉,他的荣誉和名誉不得加以攻击。人人有权享受法律保护,以免受这种干涉或攻击。

第十三条 (一)人人在各国境内有权自由迁徙和居住。……

…………

第十七条 (一)人人得有单独的财产所有权以及同他人合有的所有权。

(二)任何的财产不得任意剥夺。

第十八条人人有思想、良心和宗教自由的权利;此项权利包括他的宗教或信仰的自由,以及单独或集体、公开或秘密地以教义、实践、礼拜和戒律表示他的宗教或信仰的自由。

第十九条人人有权享有主张和发表意见的自由;此项权利包括持有主张而不受干涉的自由;和通过任何媒介和不论国界寻求、接受和传递消息和思想的自由。

第二十条 (一)人人有权享有和平集会和结社的自由。

(二)任何人不得迫使隶属于某一团体。}

第四,宣言明确指出应该保障公民政治参与的权利,包括选举权和被选举权应该得以保障。比如,宣言制定了与民主权利有关的这一条款:

\quo{第二十一条 (一)人人有直接或通过自由选择的代表参与治理本国的权利。

(二)人人有平等机会参加本国公务的权利。

(三)人民的意志是政府权力的基础;这一意志应以定期和真正的选举予以表现,而选举应依据普遍和平等的投票权,并以不记名投票或相当的自由投票程序进行。}

第五,宣言还规定了人应该享有的社会保障权、工作权、休息权以及受教育权等。比如,宣言包括相应的如下条款:

\quo{第二十二条每个人、作为社会的一员,有权享受社会保障,并有权享受他的个人尊严和人格的自由发展所必需的经济、社会和文化方面各种权利的实现,这种实现是通过国家努力和国际合作并依照各国的组织和资源情况。

第二十三条 (一)人人有权工作、自由选择职业、享受公正和合适的工作条件并享受免于失业的保障。

(二)人人有同工同酬的权利,不受任何歧视。…………

第二十四条人人有享受休息和闲暇的权利,包括工作时间有合理限制和定期给薪休假的权利。

…………

第二十六条 (一)人人都有受教育的权利,教育应当免费,至少在初级和基本阶段应如此。初级教育应属义务性质。技术和职业教育应普遍设立。高等教育应根据成绩而对一切人平等开放。

(二)教育的目的在于充分发展人的个性并加强对人权和基本自由的尊重。教育应促进各国、各种族或各宗教集团的了解、容忍和友谊,并应促进联合国维护和平的各项活动。…………}

应该说,《世界人权宣言》的条款反映 1948 年人类关于公民自由与权利的共识。一方面,宣言包括和强调了像生命权、自由权、财产权和平等权,等等;另一方面,宣言也包括和强调了人的受教育权、保障权、发展权和社会福利权,等等。由此可见,这一宣言体现了不同意识形态的融合,具有很强的包容性。

\tsectionnonum{推荐阅读书目}

戈登:《控制国家:从古代雅典到今天的宪政史》,南京:江苏人民出版社 2005 年版。

任东来、陈伟、白雪峰:《美国宪政历程:影响美国的 25 个司法大案》,北京:中国法制出版社 2005 年版。

张千帆:《宪法学导论:原理与应用》,北京:法律出版社 2008 年版。
