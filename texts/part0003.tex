\tchapter{什么是政治?}

\quo[——亚里士多德]{政治似乎就是这门最权威的科学。因为正是这门科学规定了在城邦中应当研究哪门科学,哪部分公民应当学习哪部分知识,以及学到何种程度。我们也看到,那些最受尊敬的能力,如战术、理财术和修辞术,都隶属于政治学。}

\quo[——罗伯特·古丁与汉斯-迪特尔·克林格曼]{我们认为,不受限制的权力是带有暴力性质的,并且是纯粹的和简单的。除了在一些退化的、极限的意义上事实可能如此之外,这完全不能算作政治权力的运作。纯粹的暴力更多是一种物理力量而不是政治。在我们看来,只有政治参与者行动的约束条件以及在这些约束条件下指导他们行动的策略,才构成政治的本质。}

\quo[——韩非]{医善吮人之伤,含人之血,非骨肉之亲也,利所加也。故舆人成舆,则欲人之富贵;匠人成棺,则欲人之夭死。人不贵,则舆不售;人不死,则棺不买,情非憎人也,利在人之死也。}

\quo[——卡尔·施米特]{自由主义的系统理论几乎只关心国内反对国家权力的斗争。为了实现保护个人自由和私有财产的目的,自由主义提出了一套阻碍并限制国家和政府权力的方法。……由此,我们看到了一个完整的非军事化、非政治化的概念体系。}

\tsection{政治是国家兴衰的关键}

政治的重要性毋庸置疑。每个人都无法逃避政治,无论你喜欢或不喜欢;每个人过得快乐或不快乐,通常都跟政治有关。古希腊思想家亚里士多德在《政治学》中说,人是天生的政治动物。他在《尼各马可伦理学》中还认为:

\quo{政治似乎就是这门最权威的科学(the master science)。因为正是这门科学规定了在城邦中应当研究哪门科学,哪部分公民应当学习哪部分知识,以及学到何种程度。我们也看到,那些最受尊敬的能力,如战术、理财术和修辞术,都隶属于政治学。\nauthor{亚里士多德:《尼各马可伦理学》,廖申白译注,北京:商务印书馆2008年版,第5—6页。}}

政治的重要性还体现在政治与经济的关系中。过去的教科书认为,经济决定政治;但从另一个角度看,政治也决定着经济。实际上,政治与经济本身就是一种互动或互相影响的关系,参见图1.1。

\img{../Images/image00296.jpeg}[图1.1 政治与经济的互动关系]

由于门户之见,或许大部分学者都认为自己学科较其他学科更为重要。所以,这里不必讨论政治学者认为政治重要的观点,大家不妨来听听经济学家的看法。这里介绍的第一位经济学家是曼瑟·奥尔森,他尽管没有获得诺贝尔经济学奖,但他的著述——特别是《集体行动的逻辑》——在学界引用率非常之高,他的另一部书《权力与繁荣》在中国大陆也非常流行。\nauthor{曼瑟尔·奥尔森:《集体行动的逻辑》,陈郁等译,上海:格致出版社2011年版;曼瑟·奥尔森:《权力与繁荣》,苏长和译,上海:上海人民出版社2009年版。}实际上,《权力与繁荣》的书名就隐含着一种逻辑:政治权力很大程度上决定了是否能实现经济繁荣。

1993年,奥尔森发表了一篇题为《独裁、民主与发展》的论文,讨论的是政治与经济的关系。\nauthor{参见Mancur Olson,“Dictatorship, Democracy and Development”, \italic{American Political Science Review}, Vol 87, No.3(Sept.1993), pp.567-576。为了写作更生动,作者此处对奥尔森这篇论文的思想做了自己的阐释和发挥,而非完全是对原著的引用。}有趣的是,他引用了20世纪20年代中国军阀割据的历史。在当时的中国,大小军阀的割据与混战非常厉害。一个地方今天被你占了,明天又被他占了。如果这种状况得以持续,这些地方实际上就会沦为大小军阀的流寇统治。本来老百姓在搞生产,包括种植庄稼和饲养家畜等。突然,流寇来了。流寇们不仅互相交战,而且打完之后还把附近的村庄洗劫一番。

这种流寇统治对经济的影响几乎是毁灭性的。流寇统治的最大问题是破坏了一个社会的正常激励机制。简单地说,农民之所以春天播种,是因为预期秋天能够收获。流寇统治带来了什么问题呢?种地的是一个人,收获的是另一个人。因此,流寇统治会导致整个社会生产的迅速下滑。为什么长期内战通常会死很多人?并不是说内战中被打死的人有那么多,而是内战破坏了整个社会的激励机制,其生产系统被迅速摧毁了,所以会导致大量人口的死亡。

就在这种政治混乱的情形里头,有个别地方的治理开始出现好转。原因很可能是流寇中的某个强有力者把其他流寇都赶走了,他自己摇身变为坐寇。坐寇为什么可能强于流寇?因为坐寇发现,倘若自己依靠武力以抢劫为生,就没有人从事生产了,老百姓都逃亡了,经济也就完蛋了。经过理性思考,坐寇发现这样是没有出路的,统治无法长期维持。所以,他下令任何军人不得再抢劫任何财物,但同时规定辖区内的每一住户须按一定税率给他缴税,比如大致按每亩地平均收成的10%缴税。

与流寇统治相比,坐寇统治怎样呢?无疑,其社会激励机制要好得多。所以,大家就有了从事生产活动的动力,大约只需要把10%的收成作为税收交给统治者。这个统治者由于有税收需要,所以他希望经济能够发展。当经济发展时,他的税收收入也能相应增加。因此,坐寇统治要强于流寇统治。

但是,坐寇统治仍然有它的问题。从逻辑上说,第一个问题是有些坐寇会变得极其贪婪,甚至会极不理性。比如,他可能会制定非常高的税率,甚至还采取变相的掠夺手段。对辖区内富有的阶层或较大的企业,统治者可能想征收他们的财产。这样,当统治者变得贪婪时,“统治之手”就变成了“掠夺之手”。

坐寇统治的第二个问题是每个统治者都会死,这就面临坐寇代际更替的问题。假如统治者有生之年考虑要把位置传给自己的下一代——假定他能顺利传给下一代的话,仍然存在一个问题:统治者是代际关系中的利他主义者吗?简单地说,这个统治者只考虑自己的现世享受呢,还是考虑“子孙后代江山永固”的问题?或许有统治者像关心自己的福利一样关心自己子女的福利,这样的统治者会更加深谋远虑和审慎节制。但是,大家都听过法国国王路易十五的名言——“在我之后,哪管洪水滔天”。结果,在路易十六的时代,法国大革命就爆发了。所以,统治者并非总能做到深谋远虑和审慎节制。

因此,尽管坐寇统治要比流寇统治好,但并非是一种理想的统治形式。只有把统治者的权力放到宪政民主的框架中,统治者才不会胡作非为,长期持续的繁荣才有保证。换句话说,坐寇或统治者的权力必须受到制约,统治者不是想干什么就可以干什么。这是宪政的基本原则,宪政的简单理解就是“限政”,即限制政治权力。当然,与宪政密切相关是民主的办法,即辖区内的居民对由谁来统治拥有投票和选择权,就是由被统治者来决定谁可以统治大家。因此,通过宪法和民主的方法来决定统治规则,才更有可能实现持久的繁荣。当然,从经验来看,民主政体是否更有利于经济增长和长期繁荣,目前学术界还有争议。但是,奥尔森的言说逻辑无疑是可信的。作为经济学家的奥尔森认为政治非常重要:如果统治和统治者的问题不能很好解决,持久的繁荣是不可能的。

另一位美国经济学家、诺贝尔经济学奖得主道格拉斯·诺思在中国的影响很大。他在《经济史中的结构与变革》中提出了这样一个问题:在近代欧洲,为什么是英国与荷兰,而不是法国与西班牙,较早地开始工业革命?他认为,英国的制度倾向于保护财产权利,法国的制度中君主力量过大,缺少应有的制约,所以法国君主有可能侵犯臣民的财产权。比如,诺思提到,法国南部原来有一个比较有名的市场,但由于法国国王征收与掠夺,这个市场就逐渐衰落了。这不过是当时法国政治经济模式的冰山一角。英国就与此不同,英国从1215年《大宪章》开始,就确立了制约国王政治权力的传统。后来,尽管国王和贵族就这个问题不断发生冲突,但到了1688年,英国就基本确立了宪政体制。宪政体制使得英国的财产权利得到了有效保护。这样,更有效率的产权制度最终成就了英国的工业革命。所以,诺思认为,英国工业革命的前提是从13世纪到17世纪英国政治体系的变革及宪政体制的确立。\nauthor{道格拉斯·诺思:《经济史中的结构与变革》,厉以平译,北京:商务印书馆2005年版。}

诺思及其合作者后来的研究同样证明了政治的重要性,他们在2008年出版的《暴力与社会秩序》(\italic{Violence and Social Order})一书中有如下表格(见表1.1)。该表是对1960年到2000年穷国与富国的经济增长率和经济增长年份比率的比较。该表的三个发现是:(1)在经济增长的年份,穷国的平均经济增长率要高于富国;(2)穷国经济增长年份的比率却远远低于富国;(3)在经济衰退的年份,穷国的经济衰退率也高于富国。

\tbl{../Images/image00297.jpeg}[表1.1 穷国与富国的好年头和坏年头的增长率(国家按2000年人均收入分类)]

资料来源:道格拉斯·C.诺思、约翰·约瑟夫·瓦利斯、巴里·R.温格斯特:《暴力与社会秩序:诠释有文字记载的人类历史的一个概念性框架》,杭行、王亮译,上海:上海人民出版社2013年版,第7页,表1.2。中译表格数据存在少许错误,此处参考英文原著作了调整。

表1.1中,国家是根据2000年人均收入来区分的。我们这里只关注三个数据:一是增长年份的比率;二是增长年份的平均增长率;三是衰退年份的平均衰退率。为了简化处理,此处只考察两类国家。一类是人均收入超过20 000美元的非石油国家,该表中有27个。在所有统计年份中,这些国家84%的年份是增长的,增长年份的平均增长率是3.88%;16%的年份是衰退的,衰退年份的平均负增长率是-2.33%。如果按100年计算,富国84年实现了增长,平均增长率是3.88%;16年出现了衰退,平均负增长率是-2.33%。两者相比,富国的长期经济绩效应该是不错的。

一类是人均收入300至2000美元的最穷国家组,该表中有44个国家。这些国家只有56%的年份是增长的,增长年份的平均增长率达到了5.37%(高于富国的3.88%),但是它们有44%的年份是衰退的,衰退年份的平均负增长率是-5.38%。如果按100年计算,穷国56年实现了增长,平均增长率是5.37%;但有44年出现了衰退,平均负增长率达-5.38%。两者比较,就会发现,即便在一个世纪的漫长时间里,这些国家能够实现的长期经济绩效是非常有限的。

从表1.1看出,穷国贫穷的核心是其经济增长不是连续的,有大量年份处在经济衰退中。诺思及其合作者追问:为什么穷国增长年份的比率如此之低?总的来说,他们认为,构建有效社会秩序的困难,是穷国经济增长困境的根源——而这看上去更多是一个政治问题。

所以,诺思的两项研究都证明了政治对于经济增长的重要性。诺思认为:“国家既是经济增长的关键,也是人为的经济衰退的根源。”这句话后来被称为国家问题的“诺思悖论”。如果说国家是决定经济增长或衰退的关键因素,当然就证明了政治对经济的重要性。

\tsection{中国人的政治观}

既然政治很重要,那么什么是政治呢?这里先考察中国古代对于政治概念的理解。\nauthor{关于中国古代对于政治概念的理解,部分内容参考了如下作品:孙关宏、胡雨春、任军锋主编:《政治学概论》(第二版),上海:复旦大学出版社2008年版,第一章。}《说文解字》上说“政,正也”,可以理解为端正或正确。在中国古代典籍中,“政”字有几种主要含义:(1)政可以指政事。《尚书·洪范》中有“八政”的说法,意指八种政事。(2)政可以指政权或权柄。《论语》说:“天下有道,则政不在大夫。”孔子的意思是说,如果天下有道,政权或权力不应该在大夫的手上,而应该在国君的手上。(3)政可以指政令和政策。苏轼所说“今日之政,小用则小败,大用则大败”中的“政”是指政令和政策。(4)政可以指主持政事。《宋史·欧阳修传》记载:“其在政府,与韩琦同心辅政。”是说欧阳修与韩琦两人同心协力来辅佐朝政,主持政事。(5)政可以指国家的制度和秩序。《春秋左传》有“大乱宋国之政”的说法,是指整个制度和秩序乱掉了。(6)政还可以指符合礼仪和道德的做法,这更多代表儒家的观点。《论语》说:“政者正也,子帅以正,孰敢不正?”这里的“政”可以理解为一种符合礼仪和道德的做法。这几种含义代表了中国古人对“政”的理解。

“治”在古代本为水名。《说文》记载:“治,水,出东莱曲城阳丘山,南入海。”这里提到的地方大概在今天胶东半岛的龙口一带。后来,“治”就被引申为“治水”“治理”“整治”的意思。在中国古代,“治”可以指治理和统治。《史记·循吏列传序》里面说:“奉职循理,亦可以为治,何必威严哉?”意思是说“奉职循理”就可以治理得很好。“治”还可以指政治清明、社会安定。这里的“治”与“乱”相对,成语有“治乱兴衰”的说法。《易·系辞下》也有这样的用法:“君子安而不忘危,存而不忘亡,治而不忘乱。”

那么,“政”和“治”两字合用是什么意思呢?在中国古代典籍中,“政治”主要有几层含义。第一层含义是政事得到了治理。比如,《书·毕命》有“道洽政治,泽润生命”的说法,汉朝贾谊的《新书·大政下》说:“有教然后政治也,政治然后民劝之。”这里的“政治”都是指“政”事得到有效的“治”理。第二层含义是政事治理的本身。《宋书·沈攸之传》记载:“至荆州,政治如在下口,营造舟甲,常如敌至。”这里的“政治”讲的是政事治理本身,而没有讲到政事是否得到了有效的治理。第三层含义是治理国家所实行的措施。比如,《周礼·地官·遂人》上说:“掌其政治禁令。”这里的“政治”跟治理国家的措施有关。此外,在中国古代,“政治”最通用的理解是指君主及其大臣统治国家和治理社会的活动。这也更符合现代政治学常用的表述方式。

到了近代,中文语境中又是如何理解政治呢?近代政治的概念应为英文“politics”的中译。比如,《海国图志》里就有“梭伦所定政治章程……”“罗马军旅最有纪律……勤修政治”的说法。1903年,《新尔雅》介绍“政治”词条时这样说:“(《释群》)成国家政治之中枢机关谓之统治机关,统治机关之运营谓之政治。确定表明政治之理想者谓之立法,实行政治之理想者谓之行政。”这里把政治理解为“统治机关之运营”,代表了中国人在20世纪初对政治的理解。

中国近代政治观的主要代表人物是孙中山先生,他在《民权主义》中的政治观非常经典,为人熟知——

\quo{政治两字的意思,浅而言之,政是众人的事,治就是管理,管理众人的事便是政治。有管理众人之事的力量,便是政权,今以人民管理政事,便叫做民权。}

1949年以后,国内公共政治课教科书的观点很大程度上受到20世纪30年代苏联政治教科书的影响。目前,不少公共政治教科书仍然这样定义政治:

\quo{政治是以经济为基础的上层建筑,是经济的集中表现,是以政治权力为核心展开的各种社会活动和社会关系的总和。}

在这一定义之上,通常还有三个基本观点:一是政治的根源是经济,政治是经济的集中表现,政治关系归根到底是由经济关系决定的;二是政治的实质是阶级关系,在阶级社会中,阶级性是政治的基本特性;三是政治的核心是政治权力,所以国家政权是政治权力的根本问题,任何阶级要实现自己的目的,都必须掌握对国家或社会的最高统治权。这个大概是目前主流公共政治教科书对政治概念的解读。

\tsection{孔子与韩非政治观的分野}

要真正懂得古代中国对政治的理解,还需要深入到各家各派的学说中去。这里介绍两个最有影响的流派:儒家和法家。儒家更多地从伦理角度来理解政治,强调政治的伦理观或政治的道德观。政治本身就包含了端正、正直和正确的意思,这是孔子的重要立场。《礼记·哀公问》记载着这样一个故事:

\quo{公曰:“敢问何为政?”孔子对曰:“政者,正也。君为正,则百姓从政矣。君之所为,百姓之所从也。君所不为,百姓何从?”}

孔子认为,政治就是要端正。如果上头的君王端正了,百姓就能够端正。君王的所作所为就是百姓的榜样,君王不端正,百姓怎么能端正呢?这个观点,在今天看来仍然有重要意义。孔子的说法包含了一种对于为人君王——当然包括大臣们——在行为和伦理上的严格要求。《论语》中也有类似的说法。《论语·颜渊》中说:

\quo{季康子问政于孔子。孔子对曰:“政者,正也。子帅以正,孰敢不正?”}

在这段对话中,孔子反问季康子:“如果上头的人很端正,谁还敢不端正?”翻译成现代政治学语言,可以表述为:“政府是社会的道德榜样。”借助这种视角,大家可以理解目前中国社会的很多事情。孔子在《论语·为政》中还有这样的看法:

\quo{为政以德,譬如北辰,居其所而众星共之。}

意思是说,以道德方式来治理政治,就像北极星一样,位于中央,而其他星星都围绕在它的四周。这是孔子的一种比喻,说的是“为政以德”预期的效果。所以,孔子把从政视为道德上要求很高、伦理上要求很严的一种行为。他认为,只要道德与伦理上做到了端正,政治上达致理想境界就不是什么问题。儒家学说中还有“修身、齐家、治国、平天下”的说法,“修身”被放在首位。孔子追求的人生境界则是“内圣外王”。

那么,如何评价孔子的政治观呢?这个问题比较复杂,主要有两种观点。第一种观点认为,孔子过度地强调政治的道德性和伦理性,倡导道德约束,主张个人自律,这些是没有用的。如果没有制度和法治的手段,靠道德约束就无法实现预期中的治理状态。这种观点借鉴了启蒙运动以来政治思想与现代政治学的逻辑。

但是,也有人非常肯定孔子的学说。这是基于何种理论视角呢?有人认为,任何一个社会的治理,仅仅依靠制度约束是不行的,制度无法覆盖到所有领域,所以同时需要依赖于人们的德行与操守,或者叫道德情操。实际上,经济学的奠基者亚当·斯密既写了《国富论》,又写了《道德情操论》。在他看来,一个社会的繁荣不仅依赖于自利的动机,而且同样依赖于人们的美德。

如果追溯西方传统,从古希腊到古罗马,凡是带有民主共和色彩的社会,其繁荣或多或少都跟上层阶级的德行有关。如果一个社会的上层阶级没有好的德行,即便制度优良,这个制度很快就会被腐蚀掉。所以,即便是民主共和政体下的治理,既要有好的制度,又要有好的公民德行——尤其是这个社会的精英阶层应有良好的德行。从这样的视角去理解,孔子当年的政治伦理主张有其新的价值。

然而,法家的政治观与儒家完全不同,其代表人物是韩非。韩非完全不像孔子那样用比较理想的观点看待政治。韩非说:“国者,君之车也。”他把整个国家视为国君的一个工具,目的不过是为了实现君主的欲望与抱负。韩非还说:

\quo{上古兢于道德,中古逐于智谋,当今争于气力。}

这是韩非对春秋战国时期政治局势的一种事实判断,他认为当时政治格局完全是实力角逐,秉承的是“实力哲学”。他认为,靠道德来解决问题已经是很久远之前的事情了,而当时是一个靠实力说话的年代。韩非还给国君提出一条重要的管理原则,他说:

\quo{明主之所导制其臣者,二柄而已矣。二柄者,刑德也。何谓刑德?曰:杀戮之谓刑,庆赏之谓德。}

这里的原则已经接近现代工商组织的管理哲学,即组织管理要讲究赏罚分明。赏罚分明的组织就容易管理好,国家治理也不例外。

韩非认为,人都喜欢对自己有利的东西,都不喜欢对自己有害的东西,趋利避害是人的本性。所有的统治和政事都要围绕人趋利避害的本性。那么,他到底是怎样看待人性呢?他这样说:

\quo{医善吮人之伤,含人之血,非骨肉之亲也,利所加也。故舆人成舆,则欲人之富贵;匠人成棺,则欲人之夭死。人不贵,则舆不售;人不死,则棺不买,情非憎人也,利在人之死也。}

这段话在学术界没有受到应有的重视。韩非的观点很容易让人联想起亚当·斯密在《国富论》中对人的自利心的论述(本书第3讲将会探讨相关内容)。实际上,韩非早在两千多年前就有了与亚当·斯密相类似的视角,他在某种程度上已阐述了“经济人”假设和个人主义的方法论。

既然人都是自利的,那么政治应该怎么搞呢?在西方近代的自由主义传统中,经济人假设被提出以后,他们倾向于认为掌握政治权力的人可能由于自利而胡作非为,所以有必要对政治权力和掌权者进行约束。但是,韩非与西方传统走了一条完全不同的道路。身处两千多年前的春秋战国,韩非更多地从君王的角度来看待这个问题,阐述的是君主如何巧妙地利用法、术、势来进行有效统治的学说。基于这种人性论,韩非整体上是一个现实主义者,与孔子的道德政治观区别很大。就洞察力而言,韩非的政治现实主义不亚于公元16世纪意大利思想家马基雅维利的政治现实主义。

总之,孔子和韩非是中国春秋战国时期两位最重要的政治思想家,但他们走上了两条完全不同的道路。他们的分野代表了古代中国理解政治的两种主要取向。

\tsection{古希腊人如何理解政治?}

那么,政治在西方语境中有何含义呢?政治的概念通常被认为起源于古希腊。据考证,“政治”这个古希腊词的最早记载出现在《荷马史诗》中,最初含义是城堡或卫城的意思。在古希腊,雅典人将修建在山顶的卫城称为“阿克罗波里”(acropolis),简称为“波里”(polis)。在古希腊城邦国家形成以后,“波里”就成为城邦的代名词。

那么,古希腊人怎么理解政治呢?在古希腊人看来,政治是城邦公民参与的统治和管理活动。那时的政治概念,本身就包含公民参与城邦事务的意思。只有公民对城邦统治和管理事务的参与才能被称为政治。作为古希腊城邦国家的杰出代表,当时雅典城邦的民主政体具有如下基本特征:\nauthor{参见戴维·赫尔德:《民主的模式》,燕继荣等译,北京:中央编译出版社2004年版,第15—45页。

(1)公民大会是雅典的最高权力机构,为雅典全体公民的大会,法定人数为6000人,凡年满20岁的男性公民均可出席。

(2)另一重要机构是五百人议事会,五百人议事会负责组织、提出和执行公共决策。

(3)陪审法庭由201人至6000人组成,由陪审团对案件进行判决。

(4)行政官员由抽签和选举两种办法产生。

(5)还有一些辅助的政治机构和制度安排,包括最高法院、陶片放逐法、支薪制度等。}

从这些制度安排可以看出,雅典城邦实行的是直接民主制度。尽管如此,并不是所有的雅典人都能参与城邦公共事务。第一要排除的是雅典城邦中的外邦人,他们无权参与城邦的公共事务——当时有很多地中海周围地区的人在雅典做生意或做工;第二要排除奴隶群体;第三要排除女性和未成年人。所以,那个时候参与雅典城邦事务的是20岁以上的成年男性公民。雅典城邦较兴盛时大约有30—40万人口,但只有4—5万成年男性公民能够参与城邦公共事务。

从雅典城邦民主政治的实际运作来看,公民大会是最高权力机构。公民大会也被视为雅典民主的象征,其法定人数为6000人。但有人发现公民大会人数太多,很多事情无法进行有效协商。所以,雅典城邦又设计出了一个五百人议事会的制度。雅典城邦共有10个部落,每个部落派出50人组成五百人议事会。按照今天的说法,这五百人议事会可以理解为公民大会的一个常设委员会。除此以外,雅典城邦还设立了陪审法庭,人数要求是200人以上,可以是201—6000人之间的数字,其主要职责是对案件进行判决。雅典城邦行政官员的产生主要是抽签和直接选举两种办法。除了需要专业知识或特殊技能的少数官职由公民大会直接选举外,多数行政官员的职位对任职资格并无特别要求,成年男性公民都可以出任,所以一般由抽签方式产生。

除此以外,雅典民主还有一些辅助性的政治机构和制度安排。比如,一项非常有名的安排是陶片放逐法。什么是陶片放逐法?如果公民大会、五百人议事会或部分公民认为雅典城邦的某一个重要人物——由于他的财富、权势或影响力——可能会威胁到雅典城邦的现有治理方式、甚至民主政体的时候,就可以发起一场陶片放逐投票。如果公民的陶片投票达到一定数量,就可以把这个可能会威胁雅典民主的“危险人物”流放出去,放逐时间为10年(一说为5年)。对政治上可能的“危险人物”实行流放,是一种很有创意的做法。若干年以后,实际威胁消除了,他还可以回到雅典城邦。

雅典城邦还制定了给出席城邦公共事务的公民支薪的制度,这也是强化民主的一项安排。假定出席公民大会和审判法庭的公民没有薪水,大概只有很富有的人才能参与公共事务。在当时的雅典,多数人整天需要工作,就没有时间和闲暇来参与公共事务。所以,如果不是支薪制度,政治很可能会成为富人闲暇时间的一种爱好。即便对于英国的近现代民主制来说,议员领取薪水也是后来的事情。

所以,雅典城邦的民主政体是人类民主实践的早期雏形,但这种民主跟今天的民主政体有很大的不同。雅典民主是一种直接民主,而不是间接民主或代议制民主。这当然与雅典城邦的人口规模有关,直接民主通常只能在较小的国家规模上实行。尽管五百人议事会应该算代议制的制度安排,但总体上雅典城邦实行的是直接民主制。

在古希腊的传统中,亚里士多德认为,城邦不是指一片地方,而是指一批人,正是城邦将其中的人们联结成了一个共同体,因而有着强烈的共和主义色彩。亚里士多德这样说:

\quo{人类自然是趋向于城邦生活的动物(人在本性上是一个政治动物)。凡人由于本性或由于偶然而不归属于任何城邦的,他如果不是一个鄙夫,那就是一位超人。\nauthor{亚里士多德:《政治学》,吴寿彭译,北京:商务印书馆2007年版,第7页。}}

这一观点跟今天西方主流社会的政治观念有所不同。对今天的公民来说,他可以选择参与公共事务,又可以选择不参与。在自由主义或个人主义传统中,公民个人更多地被视为一个原子,是否介入公共事务则取决于公民的个人选择。但是,古希腊政治传统强调的是公民对城邦公共事务的参与。当然,近现代政治中仍然可以找到这种政治传统的事例。

比如,法国思想家托克维尔在19世纪上半叶的美国新英格兰地区就发现了类似的传统,他称之为“乡镇精神”。托克维尔注意到,当地普通公民对乡镇公共事务的参与程度是非常高的。那时候的美国乡镇,大概是两三千人的人口规模,一些重要事务由当地居民在广场上开会讨论和投票表决来决定。但是,这样做并不容易。世界上很多其他地方,普通公民不会这样热衷于公共事务。如果多数普通公民选择回避公共事务,真正的自治就难以实现。托克维尔这样描述新英格兰地区公民的乡镇精神:

\quo{在美国,乡镇不仅有自己的制度,而且有支持和鼓励这种制度的乡镇精神。……新英格兰的居民依恋他们的乡镇,因为乡镇是强大的和独立的;他们关心自己的乡镇,因为他们参加乡镇的管理;他们热爱自己的乡镇,因为他们不能不珍惜自己的命运。\nauthor{托克维尔:《论美国的民主》(上册),董国良译,北京:商务印书馆1989年版,第74、76页。}}

这种直接民主的公共治理方式要想有效运转,跟公民的政治参与密切相关。这一点无论对两千多年前的雅典城邦,还是对19世纪美国新英格兰地区的乡镇治理,都是一样的。

总的来说,在古希腊人看来,政治的概念包含着这样几个特性:第一,政治本身就是公民对城邦公共事务的参与。如果缺少了公民对城邦公共事务的参与,就不能称为政治。第二,公私领域的区分。雅典政治区分了今天意义上的公共领域与私人领域,这意味着已经存在私人空间和个人自由的概念。当然,这种制度和观念与近代启蒙运动以后的自由主义学说还是差异很大。比如,像苏格拉底的案子中,苏格拉底就是经由当时的民主方式来审判的。他被判处死刑,罪名是他毒害了雅典年轻人的思想。但无论怎样,古希腊人已经开始区分公共领域与私人领域的界限,这是非常了不起的观念。第三,政治的目的是追求公共之善。从柏拉图到亚里士多德,他们都关心如何塑造善的社会,政治的最终目的是要实现着这样一个社会。

在当代,有的作品也从古希腊传统来界定何谓政治。英国政治哲学家肯尼斯·米诺格对政治的解读也同古希腊传统非常接近。在他著的小册子《政治的历史与边界》中,第一章标题是“政治中为什么没有专制者的位置?”很多人看到这个标题,可能会大吃一惊。他这样说:

\quo{今天我们将专制主义(连同独裁和极权)定义为一种政体。这会使古希腊人大为惊骇,因为希腊人的独特(也是他们的民族优越感)恰恰在于他们不同于那些听任专制主义统治的东方邻居。\nauthor{米诺格:《政治的历史与边界》,龚人译,南京:译林出版社2008年版,第2—3页}}

这里的东方(邻居)意指波斯。当时,古希腊的思想家提到波斯时似乎都是较为鄙视的态度。他们认为,波斯帝国的政治结构是一个高高在上的君主加上一批辅助他的臣子与将军,其他人则都是君主的“政治奴隶”。既然政治意味着公民对城邦公共事务的参与,古希腊人就不认为波斯帝国的统治和治理也是一种“政治”。在古希腊人看来,专制肯定不是政治的一种类型,专制根本就不是“政治”——只有城邦公民共同参与公共事务的活动才配得上政治的称谓。

米诺格还认为,古希腊政治传统或欧洲文明的一个关键特征是对私人领域和公共领域的区分。他进一步说:

\quo{一个众所周知的线索就是当前私人生活与公共领域之间的界限。私人领域指的是家庭生活以及个人良知的领域——个人良知即个人凭自己的意愿选择信仰和兴趣。这种私人领域存在的先决条件是:具有统治权威的国家公共领域支持着一个维护公民自主关系的法制体系。具有统治权威的公共法律体系对自己的权力进行限定,唯有这样的条件下,政治才能存在。\nauthor{米诺格:《政治的历史与边界》,龚人译,南京:译林出版社2008年版,第5页。}}

有的读者或许会产生这样的疑问:米格诺的观点是否能反映目前国际主流学术界对政治的看法呢?其实,《政治科学新手册》也认为:“‘政治’以社会权力在约束条件下的行使为基本特征。”这意味着,如果没有约束条件,就没有政治,或者就不能称为政治。

\quo{我们认为,不受限制的权力是带有暴力性质的,并且是纯粹的和简单的。除了在一些退化的、极限的意义上事实可能如此之外,这完全不能算作政治权力的运作。纯粹的暴力更多是一种物理力量而不是政治。在我们看来,只有政治参与者行动的约束条件以及在这些约束条件下指导他们行动的策略,才构成政治的本质。\nauthor{罗伯特·古丁、汉斯-迪特尔·克林格曼主编:《政治科学新手册》(上册),钟开斌等译,北京:生活·读书·新知三联书店2006年版,第8页。}}

可见,这部由美国主流政治学者编写的政治学手册,就明确区分了政治权力的运作与纯粹暴力的行使,跟上文提到的“政治中没有专制者的位置”的观点如出一辙。

\tsection{西方的现实主义政治观}

有人或许发现,上文介绍的西方传统对政治的解读过于理想主义,下面就介绍一些现实主义政治观。“政治应该是什么”跟“政治实际上是什么”是两个不同的视角。当古希腊城邦民主的时代终结之后,再讨论古希腊人如何理解政治并心向往之,更多的是一种规范意义上的东西。在现实世界中,政治并不经常是人们所设想或期待的某种理想主义类型。政治很多时候是冷酷的,甚至还充斥着暴力或血腥。

讨论政治的现实主义视角,就离不开意大利思想家马基雅维利,其名著是《君主论》。《君主论》表述的政治哲学,后来被称为“马基雅维利主义”。马基雅维利主义是指一种政治上的现实主义,它把政治和道德剥离开了。在马基雅维利者的眼中,政治是无关道德的。当然,马基雅维利主义如今更容易被理解成一种为了达到目的而不择手段的处世哲学。

马基雅维利生活在中世纪晚期的佛罗伦萨,他的基本观点是:扩大君主权力是当时意大利寻求政治出路的正确方法。因此,君主如何行事就变得非常重要。政治应该去道德化的一个例子是,马基雅维利认为,对扩大君主权力来说,统治者的恶行有时可以是好事,统治者的善行有时也可以是坏事。他这样说:

\quo{如果没有那些恶行,就难以挽救自己的国家的话,那么他也不必要因为这些恶行的责备而感到不安,因为如果好好地考虑一下每一件事情,就会察觉某些事情看来好像是好事,可是如果君主照着办就会自取灭亡,而另外一些事情看来是恶行,可是如果照办了却会给他带来安全与福祉。}

这种观点就把道德和政治剥离开来了。马基雅维利还有很多著名的言论与观点,比如——

\quo{那些曾经建立丰功伟绩的君主们却不重视守信,而是懂得怎样运用诡计,使人们晕头转向,并且最终把那些恪守信义的人们征服了。

(对君主来说)究竟是被人爱戴比被人畏惧好一些呢?抑或是被人畏惧比被人爱戴好一些呢?……如果一个人对两者必须有所取舍,那么,被人畏惧比受人爱戴是安全得多的。

因此,你必须懂得,世界上有两种斗争方法:一种方法是运用法律,另一个方法是运用武力。

君主既然必须懂得善于运用野兽的方法,他就应当同时效法狐狸与狮子。……君主必须是一头狐狸以便认识陷阱,同时必须是一头狮子,以便使豺狼惊骇。\nauthor{尼科洛·马基雅维里:《君主论》,潘汉典译,北京:商务印书馆1996年版,第75、80、83—84页。}}

很多人读到这里,感觉在道德上不太能接受。马基雅维利的观点则非常清楚,他说有些做法看上去不好,但对国家和君主有利,统治者就应该这么去做。所以,《君主论》被称为政治现实主义的代表作。值得提醒的是,国内对马基雅维利存在很多误解,要想完整地理解马基雅维利,必须要读他《君主论》以外的其他著作。

另一位现实主义学者是德国著名思想家马克斯·韦伯,他流传甚广的名著是《新教伦理与资本主义精神》,韦伯还是当时德国最重要的“公共知识分子”之一。韦伯1895年就任弗莱堡大学经济学教授时,做了一个题为《民族国家与经济政策》的演说,严厉地批评了当时德国的政治状况。要理解韦伯演讲的内容,首先要理解1895年的德国。德国当时的政治格局跟它的地理位置与历史进程有关。德意志历史上是长期四分五裂的,要从分裂变成统一,武力就非常重要。另外,德国西有法国、东有俄国这样的强邻,在地缘政治上充满了不安全感。所以,从黑格尔的国家学说到希特勒的法西斯主义,都可以在这样一个特定的历史与地理情境中找到某种缘由。

在这篇演讲中,韦伯认为德国处在危险当中。在他看来,德国在政治上是一个不成熟的民族,德国应该从经济民族转为政治民族,德国应该追求政治权力。这里的政治权力更多表现为国家与国家较量中的一种权力,它是政治实力的另一种表述。

在讨论了德国的边境问题以后,韦伯这样说:

\quo{说到底,经济发展的过程同样是权力的斗争,因此经济政策必须为之服务的最终决定性利益乃是民族权力的利益。……在这种民族国家中,就像在其他民族国家中一样,经济政策的终极价值就是“国家理由”。……我们提出“国家理由”这一口号的目的只是要明确这一主张:在德国经济政策的一切问题上,包括国家是否多大程度上应当干预经济生活,要否以及何时开放国家的经济自由化并在经济发展过程中拆除关税保护,最终的决定性因素端视它们是否有利于我们全民族的经济和政治的权力利益,以及是否有利于我们民族的担纲者——德国民族国家。\nauthor{韦伯:《民族国家与经济政策》,甘阳等译,北京:生活·读书·新知三联书店1997年版,第93页。}}

在韦伯看来,民族生存是德国国家战略的核心问题。如果忽视这个问题,德国有可能面临覆灭的危险。大家会发现,韦伯是具有强烈焦虑感和现实主义视角的思想家,他不像古希腊人那样对政治充满美好的想象。作为独立国家的德国首先要生存下去,要在欧洲谋取生存和发展的空间,这是韦伯考虑的问题。

另一位德国现实主义学者是卡尔·施米特,他是德国20世纪著名宪法学家,但由于在第三帝国时期与希特勒政权过从甚密而备受争议。他在《政治的概念》中认为,政治的核心是“划分敌友”。施米特说:

\quo{所有政治活动和政治动机所能归结成的具体政治性划分便是朋友与敌人的划分。……

任何宗教、道德、经济、种族或其他领域的对立,当其尖锐到足以有效地把人类按照敌友划分成阵营时,便转化成了政治对立。\nauthor{卡尔·施米特:《政治的概念》,刘宗坤等译,上海:上海人民出版社2004年版,第106、117页。}}

在他看来,划分敌友问题才是政治的本质。基于这样的思考,施米特认为自由主义者对政治的理解是肤浅的,他本质上也是反自由主义的——他既不喜欢自由主义,也不认为自由主义是正确的。他这样说:

\quo{自由主义的系统理论几乎只关心国内反对国家权力的斗争。为了实现保护个人自由和私有财产的目的,自由主义提出了一套阻碍并限制国家和政府权力的方法。……由此,我们看到了一个完整的非军事化、非政治化的概念体系。\nauthor{卡尔·施米特:《政治的概念》,刘宗坤等译,上海:上海人民出版社2004年版,第151—152页。}}

这样,施米特认为,自由主义完全忽视了政治中划分敌友的问题。他认为,在处理德国内部事务问题上,魏玛共和国的自由派从1919年至1933年有差不多14年的时间,但自由派认为所有问题和冲突都可以通过自由协商讨论来解决。在施米特看来,这是自由派一种天真和幼稚的幻觉。

正是因为政治是划分敌友,关乎生死搏斗,所以政治斗争的方式并不总是和平竞争或自由协商,而完全可能是暴力角逐。施米特认为,人类社会总有一些事情最终无法用说服和沟通的方法来解决,而必须诉诸暴力手段。施米特在政治与暴力关系上的观点,跟他的基本政治观是一致的。如果政治是划分敌友,那么,当人群被划分为敌友之后,人们会采取什么手段来对待敌人呢?暴力就是一个自然而然的答案。\nauthor{关于马克斯·韦伯与卡尔·施米特作为政治现实主义者的重要性,笔者在很大程度上得益于北京大学李强教授的点拨。}

上述讨论简要梳理了西方语境中的政治概念,特别是介绍了理想主义和现实主义对政治的不同解读。作者也希望借此提醒读者,观察世界有不同的方法和路径。每个人应该自己去判断和选择,从多样化的视角去理解政治、理解自己的国家以及不同肤色的人生活于其中的世界。

\tsection{理解政治的当代观点}

那么,如今国际主流学术界怎样定义政治呢?现代政治学教科书关于政治的定义不下数十种。哈罗德·拉斯韦尔认为,政治是关于“谁得到什么?何时得到?如何得到?”他还认为,政治是关于权力的配置和分享。戴维·伊斯顿认为,政治是关于“价值的权威性分配”。罗伯特·达尔则认为,政治是“影响力的运用”。按照他的说法,政治的核心因素是权力,而权力某种程度上就是影响力。弗兰克·古德诺从区分政治与行政的角度来界定政治,认为政治是国家意志的表达,行政是国家意志的执行。上面列举的是几种比较流行的政治定义。

安德鲁·海伍德认为,政治是“人们制定、维系和修正其生活的一般规则的活动”,这里注重的是政治与一般规则之间的关系。海伍德还从四个角度理解政治,分别是:作为政府艺术的政治,即政治被理解为对国家事务的管理;作为公共事务的政治,即政治是人们对公共事务的参与;作为妥协和共识的政治,即政治被视为通过妥协、调解与谈判而非武力来解决冲突的方式;作为权力和资源分配的政治,即政治意味着对权力的争夺,甚至意味着压迫与征服。\nauthor{安德鲁·海伍德:《政治学》(第二版),张立鹏译,北京:中国人民大学出版社2006年版,第4—15页。}这是一种比较全面的解读。

杰弗里·托马斯在《政治哲学导论》中提出了一个多维度的界定政治概念的框架——政治的五因素模型(参见图1.2)。

\img{../Images/image00298.jpeg}[图1.2 政治概念的现代理解:五因素模型]

资料来源:杰弗里·托马斯:《政治哲学导论》,顾肃、刘雪梅译,北京:中国人民大学出版社2006年版,第11页,图1。

这一框架从五个维度来界定政治:

第一,政治以政治共同体的存在为前提。没有政治共同体存在的时候,要么现代意义上的政治尚未产生,要么政治就沦为暴力角逐。今天的政治共同体一般是指国家,而政治是发生在这个共同体内的事情。

第二,政治产生在公共领域。这意味着存在公共领域与私人领域的区分。上文业已提及,政治不是发生在私人领域中,而是发生在公共领域中。

第三,政治与政治共同体经常面临的公共政策选择有关。每个政治共同体都会面临很多公共政策选择,比如:提高养老金的比例还是降低养老金的比例?提高退休的年龄还是降低退休的年龄?政府应该借更多的债还是借更少的债?武力解决国际争端还是和平解决?所有这些问题都涉及公共政策的选择。面对不同的公共政策方案,还存在政策竞争的问题,这些都跟政治有关。

第四,非常重要的是,政治还跟集体决策的形式有关。一个政治共同体的集体决策形式是威权方式,还是民主方式?这就跟政体有关。此外,同一种政体内部也存在具体的集体决策形式的差异。比如,议会制和总统制就是两种差异较大的集体决策形式。集体决策形式的差异会产生不同的政治效应。

第五,政治离不开行政机构与强制力机构,行政机构主要是指一般的政府官僚系统,强制力机构主要是指军队与警察。这些机构关系到国家的实际执行力与强制力。所以,行政机构或官僚体系以及军队与警察等暴力机构也是政治的核心问题。

综上所述,政治可以被理解为发生在一个国家或政治共同体内部的公共领域、涉及采取何种集体决策形式来对公共政策做选择、并以官僚机构和军队警察作为强制力支撑的一系列活动。这大概就是今天对政治概念的解读。

\tsection{推荐阅读书目}

安德鲁·海伍德:《政治学》(第三版),张立鹏译,北京:中国人民大学出版社2013年版。

迈克尔·G·罗斯金等:《政治学与生活》(第12版),林震等译,北京:中国人民大学出版社2014年版。

罗德·黑格、马丁·哈罗普:《比较政府与政治导论》,张小劲等译,北京:中国人民大学出版社2007年版。
