\tchapter{意识形态大论战}

\quo[——约翰 · 斯图亚特 · 密尔]{第一,个人的行为只要仅仅涉及自身而不涉及其他任何人的利害,他就不必向社会承担责任。……第二,对于损害他人利益的行为,个人则需要承担责任,并且在社会认为需要用这种或者那种惩罚保护它自身时,个人还应当承受社会的或法律的惩罚。}

\quo[——弗里德里希 · 哈耶克]{一个人不受制于另一人或另一些人因专断意志而产生的强制的状态,亦常被称为 “个人” 自由或 “人身” 自由的状态。……自由政策的使命就必须是将强制或其恶果减至最小限度,纵使不能将其完全消灭。}

\quo[——埃德蒙 · 柏克]{我应该中止我对于法国的新的自由的祝贺,直到我获悉了它是怎样与政府结合在一起的,与公共力量、与军队的纪律性和服从、与一种有效的而分配良好的征税制度、与道德和宗教、与财产的稳定、与和平的秩序、与政治和社会的风尚结合在一起的。}

\quo[——埃里克 · 肖]{可以说,老工党对市场力量抱有深深的疑虑,它试图通过集中化的经济计划和大量的干预主义政策限制市场的力量……老工党笃信公有制的优越性,它试图以牺牲私营部门为代价稳定地扩大公有制的范围。老工党以工会是工人阶级的代表为理由,赞同在政府中保有工会的权力……最后,老工党倾向于放任国家的财政,常常会屈从于税收、支出和借款等 “快修” 方法的诱惑,而不是寻求更为适度和审慎的方法。}

\tsection{政治观点背后的意识形态}

不同的人信奉不同的意识形态,这是人类社会的普遍现象。这一讲的主题是现代政治意识形态的交锋。先请看两段话:

\quo{我们每天所需的食料和饮料,不是出自屠户、酿酒家或烙面师的恩惠,而是出于他们自利的打算。我们不说唤起他们利他心的话,而说唤起他们利己心的话。我们不说自己有需要,而说对他们有利。……

确实,他通常既不打算促进公共的利益,也不知道他自己是在什么程度上促进那种利益。……他受着一只看不见的手的指导,去尽力达到一个并非他本意要达到的目的。……他追求自己的利益,往往使他能比在真正出于本意的情况下更有效地促进社会的利益。\nauthor{亚当 · 斯密:《国民财富的性质和原因的研究》,郭大力、王亚南译,北京:商务印书馆 2005 年版,上册第 14 页、下册第 27 页。}}

这里主要是两个观点。一是对人性的基本判断:人都是自利的,人所有行为的动机都是出于他们自利的打算。但是,这里没有说 “自利” 是好的还是坏的,作者中立地看待人的自利行为。从字里行间可以看出,自利至少在客观上可能给他人和社会带来好处。二是阐述了 “看不见的手” 原理。作者把市场机制的力量比喻成 “一只看不见的手” ——自利的个人会受到这只 “看不见的手” 的引导 “有效地促进社会的利益” 。

这是英国著名经济学家亚当 · 斯密在《国富论》中最为著名的两段话。他信奉的意识形态被称为自由主义,有人喜欢称之为古典自由主义。

再请看两段话:

\quo{工人变成了机器的单纯的附属品,要求他做的只是极其简单、极其单调和极容易学会的操作。……他们不仅是资产阶级的、资产阶级国家的奴隶,他们每日每时都受机器、受监工、首先是受各个经营工厂的资产者本人的奴役。……

现代的资产阶级私有制是建立阶级对立上面、建立在一些人对另一些人的剥削上面的产品生产和占有的最后而又最完备的表现。……从这个意义上说,共产党人可以把自己的理论概括为一句话:消灭私有制。\nauthor{马克思、恩格斯:《共产党宣言》,北京:人民出版社 1997 年版,第 34、41 页。}}

这两段话显示作者身处工业革命突飞猛进的年代,出处是马克思和恩格斯 1848 年所著的《共产党宣言》。他们对当时欧洲的阶级关系、财产制度、生产制度、国家和统治都进行了批判。尽管马克思本人受到了亚当 · 斯密的很大影响,但两人对市场机制和产权制度的主要观点完全不同。马克思在劳资关系中发明了剩余价值的概念,提出了剥削学说。在 19 世纪的欧洲,这是富有 “革命性” 的见解。当然,按照这种观点,很多毕业生争相到世界 500 强公司工作,不过是在努力争取一种被 “剥削” 的资格。而亚当 · 斯密完全不这样看。斯密认为,用劳动力去换取报酬不过是一种正常的市场交易行为;而资方并非不劳而获或平白无故就能挣钱,利润被视为经营的回报。两种不同的观点,大家可以自己去比较和判断。

马克思和恩格斯的上述见解,在政治意识形态上被称为社会主义。但是,按照现在国际学界的观点,马克思和恩格斯的社会主义是一种特定类型的社会主义,即 “科学社会主义” 或 “共产主义” 。在社会主义意识形态的内部,19 世纪晚期与 20 世纪早期又兴起了后来影响很大的一个分支—— “民主社会主义” 或 “社会民主主义” 。

最后请再看几段话:

\quo{多亏了我们对变革的坚韧抗拒,多亏了我们冷峻持重的国民性,我们还保留着我们祖先的特征。我认为,我们并没有丢掉十四世纪思想的大度和尊严,也没有把我们自己变成野蛮人。……

我们不是抛弃我们所有的那些旧的成见,而是在很大程度上珍视他们;而且大言不惭地说,因为它们是成见,所以我们珍视它们;它们存在的时间越长,它们流行的范围越广,我们便越发珍视它们。……

一种绝对的民主制,就像是一种绝对的君主制一样,都不能算作是政府的合法形式。\nauthor{埃德蒙 · 柏克:《法国革命论》,何兆武等译,北京:商务印书馆 2010 年版,第 115、116、165 页。}}

这里的 “成见” 是指一个社会过去长期形成的看法与见解。 “成见” ,通常被理解为 “陈旧的见解” ,然而这位作者似乎把它当成传统智慧的一部分。他认为应该珍视一个社会中的成见,他甚至直截了当地把变革视为一件坏事,而把保守视为一件好事。这与今天的时尚正好相悖,大家如今通常把变革视为一个褒义词。作者还认为,如果抛弃了传统,就有可能 “把我们自己变成野蛮人” 。在最后一段,作者更把 “绝对的民主制” 视为亚里士多德意义上的暴民政体。这些文字摘自英国思想家埃德蒙 · 柏克的名著《法国革命论》,柏克被视为保守主义的代表人物。

借助这三则言论,大家可以对三种主要的现代政治意识形态——自由主义、社会主义和保守主义的概貌有一个了解。比如,假设某个欧洲国家有三个主要政党——一个自由主义政党、一个保守主义政党和一个社会主义政党,一个普通选民会给谁投票呢?这在很大程度上取决于这位选民的政治意识形态。从欧洲政治史来看,上述三种主要的意识形态在不同国家的不同时期均有过较大的影响。

在现实生活中,很多人在讨论公共问题时互不相让,甚至争得面红耳赤,一个主要原因就是大家在不同的政治意识形态里看问题。比如,第一个人在古典自由主义意识形态里看问题,第二个人在保守主义意识形态里看问题,第三个人则是在社会主义意识形态里看问题。这种情况下,他们多半不会就某个政治问题或某项公共政策达成共识。

大家不妨来看几则政治意识形态的判断题,只能回答是或否:

\tbl{../images/new01.png}

一个人的政治意识形态不同,对上述问题的选择当然也不同。大家可以给出自己的选择,然后思考支持这一观点的逻辑,再想清楚反对这一观点的理由。经过正反两面观点与逻辑的比较,你还会坚持最初的选择项吗?

\tsection{现代意识形态的兴起}

主要的现代意识形态均诞生于欧洲,所以意识形态的兴起需要在欧洲的历史背景中解读。从 14 世纪开始,一直延续到 16—17 世纪,欧洲经历了起源于意大利佛罗伦萨的文艺复兴。文艺复兴的政治社会意义,一方面是对古希腊和古罗马文献的重新发现,所以被称为 “复兴” ;另一方面是对人性看法的改变,整个思潮逐步趋向于重视 “以人为本” 的人道主义。后来,在 16—17 世纪,欧洲很多地区又经历了以德国为发源地的宗教改革。随着新教的兴起,过去普通教徒与教会、与神职人员的关系被改变了。每个普通人都有权阅读《圣经》,都可以直接跟上帝沟通和对话,而新教不再认为天主教会有权垄断诠释《圣经》与上帝旨意的权力。然后,18—19 世纪又发生了以英国和法国为中心的启蒙运动。中国历史教科书通常更重视 18 世纪法国的启蒙运动,特别是伏尔泰、卢梭和孟德斯鸠等人的学说。但实际上,差不多就在这个时候,英国还出现了苏格兰启蒙运动,包括像大卫 · 休谟、亚当 · 斯密等人的哲学与政治经济学思想都产生了重要影响。由于启蒙运动的兴起,欧洲人开始对人的理性有了重新的认识。德国哲学家康德说, “启蒙运动的口号” 就是 “要有勇气运用你的理智” 。正是这种对人性前所未有的尊重和对人类理性的重新发现,人本身的地位被抬高到过去从未有过的高度。而 1789 年的法国大革命更是喊出了 “自由、平等、博爱” 的政治口号。所有这些,都对整个社会产生了重要影响,也构成了现代意识形态兴起的思想背景。

除了文化、宗教与思想的潮流,同样重要的是,理解现代意识形态还要了解从文艺复兴到当代欧洲社会(后来也包括美国)主要的历史进程。当然,这里只能做粗略的勾勒,最重要的是三方面的重大变迁。第一,欧洲从政治上讲大概经历了从封建主义的衰落到民族国家的兴起——比如英法民族君主国的出现和德国统一等等——再到政治革命——比如英国宪政革命、法国大革命、美国独立战争及选举权普及等等——这样的历史过程。简单地说,封建主义是政治和经济上比较割裂的一种状态,从这种割裂状态过渡到民族国家,是欧洲历史上非常重要的里程碑。这个历史过程跟不少政治思想家有关,比如霍布斯关于利维坦的学说强调国家的必要性,博丹的主权论强调一个国家或君主拥有至高无上的权力,马基亚维利的君主论关注的也是欧洲近代国家建设过程中君主的作为,等等。

然后,从民族国家到政治革命的过程中,有大量思想家开始呼吁大众主权,呼吁更自由的政体,呼吁更平等的政治权利,呼吁对个人权利的尊重。所以,与政治革命相对应的是,政治思想领域也发生着重大变迁。比如,卢梭对人民主权的肯定,对共和制度的肯定,对公意和公益的肯定;孟德斯鸠对宪政体制的强调,对绝对君主制和专制主义的批判,对权力分立与制衡的倡导,都跟他们身处时代的政治变革联系在一起。

但是,正是由于法国大革命的出现,少部分人开始对那个时代的激进思潮进行了审慎的反思。比如,法国大革命以后,埃德蒙 · 柏克就对激进政治思潮进行了反思和批判。另外,尽管人民主权学说支持选举权的普及,但 19 世纪英国很多保守主义思想家非常警惕选举权的普及。他们担心,如果让普通公民掌握了选举权,将会对市场机制和财产制度产生毁灭性的影响。所以,上面讲到的两种思潮在政治上存在着激烈的交锋。

第二,欧洲社会结构出现了很多重要的变化。简单地说,在工业化、现代化和城市的变迁中,这些国家的阶级结构发生着重要的变迁。工业革命到来之前,欧洲大部分人都生活在农村,城市人口大概不足 10\% 或 15\% 。随着工业革命的进展,人开始脱离土地,进入工厂,涌入城市。当时英国和法国主要城市的人口由数万至十万这样的量级,快速膨胀到 50 万、甚至是 100 万以上。在这一过程中,人的社会身份也随之发生变化,有产者和无产者两大阶级集团开始形成。显然,两个集团都有自己的意识形态,资产者需要资产者的意识形态,无产者需要无产者的意识形态。所以,工业革命带来的是欧洲阶级结构的巨大变迁,进而塑造出互相对立的意识形态。

大概到 19 世纪后期,工业革命的继续发展推动了中产阶级的兴起。特别是 20 世纪中叶以后,欧美社会逐步进入所谓的后工业社会。美国管理学家彼得 · 德鲁克把这种社会形态称为知识社会,把规模日益增大的非体力劳动者称为知识工作者。这样,整个 20 世纪,中产阶级在社会人口中的比例逐渐增加。到了 20 世纪后半叶,如果要用阶级结构去划分整个社会的话,过去那种上层阶级(有产者)和下层阶级(无产者)——两阶级对立的模式开始逐步弱化,一种新的社会阶级结构正在兴起。

现在西方研究阶级的学者用几个不同的维度来划分整个社会的阶级结构:一个维度是是否掌握生产性资源的财产权,另一个维度是科层制结构中的权力高低,第三个维度是工作所需专业技能的高低。然后,根据上述几个标准区分出更为复杂的阶级结构图(参见图 3.1)。埃里克 · 赖特的相关研究显示,今天欧美主要国家中,从事简单劳动的非技术工人阶层,总体上占整个社会就业人口的比例已经降至 40\% 以下。在最发达的欧洲国家,能够被称为白领阶层或更高阶层的人口已经占到整个社会人口比例的 60\% 以上。这样,到了 20 世纪后半叶,过去(19 世纪)马克思所定义的两大阶级对立的社会结构在欧美社会基本上已经瓦解。

\img{../images/image00299.jpeg}[图 3.1 欧美国家的阶级结构分布]

资料来源:埃里克 · 奥林 · 赖特:《后工业社会中的阶级:阶级分析的比较研究》,陈心想等译,沈阳:辽宁教育出版社 2004 年版,第 49 页,图 2.1。

由此可见,在今天最发达的欧美国家,阶级因素在整个政治生活中已经弱化。历史地看,工业革命导致了有产者和无产者的尖锐对立。大概在 19 世纪中叶《共产党宣言》发表的时候,欧洲整个社会的阶级对立是非常严重的。但是,到了 19 世纪晚期德国民主社会主义者伯恩斯坦的时代,以及后来的 20 世纪上半叶,欧美社会就发生了一个重要的变迁。德国首相俾斯曼最早开始尝试社会保险制度,让很多普通人的生活变得更有保障了。以此为先导,欧洲福利国家逐步开始兴起。这个变迁的重要结果是欧美社会阶级对立的弱化。此外,在这一过程中,还有大量的养老金开始兴起,开始控制欧美社会大型上市公司股份的较大份额。比如,2013 年美国养老金总额已经突破 16 万亿美元,这一数据与美国 2013 年 GDP 相当,约为 2014 年初美国股票市场总市值的 80\% 。尽管养老金并非全部投资于股票市场,但是通常有相当比例的养老金是投资于股市的。这样,实际上,美国主要大公司的股票均有相当比例是为各种养老金持有的,而这些养老金最终是由相当分散的很多普通美国公民持有的。这些数据意味着,美国大型上市公司资本的社会化程度已经非常高,当然这是通过市场化方式来实现的。这是一个非常有趣的事实。所以,这些变化也影响着一个社会政治意识形态的走向。

第三,欧美社会经济的重要变迁。英国工业革命启动之后,西欧经济实现了相对较快的增长。但是,后来马克思意义上的经济危机随之而来。自由资本主义创造经济繁荣和技术进步的同时,也面临不定期的经济波动问题。自由资本主义造就的经济繁荣使其拥有大量的支持者,同时由于经济不稳定或经济危机创造了大量的反对者。那么,如何解释经济不稳定呢?

比如,何种原因导致了 1929—1933 年的世界经济大萧条呢?现在有两种完全对立的观点。一种观点受到马克思学说的左右,认为是自由资本主义和市场经济本身引发了世界经济大萧条,而经济危机的机制又是资本主义本质决定的。另一种观点则认为世界经济大萧条是美联储的不当做法导致的。本来,1929 年的美国只会面临一次小小的经济波动,但美联储的不当做法使得这一经济波动演变为有史以来最大规模的经济大萧条。\nauthor{米尔顿 · 弗里德曼、安娜 · 雅各布森 · 施瓦茨:《美国货币史》,巴曙松译,北京:北京大学出版社 2009 年版。}同样面对经济大萧条这一事实,基于不同意识形态立场的学者给出了不同解释,因此也意味着不同的政策主张。如果是前一种解释,政策主张是要求加强政府对市场的调控与干预,强化对大银行和大公司的市场及金融行为的监管,主张更多的政府和更少的市场。如果是后一种解释,政策主张是要求放松管制,尤其要避免错误的政府干预。在后一种观点看来,市场波动是不可避免的,但并不会酿成严重危机,正常的经济波动是市场经济的必要代价。所以,国家与市场的关系,也构成了意识形态冲突的重要来源。

此外,欧美社会也经历了贫富差距与经济不平等方面的重大变化。工业化过程中,欧美社会曾出现较为严重的贫富差距。后来,由于经济发展所带来的 “涓滴效应” ,福利国家的建设以及累进所得税等再分配政策的实施,欧美社会的贫富差距显著缩小了。到今天为止,跟中国等很多发展中国家相比,欧美主要国家的贫富差距不是更大,而是更小。经合组织主要国家的基尼系数显示,多数发达国家的贫富差距相对比较温和(参见表 3.1)。所以,现在不少到欧美社会工作和生活的中国人,都会惊讶于多数社会阶层收入的平均程度。贫富差距与不平等程度的减弱,总体上会缓和与阶级有关的意识形态冲突。

\tbl{../images/image00300.jpeg}[表 3.1 经合组织主要国家的基尼系数(2010 年)]

备注:带*的为 2009 年数据。

数据来源:参见经合组织(OECD)网站数据频道:http://stats.oecd.org/Index.aspx?Data Set Code=IDD\#。

上述因素构成了理解现代政治意识形态的背景。

意识形态的概念是法国哲学家德 · 特雷西(Destutt de Tracy)首先提出来的,最初的用法是指 “观念的科学” (idea-ology)。马克思在 1846 年的《德意志意识形态》中,把意识形态这个概念政治化了。马克思认为,意识形态就是统治阶级的思想体系,因而具有比较强的欺骗性,其目的是为了让被统治阶级主动服从统治阶级的统治。所以,在马克思那里,意识形态有着强烈的政治色彩。\nauthor{马克思、恩格斯:《马克思恩格斯选集》第一卷,北京:人民出版社 1995 年版,第 62—135 页。}而现在的政治学研究中,通常把意识形态作为一个中性的词汇来处理,但是毫无疑问它仍然跟政治有非常密切的关系。

从概念上说,意识形态是现代社会科学体系中的一个概念和范畴,意指 “一个行动导向的信念体系,一套以某种方式指导或激励政治行动的相互联系的思想观念” 。\nauthor{安德鲁 · 海伍德:《政治学》(第二版),张立鹏译,北京:中国人民大学出版社 2006 年版,第 51—53 页。}通常来说,政治意识形态有几个主要特征:

第一,意识形态需要解释世界。为什么世界是这个样子而不是那个样子,意识形态需要给出一整套的理论解释。当意识形态给出的解释具有说服力时,这种意识形态更能产生影响。比如,自由主义对世界的解释,社会主义对世界的解释,保守主义对世界的解释,三者很不一样。世界的现状为什么是这样的?不同的意识形态有不同的解释。

第二,如同马克思所说的—— “重要的不是解释世界而是改造世界” ,意识形态通常具有改造世界的企图心。多数意识形态都支持这样的信念:现状不够完美,现状能够变得更好。所以,很多意识形态倾向于承认存在某种社会改造计划的可能性。

第三,意识形态通常还带有行动导向的色彩。特定的意识形态通常会鼓励某些特定的行动或行为,有时这种特定行动和行为可能发展为某种运动、政党甚至是革命。与一般的理论不同的是,意识形态具有强烈的行动导向。所以,意识形态不只是停留在书斋中的理论,而是有着强烈的行动导向。

第四,意识形态的另一个特点是群众取向。很多理论学说非常高深,普通民众不太容易理解。但是,意识形态——如果要成为成功的意识形态——最终一定要简化到普通民众能够理解的语言,甚至最后简化为几个口号。意识形态只有能够用非常简单的语言来表述,才具有对普通大众的政治动员能力。这是政治意识形态不同于一般理论学说的重要特征。比如,社会主义可以简化成几个简单口号,自由主义在历史上也曾简化为几个简单口号。

第五,值得提醒的是,某种意识形态和这种意识形体的实践往往是两回事。比如,一个国家据说要搞某某主义,而该国最终搞的是不是某某主义,不一定有必然的联系。所以,意识形态的口号与号称在某种意识形态指导下的政治实践是两回事,这个问题往往具有较大的迷惑性。

\tsection{什么是自由主义?}

对政治意识形态的介绍一般从自由主义开始,原因在于自由主义在流行的意识形态中是最早出现的一种,而后来的意识形态通常是站在批评、反对或修正自由主义的立场上逐渐成形的。自由主义被认为是资本主义和工业革命时期的意识形态,大概在 18 世纪晚期和 19 世纪早期趋于成熟。自由主义的主要观点可以被归纳为一项政治主张和一项经济主张。

政治上,自由主义反对两样东西:一是欧洲民族国家兴起过程中尚存的封建主义和封建制,二是绝对君主制和专制主义。自由主义主张立宪政府与个人自由。自由主义的基本政治主张是要有一个有限政府或立宪政府,政府的权力、行为和边界要受到强有力的约束。政府不能想干什么就干什么,其行为应该有明确限定。同时,社会中的普通公民和法人的权利应该享有明确的保护,特别是要防止他们受到政治权力恣意妄为的侵害。

经济上,自由主义反对重商主义。重商主义曾经是非常流行的政策主张,重商主义今天已经演变为经济民族主义或贸易保护主义。自由主义在经济上主张自由放任,它赞同亚当 · 斯密所阐述的 “看不见的手” 原理。自由主义认为,政府要尽可能少干预市场,应该由 “看不见的手” 来协调市场主体的行为,而这是最有效率的一种方式。在自由主义框架内,国家的合理角色被界定为 “守夜人” 。换句话说,一个理想国家或政府的主要角色应该是制止犯罪、防止非法的暴力及提供基本的秩序。除此之外,国家最好不要干别的事情。比如,国家要不要提供公立教育呢?最好不要。国家有没有义务提供就业岗位呢?没有这样的义务。国家应不应该提供各种福利?不应该提供福利。这些事情,国家最好都不要管。在自由主义者眼中,理想国家主要角色是警察和法官。

借鉴安德鲁 · 海伍德的论述,笔者把古典自由主义总结为对若干重要原则的倡导:\nauthor{关于自由主义的要点总结,参见安德鲁 · 海伍德:《政治学》(第二版),张立鹏译,北京:中国人民大学出版社 2006 年版,第 54—56 页。}

一是个人主义原则。 “个人主义” 过去被视为一个贬义词,但在社会科学中它是一个中性词。个人主义的概念到 19 世纪早期才逐渐形成。在这种哲学中,个人被剥离了其他身份,而成为一个抽象的人类个体。这个抽象的人类个体本身的价值和权利就是至高无上的,这些权利无须依赖基本公民身份以外的其他任何身份,这些权利不能被国家或任何集体剥夺。

因此,这里的个人主义并非日常话语中理解的不守纪律、没有组织观念或缺少团队精神,更非损人利己。真正的个人主义至少应该包括两层含义:首先,个人要优先于群体。这里的优先,并不是说只要个人、不要群体,而是思考问题的一种特定视角。比如,对于一种政策,不要首先说这对整个社会有利,而要问是否改善或促进了个人福利。在个人主义看来,抽象的社会是看不见的,社会是由一个个不同的个人构成的。只有这种政策相当程度上能实现了个人福利时,社会福利才能得以实现。所以,个人主义主张个人优先于群体,既是一种认知问题的视角,又是在价值上倡导个人的优先性。

其次,把个人因素视为整个社会创造动力的来源。一个理想的社会应该是什么样的?一种视角认为,一个理想的社会要给每个人提供有保障的就业,让每个人都过上一种有保障的生活。但是,另一种视角却认为,一个理想的社会应该激发每个人的主动性与创造力,让每个人的激励因素与个性特征成为促进社会进步的动力。只有这样,社会才会繁荣。个人主义更支持后一种观点。比如,这一代美国政治家思考的一个问题是:如何让美国产生更多的史蒂夫 · 乔布斯?乔布斯是苹果公司的创始人和缔造苹果公司辉煌的企业家。这就是说:如何让美国产生更多的优秀企业家?所以,个人主义把个人视为社会繁荣与创新的源头。

二是个人自由原则。从个人主义原则出发,可以推导出对个人自由的倡导。政治哲学意义上的自由,既非通常被理解的完全没有约束,亦非心灵或意志层面的自由,而是指政治自由或社会自由。按照密尔的说法,界定这种自由的是 “社会可以合法地施加于个人的权力之性质和界限” 。倡导个人自由原则,意味着主张来自国家或社会的强制力干预越少越好。实际上,古典自由主义会认为,国家或社会可以干预个人行为的惟一正当理由是为了保护其他人同等的自由与权利。

三是理性原则。德国哲学家康德说: “要有勇气运用你自己的理智!这就是启蒙运动的口号。” 这里的理智与理性是一个意思。自由主义相信人的理性的力量,所以在历史维度上趋近于进步史观。自由主义相信随着人理性力量的发展,对自然和社会的科学知识的进一步发掘,这个社会能够不断地获得进步和加以改善。由于这种进步史观,自由主义倾向于支持变革。在这个维度上,自由主义与保守主义就发生了分野。自由主义相信人理性的力量,信奉进步史观,而保守主义并非如此。

四是平等原则。自由主义主张的平等更多的是指形式的平等和机会的平等,而非实质的平等和结果的平等。自由主义主张每个人都有同样的权利,但并不认为同样的权利就会得到同样的结果。自由主义承认每个人的天赋存在差异,认识到每个人付出的努力是不一样的,而且还存在运气等偶然因素的差异。总的来说,这些差异是无法抹杀的,每个人最后得到的结果自然是不同的。所以,自由主义的平等更多是强调机会、形式、程序和资格方面的平等,或者说是法律面前人人平等。再进一步说,自由主义者甚至认为要追求实质的平等和结果的平等可能是危险的,甚至最终会走向自由的反面。

五是宽容原则。自由主义相信,每个人的个性与自由非常重要,所以应该倡导宽容原则。社会要允许、容纳、甚至是鼓励多样性,宽容各种不同的选择和各种可能的见解。从功利的角度看,这些不同选择和不同见解可能会对社会产生重大的裨益。比如,创新的重要特征是其不确定性。由于不能确定创新来自于何处,一个允许和鼓励多样性的社会要比另外一个禁止或遏制多样性的社会更有可能出现创新。从权利上说,每个公民都有选择的自由,所以,社会应该对这些不同的选择保持尊重和宽容。

六是被治理者同意的原则。既然个人权利优先于群体,国家权力来自个人,因此自由主义认为,所有统治都应该基于被治理者的同意。当然,这条原则的实践形式经历了复杂的演变。1258 年英格兰贵族迫使国王签订的《牛津条约》规定,国王征税等重大事务需要经过贵族会议的同意。这大概是统治要基于被治理者同意的最早形式。英格兰贵族并不认为国王职位应该由大家来选或由贵族轮流来担任,但是国王征税却必须要经过他们的同意。而到了现代社会,统治要基于被治理者同意的原则往往跟某种程度的民主形式联系在一起。上文曾提到,自由主义未必主张民主政治。但如果说自由主义可能支持民主的话,很大程度上是因为统治要居于被治理者同意的这一原则。在现代社会,怎样落实统治要基于被治理者同意的原则呢?实际可操作性的办法就是让全体或多数公民以民主投票方式产生政府。需要说明的是,尽管如此,自由主义非常警惕普通民众是否会以多数投票方式侵犯自由权利,是否会导致托克维尔和密尔所警惕的 “多数暴政” 问题。所以,自由主义通常认为,民主政治同时需要宪政原则的约束。

七是宪政原则。自由主义认为,政府权力应该受到明确的限制,这就是最简单意义上的宪政原则。据说,英国有这样一个小故事。在一个风雨交加的夜晚,有一支英格兰国王的军队途经某个农庄。士兵们非常渴望能进到这个农庄,喝口热水和享受片刻温暖。军队的先锋官就去询问农庄的主人,是否同意国王的军队进去休息一下?结果,农庄主人说出了一句后来广为流传的话: “风能进,雨能进,国王不能进。” 这句极具煽情色彩的话,乃是强调了对公民权利——特别是私人财产权利——确定无疑的尊重和保护,对应的则是对国王或政府权力明确的限定。

八是自由放任原则。古典自由主义主张自由放任,无论是经济政策上还是社会政策上。诚如杰斐逊所言—— “管得越少的政府越是好的政府” 。上文提到的亚当 · 斯密的 “看不见的手” 原理,也是强调自由放任原则。所以,古典自由主义者眼中,守夜人国家就是理想的国家类型。

\tsection{古典自由主义的大师们}

英国哲学家约翰 · 洛克是古典自由主义的早期代表人物。他认为,基于自然法和契约论,政府的权力应该来自于被统治者的共同认可或者一致同意,设立政府的唯一目的就是保护国民的生命权、自由权和财产权。洛克的学说某种程度上为有限政治和立宪主义奠定了理论基础。洛克说:

\quo{无论是绝对任意的权力,抑或是没有稳固不变的法律的统治,都是与社会和政府的目的相左。如果社会和政府不去保护人们的生命、自由和财产,如果没有关于权利和财产的一贯有效的规定来保障他们的和平与安宁,人们就不会舍弃自然状态的自由而加入社会和甘受它的约束。……

我们的论证显然是有理的,人类天生是自由的,历史的实例又证明世界凡是在和平中创建的政府,都以上述基础为开端,并基于人民的同意而建立的。……政治社会的创始是以那些要加入和建立一个社会的个人的同意为依据的;当他们这样组成一个整体时,他们可以建立他们认为合适的政府形式。\nauthor{洛克:《政府论》(下篇),叶启芳、瞿菊农译,北京:商务印书馆 1996 年版,第 85、64—65 页。此处译文根据英文原著略有调整。}}

洛克在上述两段文字分别阐述了社会的目的是为了保护人们的生命权、自由权和财产权,以及统治应该基于被治理者的同意。实际上,后世很多人受到洛克学说的影响。比如,下面这份重要文件这样说:

\quo{我们认为下面这些真理是不言而喻的:人人生而平等,造物者赋予他们若干不可剥夺的权利,其中包括生命权、自由权和追求幸福的权利。为了保障这些权利,人类才在他们之间建立政府,而政府之正当权力,是经被治理者的同意而产生的。}

这是 1776 年北美 13 个殖民地发布的《独立宣言》的开头内容。如果熟悉洛克的言论,就会清晰地看到《独立宣言》开篇即充满了洛克思想的影子。所以,从《独立宣言》可以看出,美国建国之初的政治意识形态最接近于自由主义,或者说美国就是一个以自由主义立国的国家。此后,1787 年《美国宪法》开篇则这样写道:

\quo{我们合众国人民,为了建立一个更完善的联邦,树立正义,确保国内安宁,完备共同防御,增进公共福利,并保证我们自身和子孙后代永享自由的幸福,特制定美利坚合众国宪法。}

古典自由主义的另一位杰出代表人物是上文提及的亚当 · 斯密。亚当 · 斯密总体上认为,人是自利的,其追求自利的过程客观上会在 “看不见的手” 的引导下促进社会的利益,所以他主张自由放任政策与守夜人国家。这些内容本讲开篇即做介绍,不再赘述。

有人认为,既然亚当 · 斯密认为应该自由放任,所以国家并不重要。但是,这是对斯密观点的极大误解,也是对自由主义的极大误解。亚当 · 斯密在《国富论》中用较大篇幅专门探讨了 “君主或国家的义务” 。尽管主张自由放任,但他认为君主应该履行三项主要职责,分别是:

\quo{君主的义务,首在保护本国社会的安全,使之不受其他独立社会的暴行与侵略。……君主的第二个义务,为保护人民不使社会中任何人受其他人的欺侮或压迫,换言之,就是建立一个严正的司法行政机构。……君主或国家的第三种义务就是建立并维持某些公共机关和公共工程。\nauthor{亚当 · 斯密:《国民财富的性质和原因的研究》(下册),郭大力、王亚南译,北京:商务印书馆 2005 年版,第 254、272、284 页。}}

由此可见,斯密从来没有主张政府无足轻重;相反,政府很重要。当然,斯密主张政府的职能与行为应该严格限制在 “三项义务” 的范围之内。

约翰 · 斯图亚特 · 密尔是另一位杰出的自由主义思想家,他是 19 世纪英国最重要的政治学家、经济学家和哲学家之一,他的小册子《论自由》被视为自由主义的名篇。他把《论自由》的主题与目标设定为:

\quo{本文的主题不是所谓的 “意志之自由” ……而是公民自由或社会自由,即社会可以合法地施加于个人的权力之性质和界限。……

本文旨在确立一条极简原则……人类可以个别地或集体地对任何成员的行动自由进行干涉,其唯一正当理由旨在自我保护。\nauthor{密尔:《论自由》,顾肃译,南京:译林出版社 2010 年版,第 3、11 页。}}

密尔思考的起点是假定每个人的自由都是绝对的。但如果每个人都拥有不受限制的 “绝对自由” ,当大家行使这种 “绝对自由” 时,彼此可能会发生互相侵害和冲突。所以,一个社会必须要给这种自由以一定的限制,但密尔讲得很清楚,此种限制的惟一目的是防止一个人去侵害他人。就是说,限制个人自由的惟一理由是给他人以合理的保护。这条原则也意味着,政府不能仅仅因为某事对某人自身有利或不利而强制个人从事或不从事某种行为。政府不能因为 “父爱主义” 而来干涉公民的行为、选择与自由,因为政府没有或不应该拥有这个权力。政府惟一有权对个人行为加以干涉,是由于这个人的行为会对他人造成侵害。密尔继续说:

\quo{对于文明群体中的任何一个成员,可以违反其意志而正当地行使权力的唯一目的,就是防止对他人的伤害。至于这个人自己的好处,无论是物质上的,还是精神上的,都不是充足的正当理由。\nauthor{密尔:《论自由》,顾肃译,南京:译林出版社 2010 年版,第 11 页。}}

按照密尔的看法,国家和社会无权为了你的福利而干涉你的自由,这是进一步的推导。密尔接着说:

\quo{只有涉及他人的那部分行为,才是任何人应该对社会负责的行为。从正当性上说,在仅涉及他自己的那部分行为上,他的独立性是绝对的。对于他自己,对于他的身体和心智,个人是最高主权者。\nauthor{密尔:《论自由》,顾肃译,南京:译林出版社 2010 年版,第 11—12 页。}}

密尔在《论自由》中对言论自由和思想自由的论证,不仅强调权利的视角,也借用了功利主义的方法。为什么要有言论自由呢?理由是:

\quo{第一点,就我们所能够确切知道的而言,如果任何观点被迫保持沉默,则该观点有可能是正确的。……第二点,即使被迫沉默的观点本身是错误的,它也可能,而且通常总是包含着部分真理。……第三点,即使公认的观点不仅是真理,而且是全部真理,但是,如果不允许它接受并且实际接受强有力的、认真的争论,那么,它的大多数接受者就会像持有一个偏见那样持有它,也很少理解或感知它的理性依据。……第四点,该学说本身的意义也会有丧失或减弱,失去其对品行和行为关键影响力的危险。\nauthor{密尔:《论自由》,顾肃译,南京:译林出版社 2010 年版,第 56 页。}}

此外,密尔还强调 “个性的自由发展是福祉的首要因素之一” 。比如,他认为 “天下没有一件事不是由某个人首先做出来的,现存的一切美好事物都是首创性的果实” ,所以自由的重要性不言而喻。在做总结时,密尔认为主要就是两条格言:

\quo{第一,个人的行为只要仅仅涉及自身而不涉及其他任何人的利害,他就不必向社会承担责任。……第二,对于损害他人利益的行为,个人则需要承担责任,并且在社会认为需要用这种或者那种惩罚保护它自身时,个人还应当承受社会的或法律的惩罚。\nauthor{密尔:《论自由》,顾肃译,南京:译林出版社 2010 年版,第 99 页。}}

\tsection{自由主义的演进与嬗变}

后来,自由主义遇到了很多严重的挑战。这种挑战首先来自于自由市场经济本身。间歇性的经济危机、贫富悬殊以及托拉斯和大企业垄断等问题,都是古典自由主义者未曾预见或难以有效回应的。此外,工业化和城市化的快速发展还产生了对大规模公共服务的需求。比如,19 世纪早期的巴黎一度是一个臭气熏天的城市,主要原因在于大量人口的涌入、公共设施及基本公共服务的匮乏。如果以自由主义作为立国原则,当时的政治家和政府既不认为自己有权利、亦不认为自己有义务来应对城市大规模扩张所带来的实际问题。要知道,政府在历史上没有干过也不需要干这种事情。但客观上,大规模城市化对政府产生了极大的公共服务需求。此外,欧洲发达国家 19 世纪以来的重要政治趋势是选举权的陆续普及。与选举权普及同步发生的是政党政治和政党竞争的兴起。可以料想,不同政党会主张不同的政治纲领与社会政策。由于选民人口中低收入阶层占有较高比例,福利国家政策的兴起就成了欧洲社会的大趋势。

在这种背景下,陆续兴起的其他不同意识形态开始构成对古典自由主义的挑战,特别是保守主义和社会主义。一般认为,大概在 1870 年前后,古典自由主义就开始趋向衰落。洛克、斯密和密尔等人所认同的那套价值理念和学说体系日渐式微。后来,自由主义意识形态内部兴起了现代自由主义。现代自由主义既是对古典自由主义的继承,又是对古典自由主义的改造。

现代自由主义需要应对原先的自由资本主义和自由市场经济所带来的诸种问题,需要有效回应工业革命和现代化所带来的种种挑战,所以对古典自由主义进行了大幅修正。一个最主要的变化,就是 “把政府找回来” 。现代自由主义认为,原先高度自由放任的制度安排是不恰当的,应该让政府发挥更积极的作用。因此,现代自由主义有两个基本特点:一方面,是尊重过去基本的制度安排,包括尊重私人财产权和市场经济,尊重宪政、法治与公民的自由权利;另一方面,在此基础上,主张强化政府干预——包括对市场、行业和企业的管制,加强对劳工阶层的保护以及建设福利社会。

现在多数人认为,福利国家是个褒义词,或者说福利国家才是好国家。但是,古典自由主义并不赞同这种看法。为什么呢?因为政府权力太大了,政府做的事情太多了,政府占有的资源比例也太高了。政府既有可能促进了一般意义上的社会福利,但同时却可能侵害了个人自由和选择。所以,关于政府的恰当角色,在自由主义内部存在着剧烈的争论。但从社会实践来看,强化政府干预的大趋势基本上是难以阻挡的。大概从 1870 年左右开始,到整个 20 世纪的多数时间里,政府规模和政府干预都经历了大幅扩张。

1900 年西方主要国家的财政收入占 GDP 的比例普遍低于 10\% ,但今天这些国家的比例大概是 30\% 到 50\% 。美国的比例略低,大概是 30\% 左右,但一些北欧国家的比例超过了 50\% 。这意味着,对今天欧美发达国家来说,整个经济活动中政府的比重已经超过三分之一,有的甚至已超过二分之一。所以,如果亚当 · 斯密健在的话,面对这一切,估计他是难以接受的。但是,这个趋势至少到目前来看,仍然是不可逆转的。

现代自由主义的代表人物很多,当代世界很多主流的经济学家和政治学家都持有现代自由主义的基本观点。从历史上看,美国总统富兰克林 · 罗斯福可以被视为现代自由主义的早期践行者。1929—1933 年世界大萧条时,整个美国经济遇到了严重危机。面对这种前所未有的情况,怎么办?罗斯福总统采取一系列的 “救市” 行为,主要是经济救济、改革、复兴计划(所谓 3R 计划),后世称为 “罗斯福新政” 。比如,为了缓解失业带来的压力,他搞了很多救济和公共工程,前者是直接解决苦难,后者则有助于提高就业率;为了克服金融困难,他对银行业进行了整顿,又成立了存款保险公司,以降低挤兑和其他金融风险;他还努力恢复了工业和农业生产,保护劳工权利,制定最低工资法及规定最高工时,并强化社会保障措施和进行福利国家建设。

当然,与欧洲大陆的发达国家比,今天的美国总体上仍然是福利更少而自由更多。这也与美国人强调每个人都应该成为自力成功者、而不是靠社会或国家的力量来保障个人生活的理念有关。所以,美国很晚才陆陆续续出现欧洲大陆早已流行的很多福利政策。到今天为止,美国的社会福利仍然低于欧洲大陆发达国家。

现代自由主义在经济学界的重要代表人物是被誉为 20 世纪最著名经济学家的约翰 · 梅纳德 · 凯恩斯,他于 1936 年出版了名著《就业、利息和货币通论》。\nauthor{约翰 · 梅纳德 · 凯恩斯:《就业、利息和货币通论》,高鸿业译,北京:商务印书馆 2004 年版。}凯恩斯认为,如果整个市场自由运转,采用自由放任政策,就会出现有效需求不足的问题,所以政府干预就成为必要。与古典自由主义不同,凯恩斯倾向于认为政府对经济增长和稳定负有责任。后来,经济增长率、就业率和通货膨胀率开始逐步成为欧美发达国家政府的施政目标。政府需要通过财政政策和货币政策的灵活运用,来保证适度的经济增长率、低失业率和低通货膨胀率。

当然,凯恩斯学说的完整内容要比上述介绍复杂得多。过分简化的归纳容易扭曲一个重要思想家的观点。据说凯恩斯在世时, “凯恩斯主义” 一词已开始流行,但凯恩斯则公开说: “我不是一个凯恩斯主义者。” 国内经济学界和公共政策领域对凯恩斯的误解则更多。比如,某些官员借鉴凯恩斯主义理论提出政府要加强宏观调控,但很多政策其实都是微观调控。凯恩斯的宏观调控主要是指财政政策与货币政策,所以不应该把所有的东西都往宏观调控这个筐里装。

20 世纪后半叶重要的现代自由主义政治哲学家约翰 · 罗尔斯,在其名著、出版于 1971 年的《正义论》一书开篇就说: “正义是社会制度的首要德性,正像真理是思想体系的首要德性一样。” 罗尔斯在这部著作中阐述了两条正义原则:

\quo{第一个正义原则:每个人都拥有平等的权利来享有最广泛的、与他人类似的自由相容的基本自由。

第二个正义原则:社会和经济的不平等应该:(1)有利于受惠最少者的最大利益;(2)职位和机会应该向所有人开放。\nauthor{约翰 · 罗尔斯:《正义论》,何怀宏、何包钢、廖申白译,北京:中国社会科学出版社 2009 年版,第 47 页。此处对照英文原著对译文做了调整,参见 John Rawls, \italic{A Theory of Justice}, Cambridge: Harvard University Press, 1999, p.53。}}

罗尔斯的第一条正义原则可以称之为自由原则。简单地说,一个人的自由在不侵犯他人同等自由的条件下应该尽可能的大。第一条原则似乎与古典自由主义、与密尔在《论自由》中的主要观点没有区别。但是,罗尔斯的正义论还要兼顾第二条原则,可以称为平等原则。简单地说,罗尔斯强调的平等原则是要对社会中的弱势群体较为有利。这样的话,政府干预就成为必要,政府干预也成为塑造正义社会的一部分。所以,罗尔斯一方面强调每一个人都应该享有最广泛的自由,另一方面强调还应该采取辅助性手段来改善弱势群体的处境。他的学说实际上是两者的结合,这也就是现代自由主义的基本特征。

现代自由主义涉及很多实际而重大的政策争论。比如,最近的挑战就是如何应对 2008 年的美国金融危机。其中的一个争论是,美国政府是否应该去救助那些在金融危机中濒临倒闭的大企业?——特别是房利美、房地美这样的房地产金融机构,雷曼兄弟公司这样的大型投资银行,以及通用汽车这样的大型实业公司。当时,美国最大汽车公司通用汽车面临破产倒闭的危险。那么,美国政府是否应该为通用汽车提供经济援助呢?不同的政治家与学者政策立场迥异。身为民主党人的奥巴马总统认为 “该出手时就出手” ,认为需要拯救通用汽车。共和党保守派的主要看法是政府不应该出面去拯救一家公司,美国很多经济学家也持有这种见解。那么,为什么有人支持政府出面拯救通用汽车而有人反对呢?

反对的理由来自三个方面:第一,直接援助某个企业的做法破坏了市场经济规则。有人经营不成功,就应该让它垮掉,这是一种优胜劣汰的机制。只有这样,那些原本被浪费或低效使用的经济资源才可以配置到更高效的部门中去。第二,政府拿谁的钱去拯救这些企业?纳税人的钱。问题就来了,政府凭什么把纳税人的钱花在一个特定的企业上呢?第三,政府出面拯救企业,无形中弱化了企业的主要股东和主要经营者的责任。这也违背市场经济的基本原则。

支持的理由更多是基于社会公益的视角。如果任由通用汽车彻底破产而不进行重组,会带来什么社会问题呢?首先,通用汽车本身就是一个大企业,这样会导致很多人失业。此外,通用汽车还有很多关联协作企业,特别是为数众多的主要供应商和主要经销商,他们都有可能面临经营困难,甚至是破产。这会通过连锁反应触发更严重的失业问题。基于此种考虑,政府应该出手援助。奥巴马政府最后的政策选择是给通用汽车提供经济援助,但这种援助并非无偿。但时至今日,这种做法仍然充满争议。

\tsection{新古典自由主义的复兴}

从 20 世纪 30 年代到 60 年代,是欧美发达国家现代自由主义的鼎盛时期。但是,到了 20 世纪 70 年代左右,凯恩斯主义政策开始引发很多问题,经济上主要表现是滞胀现象,即经济停滞与通货膨胀的并存;社会方面表现为福利政策导致的很多社会病。这样,实行多年的凯恩斯主义经济政策与福利国家方案遭到了大量质疑。另一个背景是,在二战之后的冷战条件下,资本主义和社会主义两大阵营发生了严重对抗和激烈论战。20 世纪 70 年代以后,新自由主义(neoliberalism)或者叫新古典自由主义(neo-classical liberalism)开始崛起。新古典自由主义从概念上看,是对古典自由主义的一种复兴。

20 世纪全球思想界的一个重要人物是弗里德里希 · 冯 · 哈耶克。2013 年国内出版了一部题为《凯恩斯大战哈耶克》的译著,这个书名大概恰如其分地反映了欧美发达国家从 20 世纪 40 年代到 70、80 年代经济公共政策领域的最重要争论,也就是哈耶克观点和凯恩斯观点的交锋。哈耶克一生出版了为数众多的经济学与政治哲学作品,1944 年出版的《通往奴役之路》是他最具影响力的一部作品。哈耶克的另一部重要著作《自由秩序原理》还直接影响了英国首相撒切尔夫人的政治观点与施政纲领。所以,要说撒切尔夫人是哈耶克最著名的粉丝,一点都不为过。

丹尼尔 · 耶金等人生动地描述了撒切尔夫人对哈耶克的推崇:

\quo{自 1974 年以来,作为反对党的领袖,她无疑也是保守党最忠实的自由市场的信徒之一。70 年代中期,在成为领袖不久之后,她访问了保守党的研究部门。当一名研究人员向她扼要介绍他自己主张保守党奉行左右之间的中间道路的一篇论文时,她粗暴地打断了他。她对于重温哈罗德 · 麦克米伦的观点毫无兴趣。相反,她从自己的公文包中抽出一本书——哈耶克的《自由秩序原理》。她把它举起来给每一个人看,坚定地说: “这才是我们所信奉的。” \nauthor{丹尼尔 · 耶金、约瑟夫 · 斯坦尼斯罗:《制高点:重建现代世界的政府与市场之争》,段宏、邢玉春、赵青海译,北京:外文出版社 2000 年版,第 151 页。}}

这个例子彰显了哈耶克在西方世界的巨大影响。那么,哈耶克到底说了什么?这里只能做粗略的介绍。首先,从方法论上说,哈耶克秉承的是个人主义方法论。他基于个人主义方法论来观察社会和世界。个人不仅被置于集体之前的优先位置,而且个人还被视为社会繁荣的真正根源与动力。他在作品中还区分了什么是真的个人主义和什么是假的个人主义。哈耶克说:

\quo{真个人主义首先是一种社会理论,亦即一种旨在理解各种决定着人类社会生活的力量的努力;其次它才是一套从这种社会观念衍生出来的政治准则。……最为愚蠢的误解……亦即那种认为个人主义乃是一种以孤立的或自足的个人的存在为预设的……

真个人主义的基本主张认为,通过对个人行动之综合影响的探究,我们发现:第一,人类赖以取得成就的许多制度乃是在心智未加设计和指导的情况下逐渐形成并正在发挥作用的;第二,套用亚当 · 弗格森的话来说, “民族或国家乃是因偶然缘故而形成的,但是它们的制度则实实在在是人之行动的结果,而非人之设计的结果” ;第三,自由人经由自生自发的合作而创造的成就,往往要比他们个人的心智所能充分理解的东西更伟大。}

其次,哈耶克的学说很大程度上基于其知识论。他的基本观点是:人在很大程度上是无知的,因为人所具有的知识和信息与其决策所需的信息相比总是少得可怜。因此,人类无法借助完全的理性来认知所有复杂事物背后的关键信息。从知识的结构来说,这些不同的有用知识总是分散在社会的各处。他这样说:

\quo{合理经济秩序的问题所具有的这种独特性质,完全是由这样一个事实决定的,即我们必须运用的有关各种情势的知识,从来就不是以一种集中的且整合的形式存在的,而仅仅是作为所有独立的个人所掌握的不完全的而且还常常是互相矛盾的分散知识而存在的。因此,社会经济问题就不只是一个如何配置 “给定” 资源的问题。……它实际上就是一个如何运用知识——亦即那种在整体上对于任何个人来说都不是给定的知识——的问题。\nauthor{F.A.冯 · 哈耶克:《个人主义与经济秩序》,邓正来译,北京:生活 · 读书 · 新知三联书店 2003 年版,第 11、12、117—118 页。}}

基于这种知识论,哈耶克有两个推论:第一,假定有一个中央机构基于理性和科学、依靠其所知的庞大信息来统一做经济决策,是完全靠不住的。理由是该机构所具有的知识和信息,与决策所需的知识和信息相比,还是少得太多。

第二,人是相对无知的,这个社会又需要创新,而创新具有高度的不确定性。因此,一个好的社会要给每一个人及组织以探索和尝试的自由。由于不能事先计划和确定创新来自于何处,自由就为创新所必需。没有此种自由,社会就会慢慢停滞和僵化。当创新不再,经济繁荣和社会进步也成了幻想。所以,哈耶克反对理性建构主义(rational constructuralism)的观点和社会工程(social engineering)的概念。他认为,人类认为可以凭借自己的完全理性与完全信息,把社会推倒并按照自己的意图重新设计一次,这种想法是完全靠不住的。

哈耶克无疑是 20 世纪最重要的政治哲学家之一,他同时也是一位不知疲倦的思想战士,同时在东西方阵营之间和西方阵营内部作战。有人把哈耶克视为 20 世纪社会主义计划经济最强悍的对手之一,因为他一生都在持续不断地批评社会主义计划经济。他主张社会主义计划经济的不可行性,因为它无法解决两大问题:信息问题和激励问题。他同时认为,计划经济会严重地侵害自由。他引用托洛茨基的话说:

\quo{在一个国家为唯一雇主的国度,反抗意味着慢慢地饥饿至死。那个不工作不得食的旧原则,现已为一新的原则所替代,即不服从不得食。}

在西方阵营内部,哈耶克的主要对手是以凯恩斯学说为代表的各式各样的国家干预理论,现代自由主义、民主社会主义和福利国家学说都是哈耶克的批评对象。总的来说,哈耶克这样定义自由及主张自由政策:

\quo{一个人不受制于另一人或另一些人因专断意志而产生的强制的状态,亦常被称为 “个人” 自由或 “人身” 自由的状态。……自由政策的使命就必须是将强制或其恶果减至最小限度,纵使不能将其完全消灭。\nauthor{弗里德里希 · 冯 · 哈耶克:《自由秩序原理》(上册),邓正来译,北京:生活 · 读书 · 新知三联书店 1997 年版,第 169、4 页。}}

另一个与哈耶克关系密切的新古典自由主义代表人物是路德维希 · 冯 · 米瑟斯,他在捍卫自由的立场上甚至比哈耶克走得更远。米瑟斯早年一部浅显易懂的作品是《自由主义》,国内译为《自由与繁荣的国度》。他在书中这样说:

\quo{谈到资本主义,就使人联想到一个心狠手毒、唯利是图的资本家,他剥削同类,无恶不作。事实上,自由主义所主张的资本主义社会秩序是:资本家如果要发财致富,唯一的途径是像满足他们自身需求一样来改善同胞的物质供应条件。}

如果一个人想致富,市场经济会给他提供机会。但惟有他设法改善其同胞的物质与生活条件,他才能实现自己的目标。谋求致富的年轻人发现,只有推动技术进步,实现产品创新,提供更好的消费者价值,才有机会致富。所以,这个逻辑跟亚当 · 斯密 “看不见的手” 原理是一样的。当然,如果政治权力进入市场,就可以通过很多别的办法牟利。

米瑟斯进一步比较了公有制与私有制。整个 20 世纪存在过两种主要的经济模式:一种以生产资料私有制为基础,一种以生产资料公有制为基础。至于后来的混合经济(mixed economy)、社会市场经济(social market economy)或社会主义市场经济(socialist market economy)等等,都是两者的变种或结合。米瑟斯这样说:

\quo{自由主义断言,在实行劳动分工的社会里,人们互相合作的唯一可行的制度是生产资料的私有制。自由主义断言,社会主义作为一个包括全部生产资料的社会制度是行不通的,尽管这种制度在只占有部分生产资料的情况下,并非完全不可能,但是它会导致生产力的下降,以至于使其非但不能够创造更多利润,不能够创造更多的财富,反而会起到减少财富的作用。}

米瑟斯认为,完全的公有制是行不通的,这种社会很快就会垮掉;如果是公有制和私有制结合的状态,就会导致生产力的下降,无法创造更多的财富。这种视角,值得大家去借鉴和思考。米瑟斯还一针见血地讨论了平等问题,他这样说:

\quo{有人指责自由主义关于法律面前人人平等的观念,他们认为法律面前的人人平等并不是真正的平等,这种指责毫无道理,想把人变得真正平等起来,这是依靠人的一切力量都办不到的事情,人与人之间本来就是不平等的,而且将继续不平等下去。……真正理智、清醒、并且合乎目的的处理办法就是争取在法律上平等待人。自由主义并不奢望得到比这更多的东西,因为超出这个范围以外的东西是不存在的,也是不可能办到的。\nauthor{路德维希 · 冯 · 米瑟斯:《自由与繁荣的国度》,韩光明等译,北京:中国社会科学出版社 2005 年版,第 53、61、69—70 页。}}

所以,米瑟斯主张的是权利的平等而非结果的平等,形式的平等而非实质的平等。实际上,如果消灭了财产或财富的不平等,就会有其他的不平等出现。比如,把财产和市场消灭掉后,社会上就会出现了一个掌握巨大政治经济权力的新阶级,而这个阶级跟普通人之间会产生新的巨大的不平等。

20 世纪后半叶,美国主流经济学家米尔顿 · 弗里德曼成了新古典自由主义复兴的重要旗手。他被视为货币主义的倡导者,《资本主义与自由》和《自由选择》则是阐释其政治哲学的通俗作品。弗里德曼认为,经济自由是政治自由的前提;私有产权和契约自由是整个市场制度的核心;政府职能必须严格限定在较小范围内;政府权力必须有限度和足够分散。

弗里德曼对大量公共政策问题都有论述,这里仅举一例。比如,他是国家垄断教育的坚定反对者。从政治上讲,教育国有化的一个弊端是政府有可能借助教育控制整个社会的思想和意识形态,所以公立教育系统可能会沦为政府的工具。另外,公立教育还会带来很多实际问题,比如,从公立学校内部管理来说,小学或中学教师们的薪水基本上是固定的,是根据等级及年资来晋升的,所以弗里德曼认为这会导致优秀教师拿得太少,而不合格教师拿得太多的问题。再比如,由于公立学校只能按学区来统一安排,就限制了家长选择教育的自由,家长在很大程度上丧失对自己子女教育的选择权。

实际上,很多美国公民都对公立教育系统不满意。那么,该怎么办呢?弗里德曼提出应该推广教育券制度,或称代金券制度。这种制度可以提高美国公民在教育问题上的自由选择权。具体来说,政府可以发行教育券,同时鼓励和允许设立特许学校(chartered school)。比如,对本地公立教育系统不满的某个社区的家长们可以发起设立特许学校,这所学校可以完全自主决定教材、选聘教师和确定招生方案。有人会问,这个学校的钱从哪里来呢?是不是全部要向学生家长们收取呢?弗里德曼说,这不公平,因为所有上公立学校的孩子们都获得政府的补贴,比如每个孩子平均获得 4000 美元的财政补贴。既然每个州政府或学区都知道本学区每个孩子的财政补贴数额,就应该根据一所特许学校的招生规模,再乘以平均每个学生财政补贴数额,来补贴给特许学校。

他认为,地方政府应该给家长发行教育券,每张教育券代表了相当数额的财政补贴数额,而家长拿到教育券后,可自行决定孩子上公立学校还是特许学校。这样,在既有财政和公立教育系统中,就引入了一种更加市场化的机制。一旦允许设立特许学校,允许不同体制的学校之间进行竞争,那些管理糟糕、教学质量低下的学校就有很大压力。如果不改进管理和教育质量的话,学生可能会逐渐流失,学校甚至会倒闭。这是弗里德曼这位经济学家基于新古典自由主义经济学对美国教育问题开出的药方,非常有意思。\nauthor{米尔顿 · 弗里德曼、罗丝 · 弗里德曼:《自由选择》,张琦译,北京:机械工业出版社 2008 年版,第 145—187 页。}

20 世纪 70 年代,新古典自由主义阵营中出现的一位重要政治学家是罗伯特 · 诺齐克,其代表作《无政府、国家与乌托邦》与罗尔斯的《正义论》同样被列为 70 年代的政治哲学经典。他主张 “最小国家论” 。在他看来,个人拥有确定无疑的基本权利。这些权利是如此重要和广泛,以至于国家及其官员还能做些什么,在诺齐克看来就是一个问题。个人权利为国家留下多大的活动空间呢?国家应有的合法功能和作用应该是什么?这些都是诺齐克关心的问题。他在书中直截了当地说:

\quo{我们关于国家的主要结论是:能被证明为正当的就是一种最小国家(minimal state),即仅限于防止暴力、偷窃、欺骗和强制履行契约等有限功能的国家;更多功能的国家(extensive state)都将被证明为是非正当的,因为这样的国家会侵犯到个人不被强迫从事某些特定事情的权利。惟有这种最小国家是令人鼓舞的和值得期待的,也是公正的,而所有超出这个限度的国家都被认为是一种恶。\nauthor{Robert Nozick, \italic{Anarchy, State and Utopia}, Oxford: Blackwell, 2001, p.iv. 该书中文版为:罗伯特 · 诺奇克:《无政府、国家与乌托邦》,姚大志译,北京:中国社会科学出版社 2008 年版。}}

比如,在美国,奥巴马政府的新医改方案要求更高比例的公民以法律强制方式缴纳医疗保险。对此,反对者提出的质疑是:政府到底有没有权力强制更高比例的公民去缴纳医疗保险?从诺齐克的原则出发,这无疑是一个问题。

\tsection{什么是保守主义?}

保守主义亦是一种影响很大的思潮。18 世纪晚期至 19 世纪早期,保守主义的兴起是对当时日益加速的社会变革和政治革命的一种 “反动” ,也是对启蒙运动和进步主义思潮中激烈变革要求的一种 “反动” 。按照安德鲁 · 海伍德的看法,早期的保守主义有两个版本,一个版本是所谓 “十足的保守派” ,即反对变化,彻头彻尾地谋求维持现状,不愿意做丝毫的实质性改变。另一个版本的保守主义则比较灵活。这种保守主义是为了保存而变革。这种保守主义主张对现状应该持维护和继承的态度,但同时认为有的方面必须要变革——因为不变革的话,既有的东西都很难保存下来。埃德蒙-柏克曾说: “一个国家,如果没有变革的手段,也就没有保守的手段。” 后来,保守主义意识形态的重大变化发生在 20 世纪 70 年代以后,它吸收古典自由主义的内核,发展成了新保守主义。

历史地看,保守主义是一种非常复杂的意识形态。比如,美国历史学者杰里 · 马勒这样写道:

\quo{因为保守主义者曾在这样那样的时间与场合,捍卫过皇权、立宪君主制、贵族特权、代议制民主和总统独裁;高关税和自由贸易;民族主义和国际主义;中央集权和联邦制;一个继承房产的社会,一个资本主义市场的社会以及福利国家的这种或那种版本。他们也捍卫普遍意义上的宗教、正统教会,及政府抵制宗教狂热分子的言论以捍卫自身的必要性。毫无疑问,今天那些自我声称的保守主义者很难想象过去的保守主义者会支持不同于其主张的制度和实践。\nauthor{杰里 · 马勒:《保守主义:从休谟到当前的社会政治思想文集》,刘曙辉、张容南译,南京:译林出版社 2010 年版,第 7 页。}}

在中国的语境中, “保守” 的日常含义与 “保守主义” 作为一种意识形态和学术话语差异很大。比如,有的女生夏天从来不穿短裙,被称为衣着风格的 “保守” ;有的家长建议子女大学毕业后不要去跨国公司或民营企业工作,而是去政府机关工作,亦被称为观念上的 “保守” 。所以,在中国日常语境中的 “保守” 跟保守主义不是一回事。

借鉴安德鲁 · 海伍德的论述,这里把作为意识形态的保守主义总结为若干重要原则:\nauthor{关于保守主义的要点总结,参见:安德鲁 · 海伍德:《政治学》(第二版),第 58—60 页。}

一是捍卫传统的原则。保守主义认为,传统当中包含了很多有益的价值与思想。 “成见” 一词在埃德蒙 · 柏克的作品中具有十分积极的含义,可以被视为传统的另一种说法。尊重成见,即是尊重传统。出于捍卫传统的考虑,保守主义都会反对激进变革。这一点本讲开篇已经重点讨论过。爱德华 · 希尔斯在《论传统》一书中开篇即讲,因为理性的力量和科学的进步,让我们觉得传统不再重要了,而该书的主要目标是阐明传统的重要性。这样的著作无疑具有浓厚的保守主义色彩。

二是经验主义原则。与经验主义相对的是理性主义,保守主义强调的是人的有界理性。理性主义认为,既然达尔文可以发现生物界的进化规律,牛顿可以发现天体运动的规律,那么人们是否可以发现人类社会的规律?当这个规律被发现后,能否根据这个规律来改造我们的社会?如果按照人的理性认知来重新设计这个社会,人类的生活会不会更美好呢?经验主义对这种理性主义——一种过度的理性主义——持高度警惕和怀疑的态度。经验主义对很多过度乐观的方案持审慎、甚至怀疑的态度。上文曾提到,哈耶克尽管被认为是一位新古典自由主义者,但他在知识论上倾向于认为人是无知的,这一见解与保守主义立场很相似。经验主义认为,人的理性是有边界的,不能想对社会做什么改造就能做什么改造,在尊重现有惯例、传统与习俗基础上做逐步改进更为合理。

三是人类的不完善原则。某些意识形态最后都会塑造一个乌托邦,但保守主义对此从来就抱有警惕和怀疑。乔万伊 · 萨托利在《民主新论》中批评过 “至善论” 的危险——这是一种与塑造乌托邦有关的思维模式。当认识到这个社会的不完美时,一种可能性是对这种不完美性和不完善性感到厌恶,由此可能会产生一种念头:这个社会应该变得更加完美、完善和纯洁。既然这样,能否按照这种理想来构建和塑造一个新社会?能否把一切与真、善、美不相关的东西统统摒弃掉或消灭掉?至善论的观念,最后往往导向一条追求乌托邦的道路,就是想要创造一个无限美好的社会。保守主义对此持否定态度,他们往往以一种冷峻的态度来思考现实问题。正如霍布斯在《利维坦》中所说: “人类的事情决不可能没有一点毛病。” 

四是社会作为有机体的原则。有机体的观点意味着社会是一个复杂的有机系统,社会内部不同部分之间是互相关联的。所以,不能凭自己的理论、臆想或理性来随意创造一种新社会。哈耶克极力反对的一个概念是 “社会工程学” (social engineering),即凭借理性对社会进行再造,就如同把整个社会视为一个建筑工程一般。保守主义认为,社会中很多重要事物是人的理性无法完全认知的,却在现实中极其重要。比如,哈耶克曾论述,我们每天使用的语言就是一种无意识进化的结果,是来自于自生自发的秩序,而非人类有意设计的产物。所以,人类自生自发的秩序中包含了很多有价值的东西。\nauthor{F.A.哈耶克:《致命的自负》,冯克利等译,北京:中国社会科学出版社 2000 年版。}如果要从头构建一个新的社会、彻底打破一切惯例和常规,最后可能会发现这个依靠理性构建起来的社会由于缺少有机体因素的支撑而难以有效运转,甚至完全陷于瘫痪。

五是重视等级、秩序和权威的原则。与保守主义相比,自由主义者更重视资格的平等和个人的自由,自由主义的理想世界是在一个自由公平的市场中进行资格平等的自由竞争。当然,自由主义承认市场竞争的结果对每个人是有差别的。但是,在保守主义者眼中,这个社会存在等级本身是很正常的。有些人更富有、更高贵、更有权势、更有影响力也很正常,社会本来就应该是这样的。回顾人类的历史,又有哪个社会不是这样的呢?如果有人想构建一个人人实现实质平等的社会,比如彻底打破了现有等级秩序和财产结构,用不了多久就又会形成一个高下有别、尊卑有序的社会。保守主义总体上倾向于维护过去业已形成的既有秩序和社会等级。所以,对于那种试图彻底打破既有秩序的社会革命思潮,保守主义是强烈反对的。如同尊重既有秩序和社会等级,保守主义也信奉尊重权威的原则。当然,这个原则不能解释过度。尊重权威并不意味着主张国家为所欲为或政府干预;相反,保守主义通常是反对上述做法的。保守主义的尊重权威往往需要跟其他原则和价值相协调,比如跟后面要讲到的尊重财产权原则等。

六是重视家庭的原则。自由主义更强调个人作为个体的价值和意义,并没有特别强调家庭的价值和意义。保守主义认为,家庭是很多重要社会功能的承担者,比如教育;同时,家庭也维系着一个社会的稳定与秩序。从更深层上说,家庭代表的是传统秩序的一部分。

七是重视和认同宗教的原则。在欧洲大陆国家,天主教徒更有可能是保守主义者。在欧洲历史上,教会与国家之间曾发生过激烈的冲突。在欧洲现代国家兴起之前,教会控制着大量的经济资源和教育资源,甚至主宰司法和婚姻。当时的保守主义者基本上倾向于维护和捍卫教会已有的地位和角色,反对现代国家接管这些事务和权力。今天的保守主义者则更认同传统宗教,通常是更为虔诚的教徒,更可能在堕胎、同性恋等政治议题上持反对立场。

八是重视道德的原则。对一个运转良好的社会来说,道德的重要性毋庸置疑。但自由主义学说对道德并不是很重视,自由主义很少专门论述道德问题——尽管创作《国富论》的亚当 · 斯密同时创作了《道德情操论》。当然,这并不是说自由主义反对道德,而是自由主义认为:每个人都倾向于追逐自己的利益,当外部法律环境公平时,自由市场机制就能有效运转,每个人追逐自我利益的行为客观上就能实现社会的共同利益。自由主义对这种基于自我利益的市场行为最后能达致社会和谐抱有充分的信心。麦特 · 里德雷(Matt Ridley)在《美德的起源》一书中干脆认为,美德是自利的个人互相博弈与长期演化的结果。换言之,美德是自利的产物。但是,保守主义则非常重视道德,认为道德是一个良好社会的重要因素。

九是尊重财产权的原则。保守主义非常重要的一条原则是强调对财产权的保护和捍卫。在这一点上,保守主义与古典自由主义的立场别无二致。保守主义不仅反对国家剥夺或征收财产,而且反对国家对私人市场的过度干预、高税收和高福利等做法。上述这些做法违背了尊重财产权的原则。保守主义在 20 世纪 70、80 年代演变为新保守主义时,在这一问题上是跟新古典自由主义是一致的。

当然,保守主义的上述几项原则并非完全能做到内在自洽。作为一个博大精深的思想体系,保守主义内部可能存在冲突。所以,要完整而准确地理解保守主义并非易事。这里再以埃德蒙 · 柏克的《法国革命论》佐证上述几项原则。柏克创作该书的背景是 1789 年的法国大革命。法国大革命爆发之后,有人给柏克写信介绍法国的情况,然后柏克给这位朋友写了一封非常长的回信,这就是《法国革命论》,书名直译应该是《法国革命的反思》。当然,上面这些细节可能仅是一个传说,柏克不过是借着这个由头要写这样一本书。

法国大革命刚发生时,很多英国人和欧洲人对法国大革命抱有热切的期待,革命意味着一种对旧秩序的打破和对一种新秩序的创造。但是,经过一段时间之后,他们发现,这个革命变得越来越暴力,越来越没有秩序,而且历时漫长。那么,柏克对法国大革命怎么看呢?这里再做进一步的介绍。柏克说:

\quo{我应该中止我对于法国的新的自由的祝贺,直到我获悉了它是怎样与政府相结合在一起的,与公共力量、与军队的纪律性和服从、与一种有效的而分配良好的征税制度、与道德和宗教、与财产的稳定、与和平的秩序、与政治和社会的风尚相结合在一起的。所有这些(以它们的方式)也都是好东西;而且没有它们,就是有了自由,也不是什么好事,并且大概是不会长久的。\nauthor{埃德蒙 · 柏克:《法国革命论》,何兆武等译,北京:商务印书馆 2010 年版,第 11 页。}}

柏克对诸种重要的社会价值有一个排序,一边放着自由或称之为 “新的自由” ,另一边放着政府、公共力量、军队的纪律和服从、税收制度、美德与宗教、财产权、和平的秩序以及整个社会过去形成的传统与风尚。他说,当同时拥有后面这些东西时,这个新的自由才是好的;如果没有这些东西,新的自由就不是什么好事。

他这段话不仅包括了对法国大革命的看法,而且含有对这场大革命前景的预言。如果有自由,但是没有上述这些重要的事情,这种自由是不会长久的。在拿破仑上台之前,法国就经历了相当长时间的混乱。为什么法国需要拿破仑?一个简单的解释是:拿破仑在当时代表了秩序和稳定,代表了对社会分歧的弥合。特别是,今天你被推上了断头台,明天他被推上了断头台——法国人后来发现再不能忍受这种生活了,整个法国渴望回归秩序。所以,在长时间混乱之后,拿破仑成了法国人民欢迎的人物。拿破仑带来的是稳定的秩序、财产的保障、有效的政府力量及军队的服从。

面对法国大革命,柏克还感到非常遗憾:

\quo{但是骑士的时代已经成为过去了。继之而来的是诡辩家、经济家和计算家的时代;欧洲的光荣是永远消失了。我们永远、永远再也看不到那种对上级和对女性的慷慨的效忠、那种骄傲的驯服、那种庄严的服从、那种衷心的部曲关系——它们哪怕是在卑顺本身之中,也活生生地保持着一种崇高的自由精神。\nauthor{埃德蒙 · 柏克:《法国革命论》,第 101 页。}}

在这段话中,柏克对欧洲传统的消逝充满遗憾和留恋。他同时还强调,法国大革命之前欧洲的时代,是一个既拥有秩序与权威、又饱含自由精神的时代。讲到反对革命与激进变革,他还说:

\quo{我对革命——它那信号往往都是从布道论坛上发出的——感到厌恶;改革的精神已经传到了国外;对一切古老制度——当其被置之于当前的方便感或与当前的倾向相对立时——的全盘鄙弃,正在你们那里风行,并且可能也要在我们这里风行:所有以上的考虑在我看来,就使得唤起我们对自己国内法律的真正原则的关注成为了并非是不可取的事。\nauthor{埃德蒙 · 柏克:《法国革命论》,第 33—34 页。}}

柏克的这段话中充满了对英格兰法律传统的强调。在该书其他部分,柏克还反复提到对家庭的尊重、对美德的尊重、对宗教的尊重,以及对社会既有传统的尊重。所有这些都从不同维度上声张了保守主义的观点。

柏克之后,保守主义思潮又经历了复杂的演变。按照杰里 · 马勒的说法,詹姆斯 · 麦迪逊、约瑟夫 · 熊彼特、卡尔 · 施米特、迈克尔 · 奥克肖特、弗里德里希 · 哈耶克等思想家身上,或多或少包含了保守主义的思想倾向。保守主义意识形态内部的一大争论是,保守主义究竟是一整套系统的价值观念和意识形态体系,还是仅仅是一种对于现状的维系与变革之间的思想倾向。这方面并无定论。更为复杂的是,按照杰里 · 马勒的说法: “保守主义的内容不仅随着时间和国家语境改变,而且常常在同一时间地点自称保守主义者的人中也内涵不一。” \nauthor{杰里 · 马勒:《保守主义:从休谟到当前的社会政治思想文集》,刘曙辉、张容南译,南京:译林出版社 2010 年版,第 36 页。}

\tsection{撒切尔夫人改革与里根革命}

20 世纪 70 年代以后,随着欧美社会经济滞胀和福利病的出现,保守主义开始迎来新的春天。以英国和美国为首,新保守主义运动开始兴起。两位发动新保守主义改革的政治家是英国首相玛格丽特 · 撒切尔夫人与美国总统罗纳德 · 里根。

英国在 20 世纪 40 年代的第二次世界大战时期已经建成福利社会,又经过 50、60 年代较快的经济增长,但到 70 年代也随整个欧美社会开始陷入一定程度的危机。福利国家的包袱越来越重,国家的经济干预越来越深,整个社会的活力与创新精神开始下降,通货膨胀与失业同时出现。面对这种情况,英国何去何从?撒切尔夫人当时对英国的经济和社会的评价是:

\quo{一种有气无力的社会主义已经成为英国的时尚。工党统治下接连不断的危机——经济、财政和工业危机不断地驱使我们思考和提出背离流行的看法和共识的思想和政策。\nauthor{参见丹尼尔 · 耶金、约瑟夫 · 斯坦尼斯罗:《制高点:重建现代世界的政府与市场之争》,段宏、邢玉春、赵青海译,北京:外文出版社 2000 年版,第 131 页。}}

在撒切尔夫人看来,工党统治所实行的政策已经造成了一系列的危机,所以这种现实逼迫作为保守党人的撒切尔夫人提出与当时主流不一样的思想和政策。那么,当时的主流看法有什么问题呢?当时英国保守党的另一位重要政治家基思 · 约瑟夫说: “我们管理得太多,支出得太多,税收得太多,钱借得太多,人员也配备得太多。” 

那么,该怎么办呢?一句话,统统要砍下来。要把多余的管理和政府干预砍下来,把过多的支出砍下来,把过多的税收砍下来,把过多的公债砍下来,把过多的政府和人员规模也砍下来。撒切尔夫人的基本想法是: “我想利用私有化来实现我自己建立一个拥有资本的民主国家的理想。” 这意味着,撒切尔夫人的理想国家是资本和市场力量充分发挥作用、政府大幅缩减规模的民主国家。这种理念就更像是向古典自由主义的回归。

撒切尔夫人的基本做法就是大力推进私有化改革,出售了很多国有企业,同时缩减政府干预的范围。这些方面尽管阻力重重,但总的来说还是取得了重大进展。撒切尔夫人还试图把社会福利砍下来,但遭到了社会的严重反弹,结果是收效甚微。主要原因在于,整个社会对于这种改革的抵抗非常厉害,现代工业社会的背景加上民主政体与普选制的制度——所有这一切都束缚了撒切尔夫人的手脚,最终使得她这种新保守主义改革无法达到她期待的结果。

里根总统的名言是: “政府不是解决问题的手段,政府就是问题本身。” 这句出自里根 1981 年总统就职典礼的演说,它仿佛吹响了里根革命的号角。一般认为,里根革命的理论依据是供给学派的经济学。里根推行的是供给学派的经济政策,比如将所得税降低了 25\% ,努力降低通货膨胀,降低利率,并对企业和市场行为放松政府管制。里根始终怀疑政府干预和管制的有效性,他的做法是撤回政府干预并降低税率和市场管制,以此让自由市场机制自动修正所面临的问题。

关于税率与财政收入的关系,里根总统的经济政策受到了美国经济学家阿瑟 · 拉弗的影响。拉弗有一个关于税率与财政收入的理论,可以用图 3.2 的拉弗曲线来表示。拉弗曲线其实并不高深,图中横轴是税率高低,纵轴是政府收入数量。如果税率为零,政府收入亦为零;随着税率的提高,政府收入也会提高;但是,当税率高到某个点之后,随着税率的提高,政府收入反而会下降。

\img{../images/image00301.jpeg}[图 3.2 拉弗曲线]

为什么呢?因为税率过高时,大家就没有动力工作了,这样经济就发展不了。所以,如果原本税率过高,通过降低税率和放松管制,使自由市场经济的活力得到激发,经济发展速度就会提高,结果是实现政府收入的提高。里根在很大程度上参考了拉弗曲线的政策。

但是,遗憾的是,里根革命并未达到里根总统的预期效果。他把税率砍了下来,结果短期内政府收入减少了,但是福利没有办法砍下来,这样财政赤字就上去了。所以,新保守主义改革在 20 世纪 80 年代给美国留下了一堆沉重的政府公债。尽管里根总统并未达到其改革目标,但有人认为里根的新保守主义改革为 20 世纪 90 年代美国新经济和经济增长奠定了基础。

\tsection{什么是社会主义?}

第三种主要意识形态是社会主义。关于平等和乌托邦的理想,可以追溯到很久以前。柏拉图在《理想国》中就有关于 “财产公有” 的描述。柏拉图尽管没有系统地论证这一观点,但他表述了这方面的思想。当然柏拉图不能被称为是一位社会主义者。社会主义作为一种意识形态的出现,是对资本主义早期发展的反应。首先出现的是空想社会主义。像欧文、圣西门和傅里叶等人都是空想社会主义的代表,他们强调集体主义和互助,批评资本主义的诸种负面现象。后来,这一派的学说就慢慢地发展为科学社会主义或者叫共产主义,这主要是指马克思和恩格斯的学说。马克思和恩格斯主张通过无产阶级革命来建立公有制,实行计划经济,建设共产主义。《共产党宣言》中动员工人阶级的著名口号是: “全世界无产者联合起来!” 

1848 年马克思和恩格斯起草的《共产党宣言》中包括了十条基本政策,这可以被视为马克思和恩格斯的政治纲领。尽管他们两人后来对该政治纲领的表述做出了修正,但这十条基本政策至少代表了他们在 1848 年的观点。他们说:

\quo{最先进的国家几乎都可以采取下面的措施:

1. 剥夺地产,把地租用于国家支出。

2. 征收高额累进税。

3. 废除继承权。

4. 没收一切流亡分子和叛乱分子的财产。

5. 通过拥有国家资本和独享垄断权的国家银行,把信贷集中在国家手里。

6. 把全部运输业集中在国家手里。

7. 按照总的计划增加国家工厂和生产工具,开垦荒地和改良土壤。

8. 实行普遍劳动义务制,成立产业军,特别是在农业方面。

9. 把农业和工业结合起来,促使城乡对立逐步消灭。

10. 对所有儿童实行公共的和免费的教育。取消现在这种形式的儿童的工厂劳动。把教育同物质生产结合起来,等等。\nauthor{马克思、恩格斯:《马克思恩格斯选集》第一卷,北京:人民出版社 2012 年版,第 421—422 页。}}

马克思与恩格斯之后,社会主义学说后来又有新的发展,被称为改良的社会主义或社会民主主义。其中的一位杰出代表性人物是德国思想家与政治活动家爱德华 · 伯恩斯坦,他认为应该放弃暴力革命,主张通过议会方式进行民主斗争,并进而改变资本主义国家的经济政策,强化再分配,实现社会平等以及建设福利国家。数十年之后,伯恩斯坦倡导的很多主张已经成为欧洲发达国家的基本政策和实际状态。所以,从某种程度上说,今天的发达国家——特别是欧洲大陆的发达国家——其政治经济模式已经融合了自由主义和民主社会主义的因素。

那么,什么是社会主义意识形态的主要特征呢?借鉴安德鲁 · 海伍德的论述,这里把社会主义意识形态视为对若干原则的倡导:\nauthor{关于社会主义的要点总结,参见安德鲁 · 海伍德:《政治学》(第二版),第 64—65 页。}

一是强调共同体原则。在个人与群体的关系上,社会主义强调的是群体,认为脱离了群体的个人是不存在的,个人脱离群体实际上也无法生存。比如,如果一个小孩生活在孤岛上,如鲁宾逊一般,不跟其他社会成员发生交往,他最后是无法成为一个社会人的,只不过是一个具有人形的动物罢了。所以,倘若脱离了群体,个体的价值就无从体现,甚至难以成为一个完整意义上的人。

二是平等主义原则。如果说自由主义更重视自由,保守主义更重视秩序与财产,那么社会主义就更重视平等。社会主义的平等观,不只是形式平等和机会平等,同样重要的是实质平等和结果平等。所以,一个符合社会主义理想的社会,其社会成员之间的贫富差距程度应该要远远小于按照其他原则构建的社会。平等的重要性与优先性,是衡量一种观点是否属于社会主义意识形态的关键标准。

三是博爱原则。社会主义主张博爱,认为应该以同等的方式去对待社会中的每一份子。社会主义的最终目标也是要让所有人或绝大多数人都过上一种更美好、更体面和更有尊严的生活。

四是阶级原则。社会主义秉承阶级分析方法,把社会分成资产阶级和无产阶级,或上层阶级和下层阶级。前者是统治阶级,后者是被统治阶级。社会主义认为,上层阶级与下层阶级之间存在压迫与被压迫关系。凡是下层阶级的一员,实际上时刻都受着上层阶级的压迫。所以,社会主义主张打破既有的阶级结构,以构建更平等的、甚至无阶级的社会作为理想。

五是财产的社会控制原则。社会主义认为,只要财产控制在私人手里,必定会带来前面提到的不平等和阶级压迫。所以,社会主义寻求的是让财产的私人占有过渡到某种形式的财产的社会控制。当然,关于财产的社会控制应该采取何种具体手段,马克思有自己的看法,其他社会主义思想家们也有各自的见解。总的来说,财产的社会控制主要存在国家控制和社会自治体控制两种不同的设想。

六是满足需求的原则。社会主义比较注重人基本需求的满足。每一个人都有很多基本需求,每个人应该拥有满足这些基本需求的权利。所以,共产主义的口号是 “各尽所能,按需分配” 。与自由主义和保守主义相比,社会主义至少在意识形态上更注重人的基本需求,这也是社会主义意识形态的独特之处。

自工业革命以来,不同类型的社会主义在欧洲知识界一直占据着较大的市场。很多杰出的知识分子都走到社会主义立场上,这一点大概跟人的同情心有关。英国哲学家伯特兰 · 罗素曾说: “三种激情支配了我的一生:爱的渴望,知识的渴求,以及对人类苦难的极度同情。” 如同罗素,很多人思考社会问题都是从观察苦难入手的。不少人在从小到大的环境里会见识了各种苦难,随后是思考苦难,由此产生了减轻社会苦难的积极愿望。从社会分层看,下层阶级的苦难往往更多一些。所以,思考如何让下层阶级的生活变得更好时,很多知识分子容易走上信奉社会主义的思想道路。

\tsection{从民主社会主义到新工党}

社会主义意识形态有很多重要的代表人物,当然最重要的是卡尔 · 马克思。但考虑到读者对马克思学说相对比较了解,本书不打算花很多篇幅讨论这个专题。总的来说,马克思认为, “一切历史都是阶级斗争的历史” ,资产阶级与无产阶级的矛盾和斗争是资本主义社会发展的动力;由于生产资料私有制与社会化大生产之间的矛盾,资本主义会面临不断的经济危机,由此从资本主义过渡到社会主义就具有必然性;资本主义的进一步发展就是通过无产阶级革命实现共产主义。\nauthor{国内流行的教科书容易过分简化地理解马克思和恩格斯的学说,马克思与恩格斯的论著参见:马克思、恩格斯:《马克思恩格斯选集》第一至四卷,北京:人民出版社 1995 年版;Karl Marx, Friedrich Engels, \italic{The Marx-Engels Reader}, eds. by Robert C. Tucker, New York: W.W. Norton \& Company, 1978。}

马克思的学说传到俄国后,列宁的创新也非常重要。按照马克思的观点,共产主义首先会在资本主义最发达的国家实现。但是,列宁认为,资本主义发达国家通过帝国主义方式把主要矛盾转移到了发展中国家,所以无产阶级革命有可能率先在资本主义最薄弱的链条上建立起来。这样,列宁就把马克思学说往前发展了一步。由此,列宁找到了在资本主义链条最薄弱的地方率先发动无产阶级革命的理论依据。列宁还提出,无产阶级革命政党应该发挥领导作用,借助革命政党的领导能把无产阶级力量团结起来,就有可能建立起社会主义或共产主义。列宁特别强调了暴力革命的重要作用,即通过暴力方式来实现无产阶级革命。\nauthor{参见列宁:《列宁选集》第一至四卷,北京:人民出版社 1995 年版。}所以,从马克思这样一个理论家和革命预言家,到列宁这样一个理论家和革命实践家,不仅是社会主义意识形态从理论到实践的转换,同时包含了理论上的创新。

但是,早在列宁领导布尔什维克发动十月革命之前,在 19 世纪的晚期,马克思的学说在欧洲大陆已经受到很多挑战。德国社会民主党的重要领袖爱德华 · 伯恩斯坦认为马克思和恩格斯的观点过于激进,转而走上了改良社会主义的道路,并在德国阐明了民主社会主义或社会民主党的政治纲领。

伯恩斯坦总体上并不赞成从马克思和恩格斯到后来列宁的无产阶级革命观点。相反,他认为应该在资产阶级宪政民主框架内,通过议会斗争的方式来实现社会主义的基本目标。当时的社会背景是,普鲁士和德国的普通民众开始逐步获得普选权。对工人阶级,普选权意味着什么呢?只要人数足够多,工人阶级就能把一个工人阶级政党或社会民主党选到议会里去。

如果这样一个政党能够掌握政治权力,就可以推行有利于无产阶级的公共政策,包括保护劳工权益、建设福利国家、强化再分配、甚至实行一定程度的财产社会化政策,等等。伯恩斯坦认为,马克思生活的年代还没有出现普选——除了 1848 年普鲁士短暂的民主实验以外,所以,马克思本人并没有看到通过选举改革让普通民众获得选举权的可能性。但是,后来快速的政治变迁揭示,普通民众和工人阶级完全可以拥有普选权。伯恩斯坦的创见在于,当社会趋势发生急剧变迁时,他在政治上看到了一种新的可能性。

与此同时,伯恩斯坦对马克思和恩格斯在《共产党宣言》中的政治纲领与社会设计充满了担忧。首先,他认为,完全的公有制将会造成劳工阶层的积极性不足。后来很多关于计划经济的研究已经为这种观点提供了理论和经验支持。其次,这种模式会导致生产过程的官僚主义化。从企业与市场由资本家主导的管理模式变成一种由行政官僚统治的管理模式——行政官僚本身缺少市场激励,整个生产过程极易变成一种官僚化运作。再次,无产阶级一党专政有可能会成为新的专制统治——伯恩斯坦认为至少有这种可能性。最后,伯恩斯坦认为马克思的工人阶级中心论有可能是反人道主义的。如果说工人阶级需要特殊优待,那么其他阶级呢?比如,管理阶层、知识阶层、农民阶层呢?实际上,伯恩斯坦的部分担忧已经为苏联斯大林统治时期的某些实践所证实。\nauthor{参见爱德华 · 伯恩斯坦:《伯恩斯坦文选》,殷叙彝编,北京:人民出版社 2008 年版。}

那么,社会民主主义主张何种政策呢?他们主张的一个主要政策是通过税收和再分配使不同社会阶层的收入更加平等化。比如,他们主张对高收入阶层征收累进所得税——收入越高税率就越高,实际上多数西方发达国家现在就是这么做的。不少国家最高个人所得税的税率已经达到 40\% —50\% 。社会福利政策也是一个重要方面,社会民主主义的早期理想是建设从摇篮到坟墓的、无所不包的福利国家。社会民主主义通常还会促成充分就业以及支持与强大的工会力量合作。这些大概是社会民主主义政策的概要表述。

大家会发现,社会民主主义的早期政治纲领在如今的很多欧洲国家已成为现实。今天,很多西方国家的制度与政策都是借鉴了这些原则。在 20 世纪的欧洲,从政治家到普通公民,很多人甚至认为这些制度与政策应该成为现代国家的通行做法。但是,在 20 世纪 70—80 年代西方社会的经济危机中,福利国家的模式遭遇了挑战。上文业已提到,正是在这种背景下,英国首相撒切尔夫人和美国总统里根开始对既有的福利国家模式和社会政策进行大刀阔斧的改革。

在此种背景下,特别是 20 世纪 80 年代以来,社会民主主义阵营内部出现了分化,其主流开始与新自由主义的部分主张合流,兴起了 “新的社会民主主义”  “第三条道路”  “中间道路”  “新社会民主党” 和 “新工党” 等概念。总的趋势是,他们部分地接受新自由主义的经济理论,承认市场经济的作用,强调企业家精神和激励因素的重要性,不赞同建设无所不包的福利国家。

在英国,过去工党更倾向社会民主主义的政策主张甚至被视为落伍陈旧的观念。有学者这样评价英国 “老” 工党的政策:

\quo{可以说,老工党对市场力量抱有深深的疑虑,它试图通过集中化的经济计划和大量的干预主义政策限制市场的力量……老工党笃信公有制的优越性,它试图以牺牲私营部门为代价稳定地扩大公有制的范围。老工党以工会是工人阶级的代表为理由,赞同在政府中保有工会的权力……最后,老工党倾向于放任国家的财政,常常会屈从于税收、支出和借款等 “快修” 方法的诱惑,而不是寻求更为适度和审慎的方法。\nauthor{英国新工党政策的兴起,参见斯图亚特 · 汤普森:《社会民主主义的困境:思想意识、治理与全球化》,贺和风、朱艳圣译,重庆:重庆出版社 2008 年版,第 134—167 页。}}

另一种说法则代表了后来英国 “新” 工党的基本政策: “市场应当在企业家的指导下起主导的作用,政府的干预应减少到最低限度;应缩减税收和公共开支;应尽可能淡化工会的作用。” 这则言论非常明确地强调,英国工党在向新自由主义的政策靠近。由此,大家可以完整地理解欧洲国家社会民主主义及其政策的历史变迁,了解社会主义这种意识形态的完整发展脉络。\nauthor{关于英国新工党,参见马丁 · 鲍威尔编:《新工党,新福利国家?英国社会政策中的 “第三条道路” 》,林德山等译,重庆:重庆出版社 2010 年版。}

\tsection{意识形态论战的场域}

不同的政治意识形态往往容易在一些重要议题上产生交锋。首要的问题是个人与群体的关系。究竟是个人优先于群体,还是群体优先于个人?不同意识形态的分歧很大。比如,下面就罗列了一次课堂讨论中两种主张的交锋:

支持个人优先的观点一:

集体权力是个人权利让渡而来的。形成集体的目的就在于保障个人。如果个人的权利与利益不能得到保障,集体的存在就没有意义。如果一味强调集体而忽视个人,实际上集体主义会沦为一个空洞的口号。只有每一个人的权利和利益得到切实的保障,这个集体才能得到保障。

支持集体优先的观点一:

如果坚持个人优先,国家政策往往是鼓励个人自由、权利与利益的最大化,但这其实自相矛盾。比如,一个人自由的最大化,往往会妨碍另一个人的自由。所以,如果想要达到对所有人都有利的结果,首先应该保证集体或群体的最大利益。如果说集体利益是个人利益的加总,那么只有保证集体利益,才能保证个人利益。

支持个人优先的观点二:

上述观点似乎在强调,个人利益被包含在集体利益中。但是,集体会不会侵犯个人利益呢?有人说,只要利益在个人间得到合理分配,集体就不会侵犯个人利益。但问题是,如何保证利益在个人之间进行合理分配呢?那些在分配利益过程中掌握更大权力的人,会不会设计一种对自己更有利的分配方案呢?这个问题在很多集体主义社会中非常严重。由于控制了集体资源的分配权,那些掌握权力的阶层日益成为 “新阶级” 。所以,这种观点并没有什么牢靠的基础。

支持集体优先的观点二:

集体权力未必就是个人权利的简单让渡。集体利益不只在于保障每一个构成集体的个人的利益,集体本身也有自己的利益。比如,如果一个群体流落到一个野兽出没、危险丛生的原始森林,大家该怎么办呢?这种时候,集体通常无法确保每一个人的安全,此时作为一个整体能存活下去是更重要的问题。所以,当身强体壮的人被组织起来去对付猛兽时,他们本身的生命可能处于更大的危险之中。但这样做,是为了让他们的妻子儿女及同类有更大的存活概率。所以,集体并非个人的简单加总。此种情形下,为了集体,个人必须做出必要的牺牲。

支持个人优先的观点三:

集体主义的弊端是集体容易吞噬个人。在按集体主义原则组织的群体中,组织的惯例与规则是非常强大的,个人进入这类群体后会自觉不自觉地遵循和服从这些东西。结果是,个性更容易被磨灭,创造精神被消磨,卓越分子可能被淘汰,群体智力水平可能会下降。长期当中,这样的群体就可能会衰落。

支持集体优先的观点三:

个人优先,会鼓励每个人都去争权夺利。这种情况下,每一个人对资源、利益和权力的争夺可能相当激烈,结果是内耗可能会很严重。人与人之间更有可能成为竞争和冲突关系,而不是合作与互助关系。这种模式下达成的均衡未必更好。\nauthor{上述观点来自于课堂学生发言,多数是 2012 年入学的复旦大学新闻学院大一本科生。文字经过作者的修改和润色,感谢这些积极参与课堂讨论的同学们!}

通过上述讨论,可以看出支持个人优先的观点有几个基本倾向。首先,社会或群体是由个体构成的,所以个人应该有优先性。其次,整个社会的基础与动力都在于个人,所以相应的制度与政策都需要落实到个人激励上。比如,当每个人都更努力和有效地工作时,这个社会才能变好;当每个人都创造价值时,这个社会才能变好;当每个人都更加行为端正、讲求礼节时,这个社会才能变好。当然,个人优先的结果一定是不同个体之间会有差距。所以,主张个人优先的观点必须接受贫富差距的事实。但这一点,可能会让不少人的同情心遭到重创。

支持集体优先的观点有几个基本倾向。首先,有人强调共同体生存是第一位的。这种意识可能会导致比较强烈的民族主义立场。比如,普鲁士当年国家主义盛行,黑格尔甚至把国家称为 “神在地上行走” ,其中的驱动力跟共同体生存环境有关。普鲁士西有法国,东有俄国,普鲁士或德国作为一个国家如何生存下去呢?以国家方式表现出来的集体的强大,被认为一个必要条件。其次,这种观点强调个人的最优选择未必会导致集体的最优选择,个人利益的最大化或可跟集体利益的最大化相冲突。一个著名例子就是 “囚徒困境” 。每个人都选择对自己最有利的,结果可能对大家都是不利的。所以, “囚徒困境” 的博弈情境似乎在论证,亚当 · 斯密所设想的那个自由竞争的和谐世界未必存在。那么,集体优先模式的问题是什么?最重要的是,凡是倡导集体优先的社会,必然都会强调一种更强组织化的手段和更强的集体控制。由此带来了两个问题:一是个人可能会被压制,个性受到压抑;二是那掌握集体控制手段的个人或集团,往往会拥有更大的权力,甚至可能侵犯其他普通个人的自由与权利。这也是集体主义模式的危险所在。

另一个重要议题是国家的角色与作用。有人认为,国家一半是天使,一半是魔鬼。怎样看待国家,往往取决于思考问题的视角。以美国为例,美国的政治文化中通常充斥着对国家、对政府和对权力的不信任。最近流行的著名美剧《纸牌屋》,把美国高层政治描绘成很不堪的样子,其实这不过是美国影视界的一贯做法。很多到美国读书的人都听过这样一个小故事:

\quo{有一个美国小男孩要过圣诞节了,却还没有收到礼物,感到很郁闷。于是,他就写了一封信,想寄给上帝。他在信中说,马上要过圣诞节了,上帝您给我 100 美金作礼物好吗?邮递员看到信后觉得很搞笑,上面写着寄给上帝,那送给谁呢?邮递员决定把信送到白宫。当时的总统收到这封信感到有些激动,本来寄给上帝的信最后送给了他。所以,他决定给这个小男孩寄点钱过去,但 100 美金太多了,所以就让秘书寄了 50 美金过去。

很快,那个小男孩收到了 50 美金,而且注意到钱是从白宫寄来的。于是,他给上帝写了一封回信。不久,美国总统收到了小男孩寄给上帝的第二封信。当这位总统开心地打开小男孩的信件时,他读到了这样的文字: “非常感谢上帝,您寄来的圣诞节礼物已经收到,但遗憾的是,您的钱是从白宫转过来的,这钱经过白宫时,被那帮混蛋克扣了 50 美金!” }

这个小男孩觉得上帝寄给他的应该是 100 美金,而他只收到了 50 美金,那 50 美金一定是被白宫征收了。这个说明美国整个政治文化中对国家、政府和权力抱有一种高度警惕的态度。很多美国人认为,国家随时都有可能干坏事,他们对政治的见解或许正如林达的书名——《总统是靠不住的》。所以,与政府有关的美国电影,在情节设计上跟中国电影有很大的不同。很多中国影片到最后,上头总归有一个好人。很多美国影片到最后,上头通常都有一个坏人。这是对国家、政府和权力的不同价值观——一种预设国家更多地会干好事,另一种预设国家更有可能会干坏事。

从更中立的视角看,国家的好处与坏处都是明显的。没有国家,就没有公共秩序,就没有基本安全,就没有国防力量,就没有最基本的公共基础设施。尽管私人部门也可以提供很多半公共品,但是总的来说,纯公共品主要依赖于国家提供。所以,对一个社会来说,没有国家是难以想象的。另一方面,国家的坏处也是明显的。国家天然地倾向于扩展其职能范围,想接管个人权利,想干预社会生活。国家或政府权力还经常容易被滥用,统治者和官员会因为拥有政治权力而变得腐败。这些都是可能的弊端。所以,一个好国家是有能力做好事、却没有能力做坏事的国家。

从美国小男孩的故事可以看出,美国的政治文化时时在提防国家和官员用权力干坏事。这样,他们更有可能从限制国家权力的角度去思考问题。但是,另一种政治文化里头,如果大家更多地想到国家的好处,甚至把国家和官员视为父爱主义者,那么国家和官员的政治权力从制度上就更有可能缺少制约。究竟孰优孰劣,留给大家自己去思考和评判。总之,不同的意识形态对国家的好处和坏处有着不同的判断。

第三个重要议题是自由及其限度。这里的自由是指政治自由,政治自由的谱系上存在两极:一极是自由至上主义(libertarianism),一极是极权主义(totalitarianism)。自由至上主义不是无政府主义,而是把国家限制在极小范围内。他们认为,凡超过这个必要的极小范围,都不是国家应该介入的领域。极权主义意味着国家试图利用政治权力控制一切,政治权力渗透到经济、社会乃至家庭等各个领域。在自由问题上,多数人的意识形态处在这一政治谱系两极中间的某个位置。

举例来说,国家应该不应该禁烟?有的女性厌恶二手烟,所以倾向于赞同国家完全禁烟。但同时还要问:抽烟者有没有抽烟的权利?如果赞同,为什么有这种权利?如果反对,为什么没有这种权利?还可以继续追问:如果有,行使这种权利有条件吗?比如,家里抽烟是否可以?街上抽烟是否可以?一般公共场所抽烟是否可以?在入住的酒店抽烟是否可以?一位大学教师上课抽烟不太好,但是下课后在操场上抽烟是否可以呢?这些问题其实不那么容易回答。

抽烟之所以成为一个有争议的问题,主要有两个原因。第一,抽烟容易影响他人。空气是到处流动的,空气产权很难界定。所以,一个人抽烟有可能伤害到他人。比如,两个人在一个房间里上班和工作。如果我抽烟你不抽烟,我抽烟时就可能侵犯了你的空气不被烟污染的权利。然而,问题的另一面是,我并没有故意要污染你的空气,而空气客观上是流动的。那么,这种情况下,我到底有没有抽烟的权利呢?第二,抽烟可能伤害自身健康。抽烟有害健康是共识。医学界普遍认为,抽烟会提高多种疾病及肺癌的发生率。那么,国家有没有权力出于对抽烟公民本身健康的考虑而禁烟呢?当然,禁烟之后,客观上更多人会有更好的身体,多种疾病及肺癌发生率会更低。但是,强调自由权利的观点认为,国家没有权力这样做,对这种事情的判断和决策权应该留给公民个人。所以,国家是否应该禁烟看似一个小问题,但我们都可以对此进行政治思考。这个问题的背后是自由的边界问题。

关于自由和权威的关系,英国思想家大卫 · 休谟曾写下一段非常经典的话,值得参考。他这样说:

\quo{在所有政府内部始终存在权威与自由之间的斗争,有时候是公开的,有时是隐蔽的,两者之中从无一方在争斗中占据绝对上风。在每一个政府中自由都必须做出重大牺牲,而限制自由的权威,绝不能而且也不应该在任何的政治中,成为全面专制,不受控制。必须承认自由乃文明社会的尽善化,但仍然必须承认权威乃其生存之必需。\nauthor{David Hume, \italic{Political Essays}, Cambridge: Cambridge University Press, 1994, pp.22-23.}}

再来探讨民主这一重要议题。有人支持民主,有人不那么支持民主,甚至反对民主。支持民主的理由是什么?反对民主的理由又是什么?关于民主,温斯顿 · 丘吉尔有一段广为流传的话:

\quo{没有人试图假装认为民主是尽善尽美或全知全能的,事实上,据说民主是最坏的政府形式——除了所有那些过去被反复尝试过的政府形式以外。\nauthor{原文是:No one pretends that democracy is perfect or all-wise.Indeed, it has been said that democracy is the worst form of government except all those other forms that have been tried from time to time。}}

丘吉尔的这段话后来经常被翻译为 “民主是最不坏的政府形式” 。实际上,这段话的意思是说 “民主是最好的政府形式” ——尽管民主并不完美。但是,支持民主观点成为一种流行见解,总体上是 19 世纪以后的事情。在此之前,很少有人为民主说好话,而且民主不乏强有力的批评者。在古希腊,一位被称为老寡头(old oliarchy)的人就对民主进行过有力的批评,柏拉图也倾向于批评民主,亚里士多德对民主亦无太多好感。反对民主的理由有很多,其中一个主要看法是,根据多数人的意见来决定一个国家的统治与公共政策不仅是不恰当的,而且可能是危险的。那么,支持民主的理由有哪些呢?主要观点有两个:第一,所有统治都应该基于被治理者的同意,而民主是实现被治理者同意的一种可操作方式。第二,在现有的人类政体类型选项中,找不到一个更好的选择。正如丘吉尔所言,民主固然不完美,但还有更好的制度选项吗?

当然,现在多数主流政治学家都是支持民主的,但不少人认为民主存在一定的弊端。那么,这些弊端能否克服呢?实际上,1787 年美国制宪会议就做出了一个很好榜样。制宪会议代表们要考虑的是如何既保证大众的权利,又恰当地限制大众的影响。他们最后在 1787 年《美国宪法》中设计了几个主要的制度安排。一是在美国总统选举规则中设计了选举人团制度,这实际上是一种间接选举制度。其早期操作是,先由拥有选举权的公民选举出选举人团,然后由选举人团自行决定给哪位总统候选人投票。制宪会议代表们认为,选举人团的政治智慧与政治技能要超过整个社会选民的平均水平,所以让他们来决定谁适合担任总统是一种更为稳妥的做法。当然,在美国后来的民主进程中,各州选举人团越来越根据本州选民的多数意见来投票,这样民主因素就得到了加强。

二是设立参议院。宪法规定,参议员由各州议会选举产生,任期 6 年。这也是一种政治精英选举政治精英的制度安排。参议院对应的是罗马共和国的元老院。要知道,一个人的任期越长,就越不容易受到一时情绪和情势的左右。

三是设立最高(联邦)法院。当然,最高(联邦)法院作为政治上具有重要影响的机构是慢慢发展起来的,其政治权力部分来自于宪法的授予,部分来自于其最初数代大法官的努力经营。后来,最高法院获得了制约行政权和立法权的力量。主要依靠违宪审查权,最高法院可以宣布总统、国会或各州的立法与决定违宪,从而使其丧失法律效力。最高法院这种巨大的政治权力也并非来自于民主的因素。

所以,美国宪法所确立的几种制度安排都带有强烈的精英主义民主色彩。罗伯特 · 达尔在《美国宪法的民主批判》一书中就认为,1787 年《美国宪法》还不够民主。实际上,左翼的政治学家希望寻求的政治平等是一种更加实质性的政治平等。不是说每个人都有投票权就够了,而是能否使每个人都获得同样或比较接近的政治影响力。平民主义民主理论和参与民主理论等,都更多地强调人民主权与民意因素。如果谁想要用精英力量来平衡大众权利、来限制大众民意,他们则认为这样的民主还不够民主。\nauthor{相关著作,参见罗伯特 · 达尔:《美国宪法的民主批判》,佟德志译,北京:东方出版社 2007 年版;罗伯特 · 达尔:《论政治平等》,谢岳译,上海:上海人民出版社 2010 年版。}但是,精英主义民主论者会认为,实质性的政治平等是不可能的——无论过去还是将来都不会实现。此外,让每个人都发挥同样的政治影响力的制度安排不仅是不可爱的,而且是危险的。所以,民主也是引发意识形态冲突的一个重要议题。

最后,还有两个重要的议题——平等以及政府和市场关系——前面已经有过较多的讨论,这里仅作简略分析。不同的意识形态对平等的看法差异很大。保守主义强调的是既有秩序和社会等级,所以保守主义并不热爱平等。自由主义强调的是自由优先,在自由主义框架中平等的价值显然要低于自由的价值。但与此同时,自由主义主张机会的平等和形式的平等,或者说是法律面前人人平等。当然,社会主义更多地主张结果的平等和实质的平等。所以,不同意识形态具有差异很大的平等观。此外,政府与市场的关系也是一个重要议题。政府与市场的边界在哪里?政府干预是否必要?如果政府干预确属必要,应该以何者为限?上文已有较多探讨,不再赘述。总之,更支持自由市场,还是更支持政府干预,代表了意识形态维度上的很大分歧。

\tsectionnonum{推荐阅读书目}

安德鲁 · 文森特:《现代政治意识形态》,袁久红等译,南京:江苏人民出版社 2005 年版。

《马克思恩格斯选集》,北京:人民出版社 1995 年版。

李强:《自由主义》,长春:吉林出版社 2007 年版。

杰里 · 马勒:《保守主义:从休谟到当前的社会政治思想文集》,刘曙辉、张容南译,南京:译林出版社 2010 年版。
