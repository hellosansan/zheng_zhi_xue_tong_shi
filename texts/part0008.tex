\tchapter{政府结构与政治制度}

\quo[——胡安·林茨]{议会制民主政体在历史上表现更加出色绝非偶然。对议会制和总统制进行认真的比较以后会得出结论:总体上前者较后者更有利于民主政体的稳定。这一结论尤其适用于存在深刻政治分裂和众多政党的国家;对这些国家来说,议会制更有可能维系民主政体。}

\quo[——乔万尼·萨托利]{(论极化多党制)一个以离心驱动力、不负责的反对党和不公正的竞争为特点的政治制度很难说是一种可行的制度。……这并不必然意味着极化政体注定是软弱的且最终是自我毁灭。然而,它们却难以应对爆炸性的或源自外部的危机。}

\quo[——莫里斯·迪韦尔热]{(迪韦尔热定律)(1)比例代表制倾向于导致形成多个独立的政党……(2)两轮绝对多数决定制倾向于导致形成多个彼此存在政治联盟关系的政党;(3)简单多数决定制倾向于导致两个政党的体制。}

\quo[——罗伯特·达尔]{简而言之,如果国家的基础条件同时存在有利与不利的情况,一个好的宪法设计就会有利于民主制度的生存;反之,一个坏的宪法设计可能导致民主制度的崩溃。}

\tsection{如何理解政府机构?}

美国是典型的三权分立国家,很多介绍政府机构的著述喜欢从美国讲起。众所周知,美国联邦政府有三个主要的政府机构:白宫、国会与联邦法院。白宫是美国的总统府,它是世界上最有权力的机构之一。白宫的决定将直接影响到美国和整个世界。白宫其实是一个规模不大的地方,但美国很多最重要的政治决定都是在那个地方做出的。当然,美国白宫只是美国政府的一个符号,它实际上领导着美国国务院、财政部、国防部、内政部等大量联邦行政机关。另一个重要机构是美国国会,美国国会以一幢白色圆顶建筑闻名于世,世人称其为“国会山”。这是美国参议院和众议院的办公场所,100个参议员和435个众议员在那里办公。第三个主要政府机构是美国联邦法院,它是美国最高司法机构,其主要人物是9位联邦最高大法官。

白宫、国会与联邦法院是美国三个最重要的政府机构。白宫拥有行政权,国会拥有立法权,联邦法院则拥有司法权。那么,世界各国的政府机构设置都跟美国相似吗?当然不是。比如,英国不仅没有总统,而且其最高法院也很少有人提及——除非是英国政治与司法问题的专门研究者。所以,英国的政府机构跟美国就有显著差异。至于日本、俄罗斯、印度和中国等国的政府机构,跟美国的差异也是相当之大。

一个国家政治生活的重要方面就是政府机构的设置,政府机构设置还对应着一整套政治制度的安排。一国政府机构设置的不同意味着该国政治制度安排的不同。政治学还关心政府机构和政治制度安排的不同会造成何种不同的政治效应?这是一个重要问题。

从概念上说,政府是制定和实施公共决策与政策的机构,政府履行着国家的基本职能。大家还经常提到政府机构,比如上海市卫生局。大家一般不说上海市卫生局是一个政府,而说是一个政府机构。政府正是由很多不同类型和层级的政府机构组成的。从政府机构类型来说,可以从两个维度进行分类:一个是职能维度,一个是层级维度。

从职能维度来说,政府机构主要有三类:行政机构、立法机构和司法机构,其中行政机构的规模通常是最大的。凡是有国家和政府的地方,必定存在行政机构,而后面两种机构——功能分离且相对独立的立法机构和司法机构——的产生则是特定环境下政治演进的产物。但是,对政治权力三种职能划分的见解,古希腊早已有之。亚里士多德就区分了政府的三种职能,即行政、立法和审议,对应的是三种类型的政府机构。审议机构在今天看来,可以大致归类为司法机构。

到了近代,英国思想家洛克认为政府有三种重要的权力:立法权、行政权和外交权。现在通常认为,外交权从属于行政权,所以洛克实际上提出了两权分立的概念,即行政权和立法权应该分开。后来的法国思想家孟德斯鸠则第一个系统地阐述三权分立的思想,即政治权力划分为立法权、行政权和司法权。孟德斯鸠认为应该实行三权分立与制衡,他最著名的论断是:当立法权、行政权和司法权集中在同一个人或同一机关之手,自由便不复存在了。

孟德斯鸠尽管提出了三权分立的思想,但当时世界上并没有一个国家的政府体系是按照这种方式来构建的。由于美国制宪会议和美国联邦党人的努力,这样一种政府体系的理想类型首先在美国成为现实。当然,美国1787年《宪法》所确定的只是文本意义上的三权分立体制,跟后来实际演进过程中的三权分立体制存有差异。比如,美国总统和国会的关系是后来慢慢地演化为今天的样子。再比如,美国的最高法院从建立到今天,实际上经历了一个司法权不断扩张的过程,从而使得最高法院最终拥有了很大的政治权力。

下面先简要介绍行政、立法与司法三大政府机构。一是行政机构。行政机构是政府的核心。为什么说它是政府的核心?因为一个政府可以没有立法机构、可以没有司法机构,但是一个政府必须要有行政机构。古代的官僚制帝国或王国,可能并没有正式的立法机构或司法机构——当然很多政府职能是融合在一起的——但它必须要有正式的行政机构。如果没有行政机构,国家就不成其为国家,政府就不成其为政府。因此,行政机关通常被视为是政府的核心,是具体制定和实施公共政策的部门。

对行政机构来说,除了个别国家的最高行政长官职位实行委员会制度以外,绝大多数国家都拥有单一最高行政长官。这个最高行政长官的头衔通常是总统、总理、首相或主席。在行政机构内部,通常都有严格的上下级等级关系。拿美国上届政府来说,担任国务卿期间的希拉里·克林顿尽管是一位非常强势的政治家,但只要美国总统奥巴马出席的活动,国务卿希拉里·克林顿讲话就非常低调,她深知整个活动重要的主角是总统奥巴马,因为奥巴马是她的上司。尽管希拉里一度是美国民主党党内奥巴马总统提名的竞争对手,但她一旦落选,并被总统奥巴马任命为国务卿,她就是在接受奥巴马的委托,应该服从奥巴马的命令,并履行奥巴马希望她履行的职责。作为美国国务卿,希拉里固然在为美国人民做事,但是她的工作直接服务于美国总统,因为是美国总统需要对选民负责。所以,从希拉里和奥巴马的关系中,可以看出行政机构是按照上下级科层制方式来组织和构建的。

在朝鲜战争期间,美国驻日本的最高司令官麦克阿瑟将军给美国总统杜鲁门提出建议,认为朝鲜半岛的地面战争很难能赢,除非与中国开战。麦克阿瑟不止一次跟白宫沟通这个想法,但当时杜鲁门总统担心,如果美国与中国开战,苏联将会参战,第三次世界战争可能马上爆发,而杜鲁门并不愿意出现这种局面。所以,他最后只能撤换麦克阿瑟将军,理由很简单:你既然不能服从我的命令,执行我的作战思路,我只能撤换你。从这个案例也可以看出,行政机构内部的上下级等级关系是非常清楚的。

二是立法机构。立法机构一般称为国会、议会、国民大会或代表大会,或直接叫立法机构。立法机构是指审议和批准法律及公共决策的机构,它是一个由较多成员组成的代议机构。中文的“代议”大概是两个意思:代表和议事。所以,立法机构就是一个代表的议事机构。那么,他们代表的是谁呢?各个议员代表的是他们各自背后的选民。选民们投票让他们到这个地方来,所以议员们代表的是那些支持和选举他们的人——就是那些投票让他们到华盛顿来的人,投票让他们到伦敦来的人,或投票让他们到巴黎来的人。同时,立法机构还是一个议事机构。代表们在这里通过审议、辩论、表决来决定法律与决策的通过与否。与行政机构不同的是,立法机构内部的所有成员通常都是平等的。所以,立法机关内部不像总统或总理领导的政府内部有等级森严的上下级关系,所有议员或代表之间是更平等的关系。当然,实际上,一部分资深议员或政党领袖的实际政治影响力会比较大。

三是司法机构。司法机构是指维护法律、确保法律执行以及解决法律争议的政府机构。司法机构通常由不同层级的法院构成,法院的主要工作人员是法官。如果一国的法治程度较高,法官应该是不存在上下级等级关系的独立工作人员。

上面讨论的是政府机构的职能分工,另一个方面是政府机构不同层级的划分。要知道,世界上只有极少国家是由一级政府管理的,比如像梵蒂冈这样的国家,该国总面积不过0.44平方公里。这么大一个国家,一级政府机构就可以了。但是,大部分规模较大的现代国家都至少需要两级政府机构来管理。对于规模较大的国家,通常会有三级政府机构来管理。最高层级的政府机构被称为中央政府,有些联邦制国家称为联邦政府。美国的联邦政府设在华盛顿,英国的中央政府设在伦敦,中国的中央政府设在北京。第二层级的政府机构是次国家(sub-national)层次的政府机构,最常见的名称是省政府、州政府或邦政府。此外,如果国家比较大的话,省、州或邦之下还会有第三层级的地方政府。不少国家只设一级地方政府,但有些国家由于国土面积过大或治理模式的原因,会设两级或两级以上的地方政府。比如,中国实际的政府层级包括:中央政府、省(直辖市、自治区或特别行政区)政府、地级市(行署)政府、县市政府、乡镇(街道)政府这五级。乡镇或街道之下,还有部分履行政府职能的村或居委会一级自治组织。所以,中国政府层次的数量要比美国更多。

在多数国家,不同层级的政府机构之间不仅是管辖范围的差异,而且职能上也存在显著的分工。比如,以早期的美国为例,当时普遍认为联邦政府主要是解决外交、防务、州际贸易及其他全国性问题,其他大量的公共事务则交由地方政府去处置。地方的道路、本地的公共基础设施及公立学校等,都应该交给地方政府去管理。像美国这样的国家,在19世纪,中央政府履行的职能非常之少,中央政府人员规模也非常之小,一方面是因为政府职能范围本身比较有限,另一方面是因为很多政府职能是地方政府在承担。

到了20世纪,西方国家政府机构演进的基本特点是:首先是政府职能范围与规模(相对于市场和社会)的大幅扩张,其次是中央政府职能范围与规模(相对于地方政府)的大幅扩张。比如,以政府收入占GDP的比例为例,一战以前西方主要国家大致停留在10%左右的水平上,而今天西方主要发达国家平均已经达到30%—50%。在这一过程中,中央政府占整个政府收支的比例也在不断提高。以美国为例,1929年联邦政府占所有政府支出的比例仅为17%,2009年则增至54%。其他西方发达国家的总趋势也大体如此。如今,很多国家都是中央政府对地方政府实行大规模转移支付政策,这无疑扩大了中央政府职能的范围与规模,也增加了中央政府对地方政府的影响力。

\tsection{政治系统与官僚系统的比较}

关于政府机构,要区分具有较强政治色彩的狭义的政治系统和负责具体行政事务的行政系统或官僚系统。通常,政治系统和官僚系统的差异是明显的。美国政治学者弗兰克·古德诺在《政治与行政》一书中说:“政治是国家意志的表达,行政是国家意志的执行。”政治系统与官僚系统的差异,可以参见表6.1。

\tbl{../Images/image00308.jpeg}[表6.1 政治系统与官僚系统的差异]

从定位来说,政治系统更多是跟政治权力获取有关的。比如,两个人竞争一个地方议员的席位,是一个政治过程。而官僚系统更多是跟公共政策执行有关的。比如,卫生局开会商讨如何应对禽流感疫情,这里更多是政治权力如何执行、具体公共事务如何管理的问题。所以,政治系统强调政治权力的获取和公共政策的制定——尽管有时也涉及公共政策的执行;官僚系统强调政治权力的行使和公共政策的执行——尽管有时也涉及公共政策的制定。需要注意的是,不同国家政治系统与官僚系统的角色分工有所不同。比如,像日本这样的国家,官僚系统的主导型要比美国更强一些,很多重要公共政策是在官僚系统内部制定的。所以,有人开玩笑说,日本一年换几个首相都没有关系——只要日本政府的处长们依然在努力工作。还有一种调侃的说法,把日本的决策模式称为“处长治国”。这种说法并不严谨,但有些道理。日本有大量政策,比如产业政策,都是一批处长级别的专业官僚——而不是一批当选的政治家——提出建议、提供草案、提交报告,然后进入决策或立法流程的。主管内阁大臣看完报告和政策建议后,与内阁协商并经首相同意,或再经国会批准,政策就这样定下来了。所以,在日本这样的国家,专业官僚完全可能会涉及公共政策的制定,而不只是公共政策的执行。大家会发现,不同国家政治系统与官僚系统的定位和关系并不完全相同。

从身份来说,在政治系统中工作的核心人员通常被称为政治家或政务官。在美国,这是随着选举结果的不同而更迭的那一部分人,在联邦层面包括总统、联邦参议员、联邦众议员以及美国总统任命的包括国务卿、财政部长、驻外大使等在内的一大批高级官员。当然,在美国的政府体系中,总统任命的人数相对是比较多的。而在另外一些国家,总统或首相任命的官员数量可能相对较少。在官僚系统里工作的人员通常被称为公务员、官僚或事务官。英国的文官系统用政务官和事务官来区分上述两种身份角色。需要提醒的是,这里的官僚是中性词,就是指行政系统内部的公务人员。

从录用方式来说,两者也是不同的。民主政体下的政治家职位或政务官的产生主要有两种方式:一种是选举的方式;另一种是经由选举产生的主要行政官来任命,亦政治任命,比如总统任命国务卿、部长或驻外大使。此种政治任命往往也是政治家贯彻自己政治意图和政策的重要方式,比如国务卿应该要贯彻总统的外交立场,财政部长应该要执行总统的财政政策。官僚系统的公务员与事务官的录用一般是考试和升迁的办法。在官僚系统中,比如英国、美国或很多西方发达国家,公务员序列一般最高能晋升到副部长这一级,这是事务官的最高行政级别。副部长以上的部长和内阁成员则是由总统或首相任命的,或直接由选举产生。这种级别的政务官一般都有政党身份,主要取决于政治竞争的结果。

从主要规则来说,政治系统强调的是回应性。政治家要对选民具有回应性。谁投票让政治家到华盛顿来、到伦敦来,政治家就应该对他们具有回应性。回应性某种程度上也跟政治系统的竞争规则有关。回应性对应的是问责制,问责制的直接表现是:如果选民们觉得某个政治家干得不好,这个政治家通常会输掉下一轮选举。官僚系统更多强调服从和科层制原则。上面怎么决定,下面就应该怎么执行,这是服从原则。整个官僚系统则是按照这种科层制的方式来组建的。按照德国著名社会学家马克斯·韦伯的论述,官僚制具有如下基本特征:

\quo{第一,一个正式的等级制结构。每个等级控制其下的等级并为其上的等级所控制。一个正式的等级制是中央规划与集中化决策的基础。

第二,用规则来管理。经由规则控制的系统使得高层决策能够被较低层级保持一致地执行。下级的基本职责就是服从上级的命令并执行上级的决策。

第三,功能专业化的组织。任务由专家来完成,所有人都基于他们任务的类型或专长技能进行分类组织。

第四,两种使命类型:关注上面或者关注内部。前者的使命是服务于股东、董事会或上级授权机构,后者的使命是服务于该组织本身。

第五,非人格化。对所有雇员与客户一视同仁,而不会受到个人差异的影响。

第六,雇佣基于专业能力。这意味着雇佣不是基于人的天然身份或人际关系,而是专业能力。\nauthor{韦伯论述官僚制的相关内容,参见马克斯·韦伯:《马克斯·韦伯社会学文集》,阎克文译,北京:人民出版社2010年版,第188—230页。}}

那么,政治系统与官僚系统之间是什么关系呢?政治系统要有效,很大程度上依赖官僚系统的有效性。在政治过程中,实际做事的是官僚机构,政治家把握的是方向和目标,政治家的意图要依赖官僚系统去贯彻和执行。如果官僚系统有效性很低,就会导致整个政治系统出现问题,整个国家的治理绩效就不会太好。另一方面,官僚系统则依赖于政治系统的指导。事务官只能到副部长这一级,政策方向与目标应该是部长、内阁及政府首脑来把握的。当然,在一些特殊领域,两者的关系更为复杂。如果某个领域的专业性程度非常高,政治家可能不得不更多地依赖专业官僚给出建议。比如,对美国来说,当年面对伊拉克或阿富汗的局势,究竟是否要采取军事行动?这里既有政治决策,又要听取美国军方的建议和意见。所以,一般来说,官僚系统依赖于政治系统的指导,但有时因为专业性的原因,官僚系统也可能会影响政治系统的决策。

\tsection{政府形式:议会制、总统制与半总统制}

政治机构有职能和层级的划分,政治制度也可以区分为不同的层次。对一个国家来说,有四个主要层次的政治制度。第一,是政府形式。对民主国家来说,狭义的政府形式通常是指议会制、总统制与半总统制。第二,选举制度。民主政体的主要选举制度包括多数决定制、比例代表制与混合制等。第三,政党体制。民主政体的政党体制主要包括两党制与多党制,也包括一党独大制,多党制还可以区分为温和多党制与极化多党制。政党体制对整个政治生活的影响通常是重大的。第四,央地关系。两种常见的制度安排是联邦制和单一制,当然还有两者之间的混合形式。与联邦制和单一制这样的制度标签相比,更重要的是中央与地方之间实际的分权安排。

先介绍政府形式。狭义的政府形式是指立法机构和行政机构的关系,主要有三种类型:总统制、议会制和半总统制。那么,三种政府形式的差异是什么?是否存在某种最佳的政府形式?或者说,何种政府形式最适合某个特定的社会?这些都是重要问题。在现实世界中,不同的政府形式既有成功的例子,又有失败的例子。总统制国家的典型是美国,议会制国家的典型是英国,半总统制国家的典型是法国。这三个国家尽管政府形式不同,但整体的政治和治理状况都比较好。但也有相反的例子。比如,大家可以看到拉丁美洲曾出现过很多总统制政体失败的案例,还可以看到其他地区议会制民主运转不灵的情形,半总统制也在一些国家引发过剧烈的冲突。为了比较三种政府形式的优劣,先要弄清楚它们各自的特点。

首先来看总统制。在总统制条件下,选民选举立法机构,即一院制或两院制国会;同时选举总统,民选总统选择与任命内阁部长并领导内阁管理政府部门。总统制的主要特征是:

第一,行政机关和立法机关均由民选产生,民选总统是政府首脑。

第二,总统任期与国会任期固定,彼此互不统属,互相均不能推翻对方。

第三,总统任命与指导内阁,并具有宪法承认的部分立法权。

总统制的政治结构可以参见图6.1。这张图简略地表示了总统制的主要特征。

\img{../Images/image00309.jpeg}[图6.1 总统制的政治结构]

资料来源:罗德·黑格、马丁·哈罗普:《比较政府与政治导论》,张小劲等译,北京:中国人民大学出版社2007年版,第381页,图15-1。

再来看议会制。议会制下,选民选举议员组成一院制或两院制立法机构,然后由立法机构(通常是下院或众议院)选举或任免首相及内阁。纯粹的议会制的主要特征是:

第一,立法机关由民选产生。

第二,由首相(或总理)与内阁成员构成的行政机关来自于立法机关。

第三,立法机关多数通过“不信任投票”可以罢免行政机关。

议会制的主要结构参见图6.2。这张图简略地表示了议会制的主要特征。在该图中,首相或总理在形式上是由国家元首(国王或总统)任命的,但这是礼仪性的。

\img{../Images/image00310.jpeg}[图6.2 议会制的政治结构]

资料来源:罗德·黑格、马丁·哈罗普:《比较政府与政治导论》,张小劲等译,北京:中国人民大学出版社2007年版,第384页,图15-2。

简要介绍总统制和议会制的政治结构后,一个有趣的问题随之而来:美国总统与英国首相,谁的政治权力更大?谁更有政治权威?有人认为,美国总统的政治权力更大。理由是总统由选民选举产生、直接对选民负责,来自于全民的授权,具有很高的合法性。但是,这种看法存有争议。从具体的权力行使过程看,总统制下的总统时时受到立法机构的制约。在议会制下,行政机关(内阁)与立法机关是高度融合的,英国首相通常由议会多数党领袖出任。这样,首相想要通过某项预算、法案或重要决定时面对的阻力反而会更小。因而,首相的政治权力与政治权威反而可能更大。相反,在总统制下,总统的预算、法案和重要决定如何能通过立法机构的批准,这是一个巨大的挑战。比如,2013年美国总统奥巴马就在两个问题上面临压力:一是医疗改革法案,二是预算与公债上限法案。所以,总统与国会之间可能会产生紧张的政治对抗关系。当然,在美国,总统所在的政党同时控制国会参议院和众议院多数席位时,总统的政治权力就会非常巨大。但是,这种情形并不经常发生。

最后来看半总统制。半总统制下,选民同时要选举立法机构和总统,总统任命总理及各部部长,但总统任命总理时必须要得到立法机构半数以上的支持。半总统制的主要特征是:

第一,总统与立法机构均由民选产生。

第二,总统拥有巨大的宪法权威,可以任免首相(或总理)与内阁。

第三,首相(或总理)与内阁必须要得到立法机关多数的信任。

半总统制的主要结构参见图6.3。这张图简要勾勒出半总统制的主要特征。总的来看,半总统制某种程度上是总统制与议会制的结合。因此,半总统制既可能集中了总统制和议会制的优点,又可能集中了总统制和议会制的缺点。

\img{../Images/image00311.jpeg}[图6.3 半总统制的政治结构]

资料来源:罗德·黑格、马丁·哈罗普:《比较政府与政治导论》,张小劲等译,北京:中国人民大学出版社2007年版,第394页,图15—4。

半总统制之下,如果总统和国会多数党或政党联盟同属一党,立法与行政之间的结构性冲突通常较小;但如果总统与国会多数党或政党联盟不是同属一党,两者的结构性冲突可能会很激烈。总统提请国会任命总理的人选,国会既可能同意,又可能不同意。半总统制下,如果总统任命的总理人选无法在国会得到多数支持,就可能会变成一个政治僵局。这里的关键问题是:总统提出一个怎样的人选能在国会获得多数支持呢?一个可能的答案是国会多数党或多数政党联盟的主要领袖。

所以,在政治实践中,成熟的半总统制民主国家——特别是法国,总统完全可能任命国会多数党或政党联盟主要领袖出任总理。这也是法国目前形成“左右共治”模式的制度原因。在总统大选中获胜的一位右派总统可能任命一位在国会拥有多数席位的左翼多数党领袖出任总理职位。这样,就出现了“左右共治”格局。这是半总统制情况下可能会出现一些情况。但是,如果是其他国家,这种格局不排除会引发比较严重的政治冲突。历史上德国魏玛共和国(1919—1933)民主政体的垮台就跟半总统制的政府形式有关,2013—2014年乌克兰的政治危机某种程度上也与半总统制的架构有关。\nauthor{关于三种政府形式的主要特征,参见Matthew Soberg Shugart,“Comparative Executive-Legislative Relations,”in R.A.W. Rhodes, Sarah A. Binder and Bert A. Rockman, \italic{The Oxford Handbook of Political Instituions}, Oxford: Oxford University Press, 2006, pp.344-365。}

举例来说,美国是典型的总统制国家。尽管宪法规定美国总统由选举人团选举产生,但目前的实际做法与选民直接选举产生无异,任期四年,可连任一次。民选总统任命内阁成员并领导政府,内阁部长是由总统任命的。但在美国,内阁部长的任命还需要经过国会的批准。当然,不同国家的总统制在这种人事任命制度上的具体安排是不一样的。美国总统拥有强大的行政权,但实际上又处处受到国会的制约。美国总统在预算、重要人事任命、法案及重要决策上都需要跟国会协商,需要经过国会的批准与同意。总统和议会任期固定,并且互相均不能推翻对方以垄断全部的权力。大家有没有听过美国总统解散议会呢?没有。大家有没有听过美国国会因为政策分歧而罢免总统呢?没有——除非是美国总统违宪而遭到弹劾。从制度上说,总统和议会都有固定任期,而且行政部门和立法机构成员之间不能互相重叠。在美国,如果希拉里要做国务卿,就需要把参议员的席位辞掉。因此,总统制是非常典型的立法权和行政权分立的制度,再加上政治上具有很强独立性的法院与司法权,美国是典型的三权分立制度。

美国国会实行两院制,拥有立法权,有权批准美国政府的预算、部长人选及法案。众议院由美国选民选举产生,任期两年,各州议员数量按照人口比例确定,共435人;参议员由美国各州议会选举产生,任期为6年,每两年改选1/3左右,每州2个席位,共100人。一般认为,美国参议院代表的是各州,众议院代表的是美国人民。在总统与国会的权力关系上,没有一方能够支配另一方。在批准财政预算、重大人事任免和立法方面,美国国会的权力是非常大的。

英国是典型的议会制模式。英国国会分为上议院和下议院,即贵族院和平民院,英国的两院制是历史演进的产物。在两院中,下议院掌握主要的政治权力,赢得下议院选举多数席位的政党组成政府,党魁一般出任首相,并挑选20多名议会同僚组成内阁。内阁来源于国会,内阁部长通常仍然是立法机构成员。与总统制不同,内阁是通常同僚合作型的。在内阁制下,首相或者总理通常是所有平等内阁成员中的第一个(first among equals)。换句话说,英国首相和美国总统的不同在于,英国首相某种程度上算不上是内阁部长们的老板。英国内阁成员之间更多的是一种合作关系,而首相只是其中为首的一个。

在议会制下,首相和内阁需要对议会负责,议会多数的不信任票可以解散内阁。英国国王是荣誉元首,国王任命首相及定期会晤首相的制度是礼节性的,国王并不掌握实际的政治权力。如果研究英国的王室历史,就会发现英国国王从17世纪到19世纪基本上是一个逐步去行政化的过程,后来国王不再掌握实际的政治权力。

法国是典型的半总统制国家。法国总统具有广泛的政治权力,担任三军总司令,有权提议全民公决,有解散议会和实行紧急状态的权力。实际上,这里的全民公决制度是为总统和议会可能的冲突所准备的。半总统制下,总统和议会都有最高的合法性与最高的政治权力,如果两者发生冲突就会导致政治僵局,全民公决就可能成为一种有效的调停机制。总统由直接选举产生,现在的任期为五年,并可连任一次。法国总统还任命总理。通常,总统和总理存在着明确的分工,法国总统往往对外交事务和国防等负有特殊责任,而法国总理大体上不怎么管外交,主要负责指导政府的日常事务。所以,法国总理和英国首相角色很不一样,法国总理更多是一个总管的角色,而且他经常需要去应付议会。从制度安排上说,总理要对议会负责,议会不信任票可以迫使总理和内阁辞职。所以,为了总理能够顺利地履行职责,法国总统后来就干脆跟国会多数派政党协商确定总理人选。

结果,法国政治的重要特色是“左右共治”。可以想象,这种“左右共治”可能的问题是政府很难实行比较强硬的政策,因为这种模式下政策都是不同力量妥协的产物,需要兼顾各个不同社会集团的利益。如果一个社会问题较少的时候,这种模式能够促成社会和谐。但是,如果一个社会面临比较严重的问题,比如内部冲突或财政危机,要想实施社会或经济改革,难度就非常大。这种模式下,改革的力量通常比较弱,因为它时时需要平衡各方的政治力量。

\tsection{议会制“大战”总统制}

从20世纪90年代开始,国际学术界出现了一场总统制与议会制的大论战,本书称之为议会制“大战”总统制。现代政治学分析不同的政府形式与政治制度通常需要考虑两个维度:一个是分权制衡维度,一个是政府效能维度。实际上,政府形式与政治制度既需要强调分权制衡,又需要注重政府效能。在这两个维度上,大家熟知的三权分立学说强调的是前者,即分权制衡的因素。有学者把自启蒙运动时代以来洛克、孟德斯鸠及麦迪逊等人强调的这种观点,称为西方政治传统中的分权学说。但其实,西方政治传统中还有另外一种传统,即强调政府效能的因素。比如,汉密尔顿在《联邦党人文集》的多处都强调政府效能的重要性,以及强有力的联邦政府的必要性。实际上,《联邦党人文集》倡导的是分权制衡和政府效能两者之间的平衡。

英国著名法学家白芝浩在《英国宪法》一书中专门讨论了英国议会制与美国总统制的优劣。他并不看好美国的总统制,认为美国的总统制是有问题的,而英国的议会制优点比较突出。他这样说:

\quo{在一个主要的方面,英国的制度远胜于美国。由议会产生并可由这个立法性机构中占多数席位的党派撤换的英国首相肯定依凭于这个议会。如果他想让立法机构支持他的政策,他就能够得到这种支持,并进而推行他的政策。但美国总统得不到这种保证。总统是某个时候用某种方式产生的,而国会(无论是哪一院)是在另外某个时候用另一种方式产生的。二者之间没有什么东西将其捆绑在一起,且从事实上讲,二者之间不断地产生冲突。\nauthor{沃尔特·白芝浩:《英国宪法》,夏彦才译,北京:商务印书馆2005年版,第42—43页。}}

有人说白芝浩是英国人,偏爱英国议会制并不奇怪。但后来一位非常著名的美国人也对白芝浩观点深表赞同,这位大人物是身兼美国总统和政治学家双重身份的伍德罗·威尔逊。他也认为美国的总统制不如英国的议会制。

到了20世纪90年代初,随着第三波民主化的进展,胡安·林茨重新挑起了议会制与总统制之争。1990年,他先后发表了《总统制的危害》和《议会制的优点》两篇论文。\nauthor{Juan J. Linz,“The Perils of Presidentialism,”\italic{Journal of Democracy}, Vol.1, No.1, Winter 1990, pp.51-69; Juan J.Linz,“The Virtues of Parliamentarism,”\italic{Journal of Democracy}, Vol.1, No.4, Fall 1990, pp.84-91.}林茨在《总统制的危害》一文中认为:

\quo{议会制民主政体在历史上表现更加出色绝非偶然。对议会制和总统制进行认真的比较以后会得出结论:总体上前者较后者更有利于民主政体的稳定。这一结论尤其适用于存在深刻政治分裂和众多政党的国家;对这些国家来说,议会制更有可能维系民主政体。}

他认为,总统制存在五个严重问题。首要的问题是总统和议会双重合法性的冲突。在总统制下,总统拥有最高的政治权力与合法性,议会也拥有最高的政治权力与合法性。当这两种政治权力不一致的时候,就会产生双重合法性的冲突。议会认为,他们拥有最高的政治权力。总统认为,他才是最高政治权力的所有者。举例来说,总统想要通过预算法案,但议会就是不让它通过,怎么办呢?如果是在美国,由于美国民主政治已运营两百多年,民主、共和两党之间已存在政治默契,到了最后关头两党往往愿意做出妥协和让步。但是,并不是所有国家都能解决这样的问题,特别是发展中世界的新兴民主国家。比如20世纪70年代初的智利,阿连德就任智利总统的1970—1973年间,他希望把整个国家社会主义化,包括土地再分配、矿产国有化、银行和大型企业收归国有,等等。但是,智利议会同意这样干吗?结果,议会和总统陷入了激烈的政治冲突。最后,总统就绕开议会,以紧急状态令的方式强行推行他的改革计划。这样,议会认为总统的做法违宪,两者的政治冲突已经陷入不可调和的境地,最后的结果是军事政变。林茨把总统与议会之间的这种冲突称为双重合法性的冲突。

其次的重要缺陷是总统的固定任期,这种固定任期或可导致政治僵局。总统任期是固定的,通常是四年或五年。那么,这会产生什么问题呢?比如,议会里有两个主要政党A党和B党,其中一党获得了略高于50%的席位,另一党获得略低于50%的席位。如果总统跟议会多数党属于一党,这种时候比较好办,总统的大部分法案都能通过。但是,如果议会多数党是总统的反对党,这个时候两者就容易产生冲突。由于总统任期固定,而且通常长达四或五年,就容易导致长期的政治僵局。如果不是两党制而是多党制,这个问题只会更加严重,原因在于总统所属的政党通常只有一定比例的议会席位。那么,在议会制下,如何解决这个问题呢?当总理和内阁不能得到议会多数支持时,总理和内阁就只能去职,进行内阁的重新选举,这样行政机关就能进行重新调整。这展示了政治上的灵活性。

第三个问题是总统制下更容易出现“赢家通吃”与“零和博弈”的局面。通常在议会制下,总理或首相的职位及内阁相关职位固然可能是一党主导,但很多时候也是不同政党妥协的产物。在这种制度安排下,不同的政治力量可以分享政治权力。但是,在总统制下,由于总统职位的惟一性以及选举方式的限制,结果往往是一个政治家或一个政党实质性地控制了行政权,这就更容易导致“赢家通吃”。在政治竞争中,关于总统职位的选举则更接近于“零和博弈”。这可能引发更大的政治冲突,更不容易激励政治合作与妥协。

第四个问题是总统制更不容易宽容反对派,这是林茨经验观察的结果。与议会制下的总理或首相相比,民选总统在执政过程中更有可能对反对派采取激烈和极端的做法,因为他自以为全民选举总统赋予了他更大的合法性与政治权威。但这更容易激化政治冲突。

第五个问题也非常重要,总统制下政治新星快速崛起的可能性更大,这种情况下更容易导致政治不稳定。法国历史上的第二共和国就实行总统制。1852年,法兰西第二共和国要进行第一次总统直接选举,选民是数百万法国成年男性公民。在这场选举中,巴黎有一位具有重要影响力的政治家卡芬雅克的呼声很高。当时,巴黎的政界、工商界、知识界和中产阶级中的很多人认为,他应该会当选法国总统。但就在此时,拿破仑的侄儿路易·拿破仑·波拿巴刚从英国流亡返回法国,他在巴黎政治生活中不过是一个不起眼的小角色,但他有一个非常显赫的姓氏,他被视为伟大的拿破仑皇帝的继承者。他决定也要参加法国总统的角逐。当然,他的参选也获得秩序党的支持。当时的情形是,法国各省的几百万农民刚获得投票权,他们根本就没有听说过巴黎那些赫赫有名的人物。所以,一开始投票,全国选民的绝大多数选票——5 434 000张选票——都投给了波拿巴,而卡芬雅克仅仅得到了1 448 000张选票。结果是,卡芬雅克这位声望极高的巴黎政治家根本无缘总统职位,而波拿巴这颗政治新星一举成为法国总统,史称拿破仑三世。那么,当这样一位政治新星快速崛起之后,他会做什么呢?这是更难以预见的事情。实际上,他当选总统之后不过几年,就把法兰西第二共和国搞成了法兰西第二帝国。这无疑增加了政治系统的不稳定性。而在议会制下,这种政治新星快速崛起的事情就更难发生。

《总统制的危害》一文发表后,学术界出现了很多支持林茨观点的学术文献。比如,美国学者约瑟·柴巴布的数量研究显示,议会制民主政体平均58年崩溃一次,每年崩溃的可能性是1.71%;而总统制民主政体平均24年崩溃一次,每年崩溃的可能性是4.16%。从两个数据的比较看出来,议会制远比总统制更加稳定。另一个数据是从1946年到2002年,全球共发生157次政体变更,而拉美一个地方就占到58次,比例为37%,而拉美大部分国家都是总统制。所以,这些经验研究的证据都支持林茨的观点,总体上讲与议会制相比,总统制更不稳定。\nauthor{Jose Antonio Cheibub, \italic{Presidentialism, Parliamentarism, and Democracy}, Cambridge: Cambridge University Press, 2007.}

柴巴布还用一个图来表示总统制民主政体不稳定的政治逻辑,参见图6.4。他认为,总统制本身意味着权力分立——即立法权与行政权的分立。通常,总统制并没有为形成政治联盟提供有效激励;而且总统制下的政党纪律往往并不严格,也导致政党力量总体上比较弱。当这种情形与多党制相结合时,就容易出现少数派政府。这意味着总统在议会中只拥有不到50%的支持率。少数派政府是一个立法权与行政权分裂的政府,从立法角度看容易成为一个无效的政府。比如,行政机构与立法机构很容易就关键人事任命、重要法案及重大公共政策发生冲突。当总统的提案常常无法在议会通过时,就陷入了某种程度的“僵局”。有人把这种行政权与立法权之间的僵局称为“宪法危机”。如果总统和议会之间的政治僵局或宪法危机长期持续,最后会导致民主政体无法有效运转,甚至就会导致民主政体的崩溃。

\img{../Images/image00312.jpeg}[图6.4 从总统制到民主崩溃]

资料来源:Jose Antonio Cheibub, \italic{Presidentialism, Parliamentarism, and Democracy}, Cambridge: Cambridge University Press, 2007, p.8,figure1.1。

尽管如此,林茨的观点也遭到很多挑战。美国政治学者唐纳尔德·霍洛维茨认为,林茨这项研究的最大问题是样本选择的地区偏差。议会制民主政体主要集中在欧洲,总统制民主政体主要集中在拉丁美洲。由于存在着显著的地区差异,所以政体不稳定的原因可能不是来自于政府形式本身,而是来自别的因素。比如,拉丁美洲从经济社会条件、政治文化到历史传统,都更不利于民主的稳定。因此,这项研究无法得出总统制不如议会制的结论。霍洛维茨还认为,总统制未必如林茨所言那般缺乏灵活性,总统制通过具体制度安排的调整可以增加其灵活性。总之,霍洛维茨挑战的是林茨研究的基本逻辑。\nauthor{Donald L. Horowitz,“Comparing Democratic Systems,”\italic{Journal of Democracy}, Vol.1, No.4(1990), pp.73-79.}

后来,另外一位学者斯科特·梅因沃林在这场争论中引入了一个新的变量:政党体制。他认为,此前关于总统制与议会制的讨论都是有价值的,但忽略了一个重要因素:即总统制是否稳定,取决于它跟何种政党体制相结合。当总统制跟多党制结合在一起时,就容易不稳定;当总统制跟两党制结合,就是一个高度稳定的民主政体。比如,美国就是总统制与两党制结合的民主稳定案例。再进一步说,即便在议会制条件下,如果议会政党数量非常多的话,也难以成为稳定的民主政体。\nauthor{Scott Mainwaring,“Presidentialism, Multiparty Systems, and Democracy: the Difficult Equation,”working paper, Notre Dame: Helen Kellogg Institute for International Studies, 1990.}

柴巴布在其研究中指出了另一种逻辑。如上文所说,他同意总统制较议会制更不稳定,但是他认为,拉丁美洲地区的总统制内生于过去的军人统治传统。换句话说,只要是长期军人统治的国家,在启动民主转型之后,通常更可能选择总统制政体。而具有长期军人统治历史的政体,本身固有的政治特征就更容易走向政治不稳定。所以,拉美地区总统制的不稳定,最关键的是此前的政治传统。这样,柴巴布又提供了一个新的视角。\nauthor{Jose Antonio Cheibub, \italic{Presidentialism, Parliamentarism, and Democracy}, Cambridge: Cambridge University Press, 2007.}

上面比较了议会制和总统制,而在第三波民主转型国家中很流行的政府形式是半总统制。按照罗伯特·埃尔杰(Robert Elgie)的统计,到2010年全球国家或地区中大约有52个半总统制的民主政体或半民主政体。\nauthor{Robert Elgie, \italic{Semi-Presidentialism: Sub-Types and Democratic Performance}, Oxford: Oxford University Press, 2011.}当然,有人认为埃尔杰界定半总统制的标准过于宽泛。但无论怎样,20世纪90年代以来,大量转型的国家选择半总统制似乎是一个重要趋势。从逻辑上讲,半总统制既可能兼有议会制与总统制的优点,又可能兼有两者的缺点。按照现有的经验研究,半总统制政体的绩效不如议会制政体,甚至也不如总统制政体。但问题是,既然半总统制政体绩效不那么好,为什么那么多新兴转型国家采用此种政府形式?这是一个有待研究的学术议题。

\tsection{公民投票与选举行为}

选举投票是公民政治参与的基本方式。选举,是指选民通过投票来选择自己代表的政治活动。这里选举的既可以是不同级别的议员,包括从国会议员到乡镇议员;又可以是不同级别的行政长官,包括从国家元首、政府首脑到县市长。当然,任何选举都需要特定的制度安排。

狭义上的选举制度,是指选票转化为席位(from votes to seats)的方式。广义的选举制度除了把选票转化为席位的方式外,还包括选区规模、当选门槛、议会规模等基本制度安排。选票转化为席位的选举公式差异,构成了选举制度基本类型的不同,后面会详细讨论。这里的选区规模,不是指一个选区的人口多寡或面积大小,而是指一个选区产生几个议员的名额。如果每个选区只产生一个名额,一般称为小选区制;如果每个选区产生较多名额,一般称为大选区制。有些选举制度还会设定当选门槛。比如,德国的半数国会议员由政党名单比例代表制选举产生,但选举制度设定政党当选门槛为至少获得5%的选票。另外,议会规模也非常重要。罗伯特·达尔曾统计世界上主要民主国家国会下议院的规模,其人数范围基本上介于150—650之间。按照他的统计,人口数量较多的10多个主要民主国家国会下议院的平均规模为412人。这也是广义的选举制度安排的一部分。

很多人关心,议会规模究竟多大比较好?这个问题很难给出标准的回答。议会规模过小可能会导致两个问题:第一是代表性不足,议会规模小意味着每一个议员要代表更多选民;第二是存在容易滑向实质性寡头统治的风险——少数几个人密谋就能决定重要公共事务。当然,议会规模不是越大越好。考虑到议会本身是一个协商议事的场所,议会人数太多就难以保证协商议事的有效性。比如,一万人开会就难以有效议事。从现有人类的政治经验来看,大国议会规模保持在300—500人是较合适的规模,小国议会规模保持在100—200人就可以。拿美国这个大国来说,参议院的人数规模是100人,众议院的人数规模是435人。人口1000多万的拉美国家智利,参议院的人数规模是60人,众议院的人数规模是120人。如果议会规模太大的话,要么议会难以有效运转,要么还需要在议会之中设立“核心议会”或“高级议会”。

这里还可以比较两个著名案例。法国大革命之前,法国曾召开三级会议,总共有1200人参加。第一个等级是僧侣阶层,代表是300人;第二个等级是贵族阶层,代表是300人;第三个等级是平民阶层,代表是600人。这1200人一起开会,最后尽管制定出一部宪法,但这部宪法很快就被推翻了。原因当然有很多,但1200人在一起开会议事本身就有问题,往往难以进行有效的协商议事,最后也无法达成对法国有利的政治决议。

再来看美国1787年的制宪会议,一般认为有55名代表参加,但这个数字只是一个大致的说法。美国制宪会议的成果是1787年《美国宪法》,这部宪法至今已实施两百多年,帮助美国成长为全球最强大的国家。除了数十个宪法修正案外,1787年《宪法》基本条款至今没有改变。所以,可以说1787年美国制宪会议是一次极其成功的会议,而这跟制宪会议的人数规模不无关系。与法国1200人的三级会议相比,美国制宪会议更有可能进行有效议事,所以结果也更好。

讲到选举投票,很多人自然关心选民投票背后的决定因素是什么?比如,在美国,为什么有的选民投票给民主党,有的投票给共和党?在德国这样的国家,政党体制比美国更为复杂,所以选民投票的多样化程度更高。按照现有研究,影响选民投票的主要因素包括阶级因素、宗教因素、族群与语言因素、代际因素、性别因素,等等。

整个20世纪中,阶级因素通常被视为影响选民投票行为的最重要因素。罗伯特·达尔甚至把西方国家的民主视为“和平的阶级斗争”。从古到今,穷人和富人在很多重要政治议题上的观念相左。比如,早在2000年前的古罗马共和国,那个时候元老院更多代表贵族的立场,平民大会和保民官更多代表平民的利益。所以,阶级身份影响政治立场和观点的概念,并非马克思发明的。

\img{../Images/image00313.jpeg}[图6.5 阶级政治的演进趋势]

当然,工业革命以来,阶级因素在政治生活中变得日益重要。图6.5说明了不同社会发展阶段上阶级政治因素的强弱。横轴代表的是从前工业社会到工业社会,再到后工业社会的时间演进,纵轴代表的是阶级政治因素的强弱。一个总的趋势是,在前工业社会,阶级政治因素并不是太强。随着工业化的进展和工人数量的增加,阶级冲突的程度随之提高。但随着后工业社会的到来、福利国家的建设及贫富差距的缩小,过去意义上的阶级政治或阶级冲突后来就慢慢弱化了。当然,这只是大致的情形。比如,金融危机到来时,由于普通民众生活艰难,阶级冲突可能会加剧。这些因素都会影响选民的投票行为。

阶级有两种比较经典的定义。马克思把对生产资料占有的不同,视为区分不同阶级的标准,比如资产阶级与无产阶级。另一种定义把一个人在社会上的收入和职业状况视为区分不同阶级的标准,这里的阶级有时被视为阶层,韦伯基本上倾向于这种分类方法。从马克思创作《共产党宣言》的19世纪中叶到现在,西方发达国家的阶级结构已发生重大变化。几个主要的趋势包括:简单体力劳动者比例的降低、中产阶级的崛起以及并不拥有股权(生产资料所有权)的高薪管理阶层的壮大。这种阶级结构的变化,加上福利国家建设和贫富差距缩小,欧美发达国家已经从二元对立的社会阶级结构,转型为社会阶层的多样化和阶级冲突的温和化。这一点在第3讲开头曾做过简要剖析。尽管如此,一个人的阶级身份还是会影响他的投票行为。

宗教也是影响选民投票行为的重要因素。与没有宗教信仰者相比,有宗教信仰的人总的来说更加保守。在欧洲大陆国家信教选民中,天主教徒比新教徒更为保守。在全球化时代的今天,不同宗教信仰者同处一国的情形将更为常见。比如,整个人口中有50%是基督徒,有35%是穆斯林,还有15%的其他教徒和无宗教信仰者。这样的国家,在选举投票过程中,很可能会形成一个主要的基督教政党和一个主要的伊斯兰教政党两大宗教政党对峙的格局。在此种选民人口结构下,议会中基督教政党可能占有主导地位,但伊斯兰教政党的席位比例也不低,两者容易产生政治冲突。

跟宗教问题相类似的是族群问题,族群问题通常还跟语言因素有关。比如,在加拿大,很多地区是讲英语的,但也有不少地区讲法语。讲英语的群体跟讲法语的群体存在较为显著的差异。所以,加拿大魁北克问题融合了族群、语言、宗教和地区等不同因素。这无疑会影响选民的投票立场。在发展中世界,一些族群和语言结构复杂的国家,族群和语言因素常常成为影响选民投票的重要原因。

第二次世界大战之后的西方发达国家,选民投票行为的另一个影响因素是现代与后现代的差异。罗纳德·英格尔哈特认为,西方发达国家在二战之后已经历了从物质主义向后物质主义的转型。\nauthor{罗纳德·英格尔哈特:《发达工业社会的文化转型》,张秀琴译,北京:社会科学文献出版社2013年版。}比如,很多选民更关心环保、动物保护、同性恋、堕胎等问题,而不是关心下层阶级的收入、政教关系等问题。影响这类选民投票行为的主要因素是后现代的价值观。受这种价值观的影响,选民在政治上会变得更加温和。无论是环保议题、堕胎议题还是同性恋议题等,都不容易引发类似阶级冲突的政治结果。基于后物质主义价值观的投票行为,与19世纪到20世纪早期基于阶级利益的投票行为相比,有着很大的不同。

目前还有一个问题正在变得越来越突出,即代际问题,这也会影响不同选民的投票倾向。考虑到西方发达民主国家沉重的公共债务危机,一种观点认为西方国家现在活着的这代人正在努力把债务转移到尚未出生的那一代人身上。众所周知,西方发达国家的政府欠了太多的公债,那么这些公债由谁来支付呢?如果目前的政策没有变化,答案只能是下一代!所以,尚未出生的一代可能正在遭受在世一代的“暴政”。此外,还有其他的代际冲突。比如,从年龄构成来说,老年人群投票时往往更看重福利政策和社会保障,年轻人群投票时则希望有更多的成长机会。但随着人口老龄化问题的加剧,政治上整体可能会趋于保守。如果把投票的左右倾向作为因变量,把人口的年龄作为自变量,进行数量分析,大家可能会看到这样的结果:年龄较大的投票者整体上更倾向于保守,年龄较小的投票者整体更倾向于激进。这是投票行为背后的代际因素。

不同性别选民的投票倾向也存在系统的差异。一种观点认为,如果更多女性政治家执政的话,国际政治与国内政治可能会变得更加温和。这种说法尚需实证研究的检验。很多人的生活经验是,男人跟女人在不同问题上的立场、思考和理解问题的视角很不一样。西方有本畅销书题为《男人来自火星,女人来自金星》,讲的是男女差异在感情与家庭关系中引起的摩擦。有研究表明,男女选民在投票方面也存在着显著的差异。

除此之外,在社会分裂的传统研究中,城乡分裂也是导致选民立场不同的一种重要类型。但对于现在的西方主流国家来说,这个问题已变得越来越次要了。主要原因是,现在西方发达国家的城乡差别已经很小,农业人口占总人口的比例已经很低。很多人还住在农村或城市郊区,但他们从事的是跟农业无关的活动,他们的收入来源也不取决于农业。但是,在今天的发展中国家,城乡因素仍然是影响选民投票的重要因素。城乡差异的背后,是职业的差异和观念的差异。

上述分析,更接近解释投票行为的社会学模式,即将投票行为与选民的团体成员身份联系起来,认为选民会采取一种与其所属团体的经济社会地位相似的投票模式。此外,还有几种较为主流的解释投票行为的理论模型。

一种是政党认同模式。这一模型认为,选民投票主要取决于政党认同。选民认同哪个政党,他就倾向于投该党的票——至于该党在此次选举中提供何种政策、推出哪位候选人,都是次要的。这一理论认为,不仅选民的政党认同非常稳定,而且还具有世代之间的继承关系。比如,在美国,一个支持共和党的选民,很有可能来自一个支持共和党的家庭——至少他的父亲很可能是一位共和党的支持者。

另一种是理性投票模型。这一模型把选民视为经济人,他会把政治家(候选人)提供的公共政策视为自身效用函数的一部分。在比较不同政党和候选人的政策之后,他会根据理性计算原则,从自身福利的最大化出发来进行投票。所以,投票既非一种心理认同,亦非一种习惯,乃是一种理性行为。这一模型是把理性选择学派的理论应用于选民投票行为的研究,安东尼·唐斯和詹姆斯·布坎南等都从这一视角讨论投票行为。

还有一种是支配型意识形态模型。这一模型认为,选民会根据自己主导性的意识形态立场来投票。这种理论强调的是政治观念对于政治行为的塑造。面对纷繁复杂的世界,其实普通选民根本无力对重要的政策议题做出自己的理性判断与选择。至于何种政策会导致何种结果,多数选民更是无力思考。在这种情况下,选民会根据意识形态立场来投票,这在政治上是一种简便的做法。比如,在英国,受自由市场意识形态支配的选民会把选票投给保守党,而主张政府干预意识形态的选民会把选票投给工党。有时,选民的意识形态立场与政党偏好是重叠的。\nauthor{关于主要的投票理论模型,参见Ken Newton and Jan W. Van Deth, \italic{Foundations of Comparative Politics: Democracies in the Modern World}, Cambridge: Cambridge University Press, 2005, pp.200-220。}

\tsection{不同选举制度的逻辑}

了解选民投票行为的基本情况后,可以来分析不同的选举制度。一般意义上的选举制度是指议会或议员的选举制度,总统或行政长官的选举制度后面还会简略介绍。粗略地说,议会选举制度有三种主要类型。这三种选举制度各不相同,政治后果也不同。\nauthor{关于选举制度的简要知识,参见安德鲁·海伍德:《政治学》(第二版),张立鹏译,北京:中国人民大学出版社2006年版,第277—282页;较为全面的研究,参见David M. Farrell, \italic{Electoral Systems: A Comparative Introduction}, 2<span class="math-super">nd</span>edition, New York: Palgrave Macmillan, 2011。}

第一种是多数决定制(plurality system)。多数决定制就是得票最多者当选,又分为两种类型。一种是简单多数决定制,即在所有候选人中得票最多者胜出(first past the post,简称FPTP)。比如,某个选区选一个议员,ABCD四人竞选,A得了35%,B得了25%,C得了20%,D得了20%。根据得票最多者胜出的规则,A当选。这种规则不需要当选者获得至少50%的选票,而只需获得相对最多的选票。到目前为止,英国、美国、加拿大、印度等不少国家在议员选举中都采用简单多数决定制。

另一种是绝对多数决定制(majority system),要求当选者至少需要获得50%的选票。当然,存在多个候选人的情况下,第一轮投票可能很难产生获得绝对多数票的候选人,这就需要对得票最多的两名候选人举行第二轮投票。在上述案例中,就是对A、B两人举行第二轮投票。通常,第二轮会产生一个达到50%得票率的绝对多数候选人。与相对多数决定制相比,绝对多数决定制的好处是当选者至少获得了50%的选票支持,具有更高的合法性,但这种制度操作起来比较复杂,成本比较高。

还有一种具有绝对多数决定制特征的选举制度是选择性投票制(alternative vote),又译偏好投票制。在这种投票制度下,选民投票时被要求给所有候选人排序。比如,选票上有A、B、C、D、E五个候选人,选民需要做的是给五个候选人排序,即区分出1、2、3、4、5的次序。然后,清点选票时需要统计每个候选人得到选民第1、2、3、4、5排序的得票比率。现在假定A获得所有投票第一选项的比率为40%,B获得所有投票第一选项的比率是25%,C获得20%,D获得10%,E获得5%。从结果来看,没有一个候选人得第一选项的选票率超过50%,那怎么办?在这种情况下,需要把第一选项得票比率最低的候选人E划掉,投E第一选项的这部分选票根据他们的第二选项,把选票分别分配给排名靠前的A、B、C、D四人。然后,再重新统计他们四人的选票。以此类推,直到其中一位候选人的得票率达到50%为止。这种选举制度目前主要在澳大利亚众议院选举中采用。美国政治学者霍洛维茨在研究高度分裂社会的制度设计时认为,这种选举制度有利于塑造跨族群的选举激励。\nauthor{Donald L. Horowitz,“Democracy in Divided Societies,”\italic{Journal of Democracy}, Vol.4, No.4(Oct.1993), pp.18-38.}因为在偏好性投票制度下,候选人不仅要谋求自己主要选民群体的支持,而且还要努力成为所有选民群体最不讨厌的那位候选人。

第二种是比例代表制(proportional representation)。比例代表制的基本原则是要尽可能让代表的结构更好地反映整个社会选民的结构。比例代表制最流行的投票方法是政党名单比例代表制。比如,某个选区可以产生10个议员名额,现在有A、B、C、D、E五个政党去竞争这10个议员名额。假如每党都提出一个包括10个候选人的政党名单,然后,所有选民根据政党名单来投票。比如,最后A党获得40%选票,B党获得20%选票,C党获得20%的选票,D党获得10%的选票,E党获得10%的选票,那么该选区议员席位分配的最终数量为:A党4席,B、C两党分获2席,D、E两党分获1席。在具体操作上,是这四个政党排名最靠前的4至1位候选人当选。实行政党名单比例代表制的国家较多,以色列、北欧国家及东欧国家在内的多数欧洲国家、拉美国家等都实行这种选举制度。

上文假设每个政党都得到了整数比率的选票,但实际上每个政党得票数量不会是整数。所以,这里还涉及很技术性的问题,即政党名单比例代表制下如何确定从选票到席位的计算规则。目前主要有两种计算规则,即顿特公式与最大余数法,参见表6.2。计数公式的不同,也会导致选举结果的不同。

\tbl{../Images/image00314.jpeg}[表6.2 政党名单比例代表制:顿特公式与最大余数法]

资料来源:David M.Farrell, \italic{Electoral Systems: A Comparative Introduction}, 2<span class="math-super">nd</span>edition, New York: Palgrave Macmillan, 2011,pp.68-70, table 4.1 and table 4.2。

在表6.5的两次不同选举中,每个选区均有5个议员席位,选民数量均为1000人,分别有蓝党(Blue)、红党(Red)、橘党(Orange)、绿党(Green)和彩党(Psychedelic)参加竞选。从选票结果来看,蓝党、红党、橘党、绿党和彩党分别获得360、310、150、120、60张选票。尽管这五个政党所获选票相同,但由于选举计数规则的不同,最后五个政党所获席位数存在差异。按照表6.5上部所示,顿特公式更强调每个席位对应的平均选票数量。所以,结果是蓝党、红党和橘党分获2、2和1个席位;而最大余数法强调的是先去除每个席位对应的足额选票数量,称为黑尔选举限额(Hare quota),在该案例中就是每个席位对应的200张选票,然后在余数选票中对政党再次进行排序。按照这一计算公式,结果是蓝党、红党、橘党和绿党分获2、1、1和1个席位。两者相比较,顿特公式更有利于大党,而最大余数法更有利于小党。这个案例也说明选举公式会影响选举结果。

在比例代表制中,选区规模也是一个重要因素。总的来说,选区规模越大,比例性就越高;选区规模越小,比例性就越低。如果一个选区的席位数量由10个变为20个,更多政党就有机会当选。如果一个选区的席位数量由10个变为5个,通常只有较大政党才有机会当选,小党当选可能性会降低。那么,如果每个选区的席位数量变为2个呢?比如,现在智利的议员选举就采用这种制度,结果是每个选区第三党当选机会大幅减少。这样,就有利于两大主要政党或两大主要政党联盟的形成。这也可以看出选举制度安排的重要性。

第三种是混合型选举制度,也就是把多数决定制与比例代表制结合起来。目前有大量国家采用混合型选举制度,其目标是结合多数决定制与比例代表制的优点。比如,德国国会选举中,一半议席由简单多数决定制产生——全国划为240个选区,每个选区只产生一个名额;一半议席由政党名单比例代表制产生,总共也是240个议席,全国为一个大选区,当选政党须达到总选票数量5%的当选门槛。目前,日本、泰国等大量国家都采用这种混合型选举制度。当然,从混合型选举制度的具体设计来看,如果多数决定制产生议席的比例较高,整个选举制度则越接近于多数决定制;反之,则越接近于比例代表制。

一种主流的观点认为,选举制度之所以重要,是因为选举制度直接影响政党体制的类型。法国政治学家莫里斯·迪韦尔热在其早期研究中提出了一项选举制度影响政党体制的定律,学界称之为“迪韦尔热定律”(Duverger's Law)。后来,迪韦尔热本人将其表述为:

\quo{(1)比例代表制倾向于导致形成多个独立的政党……(2)两轮绝对多数决定制倾向于导致形成多个彼此存在政治联盟关系的政党;(3)简单多数决定制倾向于导致两个政党的体制。\nauthor{Maurice Duverger,“Duverger's Law: Forty Years Later,”in Bernard Grofman and Arend Lijphart, ed., \italic{Electoral Laws and Their Political Consequences}, New York: Agathon Press, 1986, pp.69-84.}}

迪韦尔热认为,由于简单多数决定制下每个选区只有一个议席,“机械”因素和“心理”因素都使得小党较难当选,选民倾向于把选票投给大党。此外,政治家也倾向于加入大党而非加入小党或组建新的政党。\nauthor{Maurice Duverger, \italic{Political Parties: Their Organization and Activity in the Modern State}, London: Methuen &amp; Co Ltd, 1978, pp.206-280.}

从现有研究来看,学术界对“迪韦尔热定律”的认同程度是比较高的:从选举制度到政党体制存在着一种明确的影响机制。不少学者认为,纯粹的比例代表制容易导致极化多党制,从而不利于民主政体的稳定。但是,也有相反的观点。比如,美国政治学者阿伦·利普哈特认为比例代表制要优于多数决定制,并把比例代表制视为协和主义民主或共识民主模式的关键制度。\nauthor{阿伦·利普哈特:《民主的模式:36个国家的政府形式与政府绩效》,陈崎译,北京:北京大学出版社2006年版。}当然,利普哈特这种观点也遭到了激烈的批评。比如,乔尔·赛尔韦和卡里斯·坦普尔曼的研究认为,比例代表制更容易导致族群冲突与政治暴力。\nauthor{Joel Selway and Kharis Templeman,“The Myth of Consociationalism? Conflict Reduction in Divided Societies,”\italic{Comparative Political Studies}, Vol.45, No.12(2012), pp.1542-1571.}霍洛维茨则给族群或宗教高度分裂的社会推荐偏好性投票制,他认为这种制度有利于激励政治家去赢得不同政治集团的支持。而这种投票制被视为多数决定制的一种类型。那么,究竟何种选举制度更有利于塑造稳定而有效的民主政体?现在看来,这个问题还需要更多研究。

\tsection{如何理解现代政党?}

政党是现代政治中的重要现象。那么,什么是政党呢?学术界一般认为,政党是一个有政治愿景的、以执政为政治目标的政治组织,政党旨在通过选举或其他手段来控制政府的人事与政策。有些小型政党的选票比例和席位比例非常低,这样的政党通常没有机会实现单独执政,无法得到总统或总理这样的政治职位,但它们可以通过参加政党联盟分得一杯羹,比如得到一两个部长的职位。所以,这样的政党在实践中是以参政为具体目标的。但如果从理想角度讲,它的目标也是执政。

自政党政治出现在人类政治舞台上以来,政治界与学术界对政党的看法经历了较大的变化。美国第一任总统华盛顿在其离职演说(1796年)中提到了政党,但并无好感。他这样说:“假如政府软弱得不能抵御宗派的野心,自由……的确不过是一个名字而已。……我以最严肃的态度警告你们警惕政党精神的有害影响。”这意味着华盛顿对政党持负面看法,他在即将卸任之时还告诫美国当时的政治精英们要警惕政党。

华盛顿讲这段话的时候是1796年,几十年以后法国政治思想家托克维尔到美国考察,写出了鸿篇巨制《论美国的民主》。他在书中说:“政党是自由政府生来就有的恶。”这段话包含了两层意思,第一,政党不见得是个好东西。第二,只要是自由政府,必定会出现政党。为什么呢?如果是一个自由政体,只要存在对政治权力的正式竞争与争夺,那么政党迟早都会兴起。理由很简单,在获取政治权力的斗争中,是一个人去竞争有力量,还是组织起一帮人去竞争更有力量?答案当然是后者。所以,结果是单枪匹马的政治竞争者被淘汰了,剩下的人要么选择加入较大的政治集团,要么就远离政治。所以,托克维尔会认为,政党是自由政府的必然产物。

在19世纪中叶托克维尔讨论这一问题之后,随着民主政体的扩散,政党与政党政治越来越成为一个全球性的政治现象。迈克尔·罗斯金所著的流行教科书《政治学与生活》(又译《政治科学》)喜欢引用这样的观点:“政党创造出民主政治,现代民主政体则不容置疑地与政党互栖共生。”\nauthor{迈克尔·G.罗斯金等:《政治学与生活》(第12版),林震等译,北京:中国人民大学出版社2014年版,第194页。}所以,民主政治是离不开政党政治的,民主政治甚至在很大程度上就是由政党政治塑造的。政治参与和政治竞争都离不开政党。政党通过政治动员推动了政治参与,通过竞选公共职位来实现政治竞争。从18世纪到今天,从华盛顿到托克维尔,再到罗斯金,大致代表了人们关于政党和政党政治观点的变迁。

一般认为,政党在现代政治中扮演着特定的角色,履行着特定的功能。首先,政党有代表的功能,即代表了部分选民的意志和利益。通常,左派政党代表的是下层阶级的利益,右派政党代表的是上层阶级的利益,宗教政党代表的是特定宗教群体的利益,族群政党代表的是特定族群的利益,“绿党”代表的则是环境保护主义者的利益诉求,等等。其次,政党还有培养和录用精英的功能。英国前首相撒切尔夫人牛津大学毕业不久,就加入了保守党的地方组织,并成为那里的积极分子。在撒切尔夫人的政治生涯中,保守党给她提供机会,鼓励她参与地方政治活动,把她造就为一个地方性的政治人物,然后又通过保守党的全国性组织把她造就为一个全国性的政治家。撒切尔夫人本人固然极其出色,但她的政治生涯离不开保守党的政治平台。再次,政党还具有制定政治目标的功能。政党的政治愿景通常会表述为具体的政治目标。国家向何处去?方向在哪里?主要政党经常会提出自己的政治目标,并把这个目标“推销”给选民。此外,政党有利益表达和整合的功能。社会利益非常多元化,当这些利益诉求方向不一、甚至互相冲突时,该怎么办?政党此时可以扮演利益整合者的角色。再者,政党具有政治社会化和政治动员的功能。很多消极选民,正是由于政党的政治动员——包括通过宣传、组织、运动等多种方式——被卷入到政治过程中了。当政党发挥作用时,往往可以提高一个社会政治社会化的程度。最后,政党还具有组建政府的功能。无论是总统制还是议会制,选举之后组建政府通常是以政党为基础的。在总统制下,总统所在的政党往往是新政府的中坚力量;在议会制下,总理和内阁人选本身就是政党磋商与讨价还价的产物。这些都是现代政党的基本功能。\nauthor{关于政党功能,参见史蒂芬·E.弗兰泽奇:《技术年代的政党》,李秀梅译,北京:商务印书馆2010年版。}

现代政党通常可以区分为不同的类型。一种分类是把政党划分为干部型政党和群众型政党,美国民主党和共和党都是群众性政党,列宁当年创建的布尔什维克是干部型政党。在现代政治中,群众性政党往往是因为选举才临时组织到一起的,选举一结束就各自走散了。在美国,很多人和很多家庭宣称自己属于民主党或共和党,并不意味着他们跟民主党或共和党党部有着密切的联系,而是说他们在大选中投票支持民主党或共和党。另外,加入群众型政党并不需要什么严格的程序。相比之下,干部型政党有较为严密的组织,有较为严格的纪律,有相对完善的内部管理。从理论上讲,干部型政党应该是一个更有凝聚力的政党,具有更强的组织能力。此外,干部型政党通常有比较严格的入党手续。

另一种分类是把政党划分为宪政型政党和革命型政党。宪政型政党在现有基本政治框架内提出政治主张和诉求,革命型政党旨在颠覆现有的基本政治秩序。以1919—1933年的德国魏玛共和国为例,当年社会民主党、中央党等都属于宪政型政党。他们考虑的是如何在魏玛共和国的政治框架里提出什么样的政治诉求。当时的德国共产党尽管有共产主义纲领,但在实际行为上是一个比较温和的左派政党,而不是极端左派政党,并不以鼓吹暴力革命为主要诉求。但是,当时希特勒领导的纳粹党并非宪政型政党,而是革命型政党。该党的最终目的是要颠覆魏玛共和国的政治制度。

在现代民主政体下,多数政党的主要意图是赢得更多选票和席位。因此,政党的战略、组织和领导都是重要的问题。在这些方面做得不够成功的政党,将难以在政治市场上赢得成功。比如,假设大家还可以回到第2讲的那个岛屿上,如今全岛的人口或许已发展到数十万了。现在,大家制定了一个新的规则,决定要通过简单多数决定制来选举议员,成立议会制政府。如果你对政治感兴趣,想组建一个政党,并想通过这个政党来赢得选票和席位,甚至想成为议会的主要政党。那么,应该怎么做呢?这里最重要的是三个问题:政党的战略问题、组织问题和领导问题。

战略问题涉及政党的定位。定位是一个流行的市场营销学术语。什么叫政党的定位?就是说在整个选举市场中,政党要给自己找一个有利的位置以便吸引更多选票。一个选举市场中存在着各种各样的位置。如何能找一个合适的位置,以便有机会赢得较多的选票呢?这的确是一个问题。你可以去观察岛上的数十万人,他们的经济状况、教育背景、职业状况和意识形态,等等,在此基础上你大致可以确定一个较明确的政党定位。从策略上说,政党定位主要有两种考虑:一种是基于意识形态立场的定位,另一种是非常务实的定位。前一种做法,你信仰什么,你就成立一个什么样的政党,而不必过分考虑短期的选票得失。后一种做法,你经过考察和评估,认为何种定位更能赢得选票,你就确定此种定位。这两种策略,前一种以实现政治理想为目标,后一种以选票最大化为目标。

明确了政党定位以后,还要考虑如何发展政党组织。比如,采用何种组织类型?如何制定政党纪律?如何获取政党资金?如何进行大众动员?如何建立地方组织与进行层级建设?如何完善政党职能建设?等等。这些都是政党组织建设的关键问题。历史经验揭示,政党的兴衰成败很大程度上维系于其组织。正如亨廷顿1968年的忠告:“谁能组织政治,谁就能掌握未来。”

此外,同样重要的是政党的领导问题。一个有效的政党离不开有效的政治领导力,政党主要的政治家通常扮演着重要的角色。对独立之后的印度来说,以尼赫鲁为首的国大党政治领导阶层就发挥了有效的政治领导力,从而有利于印度民主的稳定。相反,巴基斯坦在其政治领袖真纳去世后,主要政党逐渐失去了政治领导力。很快,该国就蜕变为军人统治。按照美国华人学者邹谠对中国近代政治史的解读,在当时政治冲突激烈的背景下,凡是按照自由主义原则组建、政治领导力与组织力不强的政党后来都衰落了。因此,与政党定位、政党组织相比,政党的领导同样是一个重要问题。

\tsection{政党体制的不同类型}

政党体制或政党制度是指一个国家中政党的数量及其权力结构。美国政治学者乔万尼·萨托利在1976年政党学名著《政党与政党体制》中区分了两大类型的政党体制:一种是非竞争性政党体制,一种是竞争性政党体制,参见图6.6。

非竞争性政党体制是指政党之间不存在实质性的政治竞争关系,而某一主导政党居于支配性地位。非竞争性政党体制有两种类型:一党制和霸权党制。一党制是该国只有一个基于统治和支配地位的政党,不存在任何其他政党。霸权党制是该国存在多个政党,但有一个主要政党基于统治和支配地位;其他政党并非民主政体意义上的政党,这些政党可能也参与政治竞争,但该霸权政党控制着全部的或绝大部分的政治权力。

\img{../Images/image00315.jpeg}[图6.6 萨托利论政党体制的类型]

资料来源:G.萨托利:《政党与政党体制》,王明进译,北京:商务印书馆2006年版,第182页,图4。

与非竞争性政党体制相对的是竞争性政党体制,是指不同政党之间存在实质性竞争关系的政党体制类型。萨托利认为,根据竞争性政党数量的多少,竞争性政党体制有几种主要类型。第一种是主导党体制(或竞争性的一党独大制)。一些国家的历史上曾出现过主导党体制,比如日本从1955年到1993年间自民党的主导党体制,其选举是自由而公正的,政党之间是竞争性的,但自民党始终控制议会的多数席位。第二种是两党制。过去的英国和现在的美国都是典型的两党制。两党制不是说只有两个政党,而是说两大主要政党能够赢得绝大多数选票和席位。第三种是温和多党制。温和多党制下,议会通常有三到五个重要政党,现在的法国和德国基本上符合这种类型。第四种是极化多党制。这是指议会中重要政党有六到八个以上。德国魏玛共和国时代的政党制度符合极化多党制的类型。第五种是碎片化政党体制。从名称来看就知道这种政党体制的数量和结构比极化多党制更加碎片化,通常包括了10或20个以上的政党。当然,有人把后面两种政党体制不加区分地称为极化多党制。

在论述诸种竞争性政党体制时,萨托利认为极化多党制不利于民主政体的稳定,原因在于极化多党制具有很多不利于民主有效运转的基本特征:

1. 反体制政党的出现;

2. 双边反对党的存在;

3. 中央存在一个(意大利)或一组(法国、德国魏玛共和国)政党为特征;

4. 政治的极化体制;

5. 离心型驱动力对向心型驱动力可能的超越;

6. 存在固有的意识形态型式(ideological patterning);

7. 不负责任的反对党的出现;

8. 抬价政治(out-bidding)或过度承诺的政治。\nauthor{G.萨托利:《政党与政党体制》,王明进译,北京:商务印书馆2006年版,第184—207页。}

除了政党体制的粗略分类,拉克索(Markku Laakso)与塔格培拉(Rein Taagcpera)提出用“有效政党数目”(effective number of parties)来更精确地衡量民主国家的政党制度。\nauthor{Markku Laakso and Rein Taagcpera,“‘Effective’ Number of Parties: A Measure with Application to West Europe,”\italic{Comparative Political Studies}, 12(1979), pp.3-27.}他们提出了计算“选举有效政党数目”和“议会有效政党数目”的公式:

<div class="kindle-cn-bodycontent-div-alone100">
 <img alt="202" class="kindle-cn-bodycontent-image-alone60" src="../Images/image00316.jpeg" />
</div>

其中Nv代表选举有效政党数目,Vi代表每个政党的得票比率;Ns代表议会有效政党数目,Si代表议会中每个政党的席位比率。这样把具体的数据输入上述两个公式之后,就能计算出选举有效政党数目和议会有效政党数目。

那么,何种因素决定政党体制或有效政党数目呢?上文已探讨过选举制度对政党体制的影响。“迪韦尔热定律”被视为一种较有说服力的理论。但是,影响政党体制的不只是选举制度,社会分裂也被视为重要影响因素。一般来说,社会分裂维度的数量越多,政党数目有可能越多。在经典的两党制模型下,往往只有“左——右”意识形态这一单一社会分裂维度。如果一个社会存在“左——右”意识形态维度的分裂,同时存在不同族群、宗教、语言文化维度的社会分裂,通常就会呈现多党制的格局。所以,社会分裂维度越多以及由此导致的政治议题维度越多,政党数量可能越多。\nauthor{这方面著述很多,请参考下面两篇论文及其引述的文献:Octavio Amorim Neto and Gary W.Cox,“Electoral Institutions, Cleavage Structures, and the Number of Parties,”\italic{American Journal of Political Science}, Vol.41, No.1(Jan., 1997), pp.149-174; Rein Taagepera,“The Number of Parties as a Function of Heterogeneity and Electoral System,”\italic{Comparative Political Studies}, Vol.32 No.5, August 1999, pp.531-548。}

影响政党体制的第三个重要因素是该国的历史情境,政党体制存在着明显的路径依赖问题。比如,拿印度来说,该国尽管社会分裂维度的数量很多,但其建国之初的20年左右时间中维系了国大党一党独大的政党体制。原因在于,印度争取独立过程中诸种社会力量集聚到国大党的旗号下,与英国统治者进行了长期的政治斗争。结果是,到印度独立之时,国大党不是被视为印度某一个社会阶级或集团的代表,而是整个印度社会的代表,是整个印度民族的政治领导力量。当然,后来印度政党体制的演变,跟印度复杂的社会结构与社会分裂维度的数目有关。

今天的比较政治学很重视对政党体制的研究,因为政党政治已成为现代民主政治的核心。为什么政党体制很重要?一个主要的原因是政党政治关系到政府的稳定性和民主本身的稳定性。通常,政党不稳定的国家,政府也不会太稳定;或者说,政党数量特别多的国家,政府基本上也是不稳定的。此外,如何塑造强大的主导政党或大型政党,以及如何塑造有效的政党体制,还是民主转型的关键问题。在发展中世界或第三波民主化国家中,民主转型顺利的国家总体上政党比较强大,有一两个或两三个较有实力的主导政党;民主转型不顺利的国家通常没有有效的主导政党与有效的政党体制——这些国家的政党经常会不停地组合,走马灯式地更换政党名称,一些政党快速兴起而又快速衰落。从这个视角出发,发展中国家民主转型的关键问题是能否塑造有效的政党体制。对这样的转型国家来说,如果能塑造一党独大型政党体制(竞争性政党体制的一种,而非霸权党制)、两党制或温和多党制,那么该国更有可能维系新兴民主政体;如果是极化多党制,就更难维系民主政体。

在所有政党体制中,一般认为极化多党制会降低政府与民主政体的稳定性。迈克尔·泰勒及其合作者的跨国研究得出了三个相关的结论:(1)议会中政党体制的碎裂(fractionalisation)程度与政府稳定性呈现“显著的”负相关性,即政党体制碎裂程度越高,政府越不稳定;(2)一个主要政党执政的政府比多党联盟政府“极为显著地”更加稳定;(3)多数派政府比少数派政府“显著地”更稳定。\nauthor{Michael Taylor and V.M.Herman,“Party Systems and Government Stability,”\italic{The American Political Science Review}, Vol.65, No.1(Mar., 1971), pp.28-37.}萨托利与林茨等人都认为,极化多党制显然不利于民主的稳定。反体制政党的存在、离心激励主导、严重的意识形态冲突、不负责任的反对党,以及选举竞争中的过度承诺或抬价政治,使得极化多党制难以形成有效的执政力量,政府能力就会降低,民主稳定性也会下降。\nauthor{G.萨托利:《政党与政党体制》,王明进译,北京:商务印书馆2006年版,第284—397页。}

\tsection{央地关系:联邦制与单一制}

在政治制度安排中,中央政府与地方政府的关系也是一个重要方面。对大型现代国家而言,央地关系上有两条通行的基本原则。一方面,必须实行某种程度的中央集权,以保证国家统一;另一方面,必须实行某种程度的地方分权,以保证治理的有效性与灵活性。所以,任何一个大型政治体基本上不存在不要集权或不要分权的问题,而一定是集权和分权的某种组合。

正是由于中央集权与地方分权组合模式的不同,世界上多数国家的央地关系可以区分为两种类型:联邦制和单一制。一般来说,联邦制指主权或主要政治权力由联邦政府与州或邦政府共同分享的一种央地关系模式,联邦政府和州或邦政府同时从宪法与人民的授权中获得政治权力。在这种模式下,州或邦政府的政治权力不是来自中央政府的授予,而是独立地来自于宪法与人民的授权,邦或州政府不是联邦政府的下级或下属单位。比如,在联邦制国家美国,没有人会认为总统奥巴马是各州州长的上司。所以,联邦制下的这种制度安排,通常是单一制国家很难理解的。单一制指主权或主要政治权力掌握在中央政府手里,州或省政府的政治权力来自于中央政府的授予。这种模式下,地方政府实际上相当于中央政府的派出机构,是中央政府的下级或下属单位。在单一制国家法国,各个地方的政治权力主要来自于法国中央政府的授予。

需要提醒的是,联邦制和单一制只是央地关系的两种理想类型。在政治实践中,央地关系的实际安排更为复杂。与联邦制或单一制这样的政治符号相比,央地关系的实际政治分权更为重要。比如,美国尽管是典型的联邦制国家,但从建国至今,美国联邦政府的政治权力一直在扩张。联邦政府在履行越来越多的职能,并开始介入传统上被认为是州或地方政府事务的很多领域。在整个20世纪,美国联邦政府在整个政府收支中的比例大幅上升,如今已超过一半。所以,与地方政府相比,美国联邦政府的政治权力在大幅扩张。当然,毫无疑问,美国的基本政治架构无疑还是联邦制模式。

英国尽管是一个单一制国家,统辖着英格兰、苏格兰、威尔士、北爱尔兰等四个主要地区。但最近半个世纪中,英国地方分权的趋势一直在强化。英国地方分权强化的一个证据是2014年苏格兰议会对苏格兰独立法案发起了全民公投。

印度宪法第一条就规定,印度是一个联邦制国家。但实际上,印度1947年以后的相当长时间内都是一个高度中央集权的准联邦制国家。当时,印度联邦政府的权力极大,印度联邦议会和联邦政府甚至可以决定不同邦的行政区划与领土面积。所以,当时的印度算不上是一个真正的联邦制国家,至多只能被称为准联邦制国家。大概到20世纪80、90年代,印度对整个政府指导的管制经济模式进行改革后,邦一级政府的地方分权才得到实质性的加强,这样印度联邦制的色彩越来越浓厚。所以,对印度来说,尽管其宪法条款并未更改,但该国却经历了从高度中央集权化的准联邦制向强化实质性地方分权的过渡。

那么,联邦制与单一制两者孰优孰劣呢?一个可能的回答是:联邦制与单一制各有优劣。联邦制的主要好处包括:适应国家内部的多样化,适应各地方不同的需要,便于地方实验,有利地方自治,等等。联邦制的坏处包括:内部由于存在不同的政治权力中心,难于统一协调。单一制的主要好处包括:便于统一控制与协调,一般决策效率会比较高。当然,有人认为,中央政府行动迅速不见得治理绩效会更高。在此种条件下,固然中央政府制定和执行正确决策的速度比较快,但中央政府犯错的速度也同样很快。单一制的坏处包括:缺少多样性,不利于地方自治,等等。

除了一般意义上的优劣分析,现在联邦制与单一制问题受到重视,主要是因为央地关系还涉及国家建构与领土完整问题。特别是,对于一个高度分裂的社会来说,或者说对于一个族群、宗教、语言结构复杂性很高的社会来说,究竟是实行联邦制还是单一制好呢?对于这类国家,政治制度建设的基本目标是能否推动国家构建和国家认同,缓和不同社会集团的关系,减少政治冲突,等等。

比较政治学对于联邦制与单一制优劣的主流研究分为两大流派:一派强调权力分享(power-sharing)原则,另一派强调政治整合(political integration)原则。强调权力分享的又被称为协合民主理论或共识民主理论,其主要代表人物是阿伦·利普哈特。从原则上讲,权力分享流派主张的是不同族群、宗教群体要和谐相处,尽可能要达成共识,所有重要政策应该尽可能兼顾不同的族群与宗教群体,中央政府的权力要根据族群、宗教群体的比例做相应的安排和分配。在央地关系上,这一流派当然会强调联邦制的重要性,强调少数群体的自治权与否决权。利普哈特等人认为,对于高度分裂的社会来说,联邦制或地方分权的制度安排能够包容族群、宗教与文化的多样性,从而提高政治适应能力。\nauthor{阿伦·利普哈特:《民主的模式:36个国家的政府形式和政府绩效》,陈崎译,北京:北京大学出版社2006年版,第135—145页。}

政治整合流派以霍洛维茨的理论研究为代表,他认为不同族群、宗教群体的人不可避免地存在冲突,想让他们达成共识几无可能。所以,首要的是如何把一个国家中的不同群体整合到一起,通过制度设计等办法让中央政府形成有效的政治权威与国家能力。这种政治权威与国家能力至少能保证国家统一与基本政治秩序。在此基础上,再来考虑如何善待不同的族群与宗教群体。两者相比较,协合型民主理论希望通过权力分享促成不同群体能最终能达成共识与合作,而政治整合理论认为存在严重亚文化分裂的社会关键是要形成一个有效的国家与中央政府,首先要进行政治整合。这一派认为,在社会分裂程度高的国家实行联邦制,就容易导致国家分裂。霍洛维茨就指出,“联邦主义会强化或激化族群冲突”,从而更容易弱化中央政府能力和诱发国家分裂。\nauthor{Donald L. Horowitz, \italic{Ethnic Groups in Conflict}, Berkeley: University of California Press, 1985, p.603.}劳伦斯·安德森(Lawrence M.Anderson)也认为,联邦制给予地区政府和其他政治力量以更多机会与资源去支持分离主义运动,从而“激发地区独立的渴望”。\nauthor{Lawrence M. Anderson,“Exploring the Paradox of Autonomy: Federalism and Secession in North America,”\italic{Regional and Federal Studies}, vol.14, 2004, pp.89-112.}

\tsection{制度设计与宪法工程学}

上面关于不同层次的政治制度的讨论,已经涉及国内学界关注较少的一个问题,即民主模式的多样性。国内学界受亚里士多德政体类型学的影响,通常较为重视不同政体的类型。这里的政体类型主要是指传统意义上的君主制、贵族制与民主制,以及现代意义上的民主政体、威权政体与极权政体。然而,很多人对同一政体——特别是民主政体——内部模式的多样性却缺少应有的重视。

而国际学术界有大量研究与民主模式的多样性有关。比如,美国政治学者加布里埃尔·阿尔蒙德早在1956年就区分了民主的三种模式:盎格鲁-撒克逊政治制度、欧洲大陆政治制度以及斯堪的纳维亚政治制度。达尔在《论民主》中提出过一种民主模式多样性的划分标准。他根据选举制度和行政—立法关系的不同区分了四种主要模式:(1)英国模式,即议会制与简单多数决定制的组合;(2)欧陆模式,即议会制与比例代表制的组合;(3)美国模式,即总统制与简单多数决定制的组合;(4)拉美模式,即总统制与比例代表制的组合。此外,他把选用半总统制或混合型选举制度的不同组合统称为(5)混合模式。这样,达尔区分了民主模式的五种类型,呈现了一种简洁而准确的民主模式类型学。\nauthor{罗伯特·达尔:《论民主》,李风华译,北京:中国人民大学出版社2012年版,第109—119页。}

阿伦·利普哈特则从他自己特定的理论视角来讨论民主模式的两种基本类型:一种是多数民主模式,一种是协合型民主模式,或称共识民主模式。阿伦·利普哈特提出了协和型民主的四个基本特征:(1)大型联合内阁;(2)局部自治;(3)比例代表制;(4)少数群体否决权。从1984年到1999年,利普哈特又把协和型民主理论发展成共识民主理论。他区分了共识民主模式与多数民主模式,并论证了共识民主具有更好的政府绩效。基于这些背景,利普哈特大胆地认为学术界存在一种“共识”:即共识民主模式更有利于高度分裂社会民主的稳定性。那么,这种“共识”真的存在吗?实际上,学术界对协合型民主理论的批评就从未平息过,有不少学者对这种理论提出质疑和挑战。但无论怎样,利普哈特亦注意到了民主模式的多样性问题。

民主模式的多样性促使大家思考:是否存在最优良的民主制度模式?特别是对于高度分裂的社会来说,何种制度模式有利于民主政体的稳定?正是由于上述问题的推动,加上20世纪80年代新制度主义在政治学研究领域的兴起,比较政治学最近兴起一个被称为“宪法工程学”(constitutional engineering)的研究分支。简单地说,宪法工程学是试图通过有意识的宪法与政治制度设计来达到某些预期的政治目标。\nauthor{参见Benjamin Reilly, \italic{Democracy and Diversity: Political Engineering in the Asia-Pacific}, Oxford: Oxford University Press, 2007, pp.21-24。作者有这样的定义:“政治工程学”就是“对政治制度的有意设计以实现某些特定的具体目标”。}就目前的研究热点来说,高度分裂社会的宪法设计与制度安排是宪法工程学的重点研究领域。宪法工程学预期的政治目标往往是两个:一是实现政体稳定;二是实现政体绩效。目前的宪法工程学研究总体上呈现五个基本特征:\nauthor{包刚升:《民主转型中的宪法工程学》,载于《开放时代》2014年第5期,第111—128页。}

第一,宪法与政治制度是一个独立的变量并且是可以人为设计的。学术界有人认为,制宪过程反映了一个国家的社会结构与政治力量对比,因此政治制度不过是既有社会结构的反映。但另一种观点认为,宪法与政治制度具有相对的独立性,可以被视为一个独立的变量。一方面,任何的宪法与政治制度都包含了人为设计的成分。比如,同样社会政治条件下,政治精英的不同选择可能导致不同的宪法与政治制度。另一方面,无论宪法制定过程在何种程度上反映了既有利益结构或人为主观设计的成分,但宪法和政治制度一旦确立,它们就完全可能独立地发起作用。詹姆斯·马奇和约翰·奥尔森就认为:“政治制度具有相对的自主性和独立的作用。……我们不认为政治仅仅是社会的反映,或者仅仅是个人行为的加总效应。”\nauthor{James G. March and Johan P. Olsen,“The New Institutionalism: Organizational Factors in Political Life,”\italic{The American Political Science Review}, Vol.78, No.3(Sep., 1984), pp.734-749.}

第二,特定的宪法设计与政治制度会产生特定的政治后果。宪法和政治制度之所以重要,是因为立法—行政关系、中央—地方关系、选举以及政党问题上的不同制度安排,都可能会导致不同的政治后果。宪法与政治制度产生作用的方式是通过“制度—行为—结果”的传导机制,这种传导机制的关键是政治制度界定了政治行为者的激励结构。有学者认为,应该从激励结构角度理解宪法和宪法设计对政治的影响。\nauthor{Gary W. Cox,“Centripetal and Centrifugal Incentives in Electoral Systems,”\italic{American Journal of Political Science}, Vol.34, No.4(Nov., 1990), pp.903-935; Ernesto Calvo and Timothy Hellwig,“Centripetal and Centrifugal Incentives under Different Electoral Systems,”\italic{American Journal of Political Science}, Vol.55, No.1(Jan., 2011), pp.27-41.}

第三,与宪法的文本相比,宪法工程学更重视宪法实际的实施和运转。关于宪法的法学研究通常更关注宪法的文本与条款,以及宪法反映了何种的法学价值,而相关的政治学研究更关注宪法在政治实践中能否实施和运转起来。如果缺乏足够的政治考虑,一部文本出色的宪法可能在政治实践中是完全无法实施的。所以,宪法工程学认为,一部有效宪法的关键不在于其意图是否足够善良、文本是否足够优美,关键在于能否得以实施和运转起来,并能达成预期的政治目标。按照现有的研究,大部分宪法都是失败的。\nauthor{Zachary Elkins, Tom Ginsburg, and James Melton, \italic{The Endurance of National Constitutions}, Cambridge: Cambridge University Press, 2009.}当然,一部宪法能否有效实施和运转,还取决于宪法与其实施的社会情境是否匹配。

第四,宪法工程学非常重视高度分裂社会的宪法设计与政治制度安排问题。高度分裂社会的通病是其民主政体往往难以稳定,社会常常陷于严重的政治冲突之中。所以,如何在高度分裂社会塑造稳定的民主政体是宪法工程学的一大挑战。实际上,最近十多年中,宪法工程学非常关注如何使民主政体在高度分裂的社会成为可能。

第五,宪法工程学的目标是塑造一个稳定而有效的民主政体。民主意味着政治参与和政治竞争,但同时民主必须要具有足够的政府效能。戴蒙德认为,民主政体存在三个悖论:一是冲突与共识的悖论;二是代表性与治国能力的悖论;三是同意与效能的悖论。\nauthor{Larry Diamond,“Three Paradoxes of Democracy,”\italic{Journal of Democracy}, Vol.1, No.3, 1990, pp.48-60.}因此,宪法工程学的目标是在确保政治参与和政治竞争的条件下,如何维系民主政体的稳定性与有效性。20世纪80年代的国家理论兴起之后,国家能力(state capacity)成了一个比政府效能或有效政府更为流行的学术概念。\nauthor{彼得·埃文斯、迪特里希·鲁施迈耶、西达·斯考切波:《找回国家》,方力维等译,北京:生活·读书·新知三联书店2009年版。}从国家理论出发,如何塑造有效的政府效能或国家能力也可以被视为宪法工程学的重要目标。

因此,无论是民主模式的多样性,还是新兴民主政体的宪法设计问题,都体现了政治制度的重要性。实际上,如何通过有效的政治制度的设计来建构合理的政治秩序,过去是、现在仍然是政治学的基本问题。

\tsection{推荐阅读书目}

阿伦·李帕特(利普哈特):《选举制度与政党制度:1945—1990年27个国家的实证研究》,谢岳译,上海:上海人民出版社2008年版。

艾伦·韦尔:《政党与政党制度》,谢峰译,北京:北京大学出版社2011年版。

Arend Lijphart, eds., \italic{Parlimentary Versus Presidential Government}, Oxford: Oxford University Press, 1992.

David M. Farrell, \italic{Electoral Systems: A Comparative Introduction}, 2<span class="math-super">nd</span>edition, New York: Palgrave Macmillan, 2011.
