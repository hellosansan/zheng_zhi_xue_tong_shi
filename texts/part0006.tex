\tchapter{政治生活中的国家}

\quo[——亚历山大 · 汉密尔顿]{政府力量是保证自由不可缺少的东西。}

\quo[——马克斯 · 韦伯]{国家是这样一个人类团体,它在一定疆域之内(成功地)宣布了对正当使用暴力的垄断权。}

\quo[——西达 · 斯考切波]{(国家能力是)对力量强大的社会组织实际的或潜在的反对时,国家执行其正式目标的能力。}

\quo[——弗朗西斯 · 福山]{美国建立的是一套有限政府制度,在历史上就限制了国家活动的范围。但在这个范围内,国家制定及实施法律和政策的能力非常之强。}

\tsection{世界版图上的国家}

打开世界地图,第一印象是世界由不同的国家组成。但是,如果回到 500 年前,即亦公元 1500 年左右,世界地图完全是另一个样子。当时的中国是明朝时期;欧洲已出现一些国家,但多数地方仍然是不同于国家的政治实体;在印度,莫卧儿王朝居于统治地位,但印度并不是一个完全统一的国家;非洲存在一些小规模的王国,但很多地方是没有发展成国家形态的松散政治体;美洲曾出现三个比较古老的文明:印加文明、玛雅文明和阿兹克特文明,其他地方则是原始的部落状态。公元 1500 年的世界地图给人留下的印象是,当时世界上并没有多少地方是由国家统治的。所以,这也说明国家并非是从来就有的。

到了公元 1900 年,南北美洲的大部分国家已成型,欧洲相当比例的国家也已成型;还有很多国家——比如印度——是欧洲列强的殖民地;中国亦受到世界列强势力的支配,但基本保持着独立国家的形态;日本是为数不多的位于欧美两洲之外保持独立的国家;非洲大部分地方都是欧洲国家的殖民地。所以,1900 年的世界地图大致上是世界殖民地图。实际上,更多国家的形成是 20 世纪以来的事情,很多非洲国家从独立到现在总共是半个世纪左右的时间。可想而知,这样的国家与已有数百年国家史的少数欧洲国家相比,政治上的差距会非常大。

美国政治学家加布里埃尔 · 阿尔蒙德等人经过统计认为,在美国宣布独立的 1776 年,按照现代国家的标准,全球只有 21 个国家;在俄国革命爆发的 1917 年,世界上只有 53 个国家;在二战结束的 1945 年,世界上也只有 68 个国家;到 2002 年为止,联合国成员国已经达到 191 个。\nauthor{加布里埃尔 · A.阿尔蒙德、拉塞尔 · J.多尔顿、小 G.宾厄姆 · 鲍威尔、卡雷 · 斯特罗姆等:《当代比较政治学:世界视野》(第八版),杨红伟等译,上海:上海人民出版社 2010 年版,第 17 页。}由此可见,世界上大部分国家都是二战以后形成的。

一些国家独立以后,国家构建问题或国家性(statehood)问题不那么严重。比如,中国过去被认为是半殖民状态,特别是 1932—1945 年遭到日本的长期入侵。20 世纪中叶之后,中国需要考虑如何建设新国家的问题。但与很多发展中国家相比,中国的优势是历史上长期保持着较为完整的传统国家形态。尽管分分合合,但统一的官僚制、中央控制的军队以及全国化的税收系统在中国都已存在很长的时间。统一国家的观念更是由来已久,这块土地上多数地方的绝大部分人口对国家存在着基本认同。

相对来说,印度在国家问题上的压力要大一些,但印度的情形要优于很多非洲国家。印度的首要问题是族群和宗教结构异常复杂。当时英国考虑撤出印度时,就面临一个棘手的问题:作为国家的印度到底该怎么办?印度历史上固然有印度国家的观念,但印度的版图更多时候是四分五裂的,而且印度历史是不断被征服的历史。当英国计划撤出印度时,就有人提出过不同的国家和版图方案,最后决定印度和巴基斯坦实行分治。由于孟加拉国最初是巴基斯坦的一块飞地,后来孟加拉国也独立了。所以,1947 年之前的印度今天已成为三个国家:印度、巴基斯坦与孟加拉国。尽管印度历史上有比较成型的国家形态,一些王朝有着较长的历史,但由于复杂的族群与宗教状况,印度在国家认同与民族认同上的挑战仍然很大。

与中国和印度相比,非洲绝大多数国家在这一方面的问题更为严重。欧洲人进入非洲时,撒哈拉以南非洲的很多地方都是部落状态,有些地方存在着小规模的王国或松散的政治实体。要在这样的地区凭空建立国家和塑造国家认同是非常困难的。比如,非洲人口最多的国家尼日利亚 20 世纪初还不存在。英国 19 世纪进入尼日利亚这块地方以后,分别建立南、北尼日利亚两个殖民地。1914 年,英国才把北尼日利亚和南尼日利亚合并成今日的尼日利亚。但是,南北两个殖民地的人们并没有什么准备,他们一开始并不认为南北尼日利亚是一个完整的国家。南部和北部的不同地区为不同的主要族群占据,而且经济发展的地区差异很大,不同族群之间还存在着复杂的历史恩怨,这就使得尼日利亚在国家构建方面遭遇了巨大的挑战。后来,尼日利亚独立不久,就爆发了一场长达两年多时间的内战。所以,对该地区来说,不少国家有着类似的问题。

美国和平基金会(Fund for Peace)与美国《外交政策》杂志现在每年进行全球脆弱国家指数(Fragile States Index,2013 年前称为失败国家指数,英文为 Failed States Index)的评级。\nauthor{全球脆弱国家指数评级,参见和平基金会网站(http://fsi.fundforpeace.org/)。}2014 年脆弱国家指数的评级显示,全球各国指数大致与该国人均 GDP 的数据相当,当然并不完全一致。很多发达国家都处于可持续(snstainable)或稳定(stable)级别,非洲很多国家处于警告(warning)或危急(alert)级别。脆弱国家指数或失败国家指数衡量的不是一个国家的民主程度或自由程度,而是衡量国家本身能否有效运转。最糟糕的一种情形,是一个国家陷入内战或完全崩溃的境地。

所以,如果身处那些被评级为危险或危急的国家,那里的人们首先关心的可能不是民主与自由,而是安全和秩序。美国著名政治学家塞缪尔 · 亨廷顿在 1968 年《变化社会中的政治秩序》中认为:

\quo{首要的问题不是自由,而是建立一个合法的公共秩序。人当然可以有秩序而无自由,但不能有自由而无秩序。必须先存在权威,而后才谈得上限制权威。\nauthor{塞缪尔 · P.亨廷顿:《变化社会中的政治秩序》,王冠华、刘为译,上海:上海人民出版社 2008 年版,第 6 页。}}

亨廷顿的观点如果过度解读,就会有逻辑问题,自由和秩序并不是完全对立的。但他提供了思考政治问题的另一个视角。当然,引文中的这个观点不仅遭到很多人的批驳,而且亨廷顿本人在其后续著作中已对此做了大幅修正。但对于今天的 “脆弱国家” 或 “失败国家” 来说,这一思考视角仍然极其重要。

\tsection{国家起源的逻辑:安全与暴力}

国家的起源与人性的基本渴望有关。人性中有两种基本渴望:一种是对安全的渴望,一种是对自主的渴望。人既希望拥有安全,又希望实现自主。但是,如果二者无法兼得时,人可能首先会放弃自主,选择安全。换句话说,当人面对高度的不确定状态、其安全受到威胁时,很多人首先考虑的是安全问题。

而安全这种基本需求,跟国家能否提供秩序有关。但是,过去国家问题没有受到应有的重视。随着 20 世纪 80 年代国家理论的兴起,学术界越来越认识到国家本身就是一个重要的政治问题。

实际上,正如第 1 讲提及的,霍布斯在《利维坦》中已阐明国家问题的基本逻辑。他说:

\quo{在没有一个共同权力使大家慑服时,人们便处在所谓的战争状态之下。这种战争是每一个人对每个人的战争。……

这就是伟大的利维坦的诞生。……根据国家中每一个人授权,他就能运用付托给他的权力与力量,通过其威慑组织大家的意志,对内谋求和平,对外互相帮助抵御外敌。\nauthor{霍布斯:《利维坦》,黎思复、黎廷弼译,北京:商务印书馆 1996 年版,第 94,132 页。}}

恩格斯在《家庭、私有制和国家的起源》中的观点,也跟霍布斯的逻辑有关。恩格斯这样说:

\quo{确切说,国家是社会在一定发展阶段上的产物;国家是承认:这个社会陷入了不可解决的自我矛盾,分裂为不可调和的对立面而又无力摆脱这些对立面。而为了使这些对立面,这些经济利益互相冲突的阶级,不致在无谓的斗争中把自己和社会消灭,就需要有一种表面上凌驾于社会之上的力量,这种力量应当缓和冲突,把冲突保持在 “秩序” 的范围以内;这种从社会中产生但又自居于社会之上并且日益同社会相异化的力量,就是国家。\nauthor{马克思、恩格斯:《马克思恩格斯选集》第四卷,北京:人民出版社 1995 年版,第 170 页。}}

恩格斯的国家理论是基于政治冲突的视角,即从对冲突的考察、对秩序的需要来理解国家的起源。国家的必要性,与人类对于安全和秩序的需要有关。如果没有国家,社会就会陷入缺乏安全和秩序的状态。而安全和秩序的缺失,是人类最无法忍受的事情。尽管一个社会也许存在压迫,但与之相比,人类更不能忍受的是无秩序的状态。按照霍布斯的逻辑,没有国家意味着每个人都有可能对他人使用暴力。对每个人来说,这种状态是没有安全保障可言的。那么,如何遏制这种个人与个人之间可能的暴力呢?这就需要一个更大的暴力,需要在一个相当大的地理范围内控制更大暴力的机构,这样才能遏制其他个别的暴力。

所以,安全与暴力这两个看似对立的事物,本质上却有着相通的逻辑。众人能生活在一个相对安全的社会里,是因为一个拥有巨大暴力的机构的存在,这种巨大暴力的存在使得普通的个人与个人之间潜在的暴力被遏制了。这种情况下,没有人有权力对其他人使用或首先使用暴力。所以,安全与暴力貌似完全对立,却是相伴而生。美国社会学家查尔斯 · 蒂利把早期的国家构建过程比喻成有组织的暴力犯罪集团竞争的过程。那个最强大的有组织的暴力集团最终垄断了暴力,就完成了初步的国家构建。\nauthor{查尔斯 · 梯利:《发动战争与缔造国家类似于有组织的犯罪》,载于彼得 · 埃文斯、迪特里希 · 鲁施迈耶、西达 · 斯考克波:《找回国家》,方力维等译,北京:生活 · 读书 · 新知三联书店 2009 年版,第 228—261 页。}总之,一个社会的基本问题是需要一个最高的暴力机构来垄断特定地域范围内的暴力。这种状态无疑要比每一个人随时随地可能对其他人行使暴力要好得多。这是对国家起源的一种逻辑解读。

\tsection{从封建主义到现代国家}

那么,现代国家是在何时何地产生的呢?国际学界一般认为,现代国家起源于近代欧洲。当然,不少人不同意这种看法。按照后文介绍的现代国家的定义,中国秦汉时期的国家形态大致已经具有 “现代国家” 的主要特征。美国学者弗朗西斯 · 福山在《政治秩序的起源》中认为,中国东周至秦汉时期已经具有了国家形态。\nauthor{弗朗西斯 · 福山:《政治秩序的起源:从前人类时代到法国大革命》,毛俊杰译,桂林:广西师范大学出版社 2012 年版,第 109—134 页。欧洲则与此不同。}欧洲的中世纪在政治上处于割据状态,那是封建制度和封建主义盛行的时代。大约公元 990 年前后,欧洲大概存在几千个政治实体,包括很多王国、公国、侯国、伯爵领地及自治市等。到 1500 年,欧洲政治实体的数量减少到 500 个左右。到 1780 年,即法国大革命之前,数量减少到 100 个。到 2000 年,欧洲变成了 27 个国家。\nauthor{关于现代国家的兴起,参见朱天飚:《比较政治经济学》,北京:北京大学出版社 2005 年版,第 21—37 页。}

民族国家兴起之前,欧洲是封建主义时代。什么是封建主义?过去,国内对封建主义的概念存有普遍的误解。封建制是领主与封臣之间基于土地的恩赐而形成的一种政治经济安排。领主恩赐给封臣的土地一般称为采邑。领主把采邑分封给封臣——又称附庸,由此形成领主与附庸之间的依附关系。在这种依附关系中,领主不仅要为附庸提供土地,而且还有对附庸提供保护的义务。附庸需要向领主表示效忠,同时需要提供必要的援助——这种援助包括经济与财务的支持,但更主要是在战争期间提供军事援助。所以,领主与附庸之间既是一种等级制的人身依附关系,又有一定的契约关系。封建制度是一种融合了保护与效忠关系、人身依附与契约精神的复杂混合体。封建制度的基础是土地,领主享有某些特定的权利,同时需要承担某些特定的义务;附庸也是如此。

法国学者马克 · 布洛赫把欧洲的封建制度称为一种政体。那些拥有地产的人,同时控制政治。而封建契约某种程度上取代了过去罗马帝国时期的国家权力和正式的政治制度安排。在罗马帝国时代,中央权威是强大且具有渗透力的,但后来慢慢演变到了封建主义的结构。\nauthor{关于封建社会的一项权威研究,参见马克 · 布洛赫:《封建社会》,张绪山译,北京:商务印书馆 2005 年版。}那么,欧洲为什么会兴起封建主义?而不是出现某种类似于中国的中央集权官僚制帝国?美国中世纪史专家詹姆斯 · 汤普逊认为,封建主义的起源与罗马帝国解体之后欧洲的军事—政治—经济逻辑有关。君主们夺取了罗马帝国的权力之后,他们并没有足够的财力资源来维持官僚机构,没有钱来维持法院和司法系统,没有钱来支持军队的供养。怎么办?后来,君主们就把占领的土地分封给下面跟随他们一起带兵作战的那些人。后者慢慢就变成了拥有领主恩赐采邑的附庸。\nauthor{汤普逊:《中世纪经济社会史(300—1300)年》(上册),耿淡如译,北京:商务印书馆 1997 年版,第 251—301 页。}

从封建主义本身变迁来看,采邑的性质经历了一些重要的变化。一开始,采邑的赐予被视为临时性的,但后来,如果没有特殊情形,采邑的占有变为终身制了。尽管如此,等年迈的封臣去世以后,还需要举行一个领主重新赐予封臣继承人领地的仪式。意思是,采邑这个东西是领主的,领主通过这种隆重的仪式授予给新的封臣。但是,越是到后来,采邑慢慢地越来越不被认为是领主的,封臣的自主权力也大大增加了。当然,这并没有改变封建关系的本质:领主需要提供封地和保护,封臣需要对领主效忠并提供兵役,两者之间仍然存在一种与人身依附有关的契约关系。

图 4.1 大概描绘了欧洲封建社会的基本结构。在封建制度的顶端是国王,就是一定区域内最大的领主。国王把土地和头衔赐予贵族,贵族则要为国王提供士兵——主要是骑士。在平时,特别是在国王与其他政治体发生战争时,贵族就有提供士兵的义务。再往下,贵族和农户之间也是图中的关系,贵族为农户提供土地与法律保护,而农户则需要为贵族提供货币租与实物租。所以,正是这样一层一层的网络构成了整个封建社会的基本结构。当然,此图把很多东西忽略掉了,比如说大贵族与小贵族之间的关系。实际上,从领主到附庸的关系中,一个人的附庸又可能是另一个人的领主,这种领主—附庸的层级关系是很多的。

那么,在封建主义体系之中,有没有现代国家?没有。有没有中央集权或中央一体化的治理结构?没有。国王有没有常设的统一军队和大型官僚机构?没有。有没有统一的税收系统?没有。有没有统一的司法系统?也没有。在封建主义体系中,特别是在英格兰传统中,当时甚至还有这样的说法: “国王应该靠自己的收入养活自己。” 只有发生战争时,国王下面的贵族们——也就是大小级别不等的领主与附庸们,才有义务为国王提供军事支持。在平时,国王主要应该靠自己直接领地的收入来养活自己,而不是靠附庸们来供养他。所以,在封建主义体系中,国王与不同贵族之间的关系既是一种经济关系,又是一种政治关系;既包含了人身依附的等级制色彩,又包含一种基于传统的契约精神。

\img{../images/image00302.jpeg}[图 4.1 欧洲封建社会的结构]

资料来源:http://lifeexamination.wordpress.com。

过去,由于受到启蒙运动思潮的影响,封建主义盛行的中世纪常常被视为黑暗时代。一种比较主流的观点认为,欧洲古希腊和古罗马的文明是很辉煌的,启蒙运动之后的欧洲文明也是很辉煌的,而两者之间是一个相对衰落的时代。但是,现在这种倾向已经改变。比如,具有代表性的一种观点是,詹姆斯 · 汤普逊在《中世纪经济社会史》对欧洲封建社会有很多积极的评价。他这样说:

\quo{尽管封建制度常常有着强暴而又恶劣的性质,无可置疑,整个来说,它是一个社会进步的和社会完整化的现象,而非一个社会腐烂的现象。如果我们从远处来看,我们能看到封建时代的文明是多么有建设性的、多么有创造性的和多么有伟大性的一种文明。……封建制度不是一座跨过野蛮和文明之间的海湾上的桥梁——它本身就是文明,这一种高级文化,在 1150 和 1250 年之间,达到了它的顶峰。\nauthor{汤普逊:《中世纪经济社会史》(下册),耿淡如译,北京:商务印书馆 1997 年版,第 326—327 页。}}

封建主义对现代政体的一个特殊贡献是,近现代立宪政体是从封建制度中直接演化出来的。应该承认,英国立宪政体就有着较为明确的封建主义起源。如果不是封建主义以及国王和贵族之间的政治结构,就难以理解英国立宪政体的起源。英国立宪政体起源的标志性事件是 1215 年《大宪章》的签署,而这一事件的背景就是英国的封建主义结构。1215 年,英格兰贵族联盟打败了国王约翰,然后胁迫他签下了《大宪章》。大宪章确立了英国贵族享有的一些政治权利与自由,亦保障了教会不受国王的控制;同时改革了法律和司法,限制国王及王室的行为。最初的《大宪章》有 63 条,其中影响最为深远的是第 39 条: “除非经过由普通法官进行的法律审判,或是根据法律行事;否则任何自由的人,不应被拘留或囚禁、被夺去财产、被放逐或被杀害。” 根据该条文,如果国王要审判一个人,只能依据法律而不是他个人的好恶。这在人类历史上是具有开创性的。实际上,正是在封建主义结构下的政治冲突,为建立约束国王政治权力的制度安排创造了条件。

众所周知,后来封建主义衰落了。封建主义衰落的过程,就是欧洲近现代国家崛起的过程。为什么封建主义会衰落?学术界有不同的解释,其中一种流行的观点是技术性解释。在冷兵器时代,重装骑士的优势非常明显。但骑士的装备非常昂贵,一个重装骑士的装备相当于 20、30 户农户的全家财产。这意味着普通人是很难作为士兵出征的,只有贵族才能供养这样的骑士。这就使得贵族具有非常强的军事优势。此外,贵族的城堡本身也是一项非常重要的军事设施。在冷兵器时代,重装骑士作为主要的进攻手段,坚固的城堡作为主要的防守手段,两者结合了起来,决定封建主义时代的军事模式。总的来说,这种模式下,防守一方更为有利一些,重装骑士通常较难攻破坚固的城堡。因此,城堡某种程度上成了欧洲封建制度和领主权力的政治象征。

但是,后来战争的技术条件发生了重要变化,中国人发明的火药经由阿拉伯世界传到了欧洲,后来欧洲又出现了火炮。火炮的出现使得封建贵族的城堡不再牢不可破,城堡根本无力抵御火炮的轰击。由于火炮非常昂贵,国王的武器购买能力要远远高过贵族,所以国王与贵族的军事优势差距也变大了。这样,国王在武器方面就逐渐取得了明显的优势。所以,这种理论认为,火炮的发明和传入欧洲,成为最终瓦解封建制度的主要原因。\nauthor{比如,斯塔夫里阿诺斯在《全球通史》中也采纳了这一观点,参见斯塔夫里阿诺斯:《全球通史》(上册),董书慧等译,北京:北京大学出版社 2007 年版,第 394—395 页。}

查尔斯 · 蒂利把现代国家的兴起视为一个 “战争塑造国家,国家制造战争” 的过程。上文曾提到,欧洲最初政治实体的数量非常之多,而这些政治实体之间不断地发生战争。为了打赢战争,一个君主最想拥有强大的军队。只有强大的军队,才能让一个君主生存下来。为了建设军队,君主需要依靠有效税收系统的支持,最好还有一整套官僚系统。所以,为了赢得战争,君主需要发展军队,需要发展官僚系统,需要发展税收系统。当一个君主的军队、官僚制和税收系统得以发展起来时,他就更有可能在对外战争中取得优势。而君主的军队、官僚制和税收系统的发展过程,实际上就是一个现代国家的塑造过程。在这样一个军事竞争格局中,凡是那些没有发展出军队、官僚制和税收系统的政治实体就慢慢被消灭了。那些拥有军事优势的政治实体通过发动战争会占领更大的地盘。正是这样,在一个 “战争塑造国家,国家制造战争” 的政治军事过程中,欧洲现代国家开始兴起。\nauthor{Charles Tilly, \italic{Coercion, Capital and European States: AD}990-1992(Revised Edition), Cambridge: Wiley-Blackwell, 1992.}

\tsection{理解国家的不同维度}

那么,什么是国家?如何定义国家呢?目前国际学术界广泛认可的定义是德国社会学家马克斯 · 韦伯提出来的:

\quo{国家是这样一个人类团体,它在一定疆域之内(成功地)宣布了对正当使用暴力的垄断权。\nauthor{马克斯 · 韦伯:《学术与政治》,冯克利译,北京:生活 · 读书 · 新知三联书店 2003 年版,第 51 页。}}

韦伯的国家定义被总结为一句简单的话:国家是合法垄断暴力的组织。美国社会学家迈克尔 · 曼根据韦伯的概念发展出了一个更详细的定义,他认为:

\quo{1. 国家是一组分工合作的制度和人员;

2. 具有向心性,即与中心有双向交流的政治关系;

3. 具有明确的地域;

4. 借助某种有组织的暴力,行使某种程度的权威,确保令行禁止。\nauthor{迈克尔 · 曼:《社会权力的来源》(第二卷上),陈海宏等译,上海:上海人民出版社 2007 年版,第 64—65 页。}}

更简单地说,曼认为现代国家是具有明确地域的一种强制性组织。这跟韦伯的定义非常接近。通过对国家定义的理解,可以总结出现代国家的几个基本特征:

第一,国家要有特定疆域。就是说,国家的地理范围是确定的,而不是随意变动的。现代国家无疑都有着特定的疆域。当国家边界不确定时,会产生很大的问题,甚至会威胁到国家本身的稳定。

第二,国家包含特定人口。问题是,有的国家人口的同质性程度很高,有的国家人口的异质性程度很高。人口的同质性程度高,自然对国家较为有利。那些族群、宗教和语言构成差异大的社会,国家面临的挑战会更大些。这个问题通常会涉及如何塑造国家认同和民族认同。

第三,国家的主要特征是垄断暴力。每个国家都需要把军队和警察部门掌握在自己手里。从很多国家的政治史来看,垄断暴力是国家构建过程中的核心问题。从韦伯到蒂利,这些学者都把处理暴力问题视为国家的基本问题。

第四,国家需要一整套官僚系统。政府是国家的载体,而政府是由一整套官僚系统组成的。这套官僚系统通常从中央延伸到地方,具有统一的命令协调体系,一般包含职能分工与层级分工。没有这套官僚系统的支持,国家是无法运转的。

第五,国家依赖于税收系统。通常,国家必须要靠社会来提供税收。按照诺思的说法,国家通过提供公共服务来换取收入,这里的收入主要是税收。税收问题曾经在西方政治史上产生过重要影响。有人认为,英国立宪国家的起源跟贵族和国王关于税收权的战争有关。美国独立革命的口号则是 “没有代表权,就不纳税” 。但无论怎样,任何国家都要以税收为基础。所以,建立一个全国性的税收系统也是国家构建的重要方面。

第六,国家主权需要得到国际承认。现代国家处于现代国际体系之中,国家主权是国家的主要特征之一。主权得到其他主要国家或国际社会的承认,也是现代国家的特点。

从历史维度和全球视野来看,国家可以区分为不同类型。第一个类型区分是国家权力的特征。在欧洲国家兴起的过程中,慢慢形成了两种国家类型:一种是绝对主义国家,一种是立宪主义国家(即宪政国家)。在立宪主义国家,君主权力受到宪法和法治传统的约束。在绝对主义国家,君主权力是绝对的,君主权力和意志是不受制约的。一般认为,英国是欧洲历史上第一个立宪主义国家,其宪政历史可以追溯至 1215 年《大宪章》。而大革命之前的法国是欧洲绝对主义国家的代表。欧洲不同国家的不同历史演进路径,是非常值得研究的一个问题。同样从封建主义到现代国家的演进,为什么会造成立宪主义国家和绝对主义国家的分野?

第二个类型区分是国家不同的功能和角色。不同的意识形态流派有其相对应的理想国家类型。同时,不同时空背景下也出现过功能与角色差异较大的不同国家类型。当然,这并非严格意义上的 “国家类型学” 。比如,古典自由主义欣赏的是 “守夜人” 国家,即自由放任国家。这种类型下的理想国家主要扮演警察与法官的角色。他们相信: “管得越少的政府越是好政府。” 比如,从 19 世纪晚期到 20 世纪以来,欧洲逐渐兴起了福利国家,福利国家的关键词是社会福利。这种国家观认为,国家不仅要给社会提供基本保护,还要为公民提供基本的社会福利。今天,欧洲发达国家或多或少都具有福利国家的色彩。再比如,20 世纪 30—40 年代,希特勒统治的德国一般被称为极权主义国家,又称政治全能主义国家。极权主义国家的关键词是政治控制,即国家通过政治手段渗透到经济、社会、甚至家庭各个领域,试图对个人和社会实现无所不包的政治控制。又比如,随着东亚国家的经济崛起,国际学术界又提出了发展型国家的概念,这类国家的关键特征是发展导向。发展型国家通过适当规划和政府干预,在落后社会实现了经济增长与繁荣。最后,非洲等一些落后地区的国家则容易沦为掠夺型国家,这类国家以某种程度的 “盗匪统治” 为特征。在这类情形下,统治集团利用国家权力对社会进行肆无忌惮地掠夺。这样,经济和社会发展几乎是没有可能的。

讨论国家功能和角色,就涉及不同理论流派关于国家角色的论战。关于国家的理想角色,目前大致有三种主要观点。第一种观点认为,国家的理想角色主要是提供保护;第二种观点认为,国家的理想角色主要应该是促进发展;第三种观点认为,国家的理想角色主要应该是提供福利。启蒙运动以来的很多西方思想家认为,国家的首要角色应该是提供保护,主要是保护社会成员的生命权、自由权和财产权,国家不需要也不应该做太多的事情。国家治理的重点是保护公民自由和财产权利,提供法律和秩序,并强制执行契约,而非干预经济和社会活动。这种观点属于国家角色的自由主义传统。

然而,很多发展中国家赢得独立时面临着与欧洲发达国家完全不同的情境。比如,印度 1947 年独立时,经济发展水平非常低,人均 GDP 大约只有数十美元。以贾瓦哈拉尔 · 尼赫鲁为代表的印度领导阶层多数都在英国留过学,他们了解英国和欧洲国家的发展水平。所以,这些人掌握政治权力以后,他们的天然想法就是要用国家的力量来发展这个社会。国家不仅要扮演提供法律、秩序和保护的角色,最好还能成为经济与社会发展的促进者与推动者。所以,尼赫鲁在当时的印度就实行了既强调国家指导又容纳民营企业的某种计划经济模式。实际上,20 世纪中叶,很多发展中国家都走到了政府干预的经济道路上去。当然,有的国家做得并不成功,比如印度;有的国家做得极其出色,比如朴正熙时代的韩国。但无论怎样,很多发展中国家的政治家和精英集团试图把国家作为发展的一种工具,是完全可以理解的。

此外,从 19 世纪晚期到 20 世纪上半叶,欧洲的发达国家则走上了普遍提供社会福利的道路,福利国家成为欧洲的一种潮流。德国从俾斯麦时代开始,尝试建立起初步的社会保险和福利制度,后来这一做法扩展到其他欧洲国家。当然,这种福利国家的模式在 20 世纪 70—80 年代以后遇到了很多挑战。今天,福利国家模式已经使得很多欧洲国家政府收支占 GDP 的比例达到了 40\% —60\% 的水平,逐年累积的财政赤字更使不少欧洲国家公债占 GDP 的比例达到 100\% —160\% 。总的来说,今天的欧洲国家已经把提供社会福利视为自己当然的职责,但同时福利国家模式面临着严重问题,改革迫在眉睫。

\tsection{国家理论的不同流派}

如何从理论上理解国家呢?这是政治学的古老问题。尽管柏拉图和亚里士多德没有讨论今天意义上的国家理论,但他们的政治思想中已涉及国家问题。文艺复兴之后及民族国家兴起的过程中,英国的霍布斯和法国的博丹则把国家视为一个中心问题。从 19 世纪到今天,学术界已经形成了几种主要的国家理论。\nauthor{关于不同的国家理论流派及国家主义的理论文献,参见迈克尔 · 曼:《社会权力的来源》(第二卷上),陈海宏等译,上海:上海人民出版社 2007 年版,第 50—107 页;朱天飚:《比较政治经济学》,北京:北京大学出版社 2005 年版,第 85—103 页。}

第一,多元主义国家理论。多元主义理论沿袭自由主义传统,一般适用于现代的自由民主国家。在多元主义视角下,代表不同社会利益的政党和集团在国家这个政治舞台上进行政治竞争,背后则是民众的广泛参与和不同利益的表达。在这一理论框架中,国家本身没有自己的自主性和利益。国家就好比是一个巨型剧院的中央舞台,谁都可以在上面表演节目。你能够竞争到多少时间和多少舞台面积,你就拥有多少影响。所有的利益集团和政党都可以平等地竞争,最后国家的公共决策则被视为不同利益集团政治竞争和博弈的一种均衡。这是自由主义的国家理论。

第二,马克思主义国家理论。马克思主义国家理论的基础是阶级分析方法。按照马克思与恩格斯的说法,迄今为止所有有文字记载的历史都是阶级斗争的历史。国家是为一个社会的经济生产方式和阶级利益服务的;或者更直接地说,国家是阶级统治的工具。通过国家,统治阶级实现了对被统治阶级的政治统治,从而可以更好地实现其经济利益。马克思主义国家理论具有很强的经济决定论色彩。经济上的支配阶级,也是政治上的支配阶级。国家的本质是经济上的支配阶级实现对被支配阶级进行政治统治的工具。从这种理论视角出发,国家本身并没有自主性,国家的意志不过是统治阶级意志的反应,其最终目标是服务于统治阶级的经济利益。

第三,新古典国家理论。这是理性选择学派的国家理论,国家被模型化为一个追求统治收益最大化的统治者。这种国家理论采用的是经济人假设和新古典经济学的分析方法。经济人假设把人视为自利的、理性计算的和追求效用最大化的。而国家被等同于统治者,统治者也是自利的、理性计算的和追求统治收益最大化的。这种国家理论是道格拉斯 · 诺思在《经济史上的结构与变迁》中阐发的。诺思认为,统治者用提供基本的公共服务来换取统治收入,谋求的是统治收益的最大化。但是,统治者同时受到两个条件的约束:一是交易成本——这是新制度主义经济学的核心概念;二是竞争约束——就是说现有统治者干得不好时潜在竞争对手会出现。\nauthor{道格拉斯 · C.诺思:《经济史上的结构与变革》,厉以平译,北京:商务印书馆 2005 年版。}诺思在这样一个理论框架中理解国家和解释国家行为。

第四,国家主义国家理论。这是关于国家的精英主义视角,后面还会详细介绍。比如美国学者西达 · 斯考切波就把国家为 “一套具有自主性的机构” 。国家具有自主性,而且根据国家自身利益行事。由此可见,这种理论框架具有强烈的国家中心论视角,而非过去很多理论沿袭的社会中心论。尽管这一理论遭遇了很多挑战,但国家主义现在是一个非常流行的理论流派,其重要性至今并未减弱。

如何恰当地评价这些不同的国家理论是一件困难的事情。每一种理论都倾向于站在自己理论的优势方面对其他理论进行批评,而实际上每种理论都有其解释力的不足。因此,与其说哪种国家理论更接近真理,不如说哪种理论在何种方面、何种情境下更能帮助大家理解国家这一政治分析的中心问题。

\tsection{国家构建与国家能力}

关于国家的理论思考由来已久,但启蒙运动以来的思想传统倾向于把国家视为一种 “必要的恶” ,这是一种自由主义的视角。正因为如此,关于国家本身的理论并没有得到很好的发展。

到了 20 世纪 80 年代,国家理论研究迎来了它的春天。1985 年,彼得 · 埃文斯等学者主编的《找回国家》(\italic{Bring the State Back in})一书的出版,通常被视为政治学研究中国家主义理论复兴的标志性事件。这部著作认为,社会科学研究中以社会为中心的理论视角应该被摒弃,而国家与社会互动的视角或者以国家为中心的视角是值得倡导的。

国家主义理论常用的三个概念分别是国家自主性、国家构建和国家能力。美国哈佛大学教授西达 · 斯考切波认为其他理论——

\quo{都没有将国家看成一套具有自主性的结构——这一结构具有自身的逻辑和利益,而不必与社会支配阶级的利益和政体中全体成员群体的利益等同或融合。\nauthor{西达 · 斯考切波:《国家与社会革命:对法国、俄国和中国的比较分析》,上海:上海人民出版社 2007 年版,第 27—28 页。}}

为什么存在国家自主性呢?有学者认为,这种理论视角最有效的解释力来自于这样一个事实,即现代国家总是处在一个国家构成的全球体系中,处在国际体系和国内社会临界点上。这样,国家只能根据国家竞争和地缘政治需要来采取政治行动。因此,国家不一定会受国内社会力量的直接约束,而是从适应国际体系中国家间竞争的需要出发自主地行动。这被视为国家自主性的重要来源。\nauthor{参见朱天飚:《比较政治经济学》,北京:北京大学出版社 2005 年版,第五章。}国家自主性意味着国家或政府可以不受社会力量、阶级力量、甚至是官僚集团力量的左右,它有自己的特定意愿和诉求。所以,国家自主性也是指国家独立于社会进行自我决策的程度。

尽管如此,国家自主性的概念至今充满争议,这些争议主要来自于两个方面。首先是理性选择学派提出的问题:国家自主性的概念背后有没有微观基础?国家自主性的微观基础是什么?是官僚集团的利益吗?如果不是,那又是什么呢?如果是官僚集团的利益,那么就不是国家本身的利益,也就谈不上国家自主性的概念。这还与另一个重要问题有关,即 “几乎没有一个国家是统一的运作者” 。比如,对于 “国家按照自己的利益行事” 这种说法,基欧汉和奈等学者就追问过: “是什么样的自己,是什么利益?” 国家主义理论和国家自主性的概念对这些质疑总体上较难有效回应。其次是几乎没有哪个国家拥有执行其国家目标和意志所需的全部资源。为了实现国家目标和意志,国家统治精英几乎都需要以某种方式与社会中的强势集团进行结盟或协商。一旦国家与社会的互动贯穿其中,国家自主性就可能成为一个问题。这是另一种质疑的视角。\nauthor{参见迈克尔 · 曼:《社会权力的来源》(第二卷上),陈海宏等译,上海:上海人民出版社 2007 年版,第 58—59 页。}

另一个常用概念是国家构建。国家构建是指一个现代国家或一个有效的现代国家塑造的过程,这个过程可以从三个维度来理解。

第一,政治共同体的形成。比如,19 世纪末期的德意志有那么多不同的政治实体,能不能成为一个统一的国家就是一个巨大的挑战。1960 年,尼日利亚独立时的一个重大挑战就是该国能否真正成为一个完整统一的政治共同体。在一些国家的内部,始终存在部分地区要统一还是要分裂的问题。比如,最近发生在乌克兰克里米亚地区的政治危机,就是一场事关政治共同体完整性的危机。总的来看,一个国家内部的族群、宗教及语言的多样性程度较高时,要建立统一的政治共同体的难度就更大。

第二,国家机构与制度建设的问题。政治制度的建设、官僚系统的建设、军队的组建与集中控制以及税收系统的建设,都是国家构建的重要方面。对很多非洲国家来说,20 世纪 60 年代欧洲殖民者离开非洲时留下了一整套国家机构与制度,但这套系统并不完善。在一些国家,欧洲殖民者还把部分政府机构撤走了。所以,很多非洲国家就需要从头建设这样一套国家机构与制度,但这方面的挑战很大。在当时的非洲,有的国家只有数十名大学生,不少国家城市化率不到 10\% ,工业化程度非常低,绝大多数人口都不识字。这种情况下,要建立一整套官僚机构和政治制度就非常困难。要知道,现代官僚制很大程度上是以书写系统为基础,公共治理是依靠一整套公文和书写系统来实施的。

第三,国家能力塑造和增强的过程。上文曾经提及,汉密尔顿在《联邦党人文集》中说:

\quo{政府的力量是保证自由不可缺少的东西。……

软弱无力的行政部门必然造成软弱无力的行政管理,而软弱无力无非是管理不善的另一种说法而已;管理不善的政府,不论理论上有何种说辞,在实践上就是个坏政府。\nauthor{汉密尔顿、杰伊、麦迪逊:《联邦党人文集》,程逢如、在汉、舒逊译,北京:商务印书馆 1995 年版,第 5、356 页。}}

对很多发展中国家来说,国家能力的塑造和增强非常重要。如果国家能力低下,很多发展中国家无力应付各种基本的政治、经济与社会挑战,那么建设现代政治文明就无从谈起。对落后国家来说,国家构建的首要困难来自于经济社会条件的约束。对于经济落后、教育尚未普及、城市化和工业化程度很低的国家来说,经济与社会发展构成其国家构建的基础条件,其次才谈得上政治共同体的建设,官僚制、军队系统和税收体系的完善,以及政府能力和效能的建设。实际上,这些国家面临着国家构建与社会发展的悖论。这个悖论是:没有一个有效的国家,社会经济很难发展起来;但同时,如果没有起码的经济与社会发展,就不能形成一个有效的国家。这就使得不少国家陷入 “国家构建—社会发展” 的恶性循环。

那么,什么是国家能力呢?不同学者对此有不同的定义。总的来说有两种视角:一种是从国家与社会的关系视角来定义,一种是从国家本身来定义。斯考切波认为,国家能力是指 “面对力量强大的社会组织实际的或潜在的反对时,国家执行其正式目标的能力” 。这里强调的是国家独立于社会,国家自主地执行其目标的能力。米格代尔认为,国家能力是指国家领导层 “通过国家的计划、政策和行动来实现其改造社会的目标的能力” 。他强调的是国家对社会的改造。上述两种定义都强调了国家与社会之间的关系。\nauthor{参见西达 · 斯考切波:《国家与社会革命:对法国、俄国和中国的比较分析》,上海:上海人民出版社 2007 年版;乔尔 · S.米格代尔:《强社会与弱国家——第三世界国家社会关系及国家能力》,张长东等译,南京:江苏人民出版社 2012 年版。}

但是,蒂利对国家能力的定义更多着眼于国家本身。简洁地说,国家能力是 “国家执行其政治决策的能力” 。更具体地说, “国家能力是指国家机关对现有的非国家资源、活动和人际关系的干预,改变那些资源的现行分配状态,改变那些活动、人际关系以及在分配中的关系的程度。” \nauthor{查尔斯 · 蒂利:《民主》,魏洪钟译,上海:上海人民出版社 2009 年版,第 14—15 页。}这一定义并没有强调国家与社会之间的互动。

香港中文大学教授王绍光认为,国家能力是 “国家将自己的意志、目标转化为现实的能力” 。在他看来,国家能力的构成可以分解为六个具体的维度: “第一,对暴力合法使用的垄断;第二,提取资源;第三,塑造民族统一性和动员群众;第四,调控经济和社会;第五,维持政府机构的内部凝聚力;第六,重新分配资源。” \nauthor{王绍光:《安邦之道:国家转型的目标与途径》,北京:生活 · 读书 · 新知三联书店 2007 年版,第 61 页。}当然,王绍光对国家能力的界定具有强烈的国家干预倾向,因而也充满争议。

\tsection{国家能力的不同视角}

迈克尔 · 曼把国家能力区分为两种类型:专制性权力(despotic power)和基础性权力(infrasturatural power),国内也有人分别译为 “独裁性权力” 与 “建制性权力” 。迈克尔 · 曼认为:

\quo{专制性权力是国家精英对于市民社会的分配权力(distributive power)。这种权力源自于国家精英无须与市民社会集团进行常规性协商就能采取的行动。}

这意味着专制性权力来源于强制力。专制性权力强,意味着国家可以不必与市民社会协商,不必经过与社会讨价还价,即可自行决策及将决定强加于社会的权力。有些国家的专制性权力强,就是指政治权力可以自行其是的随意性比较大。

\quo{建制性权力是一个中央化国家(a central state)——无论专制的还是非专制——的制度能力,这种制度能力能够渗透其所辖的疆域,并在后勤保障上贯彻其决定。这是一种集体性权力,权力基于国家的基础结构,并通过社会来协调社会生活。\nauthor{Michael Mann, \italic{The Sources of Social Power, Vol.}II, \italic{The Rise of Classes and Nation-States}, 1760-1914, Cambridge: Cambridge University Press, 1993, p.59.}}

这意味着基础性权力是国家以制度化方式向社会渗透、与社会互动而在其领土范围内有效贯彻其政治决策的能力。基础性权力依赖于稳定的制度化路径,而专制性权力依赖的是随意的强制力。比较而言,前者有更高的合法性,后者的合法性更低。

基于历史的考察,曼认为存在四种不同组合:专制性权力和基础性权力都弱的类型;专制性权力强而基础性权力弱的类型;专制性权力弱而基础性权力强的类型;专制性权力和基础性权力都强的类型。曼这样区分这四种理想类型:

\quo{封建主义国家结合了很少的专制性权力和基础性权力,它几乎没有能力干预社会生活。……罗马帝国、中华帝国和欧洲绝对主义王权……都拥有断然的专制性权力,而少有基础性权力。……现代西方自由主义官僚制国家……拥有广泛的基础结构,这一基础结构在很大程度上受制于资本家或民主程序。……现代威权主义国家——顶峰时期的苏联——同时拥有专制性权力和坚实的基础结构(尽管它们的凝聚力没有我们想象的那么强)。\nauthor{Michael Mann, \italic{The Sources of Social Power, Vol.}II, \italic{The Rise of Classes and Nation-States}, 1760-1914, p.60.}}

从曼的上述分析来看,像美国这样的发达民主国家,国家能力总体上是比较强的,但这类国家强的是基础性权力,而非专制性权力。历史上的传统君主专制国家,由于其政治权力没有受到正式约束,所以专制权力通常是比较强的,但这些国家的基础权力通常都很弱。

弗朗西斯 · 福山在《国家构建》中区分了国家能力的两个维度:一是国家职能范围,二是国家力量强弱:

\quo{有必要将国家活动的范围和国家权力的强度区别开来,前者主要是指政府所承担的各种职能和追求的目标,后者是指国家制定并实施政策和执法能力特别是干净的、透明的执法能力——现在通常指国家能力或制度能力。\nauthor{弗朗西斯 · 福山:《国家构建:21 世纪的国家治理与世界秩序》,黄胜强、许铭原译,北京:中国社会科学出版社 2007 年版,第 7 页。}}

福山首先界定了国家职能的范围,他借鉴了世界银行《1997 年世界发展报告》对国家职能范围的划分:第一种是最小职能,基本上就是自由放任国家应该做的事情;第二种是中等职能,这种职能与自由放任国家不同,基本接近于 20 世纪现代自由主义理念下的国家职能;第三种是积极职能,政府要比在现代自由主义的理念下做了更多的事情。简单地说,国家做事的多少,是国家职能范围的概念。

福山强调的第二个维度是国家力量强弱。福山认为:

\quo{它表示制度能力的强度。在这个意义上的强度包括前面提到的制定和实施政策以及制定法律的能力,高效管理的能力,控制渎职、腐败和行贿的能力,保持政府机关高度透明和诚信的能力以及(最重要的)执法能力。\nauthor{弗朗西斯 · 福山:《国家构建:21 世纪的国家治理与世界秩序》,第 9 页。}}

\img{../images/image00303.jpeg}[图 4.2 国家职能的范围与国家力量的强度]

基于上述两个视角的讨论,福山根据国家职能范围大小和国家力量强弱区分出四种不同类型的国家。如图 4.2 所示,这四种国家类型组合分别是:(1)国家职能范围小,国家力量强度高;(2)国家职能范围大,国家力量强度高;(3)国家职能范围小,国家力量强度低;(4)国家职能范围大,国家力量强度低。举例来说,美国大致应该归入第一象限,即国家职能范围小但国家力量强度高。福山对美国评价是: “美国建立的是一套有限政府制度,在历史上就限制了国家活动的范围。但在这个范围内,国家制定及实施法律和政策的能力非常之强。” \nauthor{弗朗西斯 · 福山:《国家构建:21 世纪的国家治理与世界秩序》,第 11 页。}通过以上分析,大家就能较为完整地理解国家职能范围与国家力量强弱的概念。

查尔斯 · 蒂利则把国家能力与一国的政体类型进行了组合分析。他把国家政体类型区分为民主与不民主,把国家能力类型区分为高能力与低能力。国家能力对民主国家之所以重要,是因为 “如果国家缺乏监督民主决策和将其结果付诸实现的能力,民主就不能起作用” 。由此,蒂利区分了四种不同类型的国家,如图 4.3 所示。\nauthor{查尔斯 · 蒂利:《民主》,魏洪钟译,上海:上海人民出版社 2009 年版,第 14—17 页。}

\img{../images/image00304.jpeg}[图 4.3 国家能力与政体类型]

按照上图的框架,蒂利把哈萨克斯坦、伊朗等国归入高能力不民主国家类型,把索马里、刚果(金)等国归入低能力不民主国家类型,把挪威、日本等国归入高能力民主国家类型,把牙买加、比利时等国归入低能力民主国家类型。这也是一种重要的类型划分。对转型国家来说,理想目标应该是成为高能力民主国家。

\tsectionnonum{推荐阅读书目}

贾恩弗朗哥 · 波齐:《国家:本质、发展与前景》,陈尧译,上海:上海人民出版社 2007 年版。

西达 · 斯考切波:《国家与社会革命:对法国、俄国和中国的比较分析》,上海:上海人民出版社 2007 年版。

乔尔 · S.米格代尔:《强社会与弱国家——第三世界的国家社会关系及国家能力》,张长东等译,南京:江苏人民出版社 2012 年版。

弗朗西斯 · 福山:《国家构建:21 世纪的国家治理与世界秩序》,黄胜强、许铭原译,北京:中国社会科学出版社 2007 年版。
