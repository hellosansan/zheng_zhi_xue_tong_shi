\tchapter{民主转型的政治逻辑}

\quo[——亚当 · 普沃斯基]{富裕国家更有可能成为民主国家,不是因为民主的出现是威权统治下经济发展的一个结果,而是因为民主——无论民主是怎样出现的——在一个富有的社会更有可能存活下去。}

\quo[——亚里士多德]{惟有以中产阶级为基础才能组成最好的政体。中产阶级(小康之家)比任何其他阶级都较为稳定。他们既不像穷人那样希图他人的财物,他们的资产也不像富人那么多得足以引起穷人的觊觎。}

\quo[——托马斯 · 卡罗瑟斯]{在近些年被认为是 “转型国家” 的接近 100 个国家中,只有相对很少的国家——大概不足 20 个国家——正在朝着通往成功的、运转良好的民主制度的道路上迈进,或者在民主方面已经取得了某种进步和依然拥有民主化的积极力量。……迄今为止第三波的大多数国家并没有实现运转良好的民主制度,或者不能深化他们已经在民主方面取得的进步。}

\quo[——西摩 · 马丁 · 李普塞特]{民主是成功还是失败,将继续主要取决于政治领导人和领导集团的选择、行为和决策。}

\tsection{民主转型遭遇僵局?}

乌\nauthor{这部分内容曾以《民主转型僵局》为题刊载于《南风窗》2014 年第 7 期,内容有修订。}克兰如今深陷政治危机,2014 年 3 月 16 日克里米亚独立公投的结果更使该国局势雪上加霜。从 1991 年到 2014 年,乌克兰的政治转型已历时 23 年,但仍然没有形成稳定有效的民主政体。联想到其他转型国家,这两年民主的坏消息似乎多过了好消息。从乌克兰到委内瑞拉,从泰国到埃及,民主转型纷纷陷入尴尬境地。按照国际主流评级机构的报告,自 2006 年以来,全球转型国家的民主就处在轻微衰退之中,由此引发了对第三波民主化回潮的担忧。

乌克兰问题已经牵动整个欧洲的局势,但这场危机的起源乃是乌克兰国内政治的困境。尽管已经历多次大选,但乌克兰的政治转型过程并未完成,其政体类型只能被归入介于民主与威权之间的两不像政体。目前,乌克兰甚至不能排除陷入大规模武装冲突的可能。泰国看上去一直在转型,却总是无法拿到转型学校的毕业证书。选举、示威与政变是泰国政治的三个关键词。在泰国,无论谁当选,声势浩大的反对派就会涌上曼谷街头,持续抗争,甚至直至政治系统瘫痪。如果说民主政体得以维系的前提是 “败选者的同意” (losers' consent),那么泰国恰恰有一大批 “永不服输的败选者” 。2014 年 5 月,泰国军方再次宣布军事政变。埃及的政治困境至少同样严重。民选总统、穆斯林兄弟会领导人穆罕默德 · 穆尔西并没有办法应付埃及社会面临的重大问题,他无力解决穆斯林兄弟会与其他派别的冲突,无力控制强大的军方,亦无力领导国家实现政治和解与走上经济繁荣之路。在不到一年的时间里,先是兴起了大规模的民众抗争,而后是军事政变罢黜了总统。如今,埃及的军政大权又落入了赛西将军的手中。在这位军事领袖的干预下,埃及法院在 2014 年底判处 500 多位穆斯林兄弟会成员死刑。所以,美国学者内森 · 布朗甚至悲观地断言: “埃及转型已经失败。” 

上述几国的转型故事跌宕起伏,情节各异,但逻辑却是相似的。它们都陷入了不同形式的 “民主转型僵局” 。这一转型僵局由循环往复的三个阶段构成:一是威权体制的瓦解和启动转型;二是民主运转的困难及其引发的各种难题;三是威权方式作为解决问题方案的登场和威权政体的回归。这些国家摆脱威权政体之后,就启动了民主转型。但是,新兴民主政体的运转可能会遇到很大的困难。威权政体的终结,使得各种政治力量都得到了释放,普遍的政治参与和政治竞争成为常态。然而,由于种种原因,很多国家政治参与和政治竞争的结果,并不是稳定有效的政府和良好的公共治理,而是政治冲突的激化,政治家与党派竞争的不择手段,甚至可能是政治秩序的失控。对这些国家,民主带来的不是秩序与繁荣,而是混乱与停滞。一旦走到这一步,民主游戏就很难玩下去了。这样,手握实权的民选领袖或军人通过威权方式解决问题的诱惑一直在增加。最终,他们采取了行动,于是该国政体又退回到了某种威权体制或准威权体制。

第三波民主化国家的经验揭示,转型可能会出现三种不同的结局:最好的结局是完成民主转型并实现民主巩固;最坏的结局是转型失败并重新回到威权政体,在此过程中可能还伴随着秩序失控、暴力事件和流血冲突,甚至是国家分裂的危机;中间状态的结局则是新政体兼具民主因素与威权色彩,反复摇摆于民主和威权之间,但已丧失继续转型的政治动力。后面两种情况都意味着该国陷入了转型僵局。

 “好事不出门,坏事传千里。” 这句谚语说明坏消息比好消息更容易传播。一个转型国家的军事政变或严重暴力事件容易占据世界大报的头版,但另一个转型国家一场如期而至、波澜不惊的大选却常常无人问津。这种政治传播的模式,客观上使人们更容易看到民主的问题而非机会。放眼全球,民主既有好学生,又有坏学生。那些陷入转型僵局而无力自拔的国家都算不上民主的好学生。但这些国家时常爆出政治危机的重磅新闻。由于这个原因,人们容易忽略了民主的好学生。后面一类国家启动转型之后,由于政治比较平稳,选举有序进行,所以不会引起国际社会的过分关注。然而,这类国家通常以日拱一卒的精神进行着稳健的民主建设。比如,韩国启动转型至今不到 30 年时间,已成功选举 6 位总统,实现多次政党轮替。如今,韩国被视为全球自由民主程度最高的国家之一。从治理绩效上看,韩国人均 GDP 已接近 24000 美元,经济、科技与文化影响力日益增加,属于新兴工业民主国家的典范。但谁能想到,这个国家此前半个世纪的历史中充斥着强人独裁、军事政变、政治暗杀和武力镇压反对派的现象。难以置信的是,韩国如今却已以优等生的成绩从转型学校毕业了。

同样容易被人遗忘的是,在东欧,波兰、捷克等国启动转型的时间比乌克兰早不了几年,但如今都是稳定的自由民主国家。在拉美,智利、巴西、阿根廷等国在 20 世纪 80 年代后期再次启动转型,它们尽管在经济和治理方面跟韩国尚有差距,但今天已被普遍视为民主巩固的国家。实际上,这个名单可以列得很长。

即便是最近两年,民主在一些国家止步不前的同时,却在另外一些国家迈出了新的步伐。比如,东南亚的新加坡、马来西亚、缅甸等国显露出新的民主迹象,非洲的肯尼亚和马里等国举行了基本符合透明、公正、非暴力原则的新大选,等等。可以预见,一些转型仍会遭遇挫折,但另一些转型将会获得成功。所以,尽管民主转型充满不确定性,却没有理由过分悲观。按照自由之家 2013 年的评估,过去 40 年中全球自由民主国家的数量已经从 44 个跃升至 90 个,国家比例则从 29\% 增长至 46\% 。若以长时段来考察转型问题,好消息要远远多过坏消息。总之,民主的确面临着很多问题,但也面临着至少同样多的机会。

\tsection{如何理解民主转型?}

要理解民主转型,最好先了解相关概念。关于什么是民主,本书第 5 讲已有详细介绍,这里重点介绍与民主转型有关的概念。民主转型(democratic transition)现在跟民主化一般是混用的,尽管两者的侧重点不同。民主转型是指从非民主政体转变为民主政体的过程。一些早期的民主转型研究者倾向于认为,威权政体的崩溃就意味向民主的转型。但实际上,政治转型过程是高度不确定的,时间也往往很漫长。如果以短期来考察,只有少部分转型国家能够建立起巩固的民主政体,而更多的国家要么只能建立一个准民主政体,要么又回到了威权政体。

正因为如此,民主巩固(democratic consolidation)后来成为民主转型研究中的一个重要议题。简单地说,民主巩固是指一国的民主政体不断被强化的过程——通过这一过程,民主能够继续生存并能防止可能的逆转。实际上,要准确定义民主巩固是困难的。尽管很多学者给出了民主巩固的定义,但大部分定义要么标准过低要么没有标准,难以衡量。林茨和斯泰潘则给出了一个更为具体的定义,他们认为民主巩固可以从三个维度加以衡量,一是行为层面,主要的政治力量不再考虑推翻民主政体;二是态度层面,压倒性多数的公众接受民主为唯一的游戏规则;三是宪法层面,所有政治行动者都在宪法框架内解决政治冲突。 “巩固的民主是一种政治情境,在这种情境中,简而言之,民主已经成为 ‘最佳的政体选择’ (the only game in town)。” \nauthor{胡安 · 林茨和阿尔弗莱德 · 斯泰潘:《走向巩固的民主制》,载于猪口孝、爱德华 · 纽曼和约翰 · 基恩编:《变动中的民主》,林猛等译,长春:吉林人民出版社 1999 年版,第 56—81 页。}在这种情境中,应该有越来越多人确信,除了民主不能接受别的政治规则,而且没有任何有实力的政治组织或力量试图推翻民主政体。这才意味着民主的巩固。

第三个概念是民主崩溃(democratic breakdown)。民主崩溃是指从民主政体蜕变为非民主政体的过程。在中国大陆,笔者是首次在学术论文标题中使用民主崩溃概念的学者,也贡献了目前惟一一部系统研究民主崩溃的中文专著——《民主崩溃的政治学》。这部著作认为,高度的选民政治分裂与离心型政体的结合倾向于导致民主政体的崩溃。\nauthor{包刚升:《民主转型的周期性:从启动、崩溃到巩固》,《二十一世纪》2012 年 4 月号,第 17—27 页;包刚升:《民主崩溃的政治学》,北京:商务印书馆 2014 年版。}

关于民主转型,大家首先要避免的是一种过分简单化的思考。过去的民主转型公式被称为三部曲:威权政体的崩溃是第一阶段,启动民主转型是第二阶段,建立和巩固新政体是第三阶段。但实际上,这一公式过分简化了民主转型的实际过程。正如笔者业已指出的,民主转型往往是一个充满不确定性、时间漫长的过程。

\tsection{民主史:从雅典、英国到现代}

讨论这些概念之后,需要回顾一下民主作为一种政治实践的历程。这里主要探讨两方面的内容:现代民主的起源——英国的实践,与 19 世纪到 21 世纪初的三波民主化浪潮。

一般认为人类最早的民主实践起源于古希腊的城邦国家,雅典城邦则是其典型代表。本书第 1 讲与第 2 讲对此已有介绍,不再赘述。由于古希腊人在世界政治史上的首创性贡献,不少人容易误认为古希腊的民主实践是近现代民主的源头。但实际上,这两者之间既没有历史上的前后传承,又没有思想上的重要关联。在古希腊,同时代的杰出思想家中几乎没有人认为雅典城邦民主制是一种理想的政体形式。在柏拉图和亚里士多德看来,民主就是平民政体或穷人政体,容易导致暴民统治。以老寡头名义发表的作品更是对民主大加鞭挞:民主 “讨好了暴民,而不是那些值得尊敬的人” ;民主 “允许最差劲的一群人开口发言,藉此谋求自己最大的利益” ; “有些时候,就是等上一整年,500 人会议或公民大会也不能解决问题” 。\nauthor{约翰 · 索利:《雅典的民主》,王琼淑译,上海:上海译文出版社 2001 年版,第 88—92 页。}

公元前 4 世纪晚期,古希腊城邦相继为马其顿王国和罗马所征服。公元 476 年西罗马帝国覆灭以后,西欧迎来了漫长的中世纪。在中世纪,古希腊的政治实践和重要思想并没有发挥多少影响。即使在 14—16 世纪的文艺复兴时期,复兴的也主要是古罗马的拉丁文明。布克哈特在《意大利文艺复兴时期的文化》一书中认为,文艺复兴时期, “希腊学术主要限于佛罗伦萨……它始终也没有像拉丁学术那样普遍” 。

所以,古希腊的古典民主制度并非英国宪政与民主的直接源头。英国的宪政和民主,是在西欧国家间竞争体系下本国封建体制演进的一种政治结果。自美国独立革命、法国大革命以来的自由民主浪潮,其最初的影响大体都可以追溯到英国。今天已建成巩固民主制度的国家中,包括美国、加拿大、澳大利亚、印度、南非在内,大约有 30 多个国家的民主政体直接脱胎于英国的殖民统治。因此,英国无疑是现代民主的源头。

尽管如此,就英国近代政治史而言,民主基本上不是其主流价值。英国人更看重的是自由、宪政、协商政治和权力制衡,而不是普选权与人民民主。历史地看,英国民主的形成最初并不是源自政治力量对民主本身的追求,而是立宪政体和贵族政治自然演进的产物。英国人首先拥有的是宪政体制、协商政治、权力制衡和受保护的公民自由权,而民主不过是这些制度安排下自然演进的结果。按照达尔的说法,英国的政治道路是 “先实现竞争性政治而后扩大参与” 。

简要地说,英国宪政和民主的演进大约经历了四个重要的阶段。(1)第一阶段是规定国王不能做什么,标志性事件是 1215 年 6 月 15 日英格兰贵族武力胁迫国王签订的《大宪章》。这份法律文件的开创性在于,它在人类历史上第一次以契约文本方式规定了 “国王不能干什么” 。《大宪章》共 63 条。第 1 条就规定了 “永远保障英格兰教会的自由,使她享受充分的权利与自由” 。影响最为深远的第 39 条规定, “除非经过由普通法官进行的法律审判” ,否则任何人都 “不应被拘留或囚禁、被夺去财产、被放逐、被杀害” 。这就构成了对国王权力的严格限制,并成为人身保护令的起源。第 61 条还规定,由 25 个大贵族监督《大宪章》的实施,如国王有所违反,这一贵族团体可以采用包括武力在内的各种手段迫使他改过自新。这一阶段的政治贡献是塑造了英格兰立宪政治的雏形,而 1688 年光荣革命建立的君主立宪政体是这个传统的延续。(2)第二阶段是设立一个专门机构来监督国王的行为和贯彻《大宪章》。这一设想最终导致了 13 世纪英格兰议会的产生,首先是大贵族、高级教士组成的会议,后来是骑士和平民代表也有资格参加的会议,这些都是英国成为 “议会之母” 的关键步骤。议会的产生有力地推动贵族政治力量相对于国王权力的上升、协商政治和权力制衡的发展,以及地方代表选举制度的尝试。这一阶段的主要贡献是议会的产生。(3)第三阶段是责任内阁制的出现和发展。从最初的 “王在议会” 到后来的政治权力从国王向议会的转移,在 18 世纪内阁制逐步形成。英格兰内阁的起源可以追溯到中世纪的 “小会议” 和后来的枢密院,而 1742 年首席财政大臣罗伯特 · 沃波尔因得不到议会多数支持而辞职,标志着责任内阁制的形成,这是这一阶段的主要贡献。(4)第四阶段是议会改革、选举资格限制的放开和普选权的落实。尽管中间也经历了 19 世纪 30—40 年代宪章运动的重大冲击,但英国普选权的落实总体上是和平的、渐进的议会改革和选举改革的结果。早在 13 世纪中叶,英格兰议会中就有地方和自治市选派的平民代表。19 世纪之后,英国先后经历了 1832 年、1867 年、1884 年、1918 年和 1928 年五次重大选举改革,逐步放开了对选民财产资格的限制和对妇女的性别歧视,最终在 1928 年让包括妇女在内的所有成年公民获得了普选权。

上面的讨论主要着眼于宪政和民主的演进,英国政治发展的其他重要方面并未考虑在内,包括民族国家的兴起、现代官僚制的发展和文官制度的建设等等。从英国宪政与民主的演进脉络来看,英国之所以能够建立稳定、有效的民主制度,大体上有两个重要经验:一是长期存在势均力敌的政治力量——主要是贵族和国王,像 1215 年《大宪章》签订以后,贵族们是靠着武力的均衡才能迫使新的国王们不断地确认《大宪章》,政治势力的均衡是英国立宪政治和贵族政治兴起的关键;二是立宪政治和公民权利的发展优先于民主的发展,权力制衡和政治竞争的发展优先于政治参与的发展,这一政治发展的次序首先保证对政府权力实施限制,然后通过权力制衡和竞争发展出了一整套有利于现代民主制运作和实现精英控制的制度安排,包括议会、责任内阁和政党等,最后才落实普选权以保证大众的政治平等和参与。

尽管英国在立宪政治和权力制衡方面走在其他国家的前面,但它并不是世界上第一个落实普选权的国家。如果以某种程度的普选权为标准,塞缪尔 · 亨廷顿认为 1828 年的美国是世界上第一个民主国家。在此之后,他认为 “近代世界史中出现了三波民主化” 。亨廷顿把一波民主化定义为 “一组国家由非民主向民主政权的过渡,这种转型通常发生在一段特定的时期内,而且在同一时期内,朝民主化转型的国家在数量上超过向相反方向回归的国家” 。他把 1828 年美国总统选举中有选举资格的男性超过白人男性的 50\% 视为第一波民主化的开始。亨廷顿认为,第一次民主化长波是 1828—1926 年,第一次回潮是 1922—1942 年;第二次民主化短波是 1943—1962 年,第二次回潮是 1958—1975 年;第三次民主化始于 1974 年,而到他 1991 年出版《第三波》时世界还正在经历第三波民主化浪潮。\nauthor{塞缪尔 · 亨廷顿:《第三波——20 世纪后期民主化浪潮》,刘军宁译,上海:上海三联书店 1998 年版,第 11—26 页。}

尽管第三波民主化浪潮波涛汹涌,但其中也存在很大的困难,主要问题是不少转型国家的民主制度并不都是运转良好和稳定有效的。拉里 · 戴蒙德认为,拉丁美洲国家的民主在制度上是 “根基浅薄和脆弱的” , “大多数拉美民主国家都达不到自由主义民主的要求。相反,它们是选举民主国家。” \nauthor{Larry Diamond, “Consolidating Democracy in the Americas,” \italic{Annals of the American Academy of Political and Social Science}, Vol.550, NAFTA Revisited: Expectations and Realities,(Mar., 1997), pp.12-41.}还有学者认为: “在很多拉丁美洲国家,民主的质量很糟糕,公民权利保护不足,政府的责任机制也很脆弱。” \nauthor{Scott Mainwaring and Timothy R.Scully, “Latin America: Eight Lessons for Governance,” \italic{Journal of Democracy}, Volume 19, Number 3 July 2008, pp.113-127.}在谈到非洲的第三波民主化时,理查德 · 约瑟夫说,1989 年以后撒哈拉以南非洲的 47 个国家中超过半数经历了政治改革,但是, “大多数非洲国家看起来处于 ‘某种中间状态’ 。少数国家将会继续自由化和民主化;一些国家将会回到压制性的独裁统治。然而,在大多数国家,自由主义民主作为虚拟民主(virtual democracy)这一悖论将反映政治生活的状况。” \nauthor{Richard Joseph, “Democratization in Africa after 1989: Comparative and Theoretical Perspectives,” \italic{Comparative Politics}, Vol.29, No.3, Transitions to Democracy: A Special Issue in Memory of Dankwart A.Rustow(Apr., 1997), pp.363-382.}在评论原苏联和东欧地区的转型时,迈克尔 · 麦克福尔在 2002 年认为,在原苏联和东欧地区的 28 个转型国家中,仅有捷克、波兰等 8 个进入了自由主义民主国家的行列,其余的国家或者为独裁统治的阴影所笼罩,或者是某种不稳固的转型体制。\nauthor{Michael Mc Faul, “The Fourth Wave of Democracy and Dictatorship: Noncooperative Transitions in the Postcommunist World,” \italic{World Politics}, Vol.54, No.2(Jan., 2002), pp.212-244.}

卡内基国际和平基金会副总裁托马斯 · 卡罗瑟斯悲观地认为,第三波民主化浪潮中的大多数国家并没有实现成功的民主转型和巩固。

\quo{在近些年被认为是 “转型国家” 的接近 100 个国家中,只有相对很少的国家——大概不足 20 个国家——正在朝着通往成功的、运转良好的民主制度的道路上迈进,或者在民主方面已经取得了某种进步和依然拥有民主化的积极力量。……迄今为止第三波的大多数国家并没有实现运转良好的民主制度,或者不能深化他们已经在民主方面取得的进步。\nauthor{Thomas Carothers, “The End of Transition Paradigm,” \italic{Journal of Democracy}, 13(1), 2002: 5-21.}}

民主并没有在大部分第三波国家得到充分的巩固,甚至在一些国家出现了逆转。最近有不少学者都认为,一些第三波国家的民主制度已经崩溃,而且还有相当数量的国家民主政体极其脆弱,时刻都面临崩溃的危险。拉里 · 戴蒙德的统计表明,1974—2006 年间第三波民主化国家总共发生 20 次民主政体的崩溃,占到所有第三波民主政体数量的 14.2\% 。\nauthor{Larry Diamond, \italic{The Spirit of Democracy: the Struggle to Build Free Societies throughout the World}, New York: Times Books, 2008, pp.56-87.}

因此,在福山宣告历史的终结之时,民主还没有赢得最后的胜利。对第三波国家来说,民主化不仅意味着民主转型,还意味着民主巩固。只有建立巩固的民主制度,才是实现了成功的民主转型。

\tsection{现代化导致民主化?}

社会科学研究需要回答为什么的问题,解释民主转型也不例外。目前的民主转型研究主要关注两个问题:一是为什么有些国家启动了民主转型而另外一些国家没有?二是启动转型的国家为什么有的实现了民主巩固而另外一些没有?当然,有人还关心第三个问题:为什么有些民主国家出现了民主政体的崩溃而另外一些国家没有?这类问题是过去几十年政治发展和比较政治研究的核心议题之一,形成了一个庞大的智力产业。

在总结以往研究的基础上,亨廷顿认为大约有 27 个变量可以解释民主转型。在具体分析第三波民主化时,他认为主要有 5 个原因:(1)威权政体合法性的削弱;(2)长期经济增长以及对生活质量的影响,教育水平和中产阶级成长的推动;(3)天主教政治立场的转变;(4)美国、苏联外部政策的变化;以及(5)滚雪球或示范效应。\nauthor{塞缪尔 · 亨廷顿:《第三波——20 世纪后期民主化浪潮》,刘军宁译,上海:上海三联书店 1998 年版,第 54 页。}但是,他也承认 “民主化的原因因地因时而迥异” 。尽管如此,很多学者还是试图提出明确的因果理论来解释民主转型。下面将依次介绍几种主要的理论。

第一种理论关注的是一个国家的经济发展水平和富裕程度。简单地观察这个世界,会得到一个令人深刻的印象:绝大多数富裕国家都是民主国家,而绝大多数威权国家都是贫穷国家。如果把主要包含人均国民收入、预期寿命和知识水平三个要素的人类发展指数(Human Development Index,HDI)作为衡量经济发展水平的标准,那么可以看到:

\quo{在 25 个具有最高人类发展指数的独立国家中,仅有新加坡是非民主国家。在 40 个最发达的国家中,除了新加坡,仅有三个小型石油国家——科威特、巴林和文莱——是非民主国家。除此之外,在 50 个最发达国家中,还有卡塔尔和阿拉伯联合酋长国是非民主国家,但它们都是盛产石油的小国。\nauthor{Larry Diamond, \italic{The Spirit of Democracy: the Struggle to Build Free Societies throughout the World}, New York: Times Books, 2008, p.96.详细数据的来源请参阅联合国开发署发布的各国人类发展指标和自由之家发表的各国政治权利和公民权利评价系统。戴蒙德在此参考的是 2006 年的数据,他排除了古巴。}}

美国政治学者亚当 · 普沃斯基也指出,经济发展与民主政体出现的可能性之间存在强烈而稳定的关系。他对 1950—1990 年间每一年一国人均收入的高低和民主政体出现的可能性之间进行数量分析,发现人均收入能够有效预测一国是否是民主政体,其效度达到 77.5\% 。\nauthor{Adam Przeworski, Michael E.Alvarez, Jose Antonio Cheibub and Fernando Limongi, \italic{Democracy and Development: Political Institutions and Well-Being in the World}, 1950-1990, Cambridge: Cambridge University Press, 2000., 79-80.}

关于经济发展与民主关系的理论研究,最早始于美国政治学者西蒙 · 马丁 · 李普塞特 1959 年的论文《民主的一些社会条件:经济发展与政治合法性》。他认为,民主需要一定的社会经济条件,民主与经济发展水平有关。他的观点可以简单总结为: “一个国家越富有,它越有可能维持民主制度。” \nauthor{Seymour Martin Lipset, “Some Social Requisites of Democracy: Economic Development and Political Legitimacy,” \italic{American Political Science Review}, Vol.53, No.1, Mar 1959, pp.69-105.}这一观点被认为是现代化理论的核心。此后,又有很多学者从逻辑或经验的角度验证了这一理论。美国哈佛大学教授罗伯特 · 巴罗在 1999 年的一项研究中认为: “如果其他国家变得跟经济发达国家一样富裕,它们就很有可能会成为政治上民主的国家。” \nauthor{Robert J. Barro, “Determinants of Democracy,” \italic{The Journal of Political Economy}, 1999, Vol.107, No.6, pp.158-183.}

尽管如此,经济发展导致民主的理论还是遇到了重大挑战。亨廷顿在 1968 年的著作《变化社会中的政治秩序》中认为,在很多发展中国家,经济发展不仅不能导致政治民主,反而会导致政治不稳定。该书的发表在一定程度上颠覆了关于政治发展的现代化理论。政治学者吉列尔莫 · 奥唐奈在 1973 年的著作《现代化与官僚威权主义:南美政治研究》中认为,向官僚威权主义(bureaucratic-authoritarianism)转型的国家往往不是经济发展程度低的国家,而是经济发展程度高的国家。

普沃斯基及其合作者在《民主与发展:1950—1990 年全球的政治制度与福利》一书中进一步检讨了现代化理论。他们通过模型和数量研究得出结论:

\quo{富裕国家更有可能成为民主国家,不是因为民主的出现是威权统治下经济发展的一个结果,而是因为民主——无论民主是怎样出现的——在一个富有的社会更有可能存活下去。\nauthor{Adam Przeworski, Michael E.Alvarez, Jose Antonio Cheibub and Fernando Limongi, \italic{Democracy and Development: Political Institutions and Well-Being in the World}, 1950-1990, Cambridge: Cambridge University Press, 2000: 137.}}

这就是说,经济发展本身并不会导致民主。绝大多数富裕国家是民主国家,仅仅是因为民主在富裕国家更容易存活。这一理论既很好地解释了现实,又对民主的现代化理论构成了挑战。

林茨、奥唐奈和施密特等人则从其他的理论视角提出了对现代化理论的质疑。在他们看来,对于民主前提条件的研究是误导性的,因为一国的条件只是设定了政体转型的情境,而政治民主能否出现或存活下去更多地取决于政治精英的战略互动和行为选择。这一观点是对丹克沃特 · 拉斯托 1970 年研究的延续,下文还会有详细讨论。

从经验证据上看,民主的现代化理论无法解释很多特例:世界上既存在长期贫穷却维持了较为稳定民主制度的国家——印度,又存在很富裕的威权国家——新加坡。在 1950 年到 1990 年这一时期,普沃斯基等人在《民主与发展》一书中列举了 25 个人均国民收入超过 4000 美元的威权政体,但按照现代化理论,这些国家和地区更有可能是民主政体。这就难以以特殊案例作为借口。戴蒙德观察到,1990 年以后,贫穷国家当中民主政体的比例有了显著的增长。尽管多数落后国家的民主政体是脆弱的和不稳固的,但数量和比例本身也能说明问题。\nauthor{Larry Diamond, \italic{The Spirit of Democracy: The Struggle to Build Free Societies Throughout the World}, New York: Times Books, 2008, p.27.}

这些理论和事实上的挑战促使很多学者开始反思经济发展与民主的因果关系。达尔在 1971 年的《多头政体》中认为,经济发展与民主可能是一种非线性的关系。他认为,存在一个有利于民主转型的经济发展水平的理想区间。\nauthor{罗伯特 · 达尔:《多头政体》,第 79 页。}亨廷顿也持有类似的观点:

\quo{在穷国,民主化是不可能;在富国,民主化已经发生过了。在两者之间有一个政治过渡带;那些处于特定经济发展水平的国家,最可能向民主过渡,而且多数向民主过渡的国家也将在这一经济发展水平上。\nauthor{塞缪尔 · 亨廷顿:《第三波——20 世纪后期民主化浪潮》,第 70—74 页。}}

不同学者对现代化理论的挑战,使得李普塞特本人也开始从原有的学术立场上退却了。他后来承认,对于民主转型来说,社会经济条件以外的其他变量,包括政治文化、宗教传统、制度设计、公民社会、法治、政治精英的行为等等,也具有同样重要的作用。\nauthor{Seymour Martin Lipset, “The Social Requisites of Democracy Revisited: 1993 Presidential Address,” \italic{American Sociological Review}, Vol.59, No.1, Feb.1994, pp.1-24.}

\tsection{驱动民主转型的阶级力量}

第二种理论主要关注的是一个国家的阶级与社会结构。关于阶级和民主的关系,亚里士多德是最早的观察者,他发现 “平民群众与财富阶级之间时时发生党争” 。可见在古希腊,富人和穷人的斗争已经是最基本的政治冲突之一。英国政治发展的历史也表明,英国宪政与民主制度的起源,得益于长期存在一个与国王在政治上势均力敌的贵族阶级——贵族就是占有土地的封建领主。

用阶级结构和阶级斗争的视角来分析政治,一般被认为是马克思主义的方法。马克思根据所有权关系和在生产方式中所处位置的不同而把人划分为不同的阶级,并认为 “人类的历史就是阶级斗争的历史” 。吉登斯则认为: “在马克思的观念里……阶级关系是政治权力分配的轴心,是政治组织所依赖的枢纽。” 

在研究现代民主起源时,美国政治学者巴林顿 · 摩尔用阶级分析方法进行比较历史研究。他在 1966 年的名著《专制与民主的社会起源》中认为,土地贵族、农民和资产阶级在政治舞台上扮演的不同角色和力量,决定了政治发展的不同道路。在由传统社会通往现代社会的三条道路中,只有资产阶级强大的国家才有可能建成民主国家。按照摩尔的看法,民主的发展需要五个条件:一是 “建立某种均势,避免王权或土地贵族畸轻畸重的局面出现” ;二是 “向形式适宜的农业商品经济过渡” ;三是 “削弱土地贵族” ;四是 “防止建立针对工农的地主资产阶级联盟” ;五是 “以革命手段粉碎过去” 。\nauthor{巴林顿 · 摩尔:《民主与专制的社会起源》,拓夫等译,北京:华夏出版社 1987 年版,第 334—350 页。}

三百多年以来的人类政治革命也在某种程度上印证了摩尔的基本观点。这场政治革命的启动以 17 世纪的英国革命为标志,经由 18 世纪美国革命和法国革命的推动,在 19 世纪影响到整个欧洲,到 20 世纪则波及了整个世界。这场革命结束了人类天然地被分为统治者和被统治者的观念,把关于自由、宪政和民主的观念带入了政治生活。历史地看,这一政治革命与资本主义的兴起和工业革命的发展同步,因此这一政治革命又被称为 “资产阶级民主革命” 。所以,摩尔这一历史宏观理论引起了巨大的反响,甚至被称为 “一部伟大的书” 。

但与此同时,摩尔的研究也遭到很多学者的质疑。很多学者从方法论、核心论点以及国别案例三个方面对摩尔的研究提出挑战。其中一种典型批评认为,摩尔过于夸大资产阶级对民主的正面作用。艾芙琳 · 胡贝尔和约翰 · 斯蒂芬斯认为,尽管资本主义发展与民主发展存在密切的关系,但在研究了西欧、拉丁美洲和加勒比海地区的政治变迁以后,他们认为: “资产阶级不是像传统上认为的那样是完全的、正宗的民主制度的推进者。当它的利益受到来自于工人阶级和其他群众运动需要的有力挑战时,它会选择威权主义的方式。” \nauthor{Evelyne Huber and John D.Stephens, “The Bourgeoisie and Democracy: Historical and Contemporary Perspectives,” \italic{Social Research}, Vol.66, No.3,Fall 1999, pp.759-788.}

过去有一个观点认为资产阶级民主一定是虚伪的。理由在于,如果资产阶级民主是真实的,没有财产的大部分人会团结起来,通过政治手段剥夺有产者的财产。这就是说,当人数更多的无产阶级控制国家力量以后,会动用国家机器的力量把资产阶级的财产剥夺了。如果这种财产剥夺没有发生,那么资产阶级民主就一定是虚伪的。这个论证逻辑不能说没有瑕疵,但多少有些道理。按照这一逻辑,资产阶级最偏好的是宪政和有限政府,以及宪政和有限政府治理下的自由市场经济,但资产阶级本身对民主并没有特别的偏好。从阶级属性来说,资产阶级拥有那么多财产,有理由恐惧普通人获得普选权后可能会剥夺有产者的财产。当然,到了后来,直接剥夺财产的事情发生得越来越少,但给富人征收高额的所得税、财产税和遗产税成为一种替代方式。所以,从资产阶级的本意出发,如果有宪政,有有限政府,有自由市场,他们或许宁可不要大众主导的民主政治。其中的微妙,大家应该能够理解。

但是,这里还有另外一个逻辑。非民主国家实际上很少会恪守宪政、有限政府和自由市场的治理原则;相反,它们经常破坏上述原则。所以,资产阶级尽管未必赞同大众民主,但他们支持威权政体的风险也相当大。所以,资产阶级其实存在两种相互矛盾的恐惧:一方面恐惧威权统治者的胡作非为,另一方面恐惧民主政治下工人阶层的再分配政策。按照这一逻辑,资产阶级究竟倾向于支持还是反对民主,取决于他们对这两种恐惧的判断与权衡。

还有学者认为,工人阶级或有组织的工人才是民主转型过程的关键因素。约翰 · 斯蒂芬斯认为,摩尔的研究 “大大低估了民主转型过程中有组织的工人阶级的作用” 。\nauthor{John D.Stephens, “Democratic Transitions and Breakdown in Western Europe, 1870-1939: A Test of the Moore Thesis,” \italic{The American Journal of Sociology}, Vol.94, No.5,(Mar., 1989), pp.1019-1077.}有学者认为: “在完全的民主政体获得发展的任何地方,有组织的工人阶级都扮演着关键角色。” 他们甚至断言,如果不是有组织的工人阶级在争取普选权和其他公民权利过程中的积极作用, “实际上资本主义国家几乎必然是威权主义的” 。\nauthor{Jean Grugel, \italic{Democratization: A Critical Introduction}, Basingstoke: Palgrave, 2002, p.54.}鲁思 · 贝琳斯 · 科利尔则干脆认为,工人阶级是民主和民主化过程中的核心力量。\nauthor{Ruth Berins Collier, \italic{Paths towards Democracy}, Cambridge: Cambridge University Press, 1999.}

在早期欧美资本主义国家建立之初,没有哪个国家的公民是拥有普选权的。当时国家的主要特征是自由主义、立宪主义和公民受限制的选举权。这样的国家显然还不是充分的民主国家。而在早期争取普选权的过程中,欧洲的工人阶级和劳工运动发挥了核心作用。发生在 19 世纪中叶的英国宪章运动恰好证明了工人阶级对民主的推动。1836 年 6 月, “伦敦工人协会” 成立,宗旨就是 “以各种合法手段使社会上一切阶层获得平等的政治权利和社会权利” 。1838 年 5 月,在伦敦工人协会支持下,12 人委员会提出一份名为《人民宪章》的政治文件。其核心内容包括六项要求:年满 21 岁的男子享有普选权,秘密投票,废除议员财产资格限制,议员支薪,选区平均分配和议会每年改选。宪章运动的核心是工人阶级,形式包括集体请愿、集体签名、大型集会、暴力抵抗等。尽管宪章运动遭到镇压并最后归于失败,但随后英国政治的发展——特别是 1867 年的选举改革、1884—1885 年的选举改革以及 1911 年实施的议员薪酬制度,逐步实现了宪章运动提出的大部分政治要求。 “宪章运动表明工人阶级已成为英国民主的动力。” 

在第三波民主化中,工人阶级和劳工运动在不同地区的作用差别较大。总的来说,在南欧和拉美,劳工运动是重要的支持民主的组织化力量。在 20 世纪 60—70 年代的西班牙,工人阶级是佛朗哥独裁统治的最重要的反对派。在智利,铜业工人联合会是皮诺切特政体最早的大众反对力量。在 20 世纪 70 年代晚期和 80 年代早期的阿根廷,强有力的全国劳工联盟,与人权组织和其他社会集团一起,推动了威权政体的崩溃。这些国家的例子都说明工人阶级对第三波民主化的积极作用。

除了重视资产阶级或工人阶级的作用之外,还有一个既古老又时髦的观点,即重视中产阶级的力量。在亚里士多德看来,中产阶级势力足够强大的地方才能 “建立一个持久的共和政体” 。亚里士多德认为:

\quo{在一切城邦中,所有公民可以分为三个部分(阶级)——极富、极贫和两者之间的中产阶级。……惟有以中产阶级为基础才能组成最好的政体。中产阶级(小康之家)比任何其他阶级都较为稳定。他们既不像穷人那样希图他人的财物,他们的资产也不像富人那么多得足以引起穷人的觊觎。\nauthor{亚里士多德:《政治学》,吴寿彭译,北京:商务印书馆 2007 年版,第 207—215 页。}}

在亨廷顿看来,中产阶级是由 “商人、专业人士、店主、教师、公务员、经理、技术人员、文秘人员和售货员” 组成的。他认为,除了个别国家, “第三波民主化运动不是由地主、农民或产业工人来领导的。几乎每一个国家民主化最积极的支持者是来自城市中产阶级。” 在 20 世纪 70 年代中期的巴西,有 “一部分人在要求回归民主统治上喊得最响:他们就是大而发达城市中的居民和中产阶级” 。在 20 世纪 80 年代的韩国,一个庞大的城市中产阶级加入学生的抗争队伍之后,威权政体才真正受到了威胁。 “动员首尔的管理阶层和职业阶层也许是 1987 年向民主过渡的最重要的因素。” 亨廷顿还断言, “在城市中产阶级规模相对较小或相对薄弱的地方……要么民主化不成功,要么民主政治不稳定。” \nauthor{塞缪尔 · 亨廷顿:《第三波——20 世纪后期民主化浪潮》,第 76—78 页。}

\tsection{政治文化重要吗?}

第三种理论主要关注一个国家的政治文化和宗教因素。这方面的基本问题包括:是否某种特定的文化或宗教更有利于民主和民主转型?如果是,那么文化或宗教是通过何种机制影响民主转型的?如果不是,那么又如何解释世界范围内政体与不同类型的文化或宗教之间的相关性?尽管对这些问题的看法存在分歧,但有很多学者认为,政治文化或宗教是解释民主转型和巩固的一个有力因素。

较早论述政治文化影响民主政体的重要政治思想家是托克维尔。在《论美国的民主》中,他认为: “美国之能维护民主制度,应归功于地理环境、法制和民情。” 他把地理环境、法制与民情认定为美国能实现民主巩固的三大因素,但他又认为: “自然环境不如法制,而法制又不如民情。”  “我确信,最佳的地理位置和最好的法制,没有民情的支持也不能维护一个政体;但民情却能减缓最不利的地理环境和最坏的法制的影响。” 托克维尔所说的民情,是指 “一个民族的整个道德和精神面貌” ,非常接近政治文化的概念。在谈到美国地方自治时,他还特别强调: “在美国,乡镇不仅有自己的制度,而且有支持和鼓励这种制度的乡镇精神。” 乡镇精神可以被理解为 19 世纪美国新英格兰地区民众关心公共事务的政治文化,而这种政治文化对民主起了重要的支撑作用。\nauthor{托克维尔:《论美国的民主》,董果良译,北京:商务印书馆 2008 年版,第 354—358、74 页。}

到了 20 世纪 60 年代,首先对政治文化进行大型实证研究的阿尔蒙德及其合作者认为,政治文化与民主政治、民主稳定有着密切的关系。阿尔蒙德认为:

\quo{一个稳定的和有效率的民主政府,不光是依靠政府结构和政治结构;它依靠人民所具有的对政治过程的取向——即政治文化。除非政治文化能够支持民主系统,否则,这种系统获得成功的机会将是渺茫的。\nauthor{加布里埃尔 · 阿尔蒙德和西德尼 · 维巴:《公民文化——五个国家的政治态度和民主制》,徐湘林等译,北京:东方出版社 2008 年版,第 443 页。}}

那么,哪种政治文化有利于民主呢?答案是公民文化——所谓 “公民文化” 是一种参与者取向、臣民取向和村民取向混合的文化。他们在 1963 年发表的这项研究认为,美国和英国是典型的公民文化,有利于民主政治和政体稳定,而意大利、墨西哥和联邦德国都不是典型的公民文化,这种政治文化对维持稳定的民主政体会产生压力。

其他很多重要的政治学者也认同政治文化与民主的关系。达尔认为, “信念指导行动” , “在一个特定的国家里,对多头政治体制合法性的信念越强,则实行多头政体的可能性越大” 。在 20 世纪 60—80 年代理性选择范式大行其道时,罗纳德 · 英格尔哈特挺身而出对理性选择范式提出尖锐批评,重申政治文化的重要价值。他在《政治文化的复兴》一文中用四个指标衡量 “公民文化” ——个人生活满意度、政治满意度、人际信任和对现存社会秩序的支持度,并对数十个国家进行了比较研究后得出结论:经济因素固然在政治上是重要的,但不同的政治文化也有着重要的政治后果,特别是与民主制度生存的可能性密切相关,而且这种影响更为持久。\nauthor{Ronald Inglehart, “The Renaissance of Political Culture,” \italic{American Political Science Review}, Vol.82, No.4,(Dec., 1988), pp.1203-1230.}

20 世纪 90 年代以来,政治文化研究中最热门的概念恐怕是 “社会资本” 。罗伯特 · 帕特南在研究 20 世纪 70 年代至 90 年代初意大利地方政府的民主试验时发现,不同的地方政府绩效差别很大。 “为什么有些民主政府获得了成功而有些失败了呢?” 他认为,原因主要不在于经济现代性的差异,而是因为意大利的北部比南部具有更好的公民共同体的传统。他把这种公民传统视为社会资本—— “这里所说的社会资本是指社会组织的特征,诸如信任、规范以及网络,它们能够通过促进合作行为来提高社会的效率。” \nauthor{罗伯特 · 帕特南:《使民主运转起来:现代意大利的公民传统》,王列、赖海榕译,江西人民出版社 2001 年版,第 1、195 页。}

民主转型能否成功以及民主政体能否巩固,取决于政治参与者的政治行为,而政治行为的背后则是政治文化。按照海伍德的说法, “政治多发生于我们的头脑之中。” 一个社会的政治精英和普通公众对权力、政府、民主、政党、政治参与、合法性、政治竞争和暴力等重要问题的观念,很大程度上决定了一个国家拥有什么样的政治。戴蒙德、林茨和李普塞特认为,对于一个稳定而有效的民主政体来说,有些特定的价值观和信念是特别重要的——

\quo{对民主合法性的信仰;对对立党派、对立信仰和对立立场的宽容;跟政治对手妥协的意愿,以及妥协意愿背后的实用主义和灵活性;对政治环境的信任,以及互相合作,尤其是在政治竞争者之间的合作;政治立场和党派立场的温和倾向;政治沟通的礼节;基于政治平等的政治效能和政治参与,而这种政治平等又混合了臣民和村民的角色。\nauthor{Larry Diamond, Juan Linz and Seymour Martin Lipset, “Introduction: What Makes for Democracy?” in Larry Diamond, Juan Linz and Seymour Martin Lipset, eds., \italic{Politics in Developing Countries: Comparing Experiences with Democracy}, Boulder: Lynne Rienner Publishers, 1995, p.19.}}

他们认为上述政治文化与民主稳定的关系十分密切。印度战后民主的稳定与其政治领导人——特别是甘地——对自由、宽容、非暴力、政治吸纳和包容价值观的倡导是有密切关系的。而 20 世纪中叶之后土耳其、尼日利亚等国民主政体的不稳定是与这些国家国内两种政治文化的冲突有很大关系——一种是新引入的自由民主的政治文化,一种是传统的强调威权和服从的政治文化。

尽管政治文化和社会资本研究现在是一个热门学术产业,但还是有人给这种热情泼冷水。第一种疑问是政治文化与政治制度何为因何为果?比如,爱德华 · 穆勒和米切尔 · 塞列格逊在 1994 年的跨国研究中指出,大多数公民文化态度不会对民主产生任何重要的影响,而人际信任——即公民文化中的一项态度,很明显是民主的结果而非原因。\nauthor{Edward N. Muller and Mitchell A. Seligson, “Civic Culture and Democracy: the Question of Causal Relationships,” \italic{American Political Science Review}, Vol.88, No.3, Sep., 1994, pp.635-652.}此前,有人注意到民主国家人们之间的信任程度更高。有人说,信任导致民主。但是,有人反过来说,当一个国家长期实行民主制度后,人与人的信任程度就提高了。那么,何为因何为果呢?这的确是一个问题。

第二个疑问是政治文化是否真的有很强的政治效应?两位政治学者杰克曼和米勒从 1996 年开始连续发表论文对政治文化的基本命题进行攻击。他们在《政治文化的复兴?》一文中对罗纳德 · 英格尔哈特和罗伯特 · 帕特南的研究提出了质疑,认为前者只考察了工业化国家,而后者在分析手段上存在偏差。首先,文化是什么往往难以界定;其次,政治文化研究常常是事后解释;最后,政治文化研究常常借助少数案例得出普适性的理论。他们的结论正好相反: “几乎没有证据能揭示在政治文化和政治经济绩效之间存在系统性的关系。” \nauthor{Robert W. Jackman and Ross A. Miller, “A Renaissance of Political Culture?” \italic{American Journal of Political Science}, Vol.40, No.3(Aug., 1996), pp.632-659.}

第三个疑问是政治文化是否是一成不变的?尽管几乎没有政治文化学者主张政治文化是一成不变的,但他们一般倾向于认为,与经济社会指标、与政治制度相比,政治文化是更稳定、更不易改变和更难以克服的。而以第三波民主化国家经验来看,拉丁美洲就克服了早先被认为支持威权主义的文化传统,其政治精英和普通大众在过去几十年中的政治心理和价值倾向也已经发生了重大的变化。

政治文化的另一个重要方面是宗教。在有些学者的眼中,宗教可能是最重要的政治文化。实际上,汤因比和亨廷顿在划分不同的文明时,宗教是一个最重要的标准。韦伯认为欧美资本主义的兴起有赖于新教伦理的观点更是广为人知。

亨廷顿也认为宗教传统对民主化的成功具有重要的影响。他说:

\quo{在西方的基督教与民主之间存在着高度的关联。近代民主首先且主要出现在基督教国家。到 1988 年,在基督教或新教是主要宗教的 46 个国家中有 39 个是民主国家。\nauthor{塞缪尔 · 亨廷顿:《第三波——20 世纪后期民主化浪潮》,第 83 页。}}

他非常认同肯尼思 · 博伦 1979 年对 99 个国家的一项研究: “新教徒的人口比例越大,民主的程度也就越高。” 通常认为,基督教或新教强调个人尊严和宗教与国家的分离,都有利于民主的兴起。

包括李普塞特在内的不少学者认为,天主教是一种容易 “阻碍” 民主化的宗教。但是,20 世纪 60 年代以后天主教的活动和信条发生了惊人的变化,从保守的力量变为变革的力量,从支持威权的力量变为支持民主的力量。所以,亨廷顿认为,天主教教廷的这种变化实际上使它自身成为第三波民主化浪潮的重要推动因素。 “从总体上看,在 1974 年到 1989 年间过渡到民主的国家中大约有四分之三是天主教国家。” 

亨廷顿认为,其他宗教基本上都是民主转型的障碍因素。他笃定认为,流行于中国、日本、韩国和新加坡等东亚国家的儒教 “要么不民主,要么反民主” ,因为儒教 “强调团体、团队胜于强调个人,强调权威胜于强调自由,强调责任胜于强调权利” , “儒教社会缺少抗衡国家之权利的传统” 。\nauthor{塞缪尔 · 亨廷顿:《第三波——20 世纪后期民主化浪潮》,第 364—371 页。}但福山不同意这一观点,他认为很多人高估了儒教对民主制度的不利影响。尽管缺少个人主义的传统,但儒教具有平等主义精神,特别是很早就用考试制度录用人才;儒教重视教育,通常人口中具有较高的识字率;儒教非常宽容,这一点远胜于基督教和伊斯兰教。与其说儒教是主张国家优先的,还不如说是主张家庭优先的。因此,福山认为,儒教并不构成对民主的障碍。相反,他主张现代化理论长期中是成立的,即经济发展最终会推动政治民主。\nauthor{Francis Fukuyama, “Confucianism and Democracy,” in Larry Diamond, Marc F. Plattner and Philip J. Costopoulos, \italic{World Religions and Democracy}, Baltimore: The Johns Hopkins University Press, 2005, pp.42-55.}包括日本、韩国在内的东亚地区的成功民主转型案例,似乎也批驳了所谓 “儒教不利于民主” 的命题。

尽管 “伊斯兰的教规含有既有利于又不利于民主的成分” ,但 “实际上,除了一个例外(土耳其),没有一个伊斯兰国家长期维持过充分的民主政治体制” 。亨廷顿不太看好民主政治在伊斯兰国家的前景。20 世纪 70 年代以后,相当比例的伊斯兰国家和以美国为首的西方世界之间的隔阂似乎又加深了。而 “9 · 11” 事件更是强化了这种对抗。按照戴蒙德等人在 2003 年的计算,在穆斯林人口占主要比例的 43 个国家,仅有 7 个国家是民主国家,但这 7 个国家没有一个符合自由主义民主的标准。在伊斯兰教具有最强影响力的 16 个阿拉伯国家中,没有一个是民主国家。\nauthor{Larry Diamond, “Universal Democracy?” \italic{Policy Review}, Jun/Jul 2003, 119, ABI/INFORM Global, pp.3-25.}

尽管如此,仍然有学者持有不同的看法。弗瑞德 · 哈利代同意民主在伊斯兰国家阻力重重,但他认为这些国家的民主障碍与他们社会中其他的政治与社会特征有关,而不是来自于伊斯兰教本身。因此,并不能得出伊斯兰教不利于民主的结论。阿卜杜 · 菲拉利-安萨里则认为,伊斯兰社会对西方现代世俗文明、西方式民主的某种敌意,不是来自于伊斯兰教的教义本身,而是来自于 19 世纪伊斯兰世界与西方现代世界的碰撞和冲突。19 世纪后半叶伊斯兰世界的精神领袖阿富汗尼(Jamal-Eddin Al-Afghani)的思想正好是这种碰撞和冲突的反应。阿富汗尼认为,欧洲人的信仰体系和社会秩序与伊斯兰世界的信仰体系和社会秩序是对立的,因此世俗化意味着放弃伊斯兰教,向欧洲人的价值和信仰屈服。而民主作为一种社会制度,已经被贴上了 “西方世界” 的标签。\nauthor{Abdou Filali-Ansary, “Muslims and Democracy,” in Larry Diamond, Marc F. Plattner and Philip J. Costopoulos, \italic{World Religions and Democracy}, Baltimore: The Johns Hopkins University Press, 2005, pp.153-167.}这一观点强调的不是伊斯兰教教义本身,而是国际体系对伊斯兰世界国内政治的塑造。

总体上说,现在重视宗教因素对民主转型影响的观点并不是主流。胡安 · 林茨和阿尔弗莱德 · 斯泰潘在分析东欧国家的转型经验时认为,即使不考虑宗教因素,民主转型照样会发生。他们认为,宗教以外的其他因素就足以解释该地区的民主化。

\tsection{影响转型的国际因素}

第四种理论主要关注的是一国民主转型的国际环境或国际因素。从长时段来看,国际因素可能是推动世界上多数国家向民主转型的最重要因素。今日的国际经济体系、世界政治格局和全球意识形态,是以英国为首的西欧文明在过去 500 年的崛起和扩张所塑造的。从这个意义上讲,世界性的民主革命不过是全球化和全球秩序重建的一部分。在过去,新航路和美洲的发现、工业革命和技术的扩展、跨国贸易和商业活动的兴起、殖民主义和反殖民化、西方世界与其他国家的战争以及两次世界大战,都以不同方式塑造着国际体系,并对大多数国家的国内政治产生巨大的冲击。二战以后全球化加速,几乎所有国家都被深刻地卷入一定的国际政治、经济、军事体系之中,国际因素的影响有增无减。

怀特海德认为,国际因素对民主和民主转型具有巨大影响,他用简要的计算说明了这种影响:

\quo{考虑 1990 年被自由之家评级为 “自由” 的 61 个独立国家,其中 30 个国家——以美国为首——的民主制度可以追溯到摆脱大英帝国殖民统治的过程。另外的 12 个国家现今的民主制度起源于二战中盟军方面的胜利。还有 13 个国家从保守威权主义向民主的转型发生在 1973 年之后(这些国家都是美国的军事同盟国,而美国过去以冷战为理由使它们的非民主统治具有合法性)。这样,在 61 个国家中只剩下 6 个国家其民主制度既不是起源于反殖民化,也不是起源于二战,也不是起源于近期冷战的消退。……到 1995 年 1 月为止,自由之家的 “自由” 国家名单中又增加了 15 个,总数达 76 个。而这 15 个中有 9 个位于中东欧地区。这一组国家反映的是苏联的解体。\nauthor{Laurence Whitehead, “Chapter 1 Three International Dimensions of Democratization,” in Laurence Whitehead,ed., \italic{The International Dimensions of Democratization}, Oxford: Oxford University Press, 2001, pp.3-4.}}

其实早在 1971 年,达尔就注意到了国际因素的重要影响,他说: “一个国家的命运永远不会完全掌握在它自己的人民手里。……每个国家都是在与他国共处的环境中生存的。” 亨廷顿则认为: “外国政府或机构的行动也许会影响、甚至是决定性地影响到一个国家的民主化。” 对于第三波民主化国家来说,其作用机制主要是两种。第一种是重要国家和国际组织对外政策的影响,特别是梵蒂冈、欧洲共同体(欧盟)、美国和前苏联的做法。另一种是邻近国家民主转型的示范效应或 “滚雪球效应” 。 “一个国家成功地实现民主化,这会鼓励其他国家的民主化” 。\nauthor{塞缪尔 · 亨廷顿:《第三波——20 世纪后期民主化浪潮》,第 98、113 页。}

戴蒙德、林茨和李普塞特认为: “国家政治体制和政体变迁受到一系列国际因素的影响,包括殖民统治、外国干预、文化扩散和国外的示范效应。” \nauthor{Larry Diamond, Juan Linz and Seymour Martin Lipset, “Introduction: What Makes for Democracy?” p.48.}巴巴拉 · 魏奈特则对影响民主的国内发展因素和国际扩散因素作了系统的、跨越两个世纪的比较研究。她认为: “当单独评估时,(国内的)发展指标对于民主来说是强有力的预测指标。但是,当(国际的)扩散变量考虑进来以后,发展指标的预测效力就大大下降了。” 这项研究意味着国际扩散因素是民主更有效的预测指标。\nauthor{Barbara Wejnert, “Diffusion, Development, and Democracy, 1800-1999,” \italic{American Sociological Review}, Vol.70, No.1(Feb., 2005), pp.53-81.}

那么,国际因素是通过何种机制起作用的呢?怀特海德归纳总结出三种机制,它们分别是:(1)传染(Contagion),即民主经验不借助强制力的扩散;(2)控制(Control),即一国借助强制力或约束力对另一国民主的推动;(3)同意(Consent),这是国际力量通过与国内集团的一系列复杂互动产生影响的一个过程。\nauthor{Laurence Whitehead, “Chapter 1 Three International Dimensions of Democratization,” in Laurence Whitehead,ed., \italic{The International Dimensions of Democratization}, Oxford: Oxford University Press, 2001, pp.5-16.}施密特在此研究的基础上,认为还有第四种机制:(4)制约(Conditionality),即国家或国际组织通过审慎地使用强制力和讨价还价推动一国的民主。施密特把以上的四种机制做了类型区分,请参阅表 8.1。\nauthor{Philippe C. Schmitter, “The International Context, Political Conditionality, and the Consolidation of Neo-Democracies, ” in Laurence Whitehead,ed., \italic{The International Dimensions of Democratization}, Oxford: Oxford University Press, 2001, pp.28-31.}

\tbl{../Images/image00318.jpeg}[表 8.1 国际因素的影响机制]

20 世纪 90 年代以前,民主转型被认为是国内因素驱动的,而国际因素受到的重视程度不够。但如今,几乎所有政治学者都认为,国际因素对于民主转型和巩固有着重要影响。尽管如此,戴蒙德认为:

\quo{除了那些因为外国军事干预缔造民主的国家以外——比如 1983 年的格林纳达和 1989 年的巴拿马,外部因素不是决定性的。即使在那些国际武力干预的国家,如果没有对民主的国内支持,民主也无法长期存活。以 1994 年的海地为例,迫在眉睫的武装干预帮助摧垮了海地的军人政权,但该国随后又回到了威权主义统治。\nauthor{Larry Diamond, \italic{The Spirit of Democracy: The Struggle to Build Free Societies Throughout the World}, New York: Times Books, 2008, p.106.}}

由此看来,国际因素的影响也是相当不确定的。卡琳 · 冯 · 希佩尔研究了冷战以后美国对巴拿马、索马里、海地和波斯尼亚等国的军事干预后认为,想通过军事干预方式建立民主制度和新的国家的理念是一种 “危险的傲慢” 。希佩尔还引用美国前国家安全事务高级官员安东尼 · 莱克的话说,美国并不能靠美国人的力量重建另外的一个国家,美国能做的最多是帮助一个国家靠他们本身的力量重建自己的国家。\nauthor{Karin von Hippel, \italic{Democracy by Force: US Military Intervention in the Post-Cold War World}, Cambridge: Cambridge University Press, 2004, p.1.}

强调国际因素的另一个挑战是,国际因素比较接近的同一地区国家的民主转型和民主绩效的差别非常之大。拿南美洲和东欧来说,主要大国、重要国际组织对这两个地区不同国家的政策是比较接近的,民主化扩散效应的影响也比较接近,但是这两个地区的国家在民主转型和民主绩效上的差别非常大。由此看来,这些国家的国内因素才是民主转型成功与否的关键。

\tsection{转型政治中的精英行为}

第五种理论关注的是民主转型过程中政治行动者的行为。上述的四种理论重视的是国内外的结构性因素。也就是说,民主转型和巩固需要以一定的经济、社会、文化和国际条件作为前提,而一个国家能否成功地实现民主转型和巩固关键取决于这些 “结构性因素” 。这里的第五种理论强调的是民主转型和巩固的 “过程性因素” 。或者说,一个国家能否实现民主转型和巩固取决于该国政治转型的过程,特别是民主转型过程中政治精英的政治行为、战略选择和政治互动。这一理论视角现在也被称为 “转型研究” 或 “转型学” (transitology)。

如果说前四种理论是结构主义的,那么这里要介绍的是民主转型的能动理论。两者的主要分歧在于对民主转型过程中结构性因素和能动性因素看法的差异。后者认为,结构主义的理论范式有几个明显的弱点。一是带有决定论的色彩,认为经济、社会、文化、历史和国际因素能够决定政治结果,而忽视政治行为者的作用。普沃斯基批评道: “在这种论证中,结果是由条件单方面决定的,人们什么也不做,历史还是会这样发展。” 二是结构主义关注长期的历史变迁,无法很好地解释民主转型或政治变革发生的时机。三是结构主义范式更多是功能性的,而不是发生学的。正是这些不足为转型范式的兴起提供了条件。

转型研究起源于丹尼沃克 · 拉斯托 1970 年的一篇论文《向民主转型:一个动态模型》。\nauthor{Dankwart A. Rustow, “Transitions to Democracy: Towards a Dynamic Model,” \italic{Comparative Politics}, Vol.2, No.3, April 1970, pp.337-363.}作为现代化理论、政治文化理论和社会政治结构理论的批评者,他认为关键不在于解释民主制度如何能得以维持,而在于解释民主是如何产生的,因而发生学的研究更重要。他认为社会经济条件并不能单方面决定政治,政治冲突或结盟的模式是任何政治制度的中心特征,而政治选择是政治过程的中心问题。他认为,民主的产生有一个背景条件和三个阶段,分别是:

\quo{(1)背景条件:民族统一(national unity),即国家本身是一个没有争议的政治共同体。而这是民主能产生的唯一前提条件,没有别的任何前提条件。

(2)准备阶段:这一阶段存在长期的和难以解决的政治斗争,民主常常不是事先的设计,而是作为一种解决政治冲突的程序和制度而兴起的。

(3)决策阶段:这一阶段一系列政治力量进行互动,作出政治选择和达成妥协。政治家、军人和社会精英往往在这一过程中发挥主要作用。

(4)适应阶段:这一阶段政治家和全体选民要习惯和适应民主的政治规则,不仅把民主作为一种竞争公职的制度,也作为解决冲突的制度。}

拉斯托认为,民主的产生是一个动态政治过程的结果。在此之后,比较民主化的两项大型研究使转型范式日益流行。林茨和斯泰潘在 1978 年主持的研究《民主政体的崩溃》中认为,结构性环境条件越是不利,民主的生存就越需要高超的、创造性的、富有勇气的和忠于民主的政治领导力。即使障碍是巨大的,民主的崩溃也不是不可避免的。但如果领导无力、决策频频出错,民主崩溃就会加速。\nauthor{Juan J. Linz and Alfred Stepan, eds., \italic{The Breakdown of Democratic Regimes}, Baltimore: The Johns Hopkins University Press, 1978.}奥唐奈和施密特 1986 年主持的研究《从威权统治转型》也有力地推动了这一方面的研究。他们把政治行为者分为四类:执政集团的保守派和改革派,反对阵营的温和派和激进派。他们通过 “考察威权政体领导人和民主反对派之间的互动、协定和交易来关注民主化过程本身” 。他们强调, “成功的转型取决于政治精英之间的协议” ,因此, “高超的领导力” 是成功民主转型的关键。\nauthor{Guillermo O'Donnell and Philippe Schmitter, eds., \italic{Transitions from Authoritarian Rule: Teneative Conclustion about Uncertain Democrcies}, Baltimore: The Johns Hopkins University Press, 1986.}转型研究的很多学者还关注第三波民主化国家的转型模式。他们一般倾向于从政治精英与大众的关系、政治精英内部当权派和反对派的关系以及转型过程的激进程度等来区分不同的转型模式。

亨廷顿更重视政治精英的力量,他把政治精英区分为当权派和反对派,他最关心以下三项关键的政治互动: “政府与反对派之间的互动,执政联盟中改革派与保守派之间的互动以及反对派阵营中的温和派和极端主义者之间的互动。” 他根据这三种互动关系的不同,区分出三种不同的转型模式:\nauthor{Samuel P. Huntington, “How Countries Democratize,” Political Science Quarterly, Vol.106, No.4,(Winter, 1991-1992), pp.579-616.

(1)改革(transformation):执政联盟中的改革派主导的政治转型;

(2)置换(replacement):威权政体垮台后反对派主导的政治转型;

(3)移转(transplacement):执政联盟被迫与反对派谈判启动的政治转型。}

在前人研究的基础上,杰拉尔多 · 芒克和卡罗尔 · 莱夫在研究南美和东欧国家的转型后认为,政治转型存在路径依赖,正是转型过程决定了民主转型和民主巩固的前景。他们根据 “在位精英—反对派精英” 的主导权,以及两者之间 “对抗—吸纳” 关系,区分了几种不同的转型模式,请参见图 8.1。\nauthor{Gerardo L. Munck and Carol Skalnik Leff, “Modes of Transition and Democratization: South America and Eastern Europe in Comparative Perspective,” Comparative Politics, Vol.29, No.3, 1997, pp.343-362.}

\img{../Images/image00319.jpeg}[图 8.1 转型模式:南美和东欧国家的案例]

实际上,从 20 世纪 90 年代到现在,转型研究是民主化研究领域的主流范式。甚至连李普塞特都认为: “民主是成功还是失败,将继续主要取决于政治领导人和领导集团的选择、行为和决策。” 

尽管如此,这种理论的缺陷是明显的。有学者批评这种理论过于精英主义、过于经验主义、惟意志论和短期化倾向。过于精英主义,往往会忽视大众的政治力量和政治行为;过于经验主义,容易把主要由南美和东欧转型过程研究的结论普遍化;惟意志论往往会过分强调政治精英的主动选择而忽视转型国家客观的经济社会条件;短期化倾向往往过分重视短期政治变迁而忽视一个国家的长期因素。

托马斯 · 卡洛苏则更是认为,转型研究范式的若干核心假定是错误的,包括把转型理解为一定是向民主转型,民主转型有一系列确定的阶段和次序,民主政体主要就是选举,转型国家的经济水平、政治史、制度遗产、族群构成、社会文化传统和其他结构性特征并不重要,转型国家不需要面对国家建设(state building)的问题等等。因此,他认为民主转型范式已经终结。\nauthor{Thomas Carothers, “The End of Transition Paradigm,” \italic{Journal of Democracy}, Vol.13, No.1, 2002, pp.5-21.}

\tsectionnonum{推荐阅读书目}

塞缪尔 · 亨廷顿:《第三波:20 世纪后期民主化浪潮》,刘军宁译,上海:上海三联书店 1998 年版,或塞缪尔 · 亨廷顿:《第三波:20 世纪后期的民主化浪潮》,欧阳景根译,北京:中国人民大学出版社 2013 年版。

胡安 · J.林茨、阿尔弗莱德 · 斯泰潘:《民主转型与巩固的问题:南欧、南美和后共产主义欧洲》,孙龙等译,杭州:浙江人民出版社 2008 年版。

拉里 · 戴蒙德:《民主的精神》,张大军译,北京:群言出版社 2013 年版。

包刚升:《民主崩溃的政治学》,北京:商务印书馆 2014 年版。
