\tchapter{不同的政体:民主、威权及极权}

\quo[——约瑟夫·熊彼特]{民主方法就是那种为做出政治决定而实行的制度安排,在这种安排中,某些人通过争取人民选票取得作决定的权力。}

\quo[——罗伯特·达尔]{民主国家的一个重要特征,就是政府不断地对公民的选择做出响应,公民在政治上被一视同仁。}

\quo[——西蒙·李普塞特]{正是组织使当选者获得了对于选民、被委托者对于委托者、代表对于被代表者的统治地位。组织处处意味着寡头统治。}

\quo[——拉里·戴蒙德]{民主政治不仅是最广泛受到称颂的政治制度,而且也可能是最难以坚守的政治制度。在所有的政府形式中,惟独民主政体依赖于最少的强制和最多的同意。民主政府最终发现它们自己陷于内在的悖论和矛盾的冲突中。……建立一个民主政治和坚守一个民主政治是两件不同的事。……假如民主不能起作用,人们则可能宁愿选择不经他们同意的统治,他们可能选择不再忍受去作出政治抉择的痛苦。}

\tsection{全球视野中的政体类型}

本书的第2讲曾探讨过如何构建政治秩序的问题。在现代政治分析框架中,政治秩序的核心是政体问题。如何界定政体类型,是政治学一个古老问题。关于20世纪以来的全球政体类型,目前比较公认的区分是三种:民主政体、威权政体(authoritarianism)和极权政体(totalitarianism)。

一般认为,比较正式的政体类型学起源于亚里士多德。他区分政体类型有两个标准:第一个标准是最高统治权掌握在谁手中:是一个人手中,少数人手中,还是多数人手中;第二个标准是这种统治服务于部分人的利益还是城邦整体的利益。基于上述两个标准,亚里士多德区分了六种政体类型,见表5.1。

<p class="kindle-cn-picture-txt-withfewcharactors">表5.1 亚里士多德的政体类型学</p>

<table cellspacing="0" class="kindle-cn-table-body">
 <tr>
 <td class="kindle-cn-table-th" width="30\%"><br /></td>

 <td class="kindle-cn-table-th" width="30\%">正宗政体</td>

 <td class="kindle-cn-table-th" width="30\%">变态政体</td>
</tr>

 <tr>
 <td class="kindle-cn-table-dg11">一人统治</td>

 <td class="kindle-cn-table-dg11">君主政体</td>

 <td class="kindle-cn-table-dg11">僭主政体</td>
</tr>

 <tr>
 <td class="kindle-cn-table-dg11">少数人统治</td>

 <td class="kindle-cn-table-dg11">贵族政体</td>

 <td class="kindle-cn-table-dg11">寡头政体</td>
</tr>

 <tr>
 <td class="kindle-cn-table-dg11">多数人统治</td>

 <td class="kindle-cn-table-dg11">共和政体</td>

 <td class="kindle-cn-table-dg11">平民政体</td>
</tr>
</table>

第一种类型较易理解。如果是一人统治且服务于整体利益,就是一种有节制的君主政体。可以理解的是,既然君主的统治行为是不受制约的,君主政体有可能会腐败或变坏,这样就蜕变为僭主政体。第二种也比较好理解。如果是少数人统治,但这些统治精英们总体上比较有节制,不是贪得无厌,治理国家比较有分寸,就是贵族政体。但如果统治精英们变得极其贪婪,丝毫不顾及多数普通民众的利益,就蜕变为寡头政体。较难理解是第三种类型。如果是多数人统治,又服务于部分人的利益而非城邦全体的利益,这是什么意思?亚里士多德把多数人的统治视为穷人的统治,因为在任何社会下层阶级的数量总是要超过上层阶级的数量。多数人的统治只服务于部分人的利益是指,这种统治完全不考虑上层阶级的利益。亚里士多德认为,如果以平民为基础的多数人的统治试图要迫害富人或试图征收富人的财产,就会蜕变为暴民政体。所以,亚里士多德意义上的共和政体意指,统治尽管是基于多数人的意志,但这种统治兼顾了多数人和少数人、穷人阶级和富人阶级的利益;否则,就蜕变为平民政体或暴民政体。

到20世纪,现代政体类型一般被分类为:民主政体、威权主义政体和极权主义政体。三者之中,民主政体的概念是最早出现的。威权主义政体的概念是美国政治学者胡安·林茨较早提出的。林茨认为,威权主义政体既不同于西方世界的自由民主政体,也不同于苏联集团的共产主义政体,而是介于两者之间。后来,这个概念就开始流行。经验研究发现,大量国家在第三波民主化之前都属于威权主义政体的类型,既非民主政体,亦非极权主义政体。在极权主义政体的概念中,极权不同于集权——集权通常与分权相对,而极权主义(totalitarianism)英文单词的词根是“total”,是“整体”或“全部”的意思。“整体”或“全部”在这里的含义是,政治权力试图囊括一切、无所不包。极权主义意味着政治权力要渗透到社会生活的每一角落。所以,极权主义政体又译为全能主义政体。

胡安·林茨认为现代政体类型不止这三种,他在《民主转型与巩固的问题》中划分了五种现代政体类型,分别是民主政体、威权主义政体、全能主义(极权主义)政体、后全能主义(极权主义)政体和苏丹制政体。后全能主义政体是全能主义政体发生改革或演进的产物。后全能主义政体既可能转变为较为标准的威权主义政体,又可能仍然保留着比较多的全能主义特征。林茨这里的苏丹制,沿用的是马克斯·韦伯的概念,粗略地说是指完全基于个人统治的专制主义政体。林茨主要从政治动员、政治领导权、多元化和意识形态四个方面来比较五种政体类型的异同,请参见表5.2。该表清晰地展示了五种现代政体类型在四个主要维度上的差异。

\tbl{../Images/image00305.jpeg}[表5.2 林茨:现代政体的主要理想类型及其判定依据]

\img{../Images/image00306.jpeg}

\img{../Images/image00307.jpeg}

资料来源:胡安·J.林茨、阿尔弗雷德·斯特潘:《民主转型与巩固的问题:南欧、南美和后共产主义欧洲》,孙龙等译,杭州:浙江人民出版社2008年版,第46—47页,表3.1。

\tsection{什么是民主政体?}

很多教科书说,所谓民主就是人民当家作主。这种说法大概是20世纪中叶之前对民主的主流表述。民主意指人民的统治,就是人民当家作主的意思。这一定义实际上出自民主的希腊文原意,“民主”一词的古希腊文是“democratia”,“demo”是“人民”或“大众”的意思,“cratia”就是“统治”的意思。按照古希腊人的理解,民主也可以指多数人的统治。美国前总统亚伯拉罕·林肯在葛底斯堡演说中称,民主应该是民有、民治、民享的政府。这是人们对于民主的一般理解,上述定义也被称为民主的实质性定义。

但是,有人可能会对此提出疑问。一方面,究竟谁是人民?人民是指全体公民?多数公民?还是政治上正确的公民?另一方面,到底如何统治?实际上,在稍有规模的现代国家,人民几乎是无法直接统治的。这一定义引发的另一个问题是:如何区分民主政体与非民主政体?乔万伊·萨托利在《民主新论》中说,1945年以后,人类社会中再也没有人公开宣称自己是民主的敌人,绝大多数国家都自称民主国家,绝大多数政党都自称民主政党。这样,从实质性定义出发,区分民主政体和非民主政体就变得很困难。

1942年,美国经济学家约瑟夫·熊彼特在《资本主义、社会主义与民主》一书中提出了民主的程序性定义,逐渐成为政治学界的常用定义。他说:

\quo{民主方法就是那种为做出政治决定而实行的制度安排,在这种安排中,某些人通过争取人民选票取得作决定的权力。……

民主政治并不意味着人民真正在统治——就“人民”和“统治”两词的任何明显意义而言——民主政治的意思只能是:人民有接受或拒绝将要来统治他们的人的机会。\nauthor{约瑟夫·熊彼特:《资本主义、社会主义与民主》,吴良健译,北京:商务印书馆2007年版,第395—396、415页。}}

如果一个人要成为重要的政务官——无论他想成为总统、总理或首相、议员、州长或市长,他通常都要争取别人的选票。民主国家大部分最重要的政治职位都是通过争取选票的方式来获得的。按照熊彼特的这一定义,民主等同于公民广泛参与的竞争性选举制度。这种制度安排有两个要素:一是政治参与,二是政治竞争。

所以,目前学界形成了关于民主的两种定义:一种是实质性定义,另一种是程序性定义。当然,熊彼特提出的程序性定义也遭到部分学者的批评。比如,有学者认为,从这种视角定义民主容易导致“选举主义的谬误”(fallacy of electoralism),即过分强调选举的作用而忽略了民主的实质。从第三波民主转型国家的经验来看,有些国家转型后的状态是有选举而无民主,或者沦为“两不像政体”(hybrid regime)。\nauthor{美国学者拉里·戴蒙德曾专门撰文讨论这一政体类型,参见Larry Diamond,“Thinking about Hybrid Regimes,”\italic{Journal of Democracy}, Vol.13, No.2, Apr.2002, pp.21-35。}这一现象国外已经有较多研究,国内学界介绍较少。“两不像政体”顾名思义,就是既非标准的威权政体,亦非标准的民主政体,而是介于两者之间。其常见特征是:主要行政长官和议员通常由定期选举产生,普通选民的投票能发挥着实际作用,选举过程中存在不同力量的政治竞争;但是,这些国家的选举过程并没有做到自由和公正,通常存在不同程度的选举舞弊和欺诈,当选的执政者则常常利用行政资源压制反对派和媒体,进行各种政治操纵,甚至为一己之私而推动修宪。正因为这些特征,国际学界通常把“两不像政体”视为威权色彩浓厚的政体类型。

王绍光则把竞争性选举制度意义上的民主称为“选主”,即选票决定一切。他批评道,在“选主”体制下,最终多数人的利益可能并没有得到保证,还是少数人实质性地控制着政治。他还使用了“选出新的主子”这种意识形态味道浓烈的说法,他甚至认为应该“用抽签替代选举”。\nauthor{王绍光:《民主四讲》,北京:生活·读书·新知三联书店2008年版,第242—256页。}王绍光的批评,实际上涉及民主政体条件下的政治平等究竟是程序平等、还是实质平等问题。在20世纪的重要理论家中,美国政治学家罗伯特·达尔晚年也非常强调实质性的政治平等。这个问题后面还会有专门的讨论。但是,从人类已有经验来看,要通过选举来实现实质性的政治平等,即每个公民实现同等的政治影响力,几乎都是办不到的。公民参与的竞争性选择制度是人类社会目前可以实践的民主形式。

其实,罗伯特·达尔早年在《多头政体》中定义的民主或多头政体,更接近于熊彼特的程序性定义。达尔认为:“民主国家的一个重要特征,就是政府不断地对公民的选择做出响应,公民在政治上被一视同仁。”他接着问:“一种制度要成为严格民主制度,还要具备哪些其他特征呢?”他依次列出了多头政体的八个条件:

1. 建立和加入组织的自由。这里的组织当然是指政治组织,也就是说人们可以自由决定建立或加入某些政治团体或政党。这是政治自由的一个基本方面。

2. 表达自由。人们有表达自己的意愿、观点、立场和主张的自由。既然有表达自由,就应该有出版自由和新闻自由。

3. 投票权。所有成年公民都拥有投票权和普选权。但是,大家不要机械地理解普选权。研究民主的历史,大家会发现很多国家的投票权是一个逐渐普及的过程,不是所有公民一开始就享有同等权利的投票权。

4. 取得公共职务的资格。这就是说,所有成年公民大致都有取得公共职务的资格。当然,有些公职对一个人的出生地、年龄可能有限制条件,还有个别国家对一些公职还有学历限制。

5. 政治领导人为争取支持而竞争的权利。这里的竞争是公开的政治竞争。一个人可以在不同场合、以不同方式公开地呼吁选民的政治支持。所以,政治竞争是民主的一个实质性标准。在达尔看来,民主政体之下存在着公开的政治竞争和公开的反对派。

6. 可选择的信息来源。这跟第二条表达自由有关。可选择的信息来源是什么意思呢?不应该只有一个统一的机构来提供信息,所以可选择的信息来源跟新闻和出版自由有关。在互联网时代,自然还应该包括网络信息传播的自由。

7. 自由公正的选举。这是民主政体非常重要的特征,甚至是民主的核心条件。自由公正的选举,意味着它不只是选举,而是说选举的规则和过程是公正的,不同候选人和选民都有公平地参与政治和竞争的权利。

8. 根据选票和其他的民意表达制定政府政策的制度。如果选票真正起作用,那自然应该能够保证政府的政策是根据民意表达来制定的。一个议员如果不能代表他所在地区的利益、不能在政策制定过程中反映民意,他下一次被选下去的概率就非常高。\nauthor{罗伯特·达尔:《多头政体:参与和反对》,谭君久、刘惠荣译,北京:商务印书馆2003年版,第11—15页。}

与熊彼特关于民主的最低标准定义相比,罗伯特·达尔阐述的八个条件内容要丰富得多。但达尔对于多头政体的界定,从实质性条件来看,主要是政治参与和政治竞争这两个标准。

结合上述讨论,可以总结出民主政体的几个基本特征:

第一个特征是政治参与。民主意味着多数成年公民拥有投票权。基于对欧洲历史的考察,最初是财产较多的男子拥有投票权,后来是财产资格标准的降低与取消,再后来又逐步扩展到成年女子。拿英国来说,19世纪之前,仅有少数富有的男性公民拥有投票权,后来经过19世纪的几次选举改革,男性公民参加投票的财产资格逐步降低。而成年男女公民获得同等的普选权一直要到1928年才实现。在智利等不少国家,投票权普及过程中最初还伴随着教育资格和识字要求,后来这些限制条件也逐步取消了。

第二个特征是政治竞争。按照程序性定义,民主本身就包括了政治竞争的含义。不同候选人可以就公共职位展开公开角逐,通过争取选民手中的选票来获得当选的机会。现代政治中,这种政治竞争不仅指单个政治家之间的竞争,而且指不同政治团体——主要是不同政党——之间的竞争。政治竞争和上面提及的政治参与构成了民主政体的两个基本特征。

第三个特征是问责制或责任制。什么是问责制呢?就是指当一个公职人员做事情的时候,是对某个政治共同体或特定地域内的选民负有责任的。简单地说,他要对别人有个交代,他要承担自己政治行为的后果,干得不好时通常就必须走人。这就是问责制的特点。在政治实践中,各个国家的实际做法可能不一样,但有一点是共通的:在民主政体下,选举产生的或由当选政治家任命的重要官员都要对选民和共同体负责。

第四个特征是回应或响应机制。在民主政体下,政府对于公众的利益诉求有一种正式的回应或响应机制。例如,最近出现了重大的公共问题,如果很多人在报纸、电视、网络上呼吁,或以集会、示威游行方式呼吁的话,政治家和政府不会置之不理,他们通常都会做出回应。既然民主意味着政治家要根据大众的利益诉求和政治意愿来进行统治,民主政体就是一种正式的回应或响应机制。政府或政治家如果对民众诉求置之不理,通常会在下一次选举中遭到失败。

第五个特征是起码的政治平等。这里的政治平等是指具有平等的参与政治、政治表达和投票的基本权利,所有公民在这方面应该是平等的。这里的政治平等首先强调的是形式平等和资格平等,而非实质平等和结果平等。当然,对于何谓政治平等存在争议。罗伯特·达尔晚年更强调实质性的政治平等,希望实现不同公民在政治影响力上更为平等。当然,这种实质性的政治平等是否存在,可能存在很大争议。

第六个特征是多数决定的规则。这就是平时说的少数服从多数。做公共决策的时候,大家意见不一致怎么办?通常需要根据人数较多一方的意见来做决定。这里是指一个公共决策应该赢得超过50\%的支持率,这是绝对多数。还有所谓相对多数的概念,即所有方案中赢得最多支持率的那个方案胜出,而无论这一支持率是多少。当然,很多时候会发现这种多数规则未必能够满足。20世纪50年代,美国学者肯尼思·阿罗提出了著名的“阿罗不可能定理”,意思是说很多情况下多数决定规则实际上是不可能的。比如,现在有三个女生去买冰淇淋吃。冰淇淋商店给出的优惠活动是:同款冰激凌买三个就会有很大折扣,不同款冰激凌单独买价格就会比较贵。所以,三个女生决定买一款冰淇淋。但是,三个女生对不同款冰激凌的偏好次序是不一样的,如下:

第一个女生对冰激凌的偏好次序:巧克力 > ;草莓 > ;香草;

第二个女生对冰激凌的偏好次序:草莓 > ;香草 > ;巧克力;

第三个女生对冰激凌的偏好次序:香草 > ;巧克力 > ;草莓。

那么,这三个女生最终会买什么口味的冰淇淋呢?如果按照多数决定规则来投票,结果是什么呢?经过推导,就会发现这种情形下投票结果实际上取决于投票次序。所以,阿罗论证了在一些情况下多数决定规则是不可能的。

第七个特征是对少数权利的保护。与多数决定规则相关的一个问题是:多数决定规则并不意味着多数可以侵犯少数的权利。密尔在《论自由》、托克维尔在《论美国的民主》中都讨论过“多数暴政”问题。民主在尊重多数规则的同时,还要保护少数的权利。换句话说,多数统治并不意味着多数可以就任何事情做出任何决策。多数决定规则是有其明确的边界和范围的。现在一般把欧美发达国家的民主政体称为自由民主政体,其基本特征是对少数权利的保护与尊重。

第八个特征是言论自由与新闻自由。既然民主意味着政治参与和政治竞争,就必然需要政治表达和政治沟通,言论自由与新闻自由就是基本条件。只有在这种条件下,一个社会中才能听到不同的政治理念与政策主张,才存在实质性的政治竞争。在互联网时代,网络信息传播的自由也变得同样重要。

\tsection{民主政体的治理细节}

这里通过两个小案例来分析民主政体是如何运转的。第一个案例是市镇的治理。一个民主的市镇应该是怎样治理的呢?比如,该市镇有一个作为水源地和风景区的湖泊,而湖泊的水质正在恶化。那么,如果是民主治理,民主机制会对此作出何种反应呢?

首先,马上有记者会在市镇报纸上报道此事:湖泊水质正在恶化!饮用水安全受威胁!为什么湖泊水质变坏了?谁负责?然后,小镇居民——特别是积极参与政治的人士——会通过不同方式进行呼吁与抗议,包括约见自己投票支持的市镇议员、在市政厅门前的示威以及周末举行绕湖游行活动,等等。这样做的直接后果是,那些政治上的当选者——市镇议员或是市镇长——会非常着急。他们很清楚,如果不能对此采取有效行动的话,下次当选的可能性就会下降。所以,调查水污染的原因、研究应对水污染的解决方案以及为此调动所需资源,将成为他们下一步的主要工作。

当然,还有另外一种可能。市镇长或议员跟大家报告,目前还很难有办法,原因在于河道是相通的,本地并没有污染源,而是上游的污水流淌到这里,所以湖泊水质才下降了。但是,即便这种情况下,这个市镇长也会积极地跟上游政府机构协调,比如建立一个河道或水系的联席会议,并督促上游治理水污染。如果上游的政府也实行民主治理,这个事情相对来说就更容易协调。长期来看,湖泊的水污染就可能得到有效治理。

在这个市镇治理案例中,如果是民主政体,可以看到公民的基本诉求能通过一种正式的机制反馈到政府,并对其产生直接影响。无论是新闻媒体的自由报道,还是普通选民的抗议与施压,或者选民对公职人员的投票机制及其候选人之间的竞争,均有助于建立一种更为有效的公共治理机制。把这个市镇治理的案例扩展到整个国家,其政治逻辑是一样的。

再来一个公立学校管理的案例。比如,在非民主政体下,某地有些家长对当地一所公立学校的校长非常不满,那么他们能怎么办呢?实际上,对于这样的校长,教师、家长和学生都不满意。教师们私下抱怨校长如何不称职,家长们知道校长治校无方,学生们的满意度也不会很高。结果可能是,该校学生的考试成绩连年下滑,学校恶性事件时有发生,很多教师工作缺乏动力或者开始不务正业。那么,最关心本地学校教育质量的家长们能做什么呢?大家会发现缺少正式的制度化途径。在非民主政体下,公立学校校长通常是由当地教育局长或市镇长任命的,教育局长或市镇长又是由上一级政府机构任命的。这种情况下,普通公民很难对校长任免过程及治校产生有效干预。结果是,一旦本地公立学校出现一位不称职的校长,当地公民往往无力改善这一状况。

如果换一种政体框架思考这个问题呢?比如,有的民主国家基层治理单位是学区。学区怎么治理呢?每个学区设有学区委员会。当地选民每两年就投票选举一次学区委员,然后由学区委员组成学区委员会,学区委员会再任命本学区各公立学校的校长。这样,再由校长聘任本校的副校长及各部门主任——当然,通常情况下大部分教师的工作是相对稳定的。在民主治理下,如果家长们发现本地公立学校校长不能胜任校长职务,接下来会发生什么呢?通常,家长们会先找到某位学区委员反映情况——这位学区委员正是依靠这一选区选票的支持才当选的。学区委员们若同时接到若干家长对同一位校长的投诉,他们会启动调查程序。经过调查,若情况属实,他们马上会提议在学区委员会讨论该校长任职问题。经过这样的程序,学区委员会如觉必要,就会很快免去这位校长的职务,并任命新校长。

通过这个案例的比较,大家就会发现两种不同治理体系的差异。民主政体意味着有效的问责制,意味着从公民到政府的积极响应机制,这种政体具有很强的自我调适能力。通过这两个案例,大家可以看出民主治理机制是如何起作用的,以及为什么民主治理通常要比非民主治理更为有效。此外,大家还发现,这种治理方式对人的要求其实不那么高。因为信息是畅通的,最了解信息的人拥有本地的治理主权,这对治理难度和复杂性的要求就大大降低了。

所以,倘若大家关心如何把一个国家治理好,不妨来关心如何把一个市镇治理好,如何把一个公立学校治理好。一个国家的每一个市镇都治理得更好,每一个公立学校都治理得更好,这个国家才能治理得更好。上述两个案例简要说明了民主治理的细节及其优势。

\tsection{民主模式的多样性}

民主政体并非千篇一律,而是涉及复杂的多样性问题。首先,大家经常会提到直接民主和间接民主的问题,后者又称为代议制民主。所谓直接民主,就是公民直接参与政治活动和政治决策的一种制度安排;所谓代议制民主就是公民选举代表参与政治活动和政治决策的一种制度安排。

比如,对一个村庄的公共事务来说,很多事情可以采用直接民主的办法。但是,通常直接民主只适用比较小的地理和人口规模,如果这个范围过大的话就只能采用代议制民主的方法。迄今为止,雅典城邦是人类政治史上直接民主的典型。随着互联网时代的到来,现在有人提出来,如果采用网络投票的方式,是否有可能在更大范围内实行直接民主?从现在的趋势来看,发达国家有望率先使用这种电子投票方式。如果是这样,直接民主的范围就可以有效扩大。

现在的主流是代议制民主。当然,全民公决有时会作为代议制民主的一项补充性制度。以美国联邦政府为例,选民通过选举人团投票选举总统,同时选民投票选举本选区的众议员,各州议会选出参议员。然后,平时就由总统、参议员组成的参议院和众议员组成的众议院代表美国人民做出政治决策。这是一种典型的代议制民主。有人说,代议制民主实际上不够民主。因为代议制民主意味着最重要的政治权力往往掌握在被选出来的代表手中。尽管代议制民主不如直接民主更为民主,但代议制民主的优势也是显著的。首先,代议制民主可以解决国家规模和统治可行性的问题,这使得大国的民主治理成为可能。其次,代议制民主还在政治生活中恰到好处地平衡大众民意和精英治理之间的关系。比如,C国与J国就D岛发生争议,然后两国民众群情激愤,都希望派军队直接占领该岛。如果这种事情付诸全民公决的话,很有可能引发严重的国际冲突。在这种政治关头,民意有可能像脱缰的野马一样失去控制。代议制民主的一个机制是用精英治理来平衡大众民意。这类重要事务就应该交给那些更懂得国际政治和更擅长外交的人来做决定——当然,这些人是民众通过投票选出来的或是由民选的政治家任命的。因此,代议制民主包含了某种微妙的平衡,长远来看有利于民主的稳定。

跟代议制民主有关的一个问题是平民主义民主与精英主义民主的分野。熊彼特是典型的精英主义民主论者,他把民主定义为人们通过投票来选择政治家的一种制度安排。他甚至明确地说,民主只是意味着人民有选择或拒绝统治他们的人的机会。但是,并非所有人都支持精主义民主论,平民主义民主论者罗伯特·达尔在《论政治平等》中就这样问:有没有可能实现实质性的政治平等,而非形式上的政治平等呢?

一般理解的政治平等更多是形式上的平等,是一种资格的概念,即每个公民都有参与政治的平等资格。但是,当所有人都参与政治的时候,是不是每个人都发挥了同等的影响力呢?比如,财富多的人可以花钱做广告,可以提供资助,可以组建工作班子,他的政治影响力就会比较大;有的人言论影响力很大,比如他在一个发行量巨大的媒体上开专栏,也会影响很多人;有的人是主要政党的重要领导人或政治活动家,可以通过政党组织来组织动员,他对政治的影响会非常大。所以,尽管每个人参与政治的资格是一样的,但是每个人实际的政治影响力差异很大。从这个角度说,即便在民主政治条件下,影响力意义上的“政治平等”是难以实现的。罗伯特·达尔追问道:实质性的政治平等可欲吗?再进一步说,假定实质性的政治平等是可欲的,那么它可行吗?

对于这个问题,有人认为实质性的政治平等是不可能实现的,这只是一种政治幻想。每个人有参与政治的投票权,有政治表达的权利,有结社的权利,有同等的政治身份和公民身份——我们能做到这些,就已经非常好了。如果要追求实质性的政治平等,要追求每个公民政治影响力的平等,那既是无法做到的,又不值得去追求。但是,也有人认为实质性的政治平等是值得追求的东西。比如,罗伯特·达尔认为实质性的“政治平等在一个国家中是可欲的”。但是,达尔也认为实质性政治平等的阻力很大:

\quo{一个政治单位公民之间政治平等的目标总是和到处面临着可怕的障碍:政治资源、技能和动机的分配;时间不可约减的限度;政治制度的规模;市场经济的盛行;重要的但不民主的国际组织的存在;严重危机的不可避免性。\nauthor{罗伯特·达尔:《论政治平等》,谢岳译,上海:上海人民出版社2010年版,第48页。}}

对于实质性政治平等的观点,你是支持还是反对呢?每个人可以做出自己的判断。支持实质性政治平等的观点,倾向于更支持平民主义民主;反之,则倾向于支持精英主义民主。

讲到政治平等的问题,经常会提到一部名为《寡头统治铁律》的重要著作,作者是意大利政治学家罗伯特·米歇尔斯,他研究了现代民主政体中的政党组织,结论是任何组织最终都是寡头统治的。他认为,任何政党组织总是由这个组织的少数人领导和控制的,不可能实现一种真正的大众治理。这是一种非常强烈的精英主义观点。美国学者西蒙·马丁·李普塞特在《寡头统治铁律》一书的序言中总结道:

\quo{正是组织使当选者获得了对于选民、被委托者对于委托者、代表对于被代表者的统治地位。组织处处意味着寡头统治。\nauthor{罗伯特·米歇尔斯:《寡头统治铁律:现代民主制度中的政党社会学》,任军锋等译,天津:天津人民出版社2003年版,英文版前言,第1页。}}

所以,在米歇尔斯看来,人类社会说到底是精英统治的,主要权力和资源都控制在精英手中,对这个社会具有重要影响的决定也是精英做出的。

再进一步探讨,若暂且接受精英主义的视角,就是假定所有社会都是由精英统治的,那么有人会问:威权主义的精英统治和民主主义的精英统治有何区别呢?既然社会实际上都是由少数人统治的,那么两种类型的社会还会有区别吗?

从逻辑上说,两者的区别主要在于三个方面:第一,政治领导权是不是对外部开放?在民主主义的精英统治中,政治领导权是开放的。比如,一个人出生时可以是穷人,可以出生在一个普通家庭,甚至肤色与多数人不同。后来,他经过努力,上了最好的大学,在法学院获得学位,又通过做律师挣了不少钱。然后,他发现自己更大的抱负是在政治领域,他就选择投身于政治,最终成了这个国家的最高行政官。这大概就是美国总统奥巴马的人生道路。这个例子说明,民主的精英统治,其政治领导权是对外部开放的。对威权的精英统治来说,其政治权力封闭在一个规模相对较小的特定集团或特定圈子中。这大概是两者的重要区别。第二,民主的精英统治更需要兼顾公共利益,或者说是兼顾多数人的利益。至于威权的精英统治,更有可能只是为了少数人利益进行统治的。第三,普通公民固然并没有太多机会直接参与公共决策,但在民主的精英统治下,普通公民拥有选择这个还是那个政治精英、这派还是那派政治精英来统治的权利,他们可以通过投票来选择或否决政治精英。而威权的精英统治下,普通公民并不拥有这种政治权利。所以,两者的差异还是明显的。

\tsection{民主的悖论与被误解的民主}

为了完整地理解什么是民主,这里再介绍一些关于民主的最新研究。一方面,本书作为一部政治学通识读本,关于民主的基本介绍要尽可能做到完整;另一方面,目前国内公共领域流行着关于民主的很多误解。这些研究基调都是不要过分理想化地理解民主与民主政体。

《民主政治的三个悖论》是一篇流传较广的论文,作者是美国斯坦福大学高级研究员拉里·戴蒙德,他还是著名学术期刊《民主杂志》的联合主编之一。基于比较政治的经验研究,他认为民主在发展中世界所遭遇的许多问题都源自民主本性中的三种紧张和悖论:

\quo{民主在发展中世界所经历的许多问题都是源自内在于民主的本性中的三种紧张和悖论。第一种紧张是冲突与认同之间的紧张。……没有竞争和冲突,就没有民主政治。但是,任何认可政治冲突的国家都冒着这样的风险,社会变得如此紧张,充满冲突,以至于社会的和平和政治的稳定都将陷于危境。……

第二种紧张或矛盾是代表性与治国能力的冲突。民主政治意味着不愿将权力集中到少数人手中,要使领导人和政策服从于人民的代表和责任机制。但是,为了稳定,民主政治(或任何政府制度)必须有亚历山大·汉密尔顿称作‘能量’的东西:它必须能够行动,必须能够随时地、迅速地、决然地采取心动。政府不仅应回应利益团体的需要,它还必须能够抵制它们的过分要求,并在它们之间进行协调。……

第三种矛盾,即同意和效能之间的矛盾。……假如民主不能起作用,人们则可能宁愿选择不经他们同意的统治,他们可能选择不再忍受去做出政治抉择的痛苦。因此,存在一个悖论:民主需要同意。同意需要合法性。合法性需要有效率的运作。但是,效率可能因为同意而被牺牲。}

在分析上述三个悖论时,戴蒙德善意地提醒那些国家:

\quo{民主政治不仅是最广泛受到称颂的政治制度,而且也可能是最难以坚守的政治制度。在所有的政府形式中,惟独民主政体依赖于最少的强制和最多的同意。民主政府最终发现它们自己陷于内在的悖论和矛盾的冲突中。……建立一个民主政治和坚守一个民主政治是两件不同的事。……假如民主不能起作用,人们则可能宁愿选择不经他们同意的统治,他们可能选择不再忍受去作出政治抉择的痛苦。\nauthor{拉里·戴蒙德:《民主政治的三个悖论》,彭灵勇译,载于刘军宁主编:《民主与民主化》,北京:商务印书馆1999年版,第121—141页。这里的译文根据原文略作调整,参见Larry Jay Diamond,“Three Paradoxes of Democracy,”\italic{Journal of Democracy}, Vol.1, No.3, Summer 1990, pp.48-60。}}

应该说,戴蒙德一文恰到好处地反映出第三世界新兴民主国家在建设和巩固民主政治上遇到的挑战和问题。这篇论文从另外一个视角对民主在实践和运行中的很多实际问题进行了讨论,提醒大家民主并非一种完美和谐的政治状态。

此外,国内学界和公共领域对民主存在着普遍的误解。这种误解表现在七个典型的方面,分别是:\nauthor{这里的内容曾以《被误解的民主》为题刊载于《东方早报》2014年3月18日。}

误解一:民主主要是一个政治哲学命题?目前国内学界和媒体通常把民主当成一个政治哲学问题来处理。比如,最常见的讨论议题包括民主是否优于其他政体,以及民主的优势与弊端等;最经常被提及的人物包括法国启蒙思想家卢梭、法国思想家托克维尔和《民主新论》作者萨托利等;最著名的引用语包括“民主是个好东西”(哈佛大学教授塞缪尔·亨廷顿在《第三波》前言中的话),以及“多数暴政”等。这些热点内容大致反映出国内对民主问题的关注重点与普遍认知。

民主的哲学思辨当然非常重要。但是,最近半个世纪以来,民主主要是一个转型问题。离开转型谈民主,意义不是太大。与哲学思辨相比,转型研究更多关注经验世界已经发生什么和正在发生什么,而非“应该”发生什么。但实际情况却是,民主的哲学思辨是一个热门话题,转型的经验研究却鲜有人问津。很多人对乌克兰、泰国、委内瑞拉与埃及转型乱象的惊讶、困惑乃至大感失望,主要缘于大家对转型的经验知识知之甚少。如今,大众视野里的民主要么是政治哲学意义上的民主,要么是作为发达国家民主典范的英美民主。前者往往把民主理解为一个“应然”的问题,后者容易把民主过分理想化。但是,特别是对于发展中地区来说,经验世界里的民主与实际发生的转型,跟前面两种解读都相去甚远。所以,只有关注转型问题,才不会以过分简单化的思维来理解民主。

误解二:转型是一个单向线性的进程?即便进入经验世界,不少人容易把转型理解为一个单向线性的进程,众所周知的转型三部曲是:旧政体的瓦解、新政体的创建和新政体的巩固。顺利完成转型三部曲的最著名案例要算美国。美国人第一步是通过1776—1783年的独立战争赶走了英国人,瓦解了旧政体;第二步是1787年制定宪法以及随后建立联邦政府,创建了新政体;第三步是宪法的有效运转及政治制度的完善,巩固了新政体。

但是,需要提醒的是,美国通常被视为政治发展的特例。其他大国——诸如法国、德国、意大利、日本等,从传统政治向现代政治的转型都经历过较为曲折的过程,这些国家至少都经历过一次民主政体的崩溃。法国经历过共和制与君主制的反复,二战以后还遭遇了第四共和国的严重危机。后面三个国家则都经历过军国体制和第二次世界大战之后的政治改造。至于第三波民主化国家中的西班牙、韩国、智利、巴西、土耳其等无不经历过类似的曲折进程。在这些国家的历史上,政变随时可能发生,内战亦非没有可能——比如西班牙内战就与转型有关。从很多国家的经验来看,转型就如同新政体的分娩过程,可能伴随着巨大的痛苦与反复的挣扎。这样,就不难理解乌克兰的转型难题与政治危机。有的国家至今还在转型道路上不停地徘徊,比如泰国。

误解三:政体要么民主要么不民主?这是政体类型的经典两分法,这种两分法在1974年启动的第三波民主化之前并无大碍。但是,正如第5讲已经提及的,第三波以来的重要现象是出现了大量的“两不像政体”(hybrid regime)。这种政体既非标准的威权政体,亦非标准的民主政体,而是介于两者之间,但国际学界通常把“两不像政体”视为威权色彩浓厚的政体类型。借助这一概念,大家就更容易理解一些转型国家正在发生的事情。

关于乌克兰国内政治危机的争论,很大分歧就出现在对其基本政体类型的判断上。持政变论者认为乌克兰此前符合立宪民主政体的标准,相反观点则把乌克兰视为某种程度的威权体制类型。按照自由之家与“政体Ⅳ”两大国际机构的评级,乌克兰都被归入“两不像政体”的类型,也就是说乌克兰的政体具有相当程度的威权色彩。所以,乌克兰政治危机中的法理问题没有那么简单,并不像一是一、二是二这般清晰。讨论乌克兰政治危机的另一种观点认为,需要区分民主政体下的街头运动与其他政体下的街头运动,这当然是对的。但是,不能把乌克兰的政治运动简单视为民主政体下的街头政治。转型国家不能排除的一种情形是,总统或总理一旦当选并采取违反宪法或法治原则的政治行动时,现行的正式制度框架可能会失去有效制约总统或总理权力的力量。一些国家街头政治的兴起,就与此有关。更为复杂的是,尽管泰国与乌克兰同样面临街头政治的问题,但泰国总理英拉·西那瓦与乌克兰前总统亚努科维奇在很多问题上的做法存在重要差异。国际上一般认为,英拉当选总理以来大体上没有采取过与宪政或法治原则相抵触的政治行动。因此,街头政治的法理问题并没有那么简单。

误解四:不民主就是因为不民主?这种表述本身容易招来误解,但某些流行观点的逻辑正是如此。当讨论亚努科维奇的总统权力如何不受约束时,一种观点认为这是“因为乌克兰缺乏宪政”。这种见解的问题是,不能用“缺乏宪政”来解释“总统权力不受约束”,因为在这种情境下“总统权力不受约束”本身就等于“缺乏宪政”。这种解释会变成同义反复。再进一步说,倘若宪政是宪法的统治,那么宪法本身又如何统治呢?在政治上,宪政本身是无法自我实施的。

背后的深层逻辑是,不少人把民主的文本或宪法简单地视为一套可拆卸的政治装置。一旦一个国家安上这套政治装置,该国就变成民主国家或立宪国家了。但实际上,民主的文本或宪法本身不过是几张纸而已。民主的文本或宪法能否生效,能否运转起来,以及能否运转得好,全赖实际的政治过程,全赖主要政治力量的所作所为,全赖政治家的领导力与选择。所以,民主这套政治装置究竟怎样,不仅取决于这套政治装置本身,更取决于安装和操作这套装置的人。很多国家面临的问题是:为什么制定了宪法和确立了民主框架,这套政治装置仍然无法运转?或者,为什么这套政治装置启动以后,就背离了原本的设计机理和设计初衷?这是比同义反复的解释与思考更有价值的问题。

误解五:民主搞不好是因为民主本身不好?在全球范围内,有些国家的民主搞得不怎么好,比如债台高筑和陷入经济困境的希腊,民选政府经常面临政变或街头政治威胁的泰国,启动转型后陷于教派冲突和军队干政的埃及,等等。一些国家甚至由过去尽管毫无生机却拥有稳定与秩序的社会,变成了彻底的一团糟。所以,一种论调认为民主成了这些国家的祸害。但是,民主搞得好不好与民主本身好不好,是两个问题。用并不准确但容易理解的话语来说,这就好比汽车开得好不好与汽车本身好不好,是两回事。车开得好不好,既取决于车本身,又取决于谁来开以及如何开。即便是一部好车,若遇到一个糟糕的司机,同样容易出问题。所以,在马路上看到有人车开得不好,出现故障,甚至遭遇车祸,都无法得出汽车本身不好的简单结论。况且,还有大量的汽车不仅行驶速度很快,而且还相当稳定。

民主搞不好的直接问题是不会搞民主。民主要搞好,既涉及一套基于民主文本和宪法条款的制度安排,又涉及政治精英与主要政治力量的信念与行为,还涉及最初的民主实践能否常规化、惯例化与稳定化。这里的任何一个方面要搞好,都太不容易。转型困难国家的一个重大挑战,是此前的旧政体没有给新政体留下多少有利的遗产,反而是留下了很多沉重的包袱。一位美国学者在评价埃及转型时这样说:“对民主而言,威权政体是一所糟糕的学校。”以埃及为例,复杂的教派冲突、政治上强势的军队、缺乏充分民主信念的精英阶层、落后的经济社会状况都是转型的阻力,当然也都是政治搞不好的原因。但是,这些问题没有一样是民主本身造成的,而都是此前统治的遗产。所以,这样的国家民主搞不好很可能是此前的负资产过于庞大,而不能简单归咎于民主本身的问题。

误解六:民主重在选举竞争与权力制衡而政府效能无关紧要?很多人受启蒙运动以来的政治哲学影响极大,一谈到民主就马上想到“分权制衡”这几个字。英国思想家洛克和法国思想家孟德斯鸠的分权学说被视为启蒙时代以来政治理念的正统,《联邦党人文集》中更受重视的是关于联邦制与三权分立的篇章。当然,对现代民主来说,分权制衡非常重要。但是,把民主仅仅理解为分权制衡就有失偏颇。实际上,只有政治参与、政治竞争、宪政约束与分权制衡,没有相当的政治权威与政府效能,任何政府是难以为继的,民主政体将无法维系。英国宪法学家白芝浩认为,先要有权威,然后才谈得上限制权威。美国思想家汉密尔顿在《联邦党人文集》中更是大篇幅地论述有效政府如何必要,以及政府效能不可或缺。

对不少转型国家来说,无法通过民主的方式形成有效的政府能力,是民主搞不好的重要原因。政府缺乏效能的常见情形包括:行政权与立法权的冲突、无法形成多数派执政党、议会政党数量的碎片化、政治领导层阶层缺乏领导力和政治技巧,以及缺乏功能健全的官僚系统,等等。在保证政治参与和政治竞争的同时,民主政府同时还必须有所作为,这样才能维系其民主政体本身。如果民主政府缺乏效能,从消极方面讲,政府可能会陷于瘫痪,政治竞争与分权制衡将演变为不同政治家与党派的恶斗;从积极方面讲,政府将无力应对重大的政治经济问题,无法在市场改革与经济发展等关键问题上达成绩效,也就无法通过提高新政体的绩效合法性来强化程序合法性。有民主而无效能,终将损害民主本身。

误解七:不同国家的民主模式都是相似的?国内媒体讨论民主通常比较粗放,仿佛不同国家的民主模式都是相似的。经典的政体类型学区分了不同政体类型的差异,但民主政体内部的模式差异却没有受到应有的重视。实际上,不同民主国家制度模式的差异是很大的,这些国家在政府形式、选举制度、政党体制和央地关系上均有不同的制度安排和不同组合。现有研究认为,不同的制度模式有着不同的政治逻辑,同时不同的制度模式还需要考虑与一个国家经济社会条件相匹配的问题。总的来说,不同的民主制度模式可能会导致不同的政治后果。

拿乌克兰的政治危机来说,原因当然是多方面的。但是,半总统制的政治架构无疑难辞其咎,这是导致这场政治危机的重要制度成因。乌克兰经历过涉及总统、总理与议会三大核心权力机构关系的多次修宪和改革,但该国总体上属于半总统制模式。半总统制的最大问题是,总统与议会之间、总统与议会任命或选举的总理之间容易发生严重的政治对抗。出任总理的女性政治家、主要政党领导人季莫申科被总统亚努科维奇投入监狱,正是在这种制度背景下发生的。这种政治架构和此类事件逐步瓦解了亚努科维奇作为总统的权力基础与合法性基础(吊诡的是,半总统制这种饱受争议的制度模式在第三波国家中扩展还很快)。另一方面,乌克兰还面临着不同地区的认同冲突,这种认同冲突与族群和语言因素有关。如今,这种冲突的焦点出现在克里米亚。从制度视角看,民主政体下不同的制度模式设计——特别是选举制度和央地关系上的制度安排——被视为一种解决国内族群和地区冲突的工具。所以,民主制度模式的多样性也是一个关键问题。

总之,民主——特别是作为转型问题的民主——的真实逻辑不同于这些广为流传的误解。国内公共领域的通病是把民主问题口号化与简单化,结果是整个社会中民主与转型常识的稀缺。这样,对民主的理解就容易停留在“好的”或“坏的”这样的思维层次上,更需要思考的乃是全球背景下真实的民主经验与转型逻辑。

\tsection{威权主义政体的逻辑}

威权主义政体是指一个人或一个小集团的统治,居于统治地位的可能是君主、独裁者、军队或政党等。一般来说,威权政体下缺少正式的政治参与和政治竞争,政府亦非责任制或问责制政府。现代世界的威权政体主要有几种类型。一种类型是君主制,君主制一般是家族统治,有明确的家族继承关系,统治者从传统中获得一定的合法性。一些中东石油国家至今仍然保留着这种统治形式。军人统治也是过去非常流行的一种威权政体类型。20世纪60—70年代世界上有大量国家都实行军人统治,在拉丁美洲和非洲尤为常见。东亚的韩国过去也曾经出现过较长时间的军人统治。非军人统治的个人独裁也是一种常见的威权政体类型。比如,韩国在朴正熙政变之前较长时间由李承晚统治。李承晚并非军人出身,而是一位文职政治家。他统治的前期还有较多的民主参与成分,后来整个统治就越来越威权化。威权政体的另一种类型是神权统治。这种国家一般是实行政教合一的宗教国家,而非世俗国家。在这样的国家,经由政治程序产生的最高行政长官至多是该国的第二号人物,该国最重要的政治人物是宗教领袖,他的实际政治权力和影响力往往要超过最高行政长官。另外,按照萨托利的说法,一党制与霸权党制都是威权主义的统治类型。

那么,威权主义政体的基本特征是什么呢?第一个特征是政治上的非多元化。在这种政体形式下,政治参与和政治竞争受到严格的限制。当然,有些威权国家保留着政治参与的形式,甚至一些国家也有政治竞争,但这种参与和竞争或多或少受到实质性的限制,通常不会出现像民主国家那样的不受限制的、公开的政治竞争。所以,威权主义政体之下的政治领域带有一定的封闭性,它不是那么开放,一小撮重要的政治人物决定着重要的公共事务。

第二个特征是在经济和社会领域倡导多元化,这是威权政体不同于极权主义政体的地方。威权政体之下通常都存在私人企业部门,允许发展市场经济,部分威权国家还允许开办私人报纸和私人电台。威权政体应该是政治领域非多元化与经济社会领域多元化的结合。一个典型的例子就是皮诺切特时代的智利,他通过军事政变颠覆了智利已经岌岌可危的民主政体,把很多反对派的政治家投入监狱或秘密处决,最终垄断了智利的政治权力。但是,他同时在经济上推行自由主义,请美国芝加哥大学毕业的一批年轻经济学家来协助他进行自由化改革。所以,皮诺切特统治是政治上威权化和经济上自由化的结合。

第三个特征是实行一定的政治控制和政治压制。如果不实行一定的政治控制和政治压制的话,政治的非多元化和封闭性就会被打破掉,上面提及的第一个特征——即政治上的非多元化——就会出现问题。

第四个特征是意识形态控制和政治动员程度总体偏低,这也是威权政体不同于极权政体的一个重要方面。在皮诺切特时代的智利或朴正熙时代的韩国,他们基本上都没有实行大规模的意识形态控制,也没有出现政府出面实施的全面政治动员。

第五个特征是政治领导权的更迭规则完全不同于民主政体。如果是君主制的话,政治领导权的更迭规则是清晰的,子承父业或兄弟相承是主要方式。还有一些威权政体,政治领导权的更迭遵循元老政治模式。比如,由上一任领导人来指定下一任领导人,或由一个规模非常小的封闭精英团体来决定下一任领导人。还有一些威权政体则经常借助暴力方式来完成政治领导人的更迭。一个将军去世后,会有第二个将军依靠武力优势脱颖而出,成为新的领导人。甚至当前一任领导人尚健在的时候,新崛起的将军会通过军事政变来取代他。所以,这种政治领导权更迭规则带有明显的暴力色彩。而民主政体下政治领导权的更迭通常是通过公开的政治竞争完成的。

理解现代威权政体,还要注意几个关键的逻辑问题。首先是合法性问题。第二次世界大战以后,全球的政治意识形态为民主价值观所支配。国际社会的一个基本共识是:“民主是个好东西。”这是塞缪尔·亨廷顿在《第三波》序言中的一句话。反过来,不民主在意识形态上被认定为坏的。所以,在观念上,民主政体要优于非民主政体。一旦国际主流社会接受民主价值观,威权政体或多或少都会面临合法性的严峻挑战。威权政体始终不能解决程序合法性的问题。按照孔子的说法,所谓名不正,则言不顺;言不顺,则事不成。从这样的视角看,这类政体始终会面临能否维系政治稳定的困境。特别是,一旦遭遇重大危机,威权政体的合法性问题马上会浮出水面。

有人说,威权政体通过改善经济绩效可以增强自己的合法性。这种说法不无道理,但威权政体同时还面临着“发展悖论”。一方面,优质威权政体一旦推动发展,更加现代化的经济、教育与观念往往会引发更多的民众抗争,从而构成对原有体制的挑战;另一方面,如果没有发展,劣质威权政体会由于经济停滞、社会不公和治理不善而激起严重的怨愤心理,从而在另一个方向上引发抗争。所以,合法性始终是威权政体的“阿喀琉斯之踵”。

其次是国家治理问题。1997年世界银行发展报告《变革中世界的政府》发布之后,“有效治理”的概念就开始在全球流行。“Good governance”一般译为“有效治理”或“善治”。威权政体在国家治理问题上的挑战,可以从两个视角来理解。一方面,威权政体的治理方式通常是自上而下的。这种治理方式天然地会导致严重的委托代理问题。在委托代理关系中,如果是一层委托代理关系,委托人通常都能有效地监督代理人。但是,如果是多层委托代理关系,对有效治理的挑战就非常大。总的来说,委托代理关系链条越长,在链条终端的最后一级代理人能实现最初委托人意图的可能性就越低。考虑到委托代理关系链条的长度,对一个威权大国来说,委托代理链条的终端上往往难以实现有效治理。

另一方面,威权政体的治理挑战还来自于信息问题。印度经济学家阿玛蒂亚·森在《贫困与饥荒》中认为,在自由民主国家,因为新闻自由,从来没有发生过大规模的饥荒致死案例。\nauthor{阿玛蒂亚·森:《贫困与饥荒》,王宇、王文玉译,北京:商务印书馆2001年版。}他认为主要机制在于:自由民主国家若发生非常小规模的饥荒,媒体就会报道,全国就会知晓,然后应急机制就会启动。但是,在威权政体下,最大的风险是所有这些不同层级的政府与部门都可能会封闭和阻塞信息。中央政府封锁信息的原因在于担心听到批评的声音,包括防止批评者借此挑战其合法性。地方政府封锁信息的原因在于主要地方官员担心自己的乌纱帽。如果地方官员统辖下的地区出现负面消息,并为更高级别的上级政府所知,上级政府不排除会启动对该地主要官员的惩戒程序。所以,威权政体下封锁消息是可以理解的。正是因为有这样一种机制,最初饥荒发生时信息往往被人为阻断了。一旦等到饥荒信息得到大范围传播时,饥荒通常已严重到不可收拾的程度了。当然,人类在21世纪之前尚未大规模使用互联网、移动通讯、社交媒体以及各种自媒体,所以,那时的威权政府或许能够实现有效的信息封闭。而在移动互联网广泛普及的今天,这更是威权政体的治理难题。

第三个重要问题是威权国家的稳定性很多时候取决于政治领导人这一偶然因素。在任何一个政体中,都可以区分出较为优秀的政治领导人和较为逊色的政治领导人,而威权政体始终面临着这方面的巨大风险。由于威权政体产生或选择政治领导人的机制不是开放的,所以选择领导人方面的风险就尤其大。而一旦成为政治领导人,由于没有充分的权力制约,他既有可能行为端正、治国有方,又有可能不知节制、为所欲为。与民主政体不同的是,如果威权政体遭遇一位较为逊色、甚至极其糟糕的政治领导人时,内部可能缺乏有效的矫正机制。这就使得单个政治领导人与政体稳定之间的关联性会被大幅放大。

第四个问题是很多威权政体都不可避免地面临着最终的政治困境和转型问题。威权政体由于缺乏程序合法性,所以更多地依赖于政绩合法性。这意味着,大家生活比较好的时候,社会相对会比较平稳,满意度也比较高,程序合法性的问题就不那么突出了。但是,如果这些问题没有很好解决时,程序合法性问题就会冒出来。历史地看,只有极少的威权政体能够维持长期的经济增长和繁荣。所以,很多威权国家都会面临一个政治困境与转型的问题。比如,韩国就是一个威权政体下实现经济起飞、而后又通过政治转型走向民主的典型案例。

\tsection{极权主义政体的逻辑}

极权主义政体是另一种政体类型,是指国家试图“完全”控制国民和社会的一种政治体系。当然,这里的“完全”需要打引号,世界上没有哪一种政体国家能真正做到完全控制国民和社会,但极权主义试图这样做。与传统的威权主义政体不同,极权主义政体的出现是比较晚近的事情,兴起在第一次世界大战以后。按照美国学者汉娜·阿伦特的说法,该种政体的主要特征是:

\quo{极权统治的手段不仅比较严厉,而且其极权主义形式与我们所知的其他政治压迫形式(例如专制政府、僭主暴政、独裁)有本质区别。凡是在它崛起执政的地方,它建立全新的政治制度,摧毁一个国家所有的社会、法治和政治传统。无论它的意识形态来自何种具体的民族传统或特殊的精神根源,极权主义政府总是将阶级转变为群众,撤换政党制度(不是用一党制,而是用群众运动来替代政党制度),将权力中心从军队转移到警察,建立一种公开走向主宰世界的外交政策。\nauthor{汉娜·阿伦特:《极权主义的起源》,林骧华译,北京:生活·读书·新知三联书店2008年版,第574页。}}

当然,究竟哪些国家符合极权政体的标准,学术界一直存在着争议。1933年至1945年希特勒统治的纳粹德国一般被公认为极权主义政体的类型。极权主义政体的兴起与现代的技术和组织条件有关。一个重要的技术条件是媒体与通信技术的革新。在古代君主专制国家,即便是皇帝或国王要想监控一个人的行为,从技术上是很难办到的。他如何可能监控每一个臣民呢?另一方面,皇帝或国王想让臣民了解他自己的思想,想把他自己的思想灌输给臣民,也是很难办到的,因为缺乏有效的政治传播途径。但是,媒体与通信手段的革新使得这一切成为可能。所以,对政治来说,媒体和通讯手段是双刃剑。

一个重要的组织条件是诞生了军队之外的大规模组织。随着工业化和城市化的进展,大规模的政治组织和普遍的大众政治动员成为可能。在此之前,人们很难理解在军队之外还可以建立起一个数万人、甚至数十万人的政治组织,并能实现自上而下的政治动员。比如,希特勒出任总理之前,他领导的纳粹党已成为魏玛德国动员能力最强、规模最大的政治组织了。

那么,极权主义有哪些基本特征呢?简单地说,卡尔·弗里德里希和兹比格纽·布热津斯基认为,极权主义包括了六个基本特征:一个包罗万象的意识形态;单一政党;有组织的恐怖;传媒垄断;武器垄断;经济管制。\nauthor{Carl J. Friedrich and Zbigniew K. Brezinski, \italic{Totalitarian Dictatorship and Autocracy}, Cambridge: Harvard University Press, 1956.}

当然,总的来说极权主义的国家数量并不是很多。但是,由极权主义政体引发对自由的破坏,在西方世界引发了极大的恐惧,由此诞生了很多与此有关的文学及影视作品。2006年英国拍摄了一部名为《V字仇杀队》(\italic{V For Vendetta})的电影,这部电影改编自艾伦·摩尔等编绘的漫画《V怪客》。这部电影的场景设定在未来的伦敦,而当时的英国被设想为一个极权主义社会。

这部电影有一个重要情节,就是戴着面具的自由斗士V利用极权主义国家控制的国家电视台播放了一个自己录制的演讲。这是V对当时假想的伦敦市民所做的一场演讲,其内容实际上是对极权主义的反思和对抵制极权主义的呼吁。这个演讲词写得非常好,反映出人类对极权主义统治的深刻思考。V在演讲中这样说:

\quo{晚上好,伦敦。

首先,请允许我向你们道歉。我和你们很多人一样,欣赏有规律生活的舒适、熟悉面孔带来的安全感以及日复一日的平静。我跟每一个人一样享受这些,不过,就节庆角度来讲,这节庆是从美好的角度来庆祝过去的重大事件,这通常和某人的死亡,或者血腥残酷的斗争结束有关,我想我们可以通过抽出一点时间坐下来聊聊的方式来纪念今年的11月5日,一个被可悲地遗忘了的日子。

当然,有些人不希望我们讲话,我怀疑就在此时此刻,电话里吼叫着命令,全副武装的人很快就会上路,为什么?因为沉默代替了谈话,言语总能保持他的力量,言语提供了表达见解的方式,而且它也可以告诉那些愿意倾听的人们真相,而真相是,这个国家有些事情不正常的可怕,不是么?

残暴,不公,歧视和镇压,在这块土地上,你们曾经有过反对的自由,有过思考和言论的自由,而你们现在拥有的是胁迫你们就犯的审查制度和监视系统。这是怎么发生的?又应该怪谁?当然,有些人要背负比别人更大的责任,并且他们也会为此付出代价。但是,说实话,如果你们要找罪人的话,只需要照照镜子好了。我知道你们为什么如此,我知道你们害怕,谁不害怕呢?战争、恐怖事件、疾病……

无数的问题企图要摧毁你的理性,剥夺你的常识,恐惧控制了你,你在慌乱中投向了元首亚当·苏特勒。他许诺给你们秩序,给你们和平,所要的回报就是你的服从和沉默。昨晚,我决定结束这沉默。昨晚我摧毁了老巴里街,来提醒这个国家他所忘记的事情。

400多年前,一位伟大的公民打算将11月5号永远刻入我们的记忆之中,他希望以此提醒世界,公平、正义和自由不只是口号而已,他们应该是我们实现的目标。所以,如果你什么也没有看见,仍然对这个政府犯下的罪行一无所知,我建议你让这个11月5日平淡地过去;可是,如果你见我之所见,感我之所感,而愿求我之所求,我请你一年后的今晚和我并肩站到议会大厦的外面,我们将一起给他们留下一个永生难忘的11月5日。\nauthor{这段引文选用了电影脚本的网络译本,译者不详。笔者根据英文原文做了修订。}}

这篇精彩而简短的演讲词是对极权主义的反思。其中最有力的话语,是V对于这种极权主义为何能够延续的检讨。他讲到,众人因为恐惧而奔向元首,每个人都被恐惧所包围,所以没有人站出来说出真相。实际上,我们每个人都脱不了干系。这篇演讲的台词并不深奥复杂,却难以置信地有力。

在《V字仇杀队》这部电影中,有很多表现极权主义统治的重要信息。比如,国家电视台是直接连接到每一个家庭、每一个酒吧和每一个娱乐场所的,而且所有地方的电视节目都是一样的。又比如,警察们随时出现在城市的每一个角落,在他们元首领导下管理和控制着整个城市。此外,除了很多穿制服的警察,还有大量的、随时会出现的秘密警察。再比如,每个家庭的客厅里都挂着元首苏特勒的头像。这部电影恰到好处地设想了极权主义统治下一种可能的生活状态。

有理由相信,电影《V字仇杀队》的漫画底稿《V怪客》是借鉴了英国作家乔治·奥威尔在著名政治小说《1984》中的精巧构思。奥威尔在《1984》中构思了一个极权主义统治的国家,并生动地描绘了极权统治的种种细节。在该书中,世界由三个巨型国家构成——大洋国、欧亚国和东亚国。奥威尔描绘的正是大洋国极权主义下的政治生活。该书有很多有意思的细节。比如,读者会看到无处不在的老大哥头像,老大哥就是大洋国的元首。书中反复出现的一句话是:“老大哥在看着你。”又比如,在主人公温斯顿的家中,有一个被称为电幕的电子装置,它既是一个可以观看元首讲话和国家电视台节目的接收终端,又是一个随时可以监控每一个人的电子装置——相当于一个360度的摄像头。温斯顿只有在家里的某个特定位置才能避开电幕随时的监控。这就意味着,大洋国的任何公民都没有私人空间。再比如,这个国家的街头还有不停歇的负责巡逻的警察,警察不仅要负责巡视街道,甚至还要负责窥探每个家庭的窗户。此外,大洋国还有负责管理思想的警察,称为思想警察。

在《1984》中,大洋国只有一个政党,即英社。该党的著名口号是:“战争即和平;自由即奴役;无知即力量。”这个国家有四个政府部门:真理部、和平部、友爱部和富裕部。“真理部负责新闻、娱乐、教育、艺术;和平部负责战争;友爱部维持法律和秩序;富裕部负责经济事务。”在《1984》中,一个令人啼笑皆非的细节是不断重印报纸。党的一句口号说:“谁控制过去就控制未来,谁控制现在就控制过去。”为什么要不断地重印报纸呢?因为从今天的立场看,历史上所发生的一些事情政治上是不正确的,所以需要把过去的报纸重新拿出来印刷,印上过去的时期和今天认为适宜的内容。实际上,奥威尔这里作为政治调侃发明的很多细节,在个别国家的历史上就真实发生过。所以,《1984》愈发显示出其作为讽刺极权主义政治小说的重要地位。\nauthor{乔治·奥威尔:《1984》,董乐山、傅惟慈译,沈阳:万卷出版公司2010年版。}

与传统的威权统治不同,极权主义统治试图让每一个人都卷入政治,并最终能控制公民和社会的一切。这种统治方式既不同于传统的君主专制,又不同于20世纪后来的很多新式独裁。一般的威权统治都无法像希特勒和纳粹党一样获得对整个社会如此之强的控制能力和动员能力。因此,极权主义显然是不同于威权主义的一种政体类型。

\tsection{推荐阅读书目}

罗伯特·达尔:《多头政体:参与和反对》,谭君久、刘惠荣译,北京:商务印书馆2003年版。

胡安·J.林茨、阿尔弗雷德·斯特潘:《民主转型与巩固问题:南欧、南美和后共产主义欧洲》,孙龙等译,杭州:浙江人民出版社2008年版。

乔治·奥威尔:《1984》,董乐山、傅惟慈译,沈阳:万卷出版公司2010年版。

汉娜·阿伦特:《极权主义的起源》,林骧华译,北京:生活·读书·新知三联书店2008年版。
