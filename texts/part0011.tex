\tchapter{如何参与?为何抗争?}

\quo[——罗伯特 · 帕特南]{在 1960 年,有 62.8\% 达到投票年龄的美国人去投票站投票,在肯尼迪与尼克松之间做出选择。在经历数十年的下滑后,1996 年仅有 48.9\% 的美国人在比尔 · 克林顿、鲍伯 · 多尔和罗斯 · 佩罗之间做出选择,这几乎接近 20 世纪最低的投票率。总统竞选的参与程度在过去 36 年间下降了近 1/4。}

\quo[——马丁 · 路德 · 金]{受压迫的人民对待他们所受的压迫,有三种办法可行。其一便是默许:被压迫者顺从于自己的命运。……有时受压迫的人还有第二种方式对待压迫,便是诉诸身体的暴力和腐蚀人心的仇恨。……受压迫的人民在追求自由当中还有第三条道路可走,那便是非暴力抵抗的道路。}

\quo[——西德尼 · 塔罗]{他们只有激起较根深蒂固的团结一致感和身份认同,才能创造一场社会运动。因此,我们几乎可以肯定,作为运动组织的基础,民族主义和种族或宗教总是比社会阶级的绝对律令更可靠,原因就在于它们能促进团结一致和集体认同。}

\quo[——戴维 · 赫尔德]{在今天,民主要想繁荣,就必须被重新看作一个双重的现象:一方面,它牵涉到国家权力的改造;另一方面,它牵涉到市民社会的重新建构。只有认识到一个双重民主化过程的必然性,自治原则才能得以确定:所谓双重民主化即是国家与市民社会互相依赖着进行的转型。}

\tsection{什么是政治参与?}

政治参与通常是指公民通过正式途径影响统治者或公共决策的行动与过程。当然,有人认为,非正式途径的参与——甚至是突破现有法律框架的参与——也是政治参与的重要内容。按照后一种观点,政治参与可以被视为公民通过一定途径去影响统治者或公共决策的行动与过程。

政治参与的例子有很多。比如,你居住在北京市昌平区的一个镇上,镇上的道路不是太好,然后你去跟当地人大代表投诉道路状况。人大代表回复说,他会建议政府重新修筑道路。这就是一种政治参与活动。第 5 讲曾提到一个例子,某地的中学校长并不令人满意,当地的家长和中学的教师们意见非常大。你作为一位学生家长,找到当地主管教育的官员,反映某某中学的校长很不称职,你认为应该任命一位新校长。这也是一种政治参与。

又比如,美国公民参加民主党或共和党大会,也是一种政治参与活动。在选举日去投票站投票,则是民主国家公民——无论是积极公民还是消极公民——最为基本的基本政治参与活动,特别是参加总统或国会议员的选举投票。当然,即使是在民主国家,有的公民可能从未参加任何选举活动,这是可能的,但这样的人通常是少数。

再比如,几年前曾在法国巴黎和英国伦敦发生过的街头骚乱事件,也是政治参与的一种形式。在巴黎或伦敦街头,有民众——特别是年轻人——抗议政府的某些政策,一开始是示威游行,然后演变为焚烧汽车和商店。示威游行是合法的,但焚烧汽车和商店是非法的。他们这样做的主要意图是要影响公共政策,然而这却是一种非法的政治参与形式。

有学者把一种较为极端的情形——密谋暴动——也称为政治参与的一种形式。在中国近现代史上,1911 年秋天,武昌的少部分新军士兵与下级军官密谋暴动并获得了成功,史称武昌起义。这是政治参与的一种特殊形式。

\img{../images/image00320.jpeg}[图 9.1 不同类型的政治参与者]

当然,不同公民的政治参与形式和程度是不同的。借助正态分布曲线,可以把所有的政治参与者分为三类:第一类是政治积极分子,第二类是普通政治参与者,第三类是政治冷漠者,参见图 9.1。德国著名社会学家马克斯 · 韦伯有一篇题为《以政治为业》的著名演讲。那么,韦伯是哪一种政治参与者呢?他显然属于政治积极分子。在美国,民主党和共和党都有一大批主要的组织者,比如各州党部的活动家,他们也属于政治积极分子。那些在美国《纽约时报》或英国《金融时报》发表鲜明政见的评论家通常也是积极的政治参与者,他们试图通过言论影响本国或全球公共政策。至于那些在华盛顿、在伦敦担任重要职位的政治家,理所当然属于政治积极分子。

除此以外,每个社会中的大量公民都属于普通的政治参与者。在发达国家,其实普通人关心政治的程度并不高,他们平常谈论的事情大体上很少涉及政治。这些普通的政治参与者,既不算积极,又不算冷漠,而是处于某种中间状态。此外,每个社会都存在一定的政治冷漠群体。与普通公民不同,他们不仅不是政治积极分子,而且还刻意回避或逃避政治。比如,他们甚至很少或从不参加重要的选举投票活动。

通常,政治参与可以分为不同的类型。比如,政治参与是正式的还是非正式的?选民投票是正式的政治参与行为。如果你对本地道路质量不满,去找议员或人大代表个别沟通,就是非正式的政治参与。在美国或英国的国会,都有大量的所谓 “走廊议员” ——这些人并不是议员,但他们经常在国会走廊工作,目的是游说议员就某某法案投支持票或反对票。这也是非正式的政治参与。

比如,政治参与是个体的还是群体的?你一个人去说服本地议员或人大代表,要求他提出一项重新修筑本地公路的议案,这种参与是个体行为。然而,倘若你曾多次沟通,但没有任何效果。这时候,你决定采取进一步的行动,你找到附近几个大型社区的业主委员会,然后每个社区都派出一些人跟你一起去找这位议员或人大代表提出诉求。这样,这种政治参与就从个体行为变为了群体行为。

比如,政治参与是温和的还是激进的?一个社会中有大量的政治诉求是比较温和的,对应的政治参与形式通常也比较温和。刚才提到的会见本地议员或人大代表要求政府修筑公路的情形,通常是一种温和的政治参与。但是,因为对大型造纸企业的排污项目不满,很多市民冲进当地市政府,把市政府暂时占领了,甚至把市长的衣服扒下来,这是比较激进的政治参与。这种事件 2012 年 7 月就在华东某市发生过。

又比如,政治参与是合法的还是非法的?泰国这几年的政治并不平静,大规模的街头政治运动在泰国是家常便饭。该国有著名的红衫军和黄衫军,前者支持他信与英拉,后者反对他信与英拉。红衫军和黄衫军组织过很多合法的政治参与活动,比如和平的示威游行。但实际上,过去经常发生的是,整个示威游行队伍可能会失控,很多人甚至试图占领机场或者是占领政府机构。后者无疑是非法的,有可能会导致社会秩序的瘫痪。

再比如,政治参与是非暴力的还是暴力的?很多政治参与都表现为和平的方式,从投票、参加政党大会到游说议员、和平示威等都是和平的政治参与。但是,政治参与也可能表现暴力的方式。比方说,泰国的红衫军占领机场,或多或少会借助暴力手段。如果不借助暴力或以暴力相威胁,红衫军如何可能占领机场呢?只是这种暴力威胁没有发展为真正的流血事件。但有些国家出现过从普通政治参与演变为严重暴力冲突的事件,甚至演变为局部武装冲突——这就是政治参与的暴力化。特别是,在一些族群分裂程度很高的社会,当威权政治倒塌、转型为民主政体时,族群的政治诉求就会上升,演化为很多极端的政治参与形式,甚至最后酿成了武装冲突和族群屠杀。所以,政治参与也可能会演变为暴力和血腥的形式。

\tsection{政体类型与政治参与}

那么,政体类型与政治参与是什么关系呢?一个简单的事实是:不同政体类型下的政治参与差异很大。民主政体、威权政体和极权政体之下政治参与的不同特征,请参见表 9.1。

\tbl{../images/image00321.jpeg}[表 9.1 不同政体类型下的政治参与]

资料来源:罗德 · 黑格、马丁 · 哈罗普:《比较政府与政治导论》,张小劲等译,北京:中国人民大学出版社 2007 年版,第 177 页。

通常,民主政体下政治参与的数量和程度是适中的。并不是所有人都关心政治,实际上有大量的人不太关心政治,可能多数公民只是参与几年一度的重要投票。比如,他可能主要参加的是总统或国会议员选举的投票,州长或州议员选举的投票,以及跟自己更相关的本地市镇长或学区委员的投票。其余的政治活动,他通常都是不参加的。他每天过着普通的、世俗的、与政治保持一定距离的一种生活,这是一种正常状态。在民主政体下,政治参与是公民个人的自由选择。换句话说,如果他想积极参与的话,政治制度给他提供了空间和可能;但如果他不想积极参与的话,他完全可以做一个政治消极公民。这是他的个人选择。

所以,民主政体下的政治参与者是自愿的,没有人可以强迫他。当然,个别国家有强制投票的规定,法律规定这是公民的法定义务,但这也仅限于选举投票。大部分国家都没有这样的强制性要求。这意味着一个人可能从来都不参与政治,他照样可以在这个社会上生活得很好。在这种政体之下,政治参与的主要目标是为了对决策者与公共政策施加影响。从一些国家的政治实践看,比如左右两派斗争非常激烈时,通常会有更多的选民去投票,因为这些选民希望自己的选票能够发挥一些影响。总之,在民主政体下,公民的政治参与更多是一种影响公共决策的方式。

威权政体下的政治参与程度要比民主政体低很多,大量民众没有卷入政治参与活动。即便有政治参与,通常情况下这种政治参与有可能是被操纵的。按照黑格等人的看法,威权政体下的政治参与,主要目标是为了维护统治者的权力,是为了制造民主的假象。二战以后,民主政体在全球范围内获得了前所未有的合法性,所以所有政府、政党与政治领导人都倾向于把自己标榜为 “民主” 的。比如,即便是伊拉克前政治领导人萨达姆 · 侯赛因也要搞出一个投票选举的形式,而他在 “总统选举” 中能够获得超过 99\% 的选票。这一高支持率,是任何民主政体下的政治领导人望尘莫及的。所以,威权政体的特征是政治参与程度比较低,很多政治参与是受操纵的。

与威权政体不同,极权政体是一种高度动员的政体类型,所以该政体下政治参与程度是很高的。但这未必就是好事。比如,在希特勒的纳粹德国,政治参与程度就非常高。今天留下的历史档案和影像资料显示,希特勒可以在一个 10 万人的政治集会上做演讲,下面的听众表现极其亢奋,很多人的表现是发自内心的。这种政治参与的特点是高度的组织化。希特勒和纳粹党通过政党与政治团体的方式把大量的普通民众组织起来,使其卷入高度参与的政治过程。极权政体下的政治参与,往往与改造社会有关。比如,希特勒一直主张雅利安人的种族优势,犹太人应该被消灭掉,让德国成为一个更纯粹的国家——诸如此类的种族主义理论。然而,极权政体下的高度政治动员和高度政治参与的做法,实际上不过是为了展示统治者的权力而已。极权政体的特点是存在着受到政府或政党严格控制的高度的政治参与。

\tsection{投票与选举权的普及}

公民政治参与的基本形式就是选举投票。民主政体下的公民通过投票来选择反映自己政治偏好的政治家,投票是公民控制政府和落实问责制的基本形式。在选举中,一边是候选人或政治家,另一边是选民。理论上讲,政治家的政策应该反映选民的政治偏好。换言之,政治家通过给选民提供符合其政治偏好的政策,来换取选民的投票支持。上文已提及,《民主的经济理论》作者安东尼 · 唐斯把政治家视为厂商,把选民视为消费者,政治家在政治市场上通过提供适宜的政策来获得选民的选票。

在经济市场上,消费者通常用脚投票来改变厂商。比如,有人长期使用某品牌的牙膏,后来他发现这个品牌的牙膏不够好,或者他发现了更好用的牙膏品牌,就开始换用另一品牌的牙膏。这就是消费者的一种选择。正是通过这样一种过程,消费者给厂商制造了压力:如果你的产品质量不够好或性价比不高的话,我随时会用脚投票。其实,政治的机制也一样。如果选民发现当选政治家的政策没有反映自己的政治偏好,选民在下一次投票中就有可能抛弃这位政治家。大家应该很清楚,政治上的问责制如果不采用这种投票形式的话,是很难操作的。有些政治家自称衷心服务于公众利益。但问题是,谁来评判他是否服务于公众利益呢?如果多数人不满意,可以把他选下去吗?只有通过这样一种选择机制,问责制才能得以落实。因此,没有选择权的问责制不大可能是一种名副其实的问责制。

讲到政治参与中的投票行为和普选权,还要对此做一个简要的历史考察。人类近代民主的雏形,对普通人来说是一种有财产资格或教育资格限制的选举权。换句话说,近代早期的民主是有钱人的民主或受教育者的民主。 “议会之母” 英国最早就走过这样的道路,很多其他国家也走过这样的道路。比如,你的财产要达到多少标准,或你的纳税要达到多少标准,你才能享有这种投票权。当然,一些国家还有教育资格或识字资格的限制。那么,其中的逻辑是什么呢?最早的逻辑并非财产歧视,而是在于一种理论主张——只有有财产的人才会对这个社会承担起责任。如果财产跟社会直接挂钩的一个形式是纳税的话,那么只有一个纳税的人才会对社会负担起实际的责任。所以,只有这样的人,社会才应该赋予他们投票的权利。

很多人一听这个主张,马上就会提出质疑。因为这个主张跟现代民主理念——一人一票准则——是有冲突的。但是,大家应该了解这个主张背后的逻辑。从 19 世纪到 20 世纪的不少保守主义者或保守的自由派们担心,如果没有收入者也能获得投票权,这意味着可能赋予了他们通过政治手段对有产者进行剥夺的权利。比如,哈耶克在《自由秩序原理》一书中就把通过多数投票方式制定的累进所得税政策视为 “不负责任” 的 “温和的抢劫” 。\nauthor{弗里德里希 · 冯 · 哈耶克:《自由秩序原理》下,邓正来译,北京:生活 · 读书 · 新知三联书店 1997 年版,第 71—94 页。}

至于教育资格的限制,过去也有一些国家是这样做的,比如 20 世纪早期的智利。这种限制的主要考虑是一个人参与政治活动的能力或技艺。这一主张的逻辑可以在柏拉图那里找到最早的论述。柏拉图曾说: “统治是一项专门的技艺。” 所以,并不是所有人都适合参与政治,参与政治的人应该有起码的知识与技能,否则可能会造成很多问题。当然,这个说法一定会引起争议,但大家最好了解这种主张背后的逻辑。

从真实的历史过程看,最早关于投票权的观念就是这样。后来,西方社会经历一些重要的政治经济变化,投票权的财产资格和教育资格限制逐步降低,直到最后完全取消。但是,即便如此,一开始普选权仅限于成年男性公民。后来,欧洲国家又经历了主张女性普选权的男女平权运动。在英国,这个运动发生在第一次世界大战的前后。特别是一战以后,英国妇女协会等组织开始努力争取妇女的投票权。从 20 世纪 10 年代到 20 年代,英国成年女性陆续获得了跟男性公民一样的投票权。此外,一些国家的某些历史阶段上,选举资格还有族群或种族身份的限制。比如,在美国和南非,黑人或有色人种的投票权最初是受到限制的,后来这些限制条件也取消了。因此,总的来说,普选权从 19 世纪到 20 世纪是一个逐步完善的过程,一开始是财产较多或与教育程度较高的成年男性公民拥有投票权,后来经历了财产与教育资格逐步取消,随后普选权又扩展至成年女性公民及少数族群公民。这样,到了 20 世纪晚期,民主国家的基本投票权安排就是以成年公民为惟一条件的一人一票制度。

需要提醒的是,选举权的普及并不是通过一个和谐而顺畅的过程来实现的。相反,这一过程中包括了大量的政治冲突、社会运动和暴力现象。一个经典案例就是英国的宪章运动。\nauthor{R.G.甘米奇:《宪章运动史》,苏公隽译,北京:商务印书馆 2013 年版。}大概在 19 世纪早期,英国已经率先成为一个符合宪政与法治标准、少数公民拥有投票权的国家。后来,随着工业革命的推进和民众政治意识的觉醒,宪章运动开始兴起。宪章运动是一场普通劳动阶层要求政治改革的社会运动,起源于 1838 年 5 月 8 日公布的《人民宪章》,一直持续到 1848 年的欧洲革命,历时十年之久。作为宪章运动的政治纲领,《人民宪章》由六个政治主张构成:

\quo{1. 21 岁以上男子享有普选权;

2. 选区大小人数平等;

3. 选举由秘密投票决定;

4. 取消参选财产限制;

5. 给予议员年俸;

6. 进行每年一度选举。}

从直接政治后果来看,英国宪章运动并没有成功。但倘若进行更长时段的历史考察,就会发现,从 19 世纪下半叶到 20 世纪上半叶,《人民宪章》上述 6 条要求中的 5 条在主要发达国家均已实现,而惟有第 6 条 “进行每年一度选举” 未被采纳。当然,由于技术性问题,第 2 条 “选区大小人数平等” 难以完全实现。

在关于法国普选史的研究中,皮埃尔 · 罗桑瓦龙认为,法国从 1789 年大革命到 19 世纪晚期、再到 20 世纪上半叶的政治变迁中,其投票权经历了 “有产公民模式” 到成年男子普选权,再到包括妇女在内的成年公民 “一人一票” 的演进。罗桑瓦龙说,在这个过程中——

\quo{三种历史交织在一起。首先是法律和制度史,其次是认识论的历史,而第三则是文化史。……历史的这一部分并非一种平铺直叙的征服史。实际上在其起点上,也就是说法国大革命期间普遍选举即已在原则上得到认可,但随即遭到严厉的质疑。这一历史伴随着重新怀疑与大倒退的终结而止住了脚步。人们在这方面应当记住以下年代或日期:首先是 1848 年,正是在这一年里,纳税选举时期宣告终止,普遍性原则在去除了大革命时期的模棱两可后得到重新表达;其次是 1851 年 12 月 2 日,正是在这一天,1850 年 5 月 31 日所颁布的邪恶的法律被以某种方式废除;最后是 1875 年 11 月 30 日,众议院议员选举法在这一天的通过最终巩固了这些成果,并由此标志着对普遍选举的一种庄重的确认。但是,选举的法律史也被纳入到一种人类学的视野,即一种个人社会实现的视野之中。它发端于团体不再被作为政治代表的基础的 18 世纪,并因 1944 年法国妇女投票的通过得以延长。\nauthor{皮埃尔 · 罗桑瓦龙:《公民的加冕礼:法国普选史》,吕一民译,上海:上海人民出版社 2005 年版,第 365 页。}}

这一段看似轻描淡写的文字实际上掩饰不住法国人为争取普选权所经历了长期、复杂的政治斗争过程。当然,这一政治过程的背景是从 18 到 20 世纪欧洲经济社会条件和人们精神世界的重大变迁。

\tsection{独自打保龄?}

另一个有趣的问题是政治冷漠。政治冷漠泛指公民的 “政治不参与” ,表现为公民对政治不感兴趣,不愿意花时间和精力去参与各种形式的政治活动。应该说,每一个社会都存在为数不少的政治冷漠者群体,这是正常现象。然而,令人担心的是很多成熟民主国家在 20 世纪 60 年代以后出现了投票率持续下降的情形。从最近几年国会议员选举的投票率来看,美国大概只有 54\% ,印度是 58\% ,日本是 71\% ,英国是 76\% ,法国是 76\% ,芬兰是 78\% 。这意味着美国大约有四成五的选民是不参加国会议员选举投票的,日本、英国和法国也有二成多的选民不参加国会议员选举。由此看来,在成熟民主国家,部分选民的政治冷漠是普遍现象。

那么,为什么成熟民主国家的投票率那么低呢?造成政治冷漠现象的原因是什么呢?关于这一问题,存在很多不同的理论解释。比如,理性选择理论强调,不少比例的选民不参加投票是其理性决策的结果。选民参加投票通常需要支付成本——比如参加投票的时间和交通费用等,但一张选票通常并不会改变选举结果,其边际影响微乎其微。所以,经过理性考虑,选民决定不参加投票。

关于政治冷漠的政治文化理论则强调民主精神与公民文化的衰退。也就是说,人们从价值观和信念上更少关心公共事务了。这也是很多社会较为普遍的现象。关于政治冷漠与参与弱化的政治文化研究,首推著名政治学家罗伯特 · 帕特南 2000 年出版的热门作品《独自打保龄》,该书的主题是美国社会资本的衰落——另一种说法是美国社会公民政治的衰退。帕特南在这本著作中这样描述美国社会发生的现象:

\quo{在 1960 年,有 62.8\% 达到投票年龄的美国人去投票站投票,在肯尼迪与尼克松之间做出选择。在经历数十年的下滑后,1996 年仅有 48.9\% 的美国人在比尔 · 克林顿、鲍伯 · 多尔和罗斯 · 佩罗之间做出选择,这几乎接近 20 世纪最低的投票率。总统竞选的参与程度在过去 36 年间下降了近 1/4。

……公众对时事的兴趣趋向在过去二十五年间逐渐衰退了大约 20\% 。同样的,另一项长期的年度调查表明:从 1975 年到 1999 年,公民对政治的兴趣稳步下滑了五分之一。

……在过去二十年中,每年竞聘美国各类政治职务——从学校董事会到镇议会——的人数缩减了约 15\% 。这种下滑导致美国每年从这些人里损失 25 万名候选人。\nauthor{罗伯特 · 帕特南:《独自打保龄:美国社区的衰落与复兴》,刘波等译,北京:北京大学出版社 2011 年版,第 21—34 页。}}

这些信息对帕特南来说,都意味着美国的政治参与和公民联系性的减少,更多人开始选择 “独自打保龄” 。这也佐证了美国社会资本的降低。在该书中,帕特南总结出几个主要的影响因素,包括 “时间和财富压力” , “市郊化、上下班和城市扩张” , “电子娱乐——主要是电视” 及 “代际更替” 等等。帕特南的这部著作出版以后,引起了广泛关注,并继《使民主运转起来》以后再次引起美国及国际学术界对社会资本研究的重视。

尽管帕特南的作品已经成为新的社会科学名著,但这部著作并非一边倒地广受赞誉,很多人并不赞同帕特南的主要观点。一方面,帕特南较重视那些会员数量下降、活跃度降低的社团组织,但没有重视那些会员数量上升、活跃度提高的社团组织。比如,从 1963 年到 1993 年,美国退休人员联合会的会员从 40 万剧增至 3000 万。另一方面,媒介与交流形式的改变——特别是电视和网络的兴起,美国人改变了政治参与和公民参与的方式。换句话说,美国的社会资本并未降低,只是转换了形式。当然,一部优秀著作在引起关注的同时受到质疑,已成为学术界的常态。

\tsection{社会运动与非暴力抗争}

社会运动是政治参与的一种重要形式。除了选举投票,社会运动是很多普通公民卷入过的最重要的政治活动。社会运动一般是指一种特定形式的集体行动或集体行为,其动机主要来源于成员的态度和期望,通常有松散的组织框架,具有明确的诉求。社会运动一般被视为社会抗争的一种表现形式。

18 世纪以后,社会运动才开始在人类历史舞台上兴起,它与工业化和城市化对人类生活与组织方式的改变有关。在诸种社会运动中,劳工运动或工人运动是较早的大规模社会运动形式。此后,主要的社会运动还包括民族主义运动、少数族群或种族权利运动、妇女权利运动、环保运动等等。

20 世纪兴起的一种非常独特的社会运动是非暴力不合作运动。非暴力不合作运动的思想可以追溯至美国哲学家亨利 · 大卫 · 梭罗,他在早期的哲思作品中阐述了 “公民不服从” (civil disobedience)思想。\nauthor{本节多处引用梭罗的文字,参见亨利 · 大卫 · 梭罗:《公民不服从》,载于何怀宏编:《西方公民不服从的传统》,长春:吉林人民出版社 2001 年版,第 16—38 页。}按照梭罗的见解,一个人的行事准则不在于服从政府或者法律,而在于服从自己的良心。他说: “我有权承担的惟一义务,乃是不论何时,都做我认为正当的事情。” 由此,梭罗认为,当一个人认为政府不义时,便拥有反对的权利。 “每个人都承认革命的权利;这便是说,当政府沦于暴政,或它效力极低、无法忍受,就有权拒绝向其效忠,且有权对其反抗。” 沿着这样的逻辑,梭罗接着问: “不公正的法律依然存在:我们是甘心服从这法律,还是致力于修正之,在达到目的之后才来服从,甚或立即破坏了它?” 他在很大程度上倾向于最后一种立场,即每个人都有权破坏他良心上认为不公正的法律。

梭罗所主张的乃是一种公民不服从的精神。不仅如此,他强调每个公民应该身体力行这种精神。惟其如此,这种公民不服从思想才能产生实质性的力量。当他评价废奴运动时,他有这种的主张:

\quo{我非常清楚,若是在马萨诸塞州里,只有一千个人,若是只有一百个人,甚或我说若是只有十个人——若是只有十个正直的人——不,若是只有一个正直的人,能停止蓄奴,真正脱离奴隶主同伙,并因之被关入县立监狱,这便能在美国废除奴隶制度。因开始时微乎其微并不打紧,只要一次做得到,便总会有人做下去。可我们倒宁愿夸夸其谈:谈论它成了我们的使命。}

对于这种公民不服从可能的力量,梭罗深具信心。他继续论述道:

\quo{若少数服从了多数,它便失去了力量;它甚至连少数也算不上;而若它倾力来对抗,它便不可战胜。若州政府要在将所有正义之士关进监狱、与放弃战争和奴隶制之间做内战,它的选择简直就绝不会犹豫。若是有一千个人今年不纳税,这方法既非暴力,也不会流血……事实上,这便是和平革命的意义所在——如果说这种革命能够出现的话。}

梭罗在此已经阐发了一种公民非暴力不服从的政治主张。不仅如此,他本身也是这种主张的践行者。他坚持六年没有交纳人头税,因此被逮捕入狱,但他对此坦然承受。因此,梭罗也是公民不服从思想的实践者。

当然,非暴力不合作运动的最伟大实践者是印度政治家甘地。甘地从南非回到印度时,印度的民族独立运动在很大程度上已陷入困境。直接的原因是印度的精英阶层和大众阶层各自认同并采取不同的反抗方式,但这些方式看来并不能奏效。印度的精英阶层倾向于选择与英国统治者进行政治谈判,但结果是,英国政府尽管做了很多让步,但他们并不认可印度的自治权。印度的底层民众倾向于暴力反抗,其中代表性的就是 1857—1859 年的印度民族大起义,但这种直接的暴力反抗会遭到英国政府的武力镇压。此外,印度精英阶层并不十分赞同对英国统治者进行剧烈的暴力反抗。

正是在这样的背景下,甘地开创性地提出了非暴力不合作运动的主张。 “非暴力不合作” 这个词组本身已经昭示了甘地的核心观点。在甘地看来,对英国统治者,印度作为一个民族的惟一正确做法是两者的结合:一是彻底的不合作,二是彻底的非暴力。从具体政策来说,甘地呼吁以非暴力的不合作手段来抵制英国殖民者的统治。他呼吁印度人不纳税、不入公立学校、不到法庭、不承担公职以及不购买英货,等等。

甘地在他的作品中阐述了彻底的、毅然决然的非暴力精神。下面两句话反映了甘地的基本立场: “勇敢在于赴死,而不在于杀戮” ; “人类只能通过非暴力来摆脱暴力,通过爱来克服恨” 。他这样认为:

\quo{非暴力反抗成功的必要条件是:1. 非暴力反抗者不应当在心里憎恨其对手;2. 问题必须是真实和实质性的;3. 非暴力反抗者必须准备受苦到底。}

正是因为非暴力反抗,甘地认为印度已经开创了新的历史:

\quo{我们采用的手段不是暴力,不必流血,也无需采取时下人们所理解的那种外交手段,我们运用的是纯粹的真理和非暴力。我们企图成功地进行不流血革命,无怪乎全世界的注意力都转向我们,迄今为止,所有国家的斗争方式都是野蛮的。他们向自己心中的敌人报复。\nauthor{莫罕达斯 · 甘地:《论非暴力》,载于何怀宏编:《西方公民不服从的传统》,长春:吉林人民出版社 2001 年版,第 39—60 页。}}

甘地实践非暴力不合作思想的一个典型例子是 “食盐进军” 。1930 年年初,英国政府在印度颁布了《食盐专营法》,规定食盐专营,结果是食盐价格和相应税收的提高。3 月 12 日,甘地带领他的 78 名学员从萨巴尔马蒂的真理修道院出发,徒步 390 公里,开赴印度海滨小镇、食盐产地丹地,自行熬制与获取食盐。随后,数百万人响应甘地的号召,公开反对《食盐专营法》,自行熬制与获取食盐。在此过程中,多数群众恪守甘地所倡导的非暴力原则。英国政府起初并未重视,后来面对规模越来越大的 “食盐进军” 运动,决定予以坚决镇压,导致数万人被逮捕,包括甘地本人。尽管如此, “食盐进军” 总体上仍然坚持非暴力原则。最后,英国政府迫于压力,不得不取消《食盐专营法》,并释放甘地等政治领导人。正是在甘地的领导下,非暴力不合作运动最终使印度摆脱了英国的殖民统治,走向了民族独立。当然,也有人认为非暴力不合作运动之所以能够奏效,乃是因为英国政府尽管是外来的却是较为文明的统治者。

20 世纪下半叶,非暴力不合作运动的伟大实践者是美国黑人民权运动领袖马丁 · 路德 · 金。美国的黑人民权运动又被称为非洲裔美国人民权运动,兴起于 20 世纪 50 年代中期,发展延续至 60 年代末。起初,黑人反对的是美国学校、公共交通、商业场所的种族隔离措施,后来演变为一场争取全面的黑人平等公民权的运动。在此过程中,马丁 · 路德 · 金作为一名黑人牧师,逐渐成长为该运动的主要领袖。1963 年 8 月 28 日,在美国首都华盛顿,25 万包括白人在内的美国公民举行了声势浩大的公共集会,以反对美国种族隔离制度。在这次大会上,马丁 · 路德 · 金发表了著名演讲《我有一个梦想》。1968 年,马丁 · 路德 · 金遇刺。但整个运动最终促使美国南部废除了种族隔离,并使美国黑人公民获得了同等的政治社会权利。总的来说,美国黑人民权运动也是试图经由非暴力的抗议活动来达成他们的政治目标,这也是马丁 · 路德 · 金的政治主张。

他认为: “任何非暴力运动,都要包括四个阶段:收集事实,以判定不公正是否存在;谈判;自我净化;以及直接行动。” 他还以伯明翰社区为例,说明了非暴力运动的四个阶段。首先是,通过显而易见的事实判断该社区存在显著的种族不公正。随后,他们努力与本社区的头面人物们展开谈判,但收效甚微。他们在自己的希望遭到沉重打击、决定采取直接行动之前,还要先进行一个自我净化的过程。他们不断自问: “我是否能挨打而不还手?”  “我能否忍受监狱的考验?” 在这一切准备就绪后,他们开始采取直接行动,而直接行动的目标是 “制造一种充满危机的局面,以期不可避免地开启谈判之门” 。当然,目的就是要消除美国的种族隔离制度,最终并使黑人获得同等的公民权利。

马丁 · 路德 · 金非常清晰地阐明了面对压迫时可能的不同选择:

\quo{受压迫的人民对待他们所受的压迫,有三种办法可行。其一便是默许:被压迫者顺从于自己的命运。他们默然适应所遭受的压迫,因之变得顺应于这样的压迫。……有时受压迫的人还有第二种方式对待压迫,便是诉诸身体的暴力和腐蚀人心的仇恨。暴力往往能带来一时的结果。……暴力却绝不能带来长久的和平。……受压迫的人民在追求自由当中还有第三条道路可走,那便是非暴力抵抗的道路。}

他主张的当然是非暴力抵抗的道路。马丁 · 路德 · 金还认为非暴力抵抗有若干基本特征:

\quo{其一,要强调的是非暴力抵抗并非给怯懦者使用的策略,它实在是一种反抗。……第二个基本事实,是它并不企图打败或羞辱对手,而是要赢得他的友谊和理解。……第三个特征,是其进攻直接针对罪恶势力,而非行使这罪恶的人。……第四点特征,是甘心受苦而不求报复,甘心挨打而不求还击。……第五方面,是它要避免的不仅是肉体的外在暴力,还包括精神的内在暴力。……第六个基本事实,是它基于这种的确信,即宇宙乃处于正义一方。\nauthor{以上两段引文参见小马丁 · 路德 · 金:《〈寄自伯明翰监狱的信〉及其它》,载于何怀宏编:《西方公民不服从的传统》,长春:吉林人民出版社 2001 年版,第 61—115 页。}}

\tsection{如何理解社会运动?}

当然,社会运动不只是非暴力不合作运动。那么,如何理解不同形式的社会运动呢?按照美国康奈尔大学政治学教授西德尼 · 塔罗的看法,社会运动是 “以潜在社会网络和使人产生共鸣的集体行动框架为基础,能发展出对强大对手保持持续挑战力的斗争(抗争)政治事件” 。塔罗认为,社会运动的基本特征包括:

\quo{集体挑战。……运动一般通过直接破坏活动,对抗社会精英、当局和其他集团或文化规范,有代表性地发起斗争(抗争)挑战。……集体挑战常常以打断、阻碍他人的活动或导致他人的活动不确定性为标志。

共同目标。……人们在运动结为一体的较根本较平常的原因,还是为了提出他们共同的主张,反对对手、当局和社会精英。

团结和集体认同。……他们只有激起较根深蒂固的团结一致感和身份认同,才能创造一场社会运动。因此,我们几乎可以肯定,作为运动组织的基础,民族主义和种族或宗教总是比社会阶级的绝对律令更可靠,原因就在于它们能促进团结一致和集体认同。

持续的斗争(抗争)政治。……只有通过持续集体行动来反抗对手,斗争事件才演变为社会运动。……持续地开展集体行动以与强大的对手互相作用,这是社会运动与早期的斗争形势不同的标志。\nauthor{……西德尼 · 塔罗:《运动中的力量:社会运动与斗争政治》,吴庆宏译,南京:译林出版社 2005 年版,第 5—9 页。}}

查尔斯 · 蒂利在《社会运动,1768—2004》中则这样定义社会运动:

\quo{本书将社会运动视为斗争(抗争)政治的特殊形式;称之为 “斗争” ,是指社会运动的群体诉求一旦实现,就有可能与他人的利益发生冲突;称之为 “政治” ,是指无论何种类型的政府都会被诉求伸张所牵连——或是作为诉求者,或是作为诉求对象,或是作为诉求目标的同盟,或是作为斗争的监控者。\nauthor{查尔斯 · 蒂利:《社会运动,1768—2004》,胡位钧译,上海:上海人民出版社 2009 年版,第 4 页。}}

查尔斯 · 蒂利认为,社会运动有三个主要的特征。第一个特征是,它是一种不间断和有组织地向目标当局公开提出群体性的诉求声张,表现为运动的形式。这里目标当局很多时候是指政府或政府机构,但有时也可以是指一些大型工商业机构。比如,20 世纪 50、60 年代的美国黑人民权运动,主要是向美国政府当局提出诉求;但如今很多动物保护协会则是针对全球各大皮草公司提出诉求,要求他们不要捕杀动物以获取皮毛;很多反全球化的社会运动则指向沃尔玛、麦当劳这样的跨国公司。

第二个特征是,它有一系列的常备剧目,包括为特定目标组成的专门协会和联盟(包括绿色和平协会、动物爱好者协会、同性权利协会等),经常举行的公开会议,依法示威游行,大型集会,请愿活动,各种各样的声明,用于专题宣传的小册子,等等。当然,很多社会运动是上述多种常备剧目的组合运动。

第三个特征是,社会运动中的参与者协同一致地表现出特定的价值(Worthiness)、统一(Unity)、规模(Numbers),以及参与者和支持者所作的奉献(Commitment),蒂利称之为 WUNC 展示。蒂利认为这里有四个要素,首先是一种特定的价值,代表了一种价值取向,表现出某种特定的认同;其次是具有一定的统一性,比如社会运动中的统一着装就具有强烈的符号意义;再次是需要有一定的规模,几个人、几十个人通常难以成为一场社会运动,通常需要数千乃至数万以上的人群的持续参与;最后是要求参与者和支持者做出奉献或付出,这样一种参与者的主动奉献部分地能够克服集体行动中的搭便车问题。\nauthor{查尔斯 · 蒂利:《社会运动,1768—2004》,胡位钧译,上海:上海人民出版社 2009 年版,第 4—5 页。}

既然社会运动是一种重要的政治社会现象,那么如何解释社会运动呢?这里介绍几种主要的社会运动理论。有人根据马克思对阶级社会中严重的阶段分裂与对抗的观察,发展出了一种怨愤理论,或者叫相对剥夺感理论。就是说,社会运动之所以发生,是社会上有一部分人口受到了严重的不公待遇或剥削。一个社会中有人占有很多,有人占有很少,后者会产生强烈的相对剥夺感。比如,过去几年美国发生的 “占领华尔街” 运动,是因为普通民众成为金融危机的更大受害者。当金融危机到来时,他们的就业机会首先遭到威胁,家庭财富大幅缩水。这一事件的大背景还可以追溯到美国最近 30 年在融入全球经济过程中内部贫富差距的加大。所以,严重的社会不公与怨愤心理可能成为社会运动兴起的重要原因。

上述理论当然是有道理的,但该理论没有很好解释一个问题,即社会运动的领导和组织问题。相对剥夺感固然为社会运动准备了社会条件,但按照曼瑟尔 · 奥尔森的说法,这里仍然无法解决搭便车的问题。列宁在他的革命学说中分析了无产阶级革命中的组织问题,他认为为了解决无产阶级革命的领导和组织问题,必须要有无产阶级政党,也就是无产阶级革命的先锋队组织。后来,列宁的这种理论被发展为社会运动的资源动员理论。换句话说,社会上是否有人不满并不重要,重要的是能不能把不满的人群动员和组织起来。所以,这一观点认为,能否解决资源动员和资源组织问题是理解社会运动的关键。

西方新马克思主义的代表人物安东尼奥 · 葛兰西认为社会运动的核心问题是构造集体认同。后来,这一观点被发展为一种集体认同的理论。比如,在英国苏格兰地区的独立运动中,关键就是苏格兰人集体认同的构造。有人通过各种媒体或口耳相传的方式告诉苏格兰人,作为英国这个少数族群一员必须去争取自己的权利,否则苏格兰这个族群在整个英国社会中就会被压制。这样,如果有人形成了这种苏格兰人的群体认同,他就更有可能积极投入到社会运动中。所以,这里更强调的是心理机制。这是一种重要的视角,研究民族主义的学者安德森把民族视为一个 “想象的共同体” 。基于安德森的观点,与民族主义有关的政治是通过 “想象” 来构建的。

最后一种重要理论是基于新古典经济学的集体行动理论。曼瑟 · 奥尔森在其引用率极高的著作《集体行动的逻辑》中认为,集体行动始终存在一个搭便车的问题。比如,某地的环境污染很严重,有人想发起一个环保主题的社会运动,希望借此改善当地的环境状况。但一个重要的逻辑问题是:如果环境真的改善了,这种收益对所有共同体成员来说是利益均沾的。任何人作为共同体的一个成员,总能获得总人口 N 分之一的收益。但是,这个争取环境改善的成本不是由所有人共同分摊的。在这种情况下,普通公民的理性选择是什么呢?—— “我为什么要管这个事情呢?” 所以,这里就有一个搭便车的问题。奥尔森认为,在集体行动中搭便车是常见现象,克服搭便车问题的关键是能不能提供选择性激励,而一般的社会运动通常难以有效提供选择性激励。所以,后来又有人沿着奥尔森的理论路径,走到了列宁的道路上去,即通过鼓动一小部分积极分子,把他们有效组织起来,就有可能解决搭便车的问题。当然,在新古典经济学的视角看来,这并不能完全解决集体行动的问题。\nauthor{关于社会运动的相关理论,参见西德尼 · 塔罗:《运动中的力量:社会运动与斗争政治》,吴庆宏译,南京:译林出版社 2005 年版,第 13—35 页。}

\tsection{市民社会理论}

与政治参与和社会运动密切相关的是一个社会本身的特质,所以这里还要介绍目前流行的市民社会或公民社会概念。 “市民社会” 的英文是 “civil society” 。 “社会” (society)这个词在英文里有两个含义,它既可以指作为整体的一个社会共同体,比如中国社会或美国社会;又可以指一个社会共同体之下的某个社团,比如很多大学社团和很多专业协会。当我们说公民社会时,首先要注意 “社会” 的这两层含义。

弗里德里希 · 冯 · 哈耶克在《致命的自负》中专门辨析,如何理解 “社会” 这一概念直接影响到人们的政治思考。其中的一个困难在于,不同语言对 “社会” 的理解存在着较大的差异,而且有的语言定义社会时本身就含有明确的政治意涵。哈耶克认为, “社会” 一词本身让人误以为所有人 “存在着对共同目标的一致追求” ,但实际上——

\quo{如我们所知,人类合作超越个人知识界限的必要条件之一,就是这种追求的范围越来越不受共同目标的支配,而是受着抽象行为规则的支配;遵守这些规则,使我们越来越服务于我们素不相识的人们的需求,并发现与我们素不相识的人同样也满足着我们的需求。人类合作范围延伸得越广,这种合作的动机与人们心中关于一个 “社会” 中会发生什么的设想就越不一致。

在这种混乱认识中被忽视的关键差别是,小群体的行为可以受一致同意的目标或其成员意志的引导,而同样作为一个 “社会” 的扩展秩序,它形成了一种协调的结构,却是因为其成员在追求不同的个人目标时,遵守着相同的行为规则。\nauthor{F.A.哈耶克:《致命的自负》,冯克利、胡晋华译,北京:中国社会科学出版社 2000 年版,第 129—130 页。}}

所以,当我们说 “社会” 时,更多的是指一个人类共同体的整体概念还是一个社团的概念,或者说是指一个具有一致目标的群体概念还是一种具有不同目标的人群进行自愿合作的秩序概念,就代表了很大的观念分野。这种关键概念上的语义差异,会导致政治思维方式的很大不同。

既然社会的概念都充满争议,市民社会或公民社会就更是一个充满争议的概念。公民社会大体上说有两个含义:第一个含义是指私人领域,这里既可以指不包括家庭的私人领域,也可以指包括家庭的私人领域。这里的私人领域是与政治社会相对的。所以,这个含义上的公民社会,几乎等同于整个的私人领域。第二个含义是指第三部门,一般是指非营利组织和非政府组织所处的领域,也就是既非市场部门又非政府部门的一个领域。这种含义上的公民社会所具有的特征是,它既不受国家权力的控制,又不受商业利益的支配。上述两种解读,是比较典型的对公民社会的理解。所以,凡是属于国家控制或企业控制的社会组织,不应当被纳入公民社会的范围。

爱德华 · 希尔斯这样定义市民社会的概念及其主要特征:

\quo{市民社会指的是社会中的一个部分,这部分社会具有自身的生命,与国家有明显区别,且大都具有相对于国家的自主性。市民社会存在于家庭、家族与地域的界域之外,但并未达致国家。

市民社会的观念有三个主要要素。其一是由一套经济的、宗教的、知识的、社会的自主性机构组成的,有别于家庭、家族、地域或国家的一部分社会。其二,这一部分社会在它自身与国家之间存在一系列特定关系以及一套独特的机构或制度,得以保障国家与市民社会的分离并维持二者之间的有效联系。其三是一整套广泛传播的文明的抑或市民的风范。\nauthor{爱德华 · 希尔斯:《市民社会的美德》,李强译,载于邓正来、J.C.亚历山大编:《国家与市民社会:一种社会理论的研究路径》,北京:中央编译出版社 2005 年版,第 33 页。}}

尽管学界对公民社会的精确定义存在争议,但很多人认同公民社会存在着若干主要特征,包括:

第一个特征是公民社会具有明显的自主性。公民社会跟私人领域有关,独立于国家权力和国家控制,因而具有充分的自主性。这也意味着,如果大量社会团体是被国家权力控制的,那就不是真正意义上的公民社会。

第二个特征是公民社会具有比较强的组织性。公民社会的一个显著特征是它以各种社会团体的形式呈现出来,特别是经常提及的非营利组织和非政府组织。这些组织既不从属于政府部门,又不从属于企业机构。足球协会、慈善救助会、女童子军、少数族裔文化促进会、动物保护协会等等,就是典型的公民社会组织。

第三个特征是公民社会看上去属于私人领域,但它又有一定的公共性。公民社会的很多社团都会积极介入到公共领域中,它们试图影响政治社会。比如,很多 NGO 或 NPO 的存在,就是要影响政治社会。环保组织很大一部分是为了推动政府立法和政策的改变,它并非纯粹地停留在私人领域。

第四个特征是公民社会中存在着大量的集体行动,而社会运动是集体行动的一种方式。公民社会,其实是社会运动存在的一种有利空间。公民社会则借助集体行动或社会运动更有效地影响到政治社会和公共政策。由于其较高的组织化程度,公民社会往往比个人更能影响政治社会与公共政策。

公民社会之所以重要,是因为很多人关注公民社会与民主或公共治理之间的关系。很明显,公民社会的发达有利于公民自治能力的发展。这意味着公民获得了独立于国家或政府的组织化的参与方式,因而有利于塑造民主政治的社会条件;反之,则不利于塑造民主政治的社会条件。这就是通常所说的,公民社会越发达,就越有利于民主政治;公民社会越不发达,就越不利于民主政治。

按照戴维 · 赫尔德的观点,能否重新构建有效的市民社会,是民主政治得以有效运转的关键。在《民主的模式》一书中,赫尔德认为:

\quo{在今天,民主要想繁荣,就必须被重新看作一个双重的现象:一方面,它牵涉到国家权力的改造;另一方面,它牵涉到市民社会的重新建构。只有认识到一个双重民主化过程的必然性,自治原则才能得以确定:所谓双重民主化即是国家与市民社会互相依赖着进行的转型。\nauthor{戴维 · 赫尔德:《民主的模式》,燕继荣等译,北京:中央编译出版社 2004 年版,第 396 页。}}

亨廷顿在《第三波》中也认为,公民社会的发展,也就是社会的多元化和强大的中介团体的兴起,是很多国家民主化重要而有利的因素。在评价儒教是民主转型的阻碍因素时,他这样说: “儒家社会缺乏反对国家的权利传统。……最为重要的是,儒教把社会和国家合二为一,而且并不认可自治性的社会机构在全国层次上抗衡国家的合法性。” \nauthor{塞缪尔 · 亨廷顿:《第三波:20 世纪后期的民主化浪潮》,欧阳景根译,北京:中国人民大学出版社 2012 年版,第 285 页。实际上,自治性的社会结构是市民社会的主要特征。}

拉里 · 戴蒙德在评述第三波民主化发展的著作《民主的精神》中,高度评价了公民社会对民主转型与巩固的积极作用。他这样说:

\quo{由于公民社会之正式和非正式组织的成长,以及它们的能力、资源、自主性和主动性的增强——所有这些能够极大地改变权力的平衡……曾经轻松地处于主导和控制地位的威权政府被迫处于防守地位。……在世界的很多地方,上述独立组织的能力和数量的提升才是民主真正的本土根源。\nauthor{拉里 · 戴蒙德:《民主的精神》,张大军译,上海:群言出版社 2013 年版,第 115—116 页。}}

市民社会固然会影响政体运转与转型,而另一面是政体类型亦会影响市民社会的培育和发展。总的来说,在民主政体下可能更有可能发展出一种自主的市民社会。在威权政体下,市民社会即便存在,其主要影响应该被控制在非政治领域;如果市民社会——特别是强大的公民社团组织——想要介入政治领域的话,通常会受到压制。否则,公民社会的兴盛就可能会对威权统治的合法性构成挑战。对于极权和后极权社会来说,它们更容易采用的方式是,通过政府、政党或协会等手段来控制原本属于市民社会的组织。这样,在这类社会中,好像存在为数众多的非营利组织和非政府组织,但实际上这些组织在很大程度上都是由政治权力所支配的。

如果说发育健全的市民社会有利于民主的运转和巩固,但非民主政体又往往不利于健全的市民社会的成长,这就使得两者的关系陷入了某种困境。鸡生蛋,还是蛋生鸡?如何打破既有的政治循环?这是一个严肃的问题。

\tsectionnonum{推荐阅读书目}

皮埃尔 · 罗桑瓦龙:《公民的加冕礼:法国普选史》,吕一民译,上海:上海人民出版社 2005 年版。

查尔斯 · 蒂利:《社会运动,1768—2004》,胡位钧译,上海:上海人民出版社 2009 年版。

赵鼎新:《社会与政治运动讲义》,北京:社会科学文献出版社 2006 年版。

于建嵘:《抗争性政治:中国政治社会学基本问题》,北京:人民出版社 2010 年版。

邓正来、J.C.亚历山大编:《国家与市民社会:一种社会理论的研究路径》,北京:中央编译出版社 1998 年版。
