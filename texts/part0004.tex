\tchapter{政治学:智者如何思考?}

\quo[——柏拉图]{到这里我们一致同意:一个安排得非常理想的国家必须妇女共有、儿童共有、全部教育共有。……他们的王则必须是那些被证明为文武双全的最优秀人物。}

\quo[——托马斯·霍布斯]{在没有一个共同权力使大家慑服的时候,人们便处在所谓的战争状态之下。这种战争是每一个人对每个人的战争。}

\quo[——孟德斯鸠]{当立法权和行政权集中在同一个人或同一机关之手,自由便不复存在了;因为人们将要害怕这个国王或议会制定暴虐的法律,并暴虐地执行这些法律。}

\quo[——亚历山大·汉密尔顿]{但是政府本身若不是对人性的最大耻辱,又是什么呢?如果人都是天使,就不需要任何政府了。如果是天使统治人,就不需要对政府有任何外来的或内在的控制了。在组织一个人统治人的政府时,最大困难在于必须首先使政府能够管理被统治者,然后再使政府管理自身。毫无疑问,依靠人民是对政府的主要控制;但是经验教导人们,必须有辅助性的预防措施。}

<h2 class="kindle-cn-heading2" id="sigil_toc_id_8" title="2.1 岛屿的寓言:谁之统治?何种秩序?">2.1 岛屿的寓言:谁之统治?何种秩序?\nauthor{这一节曾刊载于《东方早报》2013年9月24日。}</h2>

这一讲从一个小小的实验开始。假定我们正在大学教室里上课,课堂上共有120位学生和作为大学教师的我,突然,教室里的所有人被一股神秘的力量弃置在太平洋深处的一个岛屿之上。这个岛屿无人居住,有数十平方公里大小,岛上有淡水、森林、花草、鸟类和小兽。此外,我们永远都跟外界失去了通讯和交通联络,既不会有国际救援组织来搜寻我们,也不再能够使用手机或者互联网。我们唯一的可能是靠自己的力量在岛上生存下来。

如果发生这种情况,那么接下来,你认为我们首要的问题是什么?大家如何思考这个问题?从经验来看,可能会有几种不同的代表性观点。不妨把几种可能的意见记录在下面。

有人认为:

\quo{首要的是解决如何分配资源的问题。一下子穿越到一个无人小岛之上,大家首先面对的就是生存的渴望和对未知的恐惧。这种情况下,出于人性的本能,大家可能会为了争夺生存资源而发生内斗。一旦发生内斗,那么我们在这个岛屿上生存的几率就更低。所以,首先要解决的问题是建立一种合理的资源分配机制。}

有人说:

\quo{我们首先应该选出一个领袖。因为一群人在这个岛上,有些人会想先去寻找食物,有些人会想先去寻找水源,可能还会出现突发的紧急状况,如果众人各有主张的话,我们可能会陷入危机当中。所以,应该有一个领袖来统领大家,他能代表众人来做决定。}

有人则强调:

\quo{最重要的应该是分工和产权,明确分工和界定产权是关键。资源分配固然很重要,但我们现在应该还没有什么资源。我认为,每个人都应该干什么才是最重要的,需要建立一个分工的制度。另外,怎样界定产权也是一个重要问题,是先见先得的私人产权制度,还是大家共有的产权制度安排?这个也非常重要。}

还有人并不同意上述观点,他这样说:

\quo{首先要解决的问题是建立一个合理的生存秩序。处在这样一个岛屿之上,我们从长远来说需要解决生产和繁衍的问题,从近期来说需要解决每个人的安全问题。如果我们没有建立起一个合理的秩序,那么无论是安全,还是生产或繁衍都是不能实现的。}

上面几种主张各不相同。无疑,这些问题都很重要。但是,到底什么是首要的问题呢?

有人会想,突然穿越到太平洋的一个海岛上其实是一件很惬意的事情。空气清新,风景如画,白色的沙滩加蔚蓝的大海,听上去似乎是一个不错的选择。但是,问题接踵而至,最大的问题可能是没有政治秩序了。固然,上面几位同学的看法都很重要,但这些问题都跟一个更根本的问题有关,那就是构建政治秩序的问题。

无论是资源分配也好,还是领袖也好,或是分工和产权也好,当然还有更接近的表述即生存秩序的问题——如果没有一个基本的政治秩序,其他问题都无法解决好。更现实地说,如果没有一个基本的政治秩序,上面提到的内斗与暴力的问题,甚至是互相残杀问题,都有可能会出现。按照英国政治哲学家霍布斯的说法,就可能会陷入每一个人与每一个人的战争状态。所以,首要的问题是我们应该构建一种怎样的政治秩序。

那么,在这样一个岛屿上,在原先的政治秩序消失之后,我们应该构建一种怎样的政治秩序呢?任何政治秩序首先要解决统治的问题,统治方式的差异也决定着政治秩序的差异。那么,可能的统治方式是什么?如何决定由谁来统治呢?

有人提出来,要不要遵循传统来统治呢?换句话说,过去怎样,我们现在也怎样。在这个教室里,讲台上的老师是较有权威的人。现在突然穿越到一个岛屿上,有人提出来,我们要不要基于这样的传统来进行统治呢?过去,讲台上的老师给我们讲课;到了这个岛屿上,还是让老师说了算行不行?当然,有人可能会提出质疑,老师可能做学问还不错,讲课还可以,但统治岛屿和领导众人到底行不行?但同时,还有人考虑得比较现实。他们认为,尽管讲台上的老师未必是一个完美的统治者,却是现实当中一个“可得的”统治者。因为如果换一个人的话,很多人可能会不服,可能会提出巨大的异议,可能引发严重的冲突,这样我们在岛上生存的几率就会降低。基于传统进行统治,就是遵循“历来如此”的惯例:我们过去怎么统治,现在还怎么统治。

先假定我们遵循传统进行统治,讲台上的老师成了统治者,接下来大家最关心的是什么呢?估计大家不再会关心我是否有学术能力和是否善于授课,大家最关心的应该是:我是不是一个公正的统治者?与智慧、学识、才干、精力相比,一个统治者是否公正可能更为重要。

那么,我是不是一个公正的统治者呢?我能不能建立一种公正的秩序呢?这种公正的政治秩序应该让每个人都得到合理的份额,让众人各安其位、各尽其责。在此基础上,我能不能建立起一套有效的为众人服务的行政机构?还有,我能不能做到知人善任和用人所长呢?比如说,能不能请强健有力的人来负责治安?能不能请富有管理才干的人来负责行政?能不能让为人公正且懂得规则的人来负责司法?能不能请学识渊博的人来负责下一代的教育?能不能请技艺高超的人修建房屋和船只?这些,可能都是大家非常关心的问题。

也许一开始我干得还不错,我要求自己努力成为一个公正而有为的统治者,做到了众人所期望的一切。那样,尽管在这个岛上并不富有,但我们能拥有基本的安全和秩序,有必需的食物可以充饥,有结实的木屋可以御寒,众人能够安居乐业。如果这样的话,大家会满意现状吗?大家会认同讲台上的老师作为统治者的所作所为吗?

我想,应该有不少人会表示赞同。但是,即便我作为统治者做到了刚才所说的一切,大家会发现仍然有不少的问题。比如,也许我在短期内(可能两三年)做得很好,那是因为存在迫在眉睫的共同危险。我们首先必须要建立一个政治秩序,必须要形成一个共同体,才能在这个新的岛屿上生存下去。但接下来的问题是,我会不会一直是一个公正而有为的统治者呢?

我是一个普通人,我身上具有所有人与生俱来的弱点。成为统治者以后,我完全可能变得贪图享乐和不知节制,我甚至变得贪婪、骄横和暴虐。当我建立了护卫机构(也就是军队警察部门)和官僚机构以后,就拥有了强大的统治力量,拥有了压制反对意见和反对派的强制力。这样,我可能开始不太在乎众人的意见了,不再去认真倾听大家的诉求了,过去的约束机制也不再起作用。久而久之,我的私欲会进一步地膨胀,我甚至喜欢所有人都来讨好我。这是完全有可能发生的事情。如果一个统治者发生了这样的蜕变,我就偏离了一个公正而有为的统治者的标准。结果是,这个岛屿上已经创造出来的政治秩序就会逐渐败坏掉,整个治理就会变得越来越糟糕,众人开始发出抱怨、甚至是反抗的呼声。

当然,可能还存在第二种更好的情形。穿越到这个岛屿上以后,我一直是一位非常自制的统治者。经过两三年时间,我带领大家实现了刚才所设想的良好秩序和有效治理。此后,我还时刻提醒自己要成为一个卓越的统治者,要努力做到深谋远虑、公正守法和自我克制。如果一个统治者一直能这样做,岛屿更有可能实现长治久安。但即便如此,统治的问题仍然没有从根本上解决。

任何一个统治者都会死,统治者死了又该怎么办呢?比如,我作为统治者有一天死了,那么这个岛屿又该怎么办呢?我们仍然面临由谁来统治的问题。有人会说,你不是有儿子或女儿吗?是不是让你自己的后代来直接继承统治者的位子呢?还有人会说,要不要请老师来指定下一任的统治者?在单个统治者可以说了算的政治秩序里,这些都是可能的选项。但是,总的来说,当一个优秀、开明、自制、公正、有为的统治者去世以后,问题总会反复地出现,正如我们在历史上所看到的那样。

所以,大家会发现,基于传统的统治方式总会存在问题,没有人能保证一个政治共同体一直能拥有一位公正而有为的统治者。

这时,也许有人在想,要不要试试另一种办法?如果有一天统治者年岁已高,丧失了履行统治的能力,或者干脆死了,有人提出来我们要不要尝试根据第二种原则来选择我们的统治者?这种原则要求选择那些具有卡里斯玛特质的人来担任统治者——这里的卡里斯玛是德国思想家马克斯·韦伯的概念,即所谓超凡魅力型的统治者。

我们教室里也许有一位同学恰好符合卡里斯玛型人物的特质。这位同学不仅学识渊博、才华横溢、器宇轩昂,而且具有令人愉悦的个性。这位同学不仅本身出类拔萃,而且还具有鼓舞人心的力量。这样一个人就具有了令众人折服的魅力,恐怕可以被称为岛屿上最具魅力、众望所归的人物。既然原先的统治者会老去,有人提议就不妨根据卡里斯玛原则来选定一位超凡魅力型的领袖。这种主张应该会有相当的市场,既然我们必须被统治,那么为什么不找一个具有超凡魅力、众望所归的人来领导我们呢?这种见解无疑是有理的。

但问题是,一个魅力型统治者会不会更好地统治这个岛屿呢?我们可以设想一下,假定教室里的某一位同学符合这样的要求,属于具有超凡魅力的人物,这个教室里的多数人倾向于让他来做我们的统治者。但是,当他从一个众人意向中的统治者变成一个实际的统治者时,随着他身份的变化,他的心态和行为都可能会发生巨大的变化。无法回避的问题仍然是:成为统治者之后,这位具有超凡魅力的人物会不会变得贪婪?会不会破坏已有的公正规则?会不会丧失原本具有的美好品格?再进一步说,他会不会有一天变成一个暴君?大家会发现这是完全可能发生的事情。如果是这样,由于这位统治者是一位超凡魅力人物,所以他对岛屿造成的危害可能还比普通统治者为大。

很多历史就是这样重复的。一开始,出现了一个众望所归的超凡魅力型领袖,众人对他抱有很高的期待;然后,在这位卡里斯玛型领袖统治五或十年后,大家发现他变得越来越糟糕了,甚至完全沦为一个暴虐的统治者。这样,众人不再能够心悦诚服地找到“服从的道理”;相反,很多人甚至开始寻找“反抗的缘由”。此时,岛上肯定会流传很多私下的或公开的对统治者表示不满的传闻,关于这位统治者胡作非为的半真半假的“谣言”也会遍布整个岛屿。实际上,到这种地步,这位卡里斯玛型统治者的合法性已经削弱,甚至完全丧失了。

就在这个时候,岛上的护卫队在政治上变得越来越重要。如果我们是一百余人的共同体(暂不考虑人口繁衍因素),有一个五到六人的护卫队就可以了。护卫队成员是体格比较强健的人,而其中最强者是护卫队的队长。比较碰巧的是,这位队长不仅身强体壮,而且还有相当的领导力。听到民怨沸腾的呼声之后,他觉得这个岛屿再也不能让这样一位统治者继续统治下去了。

于是,在某一个月黑风高的夜晚,他说服护卫队的其他五位成员,发动了一场政变。他们把这位原本具有超凡魅力的统治者囚禁起来了,或者用一条独木舟把他流放出去,或者就直接处死了。然后,这位护卫队长还发表了一个简短的政治声明。政治声明这样说:

\quo{尽管我们的岛屿过去在这位统治者的领导下,曾经有过令人倾慕的生活,但随着时间的流逝,他变得越来越贪婪、无能和败坏。如今,他实际上已经沦为一个暴君。我们现在再也不能忍受他的统治了,所以我们必须要用正义的力量推翻他。我们推翻他没有别的目的,目的就是要重建本岛的公正与秩序!}

这个消息传出去后,在很短的时间内,很多人跑到岛屿的广场上去庆祝和欢呼,因为一种忍无可忍的统治终于结束了,一个暴虐的统治者终于被推翻了。但问题是,新的统治会比旧的统治更好吗?新的统治者会比旧的统治者更优秀吗?大家不要忘记,新建立的政治秩序的基础是什么?暴力。实际上,政治游戏的规则已经变成了“谁有力谁统治”。在一个一百余人的共同体中,如果有一个六人或八人的护卫队,他们本身体格强健,又能组织在一起。这样的话,我们其他人就很难去挑战他们。所以,这位过去的护卫队长靠着这个武力组织,就能建立起对这个岛屿的统治。

那么,大家对这种统治抱有多少期待呢?这位护卫队长既可能会成为一个优秀仁慈的独裁者,又有可能成为一个腐败骄横的独裁者。一句话,他既可能成为明君,亦可能成为暴君。但是,即便他是一个开明君主,仍然不能解决他死之后的统治问题。

此外,这种直接基于强力的统治还有一个问题。这位护卫队长自己成为统治者之后,他还要任命一位新的护卫队长。由于过去这场政变,护卫队长成了一个政治上极重要的职位,因为他直接控制着暴力工具。由于这位统治者的职位本身就是在护卫队长这个职位上通过政变而获得的,所以,他会时刻提防新的护卫队长是否会发动政变。为了稳妥起见,他甚至可能设立两个护卫队:左护卫队和右护卫队,使他们保持权力的均衡。那么,这种均衡会很稳定吗?这个就很难讲了。如果出现更强有力的护卫队长,他就完全可能通过政变取代现有的统治者。另外,现有的统治者终有一天会年老体衰,这也会加大下一次政变的风险。

所以,美国政治思想家汉密尔顿在《联邦党人文集》的第一篇就提出了一个严肃的问题:“我们人类社会是否真正能够通过深思熟虑和自由选择来建立一个良好的政府,还是他们永远注定要靠机遇和强力决定他们的政治组织。”这里的机遇就是运气,强力就是暴力。这个问题是汉密尔顿在1787年致纽约州人民的信中提出来的。这个问题还可以拿来问教室里的每一个人,所有人都应该去思考这个问题。

我们再回到岛屿之上。如果上面讲到的政治情形不断重复,也许有一些人会开始更深入地思考政治问题。其中一些人也许活得比较长寿,他们见多识广,过去又受过政治学或法学训练,这样他们的思考也许更有价值。随着时间的流逝,其中一位长者开始觉得对这个岛屿上的很多事情大彻大悟了,所以,他就开始在一些私下的场合不停地唠叨一个道理。他想要说的道理是什么呢?这位长者这样说:

\quo{我们岛屿的统治再也不能沿袭过去的做法了,无论基于传统、基于卡里斯玛还是基于暴力,政治上都是不可靠的。这几种统治方式都无法带来真正的善治和实现长治久安。很多人对我们这里过去发生的事情都是亲眼所见,比如,统治者开始可能是好的,但后来就变坏了;上一任统治者可能是好的,但下一任统治者就变坏了;更不用说统治者与统治者权力交替时发生的惨不忍睹的一幕幕悲剧了。所以,我们再也不能这样过了。}

当这样一位长者不断地跟岛上的众人去阐发这些道理的时候,有人可能会来向他继续请教:“先生,您说我们应该怎么办呢?照您的说法,我们怎样才能在岛上构建优良而长久的政治秩序呢?”这位长者说出了下面这样一番话:

\quo{我们岛屿的统治应该要基于法理。惟有法理型统治,才是真正的长久之道。那么,什么是法理型统治呢?可能岛上的很多年轻人没有听过这个说法。事情还要从很久以前说起。

老朽来到这个岛屿之前——大概是几十年前了,跟大家曾经生活在一片大陆上。在那片大陆上,存在着不同的统治方式和政治秩序,它们有的非常优良,有的并不理想,有的则糟糕之极。但是,根据老朽的观察,凡是最优良的政治秩序都是法理型统治。观察了本岛过去数十年政治上的起起落落和治乱兴衰,老朽更确定无疑地以为,只有法理型统治才能真正改善本岛的政治状况,才会有利于本岛长久的福祉。

你们现在肯定很关心什么是法理型统治。简单地说,法理型统治要遵循三个原则:

第一个原则是,不管谁统治,首先要建立基本的规则。这个基本规则决定了无论是谁掌握政治权力,他只能按确定的规则来统治,统治者不能随心所欲、为所欲为。我们应该把一个确定的规则放在谁来统治这个问题之前。就像过去英格兰的著名法学家柯克爵士对当时的英格兰国王所言:国王啊国王,您尽管在万人之上,但仍然在上帝与法律之下。所以,基本的法要优先于统治本身,基本的法要优先于统治者本身。这种传统过去被称为“立宪主义”。我们首先要明确基本规则,明确统治者和统治机构能够做什么和不能做什么,然后再来讨论谁来统治和具体如何统治的问题。

第二个原则是,所有岛上公民的基本权利要有切实的保障。无论谁做统治者,他都不能破坏岛上公民的基本权利,其中最重要的是生命权、财产权与自由权。统治者的暴虐与对公民权利的侵害,往往是一个硬币的两面。当公民权利得到确定无疑的保护时,统治者也不应该和不能够为所欲为了。这样,统治者也更有可能成为优秀的统治者,岛上公民们才能获得一种有保障的生活。

第三个原则是,要通过一种和平的、寻求岛上多数公民支持的方式来选择统治者。当然,这意味着一种投票制度,但投票制度的具体形式则存在多种选择。一种办法是直接选举,所有岛上的成年公民——比如,18岁或20岁以上——可以在一个预先确定的时间,到广场上来投票决定未来两年中谁将成为我们的统治者,这是一个办法。但也有人会对这种选举方式表示担忧,众人真的能选出对本岛有益的领袖吗?他们的主要顾虑是,这里的好多年轻人没有政治经验,可能缺少政治判断力。因此,让所有18岁以上的公民来直接选举统治者,未必是一个审慎稳妥的做法。那么,怎么办呢?我们还可以先让18岁以上的公民选出一个议事机构——比如15位或25位公民代表,再由这个议事机构来决定谁将成为我们岛屿的统治者。我估计,大家会选择更有能力和智慧的年长公民来组成议事机构,这样可能会形成更审慎的决定。

当然,法理型统治的具体安排要比这三条原则复杂得多,但老朽以为这三条原则是最重要的。实际上,过去数十年中我们岛屿已经尝试多种统治方式和政治秩序,结果都不能令人满意。所以,我们现在的出路只有一条,那就是法理型统治。惟其如此,本岛才能长治久安。}

如果你是岛上一位公民,听到长者这一番语重心长的话,再去思考本岛过去政治上的纷扰,你会不会赞同这位长者的见解呢?用他所倡导的法理型统治原则来构建我们岛屿的政治秩序,是否更合理呢?

这位长者的见解在岛上激起了很多秘密的讨论,尤其是那些富有经验的年长公民对此讨论更为热烈。一部分人说,在尝试了诸种并不理想——最后往往很糟糕——的统治方式之后,法理型统治是本岛惟一的出路。若不能实行法理型统治,本岛将继续在一批优秀的统治者与一批糟糕的统治者之间来回摇摆,将无法摆脱这种治乱交替的命运。

但另一部分人则认为,这种法理型统治的政治秩序固然是“可欲的”,但是它也是“可得的”吗?住在岛屿另一头的另一位年长智者就持有这种见解。他这样说:

\quo{听到上面那位老先生的见解,我感到有些忧虑。我已经同他争论好多年了,你们众人还是不要轻信了他的看法。在我们曾经生活的那片大陆上,出现过很多类似的情形。有的国家就采用刚才那位老先生倡议的方式来构建政治秩序。固然,有的国家实施得非常之好,但有的国家实施下来却是一场政治悲剧。照我看来,一个地方的政治如何,完全不取决于他们所实施的制度,而取决于实施这些制度的人。他们有什么样的人,比他们有什么样的制度,往往更重要。

在那片大陆上——如果我没有记错的话——有不少国家实行这种法理型统治的政治秩序后,并没有得到他们想要的结果。如果你让公众投票来决定谁统治他们,结果是该国的精英阶层马上四分五裂了,形成大量的派系。富人有富人的派系,穷人有穷人的派系;有神论者有有神论者的派系,无神论者有无神论者的派系;甚至东南西北的人群还各有各的东南西北的派系。结果是,一开始各个派系之间只是互相竞争,到后来有的就变成互相恶斗了。在不少国家,投票活动常常都演变为暴力角逐。极常见的情形是,那得势的一方往往想方设法压制失势的一方,而那失势的一方总喜欢制造混乱的局面,使那得势的一方也难以统治。那样的话,不要说好的治理,就是连和平与秩序都难有保障了。

所以,要以我的经验来看,统治的问题,不是你想要怎样设计就能怎样设计。实际上,现有的政治就是我们这些年来自己造成的。不是我过于悲观,我只是一个务实的人。要我说,我们大概只能在现有的状态里生活。你们若问我如何变得更好,我直接的想法是我们可能很难有办法变得更好。我知道,你们未必同意我的见解,特别是那些憧憬未来的年轻人。但是,你们也要知道,人本身就有缺陷,所以人类社会怎么可能没有缺陷呢?}

岛屿的故事讲到这里,大家又怎么看待这个问题呢?我们的讨论没有标准答案,政治不是数学,很多时候并不存在惟一的最优解。我只希望,诸位都能有自己的判断。

两位年长智者关心的是同一个问题:一个社会如何构建合理的政治秩序?然而,前者更多关注“什么是可欲的政治秩序”,后者更多关注“什么是可得的政治秩序”。用更学理的方式来说,第一个问题是我们在岛上应该构建何种政治秩序?第二个问题是我们在岛上能够构建何种政治秩序?前者是“应然”的问题,关注应该怎样;后者是“实然”的问题,关注事实怎样。诸位现在应该很清楚地知道,这两种思考问题的路径差异很大。

的确,当思考我们应该拥有何种政治秩序的时候,还必须考虑我们能够拥有何种政治秩序。如果回到经验世界,大家还会发现,一种政治秩序的构建较少取决于智者的思考,较多取决于政治参与者的行动。不是政治哲学原理决定了一个国家的政治秩序是怎样的,而是主要政治集团的观念、行为、选择以及互相之间的政治博弈决定了一个国家的政治秩序是怎样的。这正是现实政治的冷峻之处!

如果想作进一步的讨论,我们还可以超越一个岛屿的政治秩序问题,转而来思考诸岛竞争的问题。假定今天不只是我们这样一个班级和教室,而是我们有五个类似的班级和教室,每个教室里都有一位老师和120位学生,同样的规模,类似的男女比例,大家的智商和知识程度相当。此时,我们突然被一股神秘的力量同样弃置在太平洋深处五个规模与资源相当的不同岛屿之上。然后,经过50年、80年、100年的时间,那个时候的造船技术或别的技术也许发达一些。我们五个不同的岛屿能互相接触、彼此发现对方的时候,就存在一个诸岛竞争的问题。

那么,是什么因素决定了50年、80年、100年以后,我们的岛屿可能会比别的岛屿更富有和发达一些呢?我们当年的人口条件是相当的,岛屿的资源条件也是相当的。那么,是什么因素决定了诸岛发展程度的差异呢?

倘若我们的岛屿是其中最发达的一个,如果开放签证的话,其他岛屿的公民可能很乐意来我们的岛屿经商或者工作,乐意让他们的子女来我们这里接受教育,而我们会给他们当中条件较优秀的人发放绿卡,或者长期居留证。如果是这种情形,我们必须要问:我们是凭什么胜出的呢?因为有的岛屿可能跟我们完全不同。最糟糕岛屿的公民们发现还有别的岛屿之后,甚至还会发生大规模的逃亡。为什么诸岛之间会形成如此巨大的差异?如果深入探究,大家应该能发现,这种差异的根源在于政治秩序的不同。到那个时候,作为学者的我如果还活着的话,也许还能写出《岛富岛穷》或《为什么有的岛屿会失败?》这样的学术畅销书。

总的来说,政治秩序的好坏,直接关系到一个岛屿的福祉。每个人的生活好坏,每个人的职业成就高低,每个人的才干知识是否有用武之地,每个人是否具有努力工作的动力,每个人是否珍惜自己的德行与名声,每个家庭是否更安稳和幸福,整个社会是否拥有和平与安定,所有这些方面都跟岛屿的政治秩序有关。政治秩序的优劣,除了关系到一个岛屿的福祉,还关系到诸岛之间的竞争。一种优良的政治秩序,更能使一个社会产生发达的文明和强大的竞争力,从而使得这个岛屿不仅不会落后,反而还会遥遥领先,成为诸岛竞争中的胜出者,成为人类文明的领导者。

上面讲的故事尽管只是一个政治寓言,但这大概跟人类社会过去上千年走过的道路是有关系的。如何构建合理的政治秩序,是人类政治生活的基本问题。政治学思想与研究的演进,很大程度跟这个问题有关。

\tsection{什么是政治学?}

人类为了生存而过着群居的生活,有效的群居生活需要一定的社会组织和权威机构。刚才关于岛屿的情境研讨中,有人担心人性中恶的一面会泛滥,那怎么办呢?这就需要一个具有权威的机构。自古以来,就有少数人开始基于理性思考人类社会的政治权力、政治制度和政治秩序问题。在中国,春秋战国时期的诸子——特别是孔子、韩非子和老子——是其中最出色的思考者。在西方,古希腊的苏格拉底、柏拉图和亚里士多德等人则是最早思考这些问题的人。

最早的政治学通常被认为是哲学或历史学的分支,其主要的目的是发现人类社会构建政治秩序的基本原则。比如,柏拉图和亚里士多德的政治著作探究的主要问题就是:一个人类共同体应该立足于何种原则构建何种政治秩序?在古希腊的传统中,这是早期政治思考的主要问题。

一般认为,政治学作为一个学科经历了三种不同的传统,分别是哲学传统、经验传统与科学传统,这大概勾勒出了政治学的发展脉络。\nauthor{关于这三种传统更详细的介绍,参见海伍德:《政治学》(第二版),第15—18页。}第一种是哲学传统。哲学传统重视规范研究,更多地进行哲学思辨式的探索,关注应该怎样的问题。比如,柏拉图和亚里士多德都问过类似的问题:什么是最优良的政体?什么是善的城邦?什么是善的社会?应该由谁来统治呢?这些问题一般都是规范性思考。他们更多地以哲学思辨方式来研究应该怎样,其核心仍然是人类社会应该构建何种政治秩序。

第二种是经验传统。经验传统关心的不是应该怎样,而是现实世界中的不同国家、不同城邦、不同政体的政治实际上是怎样的。这种研究以经验事实为基础,关注的“是什么”,而非“应该是什么”。亚里士多德是政治学经验研究的开创者,他的《政治学》和《雅典政制》两本书关心的是当时希腊诸城邦不同政治体制实际上是怎样的。当然,在亚里士多德的著作中,还可以找到很多与应该怎样有关的内容。

后来,又兴起了第三种传统即科学传统。应该说,科学传统并不是独立于经验传统的路径,而是在经验研究中更多地采用科学方法。科学革命以来,牛顿和达尔文等自然科学先驱也为社会科学的发展开辟了道路。在科学传统中,学者们更关心对因果关系的探究,即何种原因导致何种结果,通常还包括对因果机制与过程的解释。好的社会科学研究基本上都是从问为什么开始的,关注的是为什么某种特定的原因会导致某种特定的结果。关于政治科学的研究方法,最后一讲会专门介绍。

从时间上说,哲学传统出现最早。到今天为止,政治学研究的哲学传统仍然保留着。经验传统起源于亚里士多德,比较鼎盛的时期大约是19世纪。20世纪之后,政治学研究更多地走向科学传统,政治学研究更加专业化,政治科学成为整个政治学研究的主流。如果今天去看美国排名最靠前的政治学学术期刊——比如《美国政治科学评论》(\italic{American Political Science Review})、《美国政治科学杂志》(\italic{American Journal of Political Science})等,大家会发现很多论文的数理化程度非常高,里面有大量的模型、数据和定量分析等。这也佐证了政治科学是目前政治学研究的主流。

那么,今天的政治学研究包括哪些主要领域呢?按照欧美一流大学政治学的学科体系,政治学通常包括四个主要领域:第一个领域是政治哲学,有时又被称为政治理论——特别是在英国。政治哲学关心的仍然是自西方的柏拉图和亚里士多德以及中国的孔子和韩非以来经久不衰的重要问题,包括:应该建立何种政治秩序?什么是最优良的政体?如何看待自由?自由和权威是何种关系?如何看待民主?个人和群体应该是何种关系?什么是正义?政治哲学研究经常借助思想史的方法,即探究政治思想史上不同重要人物对这些问题的观点,并对这些观点进行梳理和比较。思想史之所以重要,是因为今天大家所思考的问题通常都是过去的杰出头脑反复思考过的。当然,政治哲学研究还有其他方法,比如罗尔斯的《正义论》是一部从政治哲学视角探讨正义问题的杰出著作。

第二个领域是比较政治学,这是目前政治学研究最重要的领域。比较政治学是对不同国家和地区的政体制度、政治运行、公共治理与治理绩效的比较研究。最近二三十年的比较政治学非常重视与政体有关的研究。除了政体,比较政治学还关心很多其他的重要议题。比如,在20世纪60、70年代,学界就关心为什么有的国家政治稳定而有的国家政治不太稳定——这是关于政治稳定的研究。目前,在欧美的一流大学,比较政治是一个规模较大的研究领域。

第三个领域是本国政治。在美国,本国政治就是美国政治;在中国,本国政治就是中国政治。有人认为,本国政治研究是比较政治学的一部分。本国政治研究,不过是用比较政治学的理论与方法来研究自己的国家而已。那么,为什么本国政治通常又被归入一个独立的研究领域呢?主要是两个原因:一是任何政治学研究都存在“以我为主”的问题,所以本国政治研究通常会跟国别比较研究区别开来;二是本国政治研究一般会做得比较精细,不仅论文和著作的产量非常大,而且通常会比跨国研究更为专门和深入。

第四个领域是国际政治,又称国际关系。这是很多人都会关心的领域。对中国来说,中美关系、中欧关系、中俄关系以及中国与东亚、东南亚邻国的关系都是极重要的国际关系,也是国际政治研究的内容。当然,国际政治研究不只是关心本国与他国的国际关系,同时还研究别的国家和地区之间的国际关系。比如,利比亚发生内战以后,为什么法国会首先卷入,这也是国际政治关心的问题。

在欧美一流大学中,政治学一般被划分为上述四个领域:政治哲学、比较政治、本国政治和国际政治。当然,有人认为,政治经济学研究日益成为政治学中成长较快的专门领域之一。这里的政治经济学不是大家过去熟悉的马列主义政治经济学,而是一个政治和经济交叉研究的领域。在国内,朱天飚所著的《比较政治经济学》一书对此有系统介绍。\nauthor{朱天飚:《比较政治经济学》,北京:北京大学出版社2006年版。}此外,还有人把政治科学的研究方法作为一个专门的领域去研究。这个问题也非常重要。

\tsection{古希腊与古罗马的传统}

从这一节开始,本讲将会对西方从古希腊至今的政治学脉络作一概要介绍。西方的政治学传统可以追溯至古希腊。两位古希腊历史学家希罗多德和修昔底德就讨论过政治学的基本问题,被视为西方政治学的重要源头。上文业已提及,政治学最早被视为历史或哲学的分支。希罗多德所写的古典历史名著《历史》,主要记述的是古希腊与波斯帝国的战争。对于希罗多德《历史》一书是否可信,国际历史学界充满争议,暂且不论。但希罗多德在这部书中阐述了很多政治观点。特别是,他记录了波斯帝国内乱以后几位主要政治人物对于政体问题的讨论,这是有文字记载的政体类型学的最早起源。事件的背景是公元前6世纪波斯帝国发生内乱,内战结束后几个主要政治人物围绕建立何种统治秩序展开了争论,其中的代表人物是欧塔涅斯、美伽比佐斯和大流士。

欧塔涅斯是民主制的支持者、独裁制的反对者。他认为独裁统治带来了种种坏处,主张让全体波斯人参与国家的统治和管理。他这样说:

\quo{我以为我们必须停止使一个人进行独裁的统治,因为这既不是一件快活事,又不是一件好事。……当一个人愿意怎样做便怎么做而对自己所做的事情又可以毫不负责任的时候,那末这种独裁的统治又有什么好处呢?把这种权力给世界上最优秀的人,他也会脱离他正常的心情的。……不过,相反的,人民的统治的优点首先在于他的最美好的名声,那就是,在法律面前人人平等。其次,那样便不会产生一个国王所容易犯的任何错误。一切职位由抽签决定,任职的人对他们任上所做的一切事情负责,而一切意见均交由人民大众加以裁决。因此我的意见是,我们废掉独裁政治并增加人民的权力,因为一切事情是必须取决于公众的。\nauthor{希罗多德:《历史》(上册),王以筹译,北京:商务印书馆2005年版,第231—232页。}}

美伽比佐斯并不看好民主制,他是寡头制的拥护者。他既不信任独裁的君主,亦不信任普通大众。他这样说:

\quo{我同意欧塔涅斯所说的全部反对一个人的统治的意见。但是,当他主张要你把权力给予民众的时候,他的见解便不是最好的见解了。没有比不好对付的群众更愚蠢和横暴无礼的了。把我们自己从一个暴君的横暴无礼的统治之下拯救出来,却又用它来换取那肆无忌惮的人民大众的专擅,那是不能容忍的事情。……只有希望波斯会变坏的人才拥护民治;还是让我们选一批最优秀的人物,把政权交给他们罢。\nauthor{希罗多德:《历史》(上册),王以筹译,北京:商务印书馆2005年版,第232—233页。}}

大流士是后来波斯帝国历史上最重要的人物之一。在当时,大流士认为应该推行独裁统治,理由在于无论民主制还是寡头制都会带来不好的结果,而等到这种结果显现的时候都要靠独裁制来解决问题。他这样说:

\quo{现在选择既然在这三者之间,而这三者即民治、寡头之治和独裁之治之中的每一种既然又都指着它最好的一面而言,则我的意见,是认为独裁之治要比其他两种都好得多。没有什么能够比一个最优秀的人的统治更好了。……若实施寡头制,则许多人虽然愿意给国家做好事,但这种愿望却常常在他们之间产生剧烈的敌对情绪,因为每个人都想在所有的人当中为首领,都想使自己的意见占上风,这结果便引起了激烈的倾轧,互相之间的倾轧产生了派系,派系产生流血事件,而流血事件的结果仍是独裁之治;因此可以看出,这种统治方式乃是最好的统治方式。再者,民众的统治必定会产生恶意,而当在公共的事务中产生恶意的时候,坏人便不会因为敌对而分裂,而是因巩固的友谊而团结起来;因为那些对大众做坏事的人是会狼狈为奸地行动的。这种情况会继续下去,直到某个人为民众的利益起来进行斗争并制止了这样的坏事。于是,他便成了人民崇拜的偶像,而既然成了人民崇拜的偶像,也便成了他们的独裁的君主;在这样的情况下也可以证明独裁之治是最好的统治办法。\nauthor{希罗多德:《历史》(上册),第233页。}}

希罗多德这里记述了关于民主政体、寡头政体和独裁政体的讨论。争论的焦点是:究竟何种政体最好?何种政体最适合波斯帝国?实际上,这是政治学的基本问题。这里首次出现的政体类型学讨论,可以被视为后来亚里士多德发展更完善的政体类型学的基础。当然,这一争论到今天都还没有完全结束。

修昔底德所著的《伯罗奔尼撒战争史》则是对古希腊内部发生于公元前431—前404年的两大城邦联盟——以雅典为首的提洛同盟和以斯巴达为首的伯罗奔尼撒同盟——之间战争的记述。这部书的主题固然是城邦间战争与政治关系(今天被理解为国际关系),但是该书有较大篇幅记述了古希腊——特别是雅典城邦——的很多政治细节。所以,这部书是今天研究古代雅典民主制的重要史料。其中,雅典政治家伯里克利在将士葬礼上的演说,则是论述雅典民主的名篇。伯里克利这样说:

\quo{我们的宪法没有照搬任何毗邻城邦的法律,相反地,我们的宪法却成为其他城邦模仿的范例。我们的制度之所以被称为民主制,是因为城邦是由大多数人而不是由极少数人加以管理的。我们看到法律在解决私人争端的时候,为所有的人提供了平等的公正。在公共生活中,优先承担公职考虑的是一个人的才能,而不是他的社会地位,他属于哪个阶级;任何人,只要他对城邦有所贡献,绝对不会因为贫穷而湮没无闻。我们在政治生活中享有自由,我们在日常生活也是如此。……我们在私人关系上是宽松和自在,但是作为公民我们是追求法律的。对当权者和法律的敬畏使我们如此。我们不但服从那些当权者,我们还遵守法律,尤其是遵守那些保护受伤害者的法律,不论这些法律是成文法,还是虽未形成文字但是违反了就算是公认的耻辱的法律。……一言以蔽之,我们的城邦是全希腊的学校。\nauthor{修昔底德:《伯罗奔尼撒战争史》,徐松岩译,桂林:广西师范大学出版社2004年版,第98—101页。}}

伯里克利的演说片段至少涉及三个关键内容:一是雅典民主制是多数人的统治,“城邦是由大多数人而不是由极少数人加以管理的”;二是公共领域与私人领域的划分,他提到了“公共生活”与“私人关系”两个不同概念;三是雅典公民既享有自由,又懂得服从法律和权威。伯里克利身为雅典的政治领袖,对此感到非常自豪。

古希腊的政治学传统自然离不开著名哲学家柏拉图。柏拉图是一位众所周知的人物,在西方哲学史上的地位极高。到了20世纪,英国数学家与哲学家阿尔弗雷德·怀特海甚至这样说,后来的哲学研究不过是对“柏拉图的一系列注脚”。通过这一非常夸张的表述,可以看出柏拉图对整个西方哲学的影响。

柏拉图在《理想国》中开篇就讨论“什么是正义”,这是政治哲学的经典议题之一。通过苏格拉底与他人的对话,柏拉图一一抛出了诸如“欠债还债就是正义”“以善报友、以恶报敌就是正义”“正义是强者的利益”等不同观点,然后他借苏格拉底之口提出了自己的正义观,即公平对待每个人,服从等级秩序,以及使每个人各安其位。

柏拉图非常关注的是什么是最好的政体,以及什么是善的社会。那么,理想的统治秩序应该是怎样的呢?理想的统治者应该是怎样的呢?他描绘了一个理想的社会:

\quo{到这里我们一致同意:一个安排得非常理想的国家必须妇女共有、儿童共有、全部教育共有。……他们的王则必须是那些被证明为文武双全的最优秀人物。\nauthor{柏拉图:《理想国》,郭斌和、张竹明译,北京:商务印书馆2002年版,第312页。}}

柏拉图的一个招牌概念就是哲学王的统治。柏拉图何以对政治持有这样的见解呢?柏拉图的立论有两个基础:第一是对人性的基本判断。柏拉图认为,人性当中有向善的成分,又有堕落的成分。就德行而言,凡人皆有弱点,而多数人易于堕落。实际上,不同的人性观很大程度上决定了思考政治问题的不同起点。柏拉图在不同地方都讲到对人类易于堕落的担忧。所以,他认为,如果一个城邦按照普通人的德行水准来统治,很有可能是糟糕的统治。由此可见,柏拉图并非民主的朋友。

第二是他认为任何事情都需要专门的技艺,统治亦不例外。柏拉图举过很多例子,比如海上航行的船应该选谁作舵手或船长呢?是选一个很受欢迎的人,还是找一个掌握航海技艺的人?显然是后者。在他看来,航海、医疗、手工活等都需要专门的技艺,而统治同样需要专门的技艺。所以,他的结论是应该由哲学王来统治,亦即上面提到的掌握统治技艺的文武双全的人物。为了建成善的社会,当哲学王成为统治者以后,还应该把所有人都纳入到国家权力统一的安排当中,教育、子女、家庭、婚姻甚至性都应该纳入国家的统一安排。这是柏拉图的理想政体类型。当然,柏拉图的这种观点也遭到后世学者的激烈批评。特别是,卡尔·波普尔在《开放社会及其敌人》中把柏拉图视为现代极权主义思想的滥觞。\nauthor{卡尔·波普尔:《开放社会及其敌人》,郑一明等译,北京:中国社会科学出版社1999年版。}后来,柏拉图在《政治家》和《法律篇》等著作中则退了一步,他也认为哲学王的统治实际上是较难实现的。所以,混合政体被他视为一种次优选择。

在古希腊的政治学传统中,亚里士多德的开创性贡献是对政体类型进行了研究,并首创了比较研究的方法。他在《政治学》第3卷中认为:

\quo{最高治权的执行者既可以是一个人,也可以是少数人,又可以是多数人。这样,我们就可以说,这一人或少数人或多数人的统治要是旨在照顾全邦共同的利益,则由他或他们所执掌的公务团体就是正宗政体。反之,如果他或他们所执掌的公务团体只照顾自己一人或少数人或平民群众的私利,那就必然是变态政体。……政体(政府)的以一人为统治者,凡能照顾全邦人民利益的,通常就称为“王制(君主政体)”。凡政体的以少数人,虽不止一人而又不是多数人,为统治者,则称“贵族(贤能)政体”。……以群众为统治者而能照顾到全邦人民公益的,人民称它为“共和政体”。……

相应于上述各类型的变态政体,僭主政体为王制的变态;寡头政体为贵族政体的变态;平民政体为共和政体的变态。僭主政体以一人为治,凡所设施也以他个人的利益为依归;寡头(少数)政体以富户的利益为依归;平民政体则以穷人的利益为依归。三者都不照顾城邦全体公民的利益。\nauthor{亚里士多德:《政治学》,吴寿彭译,北京:商务印书馆2007年版,第136—137页。}}

在上述简短精练的文字中,亚里士多德根据两个标准区分了政体类型:一是统治者数量的多寡,二是统治的目的是否服务于全城邦的利益。这样,他划定了六种政体类型,分别是三种正宗政体:君主政体、贵族政体和共和政体,以及相应的三种变态政体:僭主政体、寡头政体和平民政体。

亚里士多德在《政治学》中甚至还颇有远见地讨论了19世纪以来愈显重要的阶级斗争问题,论述了贫富冲突的危害以及中产阶级的重要性。他在《政治学》第4卷中说:

\quo{平民群众和财富阶级之间时时发生党争;不管取得胜利的是谁,那占了上风的一方总不肯以公共利益和平等原则为依归来组织中间形式的政体,他们把政治特权看做党争胜利的果实,抢占到自己的手中后,就各自宁愿偏向平民主义或寡头主义而独行其是。

惟有以中产阶级为基础才能组成最好的政体。中产阶级(小康之家)比任何其他阶级都较为稳定。他们既不像穷人那样希图他人的财物,他们的资产也不像富人那么多得足以引起穷人的觊觎。……很明显,最好的政治团体必须由中产阶级来执掌政权。\nauthor{亚里士多德:《政治学》,第211—212、209页。}}

即便是今天,亚里士多德对阶级问题的观察仍然很有意义。读亚里士多德《政治学》中的论述,就会知道阶级斗争并非19世纪的全新事物,亦不会随着1991年苏联解体和弗朗西斯·福山预言的“历史的终结”而归于消逝。

“光荣属于希腊,伟大属于罗马。”这行被误译却颇为传神的诗,赞美的是古希腊与古罗马的荣耀。古罗马产生了不少重要的政治思想家,这里主要介绍西塞罗。西塞罗(公元前106年—前43年)生活在古罗马共和国时代,他是古罗马较重要的政治家,同时还是一位哲学家。总的来说,西塞罗作品的原创性并不突出,但他的作品在同时代最为流行。

1888年,甚至还出现了一幅以西塞罗为主角的油画,名为《西塞罗谴责喀提林》。这幅油画描绘的就是古罗马元老院里的场景,元老院的座位呈半圆形摆放,一批身穿白色袍子的元老们围坐四周,而西塞罗正站在中间对其他元老们进行演讲。当时有人试图发动政变,西塞罗的演讲重点是谴责贵族试图发动针对共和国的政变阴谋。在今天的罗马,古罗马市场遗迹群中仍然可以看到有一个保存较好、规模不大的房子,是当年古罗马元老院的遗址。

西塞罗较为重要的贡献是对混合政体思想的发展,他是混合政体的忠实信徒。什么叫混合政体?亚里士多德和柏拉图在论述君主制、贵族制和民主制时认为,三种政体各有缺陷,最优良的政体是把这三种政体的有利特性结合起来。西塞罗则在古罗马发展了这种观点。他以问答的方式提出了自己的主张,有人问:“你认为这三种形式中哪一种最好呢?”他这样说:

\quo{如果仅仅采用其中一种,我不赞成其中的任何一种,我认为它们三者结合的形式优于其中任何单独的一种。……君主制吸引我们是由于我们对它们的感情,贵族制则由于他们的智慧,民众政府则由于它们的自由。\nauthor{西塞罗:《国家篇法律篇》,沈叔平、苏力译,北京:商务印书馆2005年版,第42—43页。}}

实际上,古罗马共和国的政体构造就体现了混合政体的思想。古罗马共和国的执政官就代表君主的因素,元老院代表贵族的因素,公民大会及保民官则代表了平民的因素。古罗马历史学家波里比阿认为,罗马能在半个多世纪的时间里从一个小小的地方扩张到整个地中海,主要原因就在于古罗马共和国在政体上的巨大优势。执政官、元老院和公民大会及保民官三者的共同治理,使得各方的利益都够得到兼顾,这是混合政体的最早实践。\nauthor{波里比阿:《罗马帝国的崛起》,翁嘉声译,北京:社会科学文献出版社2013年版。}

\tsection{从“黑暗时代”到启蒙时代}

西罗马帝国覆灭之后,欧洲就进入了漫长的中世纪。过去中世纪被称为“黑暗时代”,但今天的很多历史学家正在改变这一观点,甚至有人把中世纪视为欧洲从古希腊、古罗马的文明到现代文明的一个重要转换。比如,美国历史学家詹姆斯·汤普逊就盛赞中世纪及封建主义在欧洲现代文明孕育过程中的积极作用。

欧洲中世纪出现了一位著名的神学政治哲学家托马斯·阿奎那(1227—1274),他的重要工作是把亚里士多德的作品系统地介绍给西方基督教世界。在此过程中,阿奎那本人也阐述了很多政治思想,开始对此前和逐渐形成中的自然法思想有很多探索。后世有一幅关于阿奎那的著名油画,他一手托着教堂,一手拿着书籍。这大概就是他一生中两个最重要的方面。托马斯·阿奎那把亚里士多德的学问引入中世纪拉丁文世界的努力,使得亚里士多德后来的影响非常之大。列奥·斯特劳斯主编的《政治哲学史》这样评价阿奎那的贡献:

\quo{他的著述生涯与亚里士多德的著作在西方世界被发现并开始产生巨大影响的时期大致相合。《政治学》尤其是全本《伦理学》(意指《尼各马可伦理学》)在其有生之年首次被译成拉丁文。他详加评注了亚里士多德的几乎全部主要论著,并在其神学著作中广泛利用亚里士多德的思想资料,在把亚里士多德确立为基督教西方世界中占主导地位的哲学权威方面,阿奎那的贡献是最大的。阿奎那自己的政治哲学最好看做是根据基督教的启示对亚里士多德的政治哲学所做的修正,或更准确地说,最好看做是这样的一种努力,即努力把亚里士多德同西方政治思想的早期传统结合起来……\nauthor{列奥·斯特劳斯、约瑟夫·克罗波西主编:《政治哲学史》(第三版),李洪润等译,北京:法律出版社2009年版,第232页。}}

马基雅维利是另一位众人熟知的政治思想家。他生活在欧洲文艺复兴时期的佛罗伦萨,著有《君主论》和《论李维》等重要著作,是政治现实主义的代表人物。他阐述了将政治与道德相分离的原则,同时也敏锐地嗅觉到了欧洲民族国家革命即将拉开序幕。萨拜因这样评价马基雅维利:

\quo{在他那个时代,没有任何人能够像他那样清楚地洞见到欧洲政治演化的方向。没有任何人能够比他更了解那些正在被淘汰的制度的过时性质,也没有任何人能够比他更承认赤裸裸的强力(naked force)在这一进程中所具有的作用。\nauthor{乔治·萨拜因著,托马斯·索尔森修订:《政治学说史》(第四版下册),邓正来译,上海:上海人民出版社2010年版,第7页。}}

上一讲曾讨论《君主论》的主要观点,此处不再赘述。由于《君主论》的知名度,马基雅维利的思想经常遭到误解。实际上,马基雅维利并不能被简单地视为一个君主论者——即便他在《君主论》一书中的思想亦非完全如此。可以确定无疑地说,马基雅维利有着清晰的古典共和主义思想。特别是,他在《论李维》一书中“表现出他对罗马共和国自由和自治的巨大热诚”。比如,他这样认为:“民众比君主更聪明、更忠诚”;共和国比君主更信守承诺,因而比君主更值得信赖;“与君主国相比,共和国有更强盛的活力,更长久的好运”;等等。总之,就本意而言,马基雅维利认为共和国是优于君主国的。\nauthor{尼科洛·马基雅维利:《论李维》,冯克利译,上海:上海人民出版社2006年版。}

欧洲政治思想的重要转换是16—17世纪宗教改革的兴起。宗教改革的背景是中世纪以来天主教会的腐败以及君主与教会权力的争执,同时宗教改革受到了文艺复兴中人本主义思潮的影响以及印刷术这一新技术兴起所带来的冲击。宗教改革之前,天主教会的突出问题是在欧洲各地出售赎罪券——实际上,这是以上帝的名义兜售自己的生意。当时,普通信徒是没有多少机会接触《圣经》的,阅读和诠释《圣经》是教会神职人员的特权,然后他们教会就充当了上帝与普通信众之间的中介。这样,天主教会成了一个特权机构——既是一个思想控制机构,又是一个利益分配机构。结果是,当时的天主教会沦为了十分腐败的特权机构。

教会的腐败促使很多人开始反思。两位杰出的宗教改革领袖马丁·路德(1483—1546)和约翰·加尔文(1509—1564)对天主教会当时的很多做法和腐败提出了尖锐批评,他们希望重新定义信徒与教会的关系、国家与教会的关系。有一次,马丁·路德神父在研读《圣经》时读到了“义人必因信得生”,突然认识到原来人的得救只是因为他对上帝的信仰以及上帝的恩赐,其他一切的律法都不能保证使人得以“称义”。这种见解后来被称为“因信称义”。随后,1517年10月31日,马丁·路德将批判赎罪券和天主教会的《九十五条论纲》张贴在威登堡大学的教堂门口。这标志了宗教改革的兴起。从路德到加尔文,他们认为每个人都可以读《圣经》、都可以直接跟上帝对话,从而重新构建了普通信徒与教会关系,也为重新构建国家与教会的关系提供了思想资源。同样重要的是,有人认为,宗教改革也为科学革命提供了可能。

法国思想家让·博丹(1530—1596)的主要贡献是他的主权学说。他认为,主权是超越其他权力之上的、不受法律约束的最高权力。博丹的这种主权思想跟古希腊和古罗马的传统很不一样,跟后来启蒙运动时代的政治哲学差异就更大了。但是,博丹的这种主权学说对当时民族国家的兴起,对后来国家理论的出现应该产生了重要影响。萨拜因把博丹的主权学说总结为几个简要的原则:

\quo{(1)他认为,主权的出现乃是把国家同包括家庭在内的所有其他群体区别开来的标志。因此,他一开始便把公民身份定义为对主权者的服从。……

(2)博丹的第二个步骤乃是把主权定义为“不受法律约束的、对公民和臣民进行统治的最高权力”,并对最高权力的概念进行分析。这种最高权力首先是永恒的。……它不是授予的权力,或者说它是一种无限制的或无条件的授权。……它不受法律的约束。国家的法律就是主权者的命令。……

(3)他认为,凡是未陷于无政府状态的统治,凡是“秩序良好的国家”,在其间的某处肯定存在着这样一种不可分割的权威渊源。……

(4)博丹还把主权理论适用于他对国家从属机构的讨论。在一个君主制的国家里,议会的职能必须是咨询性质的。同样,行政官员所行使的权力也是主权者授予的。再者,国家内部所存在的所有法人团体……之所以拥有权力和特权,也是因为主权者的意志所致。\nauthor{乔治·萨拜因著,托马斯·索尔森修订:《政治学说史》(第四版下册),邓正来译,上海:上海人民出版社2010年版,第81—87页。正文插入部分内容的序号为本书作者所加。}}

然而,博丹的主权学说中还包含着自相矛盾和含糊不清的成分。他总体上认为,主权是不受任何约束的最高权力,但他同时认为“主权者是受上帝之法和自然法约束的”;他并不赞同国王可随性而为,“有些事情由法国国王去做是不合法的”;他信奉“私有财产权不可侵犯”的原则。正如很多杰出思想家一样,博丹的作品中也充满着内在冲突。

此后,英国又出现了一位重量级政治思想家托马斯·霍布斯(1588—1679),他的名著是《利维坦》。霍布斯在《利维坦》一书中最为杰出的贡献是从学理上阐明了国家的必要性。他的论证从自然状态出发,开创性地引入了个人主义方法论,结论是国家乃人类社会所必需。他这样说:

\quo{任何两个人如果想取得同一东西而又不能同时享用时,彼此就会成为仇敌。他们的目的主要是自我保全,有时候只是为了自己的欢乐;在达到这一目的的过程中,彼此都力图摧毁或征服对方。……

根据这一切,我们就可以显然看出:在没有一个共同权力使大家慑服的时候,人们便处在所谓的战争状态之下。这种战争是每一个人对每个人的战争。……

这种人人互相为战的战争状态,还会产生一种结果,那便是不可能有任何事情是不公道的。是和非以及公正不公正的观念在这儿都不能存在。没有共同权力的地方就没有法律,而没有法律的地方就无所谓不公正。暴力和欺诈在战争中是两种主要的美德。……这样一种状况还是下面情况产生的一种结果,那便是没有财产,没有统治权,没有“你的”“我的”之分;每一个人能得到手的东西,在他能保住的时期内便是他的。\nauthor{霍布斯:《利维坦》,黎思复、黎廷弼译,北京:商务印书馆1996年版,第93—96页。}}

面对这样的艰难境况,霍布斯认为出路在于——

\quo{那就只有一条道路:——把大家所有权力和力量托付给某一个人或一个能够通过多数的意见把大家的意志转化为一个意志的多数人组成的集体。……在这一点办到以后,像这样统一在一个人格之下的一群人被称为国家,在拉丁文当中被称为城邦。这也就是伟大的利维坦的诞生。……用一个定义来说,这就是一大群人互相订立信约、每人都对它的行为授权,以便使它能按其认为有利于大家的和平和共同防卫的方式运用全体的力量和手段的一个人格。承当这一人格的人就称为主权者,并被说成是具有主权,其余的每一个人都是他的臣民。……

取得这种主权的方式有两种:一种方式是通过自然之力获得的……另一种方式则是人们互相达成协议,自愿地服从一个人或一个集体……后者可以称为政治的国家,或按约建立的国家;前者则称为以力取得的国家。\nauthor{霍布斯:《利维坦》,第131—132页。}}

从这几段关键表述中,可以领略霍布斯这位杰出政治思想家对人类基本政治问题的思考。

霍布斯之后的杰出政治哲学家往往知名度更高。英国哲学家约翰·洛克(1632—1704)被视为早期自由主义的杰出代表,他认为统治应该基于被治理者的同意,提出了立法权与行政权两权分立的思想,认为政府的首要职责是保卫人们的生命权、自由权与财产权。由此,洛克确立了古典自由主义奠基者的历史地位。洛克的具体政治观点,下一讲还会择要介绍。

法国启蒙运动造就了两位享誉世界的杰出政治思想家孟德斯鸠(1689—1755)和卢梭(1712—1778)。孟德斯鸠的政治思想某种程度上是对洛克的继承,他一方面反对专制和捍卫自由,另一方面论证了三权分立的必要性。孟德斯鸠的学说后来启发了美国制宪会议及联邦党人,直接影响了1787年《美国宪法》的起草与美国政体的创建。关于政体类型、自由与三权分立,孟德斯鸠这样说:

\quo{政治自由并不是愿意做什么就做什么。在一个国家里,也就是说,在一个有法律的社会里,自由仅仅是:一个人能够做他应该做的事情,而不被强迫去做他不应该做的事情。……自由是做法律所许可的一切事情的权利;如果一个公民能够做法律所禁止的事情,他就不再有自由了,因为其他的人也同样会有这个权利。……

当立法权和行政权集中在同一个人或同一机关之手,自由便不复存在了;因为人们将要害怕这个国王或议会制定暴虐的法律,并暴虐地执行这些法律。……如果司法权不同立法权和行政权分立,自由也就不存在了。如果司法权同立法权合而为一,则将对公民的生命和自由施行专断的权力,因为法官就是立法者。如果司法权同行政权合而为一,法官将握有压迫者的力量。\nauthor{孟德斯鸠:《论法的精神》(上册),张雁深译,北京:商务印书馆1961年版,第154、156页。}}

卢梭的作品文字优美,具有极强的感染力。卢梭在《社会契约论》开篇的一段话,很多人都耳熟能详——

\quo{人是生而自由的,却无往不在枷锁之中。自以为是其他一切的主人的人,反而比其他一切更是奴隶。\nauthor{卢梭:《社会契约论》,何兆武译,北京:商务印书馆2005年版,第8页。}}

卢梭的重要贡献是完善了社会契约论,并提出了主权在民的学说。所以,卢梭被视为系统阐明民主理论的重要源头。他把社会契约视为组织政治社会的条件:

\quo{要寻找出一种结合的形式,使它能以全部共同的力量来卫护和保障每个结合者的人身和财富,并且由于这一结合而使得每一个与全体想联合的个人又只不过是在服从其本人,并且仍然像以往一样的自由。……这个社会公约一旦遭到破坏,每个人就立刻恢复了他原来的权利,并在丧失约定的自由时,就又重新获得了他为了约定的自由而放弃的自己的天然的自由。\nauthor{卢梭:《社会契约论》,何兆武译,北京:商务印书馆2005年版,第23页。}}

上述讨论,被卢梭视为社会契约所要解决的根本问题。在此基础上,卢梭进一步阐明了主权在民原则。卢梭所谓的“主权者”,是指合法性源自人民,而非源自君主、贵族或任何的少部分人。在他看来,这种主权是不可让与和不可分割的。任何个人或任何团体都不能取代人民总体而被给予立法的权力。所以,后人把卢梭视为启蒙时代人民主权学说的主要阐发者。

当然,卢梭的这种民主观意味着他更强调直接民主,而非代议制民主。此外,他强调了公意及其不可分割性:

\quo{……社会公约可以简化为如下的词句:我们每个人都以其自身及其全部的力量共同置于公意的最高指导之下,并且我们在共同体中接纳每一个成员作为全体之不可分割的一部分。\nauthor{卢梭:《社会契约论》,何兆武译,北京:商务印书馆2005年版,第24—25页。}}

卢梭的这一表述引起过很多争议,有人认为这意味着某种集体主义的社会方案。所以,有人甚至担心,卢梭这一观点为现代极权主义统治打开了一条幽暗的通道。

在19世纪之前,人类政治思想史上最具有实践智慧的著作无疑要数《联邦党人文集》。三位杰出的联邦党人——亚历山大·汉密尔顿、詹姆斯·麦迪逊和约翰·杰伊——撰写《联邦党人文集》所列85篇文字的初衷,是为了说服纽约州人民批准新的美国宪法。他们三人不仅是杰出的政治思想家,而且是杰出的政治实践家——三位作者一人出任美国总统,一人出任财政部长,一人出任最高法院大法官。因为后来美国以自由民主政体闻名于世,而联邦党人的主张又与美国自由民主政体关系密切,所以多数研究者往往很重视联邦党人的共和制和联邦制思想——前者是指自治政府、代议制、三权分立、自由学说以及共和制的主张,后者是指中央与地方政府合理划分权力的主张。《联邦党人文集》的名言是“野心必须用野心来对抗”,他们这样阐述这方面的政治思想:

\quo{用这种种方法来控制政府的弊病,可能是对人性的一种耻辱。但是政府本身若不是对人性的最大耻辱,又是什么呢?如果人都是天使,就不需要任何政府了。如果是天使统治人,就不需要对政府有任何外来的或内在的控制了。在组织一个人统治人的政府时,最大困难在于必须首先使政府能够管理被统治者,然后再使政府管理自身。毫无疑问,依靠人民是对政府的主要控制;但是经验教导人们,必须有辅助性的预防措施。……

立法、行政和司法权置于同一人手中,不论是一个人、少数人或许多人,不论是世袭的、自己任命的或选举的,均可公正地断定是虐政。\nauthor{汉密尔顿、杰伊、麦迪逊:《联邦党人文集》,程逢如译,北京:商务印书馆1995年版,第264、246页。}}

由于被视为分权思想的经典作品,《联邦党人文集》常常被忽视的是这部作品对强有力的联邦政府的倡导和对政府效能的重视。实际上,联邦党人——特别是汉密尔顿——用大量篇幅论述了强有力的联邦政府和政府效能的问题。比如:

\quo{明智而热情地支持政府的权能和效率,会被诬蔑为出于爱好专制权力,反对自由原则。……(但是)政府的力量是保障自由不可缺少的东西。……

决定行政管理是否完善的首要因素就是行政部门的强而有力。……软弱无力的行政部门必然造成软弱无力的行政管理,而软弱无力无非是管理不善的另一种说法而已;管理不善的政府,不论理论上有何种说辞,在实践上就是个坏政府。……

使行政部门能够强而有力,所需要的因素是:第一,统一;第二,稳定;第三,充分的法律支持;第四,足够的权力。\nauthor{汉密尔顿、杰伊、麦迪逊:《联邦党人文集》,第5、356页。}}

一方面强调分权制衡,另一方面强调政府效能,两者的结合才是联邦党人的完整思想表述。通过上述两组言论的比较,大家就可以较全面地理解联邦党人的政治观点。

\tsection{经验研究范式的兴起}

19世纪政治学的主要特征是整体上向经验研究的转向。这一转向得益于欧美社会16—18世纪的很多积累,特别是宗教改革、启蒙运动和科学革命为这种转向开辟了可能性。这样,到19世纪,实证主义(positivism)哲学开始兴起。法国哲学家奥古斯特·孔德所著的《实证哲学讲义》把整个人类对重要问题的思考分成三个阶段,分别是神学阶段、玄学阶段和科学阶段。他认为,到了19世纪整个社会科学应该进入到科学阶段。

那么,什么是实证主义呢?简单地说,这是一种以实际验证为核心的哲学思想,注重以科学方法来观察和研究经验事实,通过这种观察和研究来探究事物的本源及事物与事物间的联系。相比而言,霍布斯在《利维坦》中的方法——先假设存在一个自然状态,分析这种状态里人和人可能是一种怎样的关系,然后推导出他的结论——就不是经验研究或实证研究的方法。到了19世纪,人们把对政治和社会的研究看作是科学,可以把政治现象描述出来,然后探讨现象背后的因果关系和因果机制。

到了19世纪30年代,托克维尔的名著《论美国的民主》开启了关于美国民主的政治社会学研究。\nauthor{托克维尔:《论美国的民主》,董国良译,北京:商务印书馆1989年版。}托克维尔的写作方法完全不同于柏拉图、霍布斯或卢梭,他深入美国社会去做实地调查,考察和了解美国民主体制的运转情况。托克维尔把关于这项研究的目标确定为:探究有助于美国能够维护民主共和制度的主要原因。托克维尔的研究主要不是基于哲学思辨,而是基于对美国社会的实地考察,在实地考察的基础上再做归纳和分析。

《论美国的民主》一书认为,美国之能维持民主制度,应归根于地理环境、法制和民情。立足经验研究,托克维尔认为,固然自然环境、法制和民情三者都很重要,但是,自然环境不如法制,法制不如民情。托克维尔为此提供了大量的经验证据,包括对美国地理的描述,对美国宪法和制度的考察,对北美13个殖民地——特别是新英格兰地区——的社会风俗研究,对美国民主实际运行状况的调查,等等。所以,《论美国的民主》一书是一部对美国民主的经验研究作品。

英国哲学家、社会学家赫伯特·斯宾塞的《社会学研究》一书也部分地采用这种研究方法。\nauthor{赫伯特·斯宾塞:《社会学研究》,张红晖、吴江波译,北京:华夏出版社2001年版。}作为一位著名的社会进化论者,斯宾塞已经注意到社会进化过程中文化和政治因素的作用,他把地理环境的物理特征(如山区和平原地势的差别)视为一个国家政治上分权或集权的解释变量,他还认为民主和政治平等的发展会受到社会经济条件变化及城市化的直接影响。所有这些都是经验研究。

后来出版的《关于政治的高尚科学》一书这样评价当时的研究趋势:

\quo{在19世纪,关于政治现象的本质以及如何进行解释的观点,逐步更多地建立在历史归纳的基础上,而不是基于对人类属性的假设。这在很大程度上归因于与当代和历史社会有关的知识的增长。实证主义和殖民主义逐步地把广泛的、复杂的文化(如印度),以及小范围的、较为原始的社会(如美国印第安和南非社会),纳入到欧洲的学者和知识分子进行学术研究的范围。与马基雅维利和孟德斯鸠所处的时代相比,外来的世界变得更容易到达,这使得人们变得越来越好奇,并在因果推论方面进行了不懈的努力。\nauthor{转引自罗伯特·古丁、汉斯-迪特尔·克林格曼主编:《政治科学新手册》(上册),钟开斌等译,北京:生活·读书·新知三联书店2006年版,第85页。}}

这样,基于经验观察基础上的研究,而非纯粹的逻辑演绎和推导,成为19世纪政治学的新趋势。与19世纪之前的政治哲学和政治思想研究不同,政治学的经验研究试图在事物和事物之间、现象和现象之间建立一种因果联系,发掘政治现象背后的因果机制。

\tsection{从政治科学到研究范式的多样化}

20世纪以来,政治学研究迎来了政治科学的时代。卡尔·波普尔在《猜想与反驳》中认为,社会科学研究的一个标准是研究者提出的理论假说应该具有“可证伪性”(falsifiability),也就是具有证伪机制。波普尔认为,科学与哲学的分界不是归纳方法与思想方法,而是:

\quo{我建议应当把理论系统的可反驳性或可证伪性作为分界标准。按照我仍然坚持的这个观点,一个系统只有作出可能与观察相冲突的论断,才可以看作是科学的;实际上通过设法造成这样的冲突,也即通过设法驳倒它,一个系统才受到检验。\nauthor{卡尔·波普尔:《猜想与反驳——科学知识的增长》,傅季重等译,上海译文出版社1986年版,第36页。}}

波普尔把可证伪性视为科学研究的核心特征。一个经典的例子是,有人提出这样的一个假说——“所有天鹅都是白的”。对于这个假说,证伪就在于找到一只黑天鹅(严格地说应该是一只非白天鹅),中间的证伪机制是明确的。再比如,有人提出这样一个假说——“凡是经济富裕的国家都是民主国家”。证伪机制也很清楚,只要找到一个富裕国家但它同时不是民主国家,这个假说就被证伪了。在波普尔看来,如果一个理论观点不存在证伪机制,就不符合一项好的科学研究的标准。

政治科学研究的核心是探索政治现象背后的因果关系,而不是别的什么研究。像自由是什么、民主是什么这样的问题也很重要,但政治科学研究最感兴趣的并不是这样的问题,而是发掘事物之间的因果关系。比如,政治科学领域有一项著名的研究:为什么有些国家是民主国家而有些国家是非民主国家?为什么有些国家实现了民主转型与巩固而有些国家没有?这就是一项试图揭示因果关系的研究。研究者试图把这一政治现象背后的原因找出来,而不只是简单描述这种现象。在政治科学研究中,这种因果关系的理论形式经常表述为一个理论假说——即何种原因导致何种结果。

经过19世纪经验研究及实证主义的兴起,再到20世纪向政治科学研究的转向,后来的政治科学越来越专业化了,美国开始取代欧洲成为政治学研究的中心。20世纪10到30年代,美国政治学研究中开始兴起了芝加哥学派。经济学领域的芝加哥学派是新古典自由主义的重镇,并成为大量诺贝尔经济学奖获得者的摇篮。政治学领域的芝加哥学派则是用交叉学科的研究战略,开始在政治学领域引入定量研究方法,同时开始为定量研究收集大规模的调查数据。这一时期出现的新现象是有人开始为这类研究提供资助。

这场政治科学的革命也被称为行为主义革命。过去,政治学的研究重点是政治制度和政治秩序等。到了行为主义革命阶段,研究重点变成人的政治行为。行为主义既不赞同政治哲学的思辨方法,亦不认可对政治制度的静态描述,而是认为政治学应该研究实际存在并且可以观察到的人的政治行为。因此,行为主义比较重视数据的收集和整理,常常运用抽样调查、数理模式、模拟实验、统计分析等手段进行研究,一般强调精确性、科学性、量化及价值中立等原则。比如,选民为什么支持共和党或民主党?这是行为主义最为常见的研究议题。由此可见,这种研究路径的转向是很大的。

一个具有里程碑意义的重要事件是1923年芝加哥大学政治学者对芝加哥6000个选民进行的抽样调查。这项抽样调查的主要内容包括:一是被调查者的个人背景,二是被调查者投票支持谁,三是被调查者在政策问题上的看法。在当时,这种研究在整个政治学领域是闻所未闻的。1929年他们又做了一个研究项目,主题是“为什么美国最优秀的人不从政”,同样是基于大样本的调查问卷。后来,政治学家哈罗德·拉斯韦尔开始在政治心理学研究中使用问卷调查的数据。

基于大型抽样调查的政治科学研究有其显著的优势。比如,现在要做一项研究:美国虔诚的基督徒选民更支持共和党还是民主党?如果没有数据,这个问题就很难说清楚。研究者可能会说虔诚的基督徒有什么特征,具有什么样的意识形态特点,然后再比较民主党和共和党的意识形态与政策主张。在此基础上,研究者得出结论:虔诚的基督徒更有可能支持共和党或民主党。但是,这种研究成果一旦发表出来,可能马上有人会出来质疑。为什么呢?主要原因在于缺乏“过硬”的证据。然而,有了大型抽样调查,这个问题就迎刃而解了。比如,可以在美国若干个州发放10000份调查问卷,问题主要分为三组。一组问题调查受访者的信教情况,去教堂或参加宗教聚会的频率,等等;一组问题调查受访者在上次总统或国会选举中的投票倾向;一组问题调查受访者的职业、收入、年龄、性别和族群等情况——最后一组问题可以作为控制变量。通过这些问卷调查所采集到的大量数据,可以看出虔诚的基督徒是否更可能投票支持共和党或民主党;然后再看看其他变量,比如职业、收入、年龄、性别和族群等,对选民投票倾向的影响是否很显著。这样的研究,在证据方面就非常可靠。这个例子说明了大型抽样调查在政治科学研究中的优势与潜力。

后来,芝加哥学派的研究方法在美国一流高校里开始传播。很多早期在芝加哥大学受过训练的学者后来前往密歇根大学任教,并把密歇根大学发展成了美国政治学调查研究的重镇和行为主义革命的基地。直到今天,密歇根大学仍然维持着这一领域的强大优势。1947年,密歇根大学搞了一个培训学会,实际上就是对这种研究调查方法的大规模推广。1977年,密歇根大学获得了全美选举调查研究资助,由此拥有了美国最成熟、最完备的选民调查数据库。后来,其他机构开始做欧洲主要国家的选民调查数据库。如今,政治文化研究领域的一个重要调查——世界价值观调查(world value survey)——的开创也依托于密歇根大学的研究力量。

所以,这场政治科学革命的重点已经从对政治秩序或政治制度的研究,转向对人的政治行为的研究。在西方发达工业民主国家,政治行为首先是投票行为。这种研究聚焦于对选民投票行为的研究,并试图解释投票行为差异的原因,研究方法上则更多采用定量研究的方法。基于这种研究路径,1960年密歇根大学的安格斯·坎贝尔等人出版了该领域的一部重要著作《美国投票者》。\nauthor{Angus Campbell, Philip Converse, Warren Miller, and Donald Stokes,\italic{The American Voter},Chicago: University of Chicago Press, 1980.}该书的研究团队基于一个庞大的选民调查问卷数据库,试图完整地展示美国选民在投票与政治行为上的特征及其原因。该研究的一个重要结论是:多数美国选民根据党派立场来投票,而这种党派立场很大程度上受到其家庭背景的影响。

但是,行为主义革命的进一步发展也引起了学术界的反思。行为主义革命强调恪守价值中立原则,反对做价值判断,认为应该专注于事实和经验。但政治哲学家列奥·斯特劳斯认为:“(那种认为)价值判断不是主观的,归根到底是受理性控制的观点,导致了在涉及正确与错误、善与恶时出现一种做出不负责任的判断的倾向。”行为主义研究意味着价值被放弃了,而这恰恰是斯特劳斯担心的事情,他认为,人们对政治的看法本身非常重要,什么是政治之善?什么是政治之恶?这都会对实际的政治产生显著的影响。另一项对行为主义的批评来自于戴维·里西。他在1984年出版的《政治科学的悲剧》中,批评20年代到60年代美国政治科学中所出现的对政治“科学”所持的幼稚看法。作为一门实证科学,如果政治科学无法系统性地吸收道德和伦理价值的因素,也无法对政治行为承担责任,它注定是要令人失望的。\nauthor{罗伯特·古丁、汉斯-迪特尔·克林格曼主编:《政治科学新手册》(上册),钟开斌等译,北京:生活·读书·新知三联书店2006年版,第108—112页。}

另一个现象是,一部分政治科学学者由于过分重视调查数据和量化分析方法本身,忽视研究议题与理论构建的重要性,学术期刊上也出现了不少意义不大的研究论文。一些研究在数据部分处理得非常精彩,但最后的结论要么不重要,要么是过去早已知道的。所以,政治科学研究过程中的理论导向和理论构建,与研究方法和量化技术,两者最好要兼顾。当然,对中国而言,基于调查问卷的研究和量化分析是目前做得远远不够的,跟国际学界相比差距还很大,所以亟待加强。

尽管对于政治科学研究大规模向量化研究转向充满了争议,但是目前美国最好的政治学刊物——特别是《美国政治学评论》和《美国政治科学杂志》等,每期都有半数以上的学术论文是借助调查数据与定量研究来完成的。如今一流政治学学术期刊上很多论文的体例、格式与呈现形式跟经济学论文非常相似。

行为主义革命之后,到20世纪70年代左右,经济学研究方法在其他社会科学研究领域的扩展,推动了理性选择范式的兴起。这股潮流被称为“经济学帝国主义”。当时,有一些经济学家声称,经济学不是受限于研究稀缺资源配置这一特定议题的学科,而是一整套与人类行为有关的研究范式与方法。经济学家和社会科学家可以用经济学方法来研究一般意义上的人类行为,从而实现了经济学方法在其他社会科学领域的大规模应用。比如,在今天,《国家为什么会失败》(\italic{Why Nations Fail})的两位作者、经济学家德隆·阿西莫格鲁和詹姆斯·罗宾逊就发表了大量与政治、转型有关的经济学论文。

理性选择理论(Rational Choice Theory)从20世纪70年代到90年代逐步成为美国政治科学领域最有影响的研究范式与理论流派。理性选择学派的核心是用经济学理论和方法研究政治,特别是借鉴新古典经济学的视角,它把政治领域类比为市场,把政治活动(比如选举)视为交易。经济学认为市场中有两种主要角色,一种是厂商,一种是消费者,他们之间的交易构成了市场。理性选择学派认为,这种交易关系在政治中也存在。政治家为选民提供某些受到欢迎的公共政策,选民则把选票投给符合自己政治偏好的政治家。选民参与政治交易的目的,是为了获得对自己有利的政策,就像消费者想通过市场交易获得面包一样。政治家参与政治交易的目的,是为了获得更多选票和席位,就像厂商想通过市场交易获得收入和利润一样。

理性选择范式基于经济人假设,经济人假设认为人有三个基本属性:(1)人是自利的;(2)人是理性的;(3)人追求效用最大化。有了经济人假设,理性选择学派把选民、官员、政治家和统治者都视为经济人。过去不少人认为:政治家和官员应该比普通人更加高尚一些。按照这种分析框架,政治家和官员并非更加高尚的特殊物种,他们跟普通人一样也是经济人,是自利的、理性的和追求效用最大化。

美国政治学者安东尼·唐斯出版于1957年的《民主的经济理论》是这一领域的奠基作品之一。\nauthor{安东尼·唐斯:《民主的经济理论》,姚洋、邢予青、赖平耀译,上海:上海人民出版社2005年版。}唐斯在书中把民主政治过程视为政治家和选民之间的理性选择与市场交易的过程。在政治市场上,政治家为了获得选票,选民为了获得政策收益,两者之间形成了类似于政治市场的交易行为。这是把新古典经济学的方法应用于对选举和民主的早期研究之一。

理性选择学派的另一位代表人物是经济学诺贝尔奖得主詹姆斯·布坎南,他在研究中认为,政治家、官僚和选民都是理性经济人,政治过程同样被视为政治家作为厂商与选民作为顾客之间的交易。布坎南在一项研究中讨论了西方发达工业民主的财政赤字与公债问题。大家都知道,现在西方国家公共债务危机日趋严重。学术界对财政赤字和政府公债问题已经有很多分析,而布坎南的分析则非常独特。他认为,财政赤字植根于民主政治的运作机制当中,是民主政体选举竞争条件下政治家与选民互相博弈的结果。在民主政体下,选民希望福利越多越好,比如免费教育、免费医疗、各种政府补贴等等;同时,选民希望税收越少越好。为了赢得更多的选票和选民支持,政治家倾向于尽可能扩大福利支出,同时不增加或少增加税收。既扩大福利开支,同时又不增加税收,如何能做到呢?惟一的办法就是财政赤字,赤字的累积就是沉重的政府公债。布坎南对财政问题的分析是政治经济学的视角,其分析范式属于理性选择学派。\nauthor{詹姆斯·布坎南:《民主财政论》,穆怀朋译,北京:商务印书馆2002年版。}

理性选择学派还跟一个重要的新古典经济学研究分支有关,就是新制度主义经济学。新制度主义经济学的代表人物之一道格拉斯·诺思还提出了“新古典国家理论”。什么是新古典国家理论?诺思把国家视为“使福利或效用最大化的统治者”,具有三个特征:

\quo{首先,国家用一组服务——我们可以称作保护和公正——来交换岁入。……其次,国家试图像一个有识别力的垄断者那样行动,将每一个选民团体分开,为每个选民团体发明产权以最大限度增加国家的岁入。……第三,既然永远存在着能够提供同一组服务的潜在的竞争对手,国家是受其选民的机会成本所制约的。竞争对手有其他国家,另外还有在现存的政治经济单位内可能成为统治者的个人。\nauthor{道格拉斯·C.诺思:《经济史上的结构与变革》,厉以平译,北京:商务印书馆2005年版,第29页。}}

所以,在诺思看来,国家通过作为一个统治者,追求的是通过提供安全与秩序来获取统治租金收入的最大化。

与理性选择范式有关的是博弈论在政治科学研究中的应用。博弈论把政治视为不同政治参与者博弈的过程。博弈论最简单的应用是囚徒困境,而多数博弈论研究采用的是比较复杂的数理形式。作为通识读物,这里仅简要介绍一项关于内战的博弈论研究。

按照芭芭拉·瓦尔特的统计,1940年到1990年间全球爆发的41场内战中,仅有17场内战交战各方达成了和平协议。但是,在达成和平协议之后,仅有8场内战的和平协议得到真正执行,还有9场内战的交战各方又重新回到了内战。换言之,仅有19.5%的内战是以和平方式解决的,八成以上的内战最后以暴力竞争和一方决定性胜利的方式解决。瓦尔特认为,之所以交战各方不愿意达成和平协议,或者达成和平协议之后也不愿意执行,很大程度上是由于缺乏“可信承诺(credible commitments)”机制。简单地说,任何一方首先放弃或削减自己的武力,如果另一方反悔,就会给前者造成毁灭性的打击。这个逻辑非常简单,但非常有说服力。所以,瓦尔特注意到,那些成功地达成并执行和平协议的案例中,通常存在一个强有力的外部干预者,这个外部干预者拥有更强大的武力能迫使交战各方强制执行和平协议。这种情形下,和平协议就具有了一种可信承诺机制。缺少可信承诺机制的条件下,保存武力甚至强化武力是一种理性的选择。\nauthor{Barbara F. Walter, \italic{Committing to Peace: The Successful Settlement of Civil Wars}, Princeton: Princeton University Press, 2002.}

瓦尔顿对于可信承诺机制的分析就是一种博弈论的理论应用。此处博弈论分析的要点是,如果A有很多军队,B也有很多军队,双方开始打内战,当内战长期持续时,双方处在焦灼状态,彼此都不太好受,双方终于同意坐下来签订一个和平协议。但问题是,接下来和平协议怎么执行呢?最大的困难是,执行和平协议意味着A和B都要放弃军队,他们都要放弃单独控制军队的做法,内战才能真正结束。所以,要么A把军队控制权交给B,要么B把军队控制权交给A,要么双方都把各自军队的控制权交给一个双方共同产生的国家级机构。而这里最大的风险是:如果有哪方先交出军队,他们这边的风险就变得巨大。一旦对方反悔,对他们来说就是灭顶之灾。所以,更多情况下是,没有哪一方会率先交出军队的控制权,结果和平协议就很难真正执行。所以,除非存在强有力的外部干预者,其主要作用是保证协议得到强制执行。除此之外,内战往往无法以和平方式终结。

到了20世纪90年代以后,政治学的研究范式又出现了很多新的趋势,总体上越来越多元化。这样,关于政治学研究方法的争论某种程度上已经陷入了僵局。比如,关于质性研究与量化研究的争论,关于政治哲学研究与政治科学研究的争论,并没有什么结果。现在总的态势是基于量化方法的实证研究是政治学研究的主流,但不同研究领域的学者根据自己的理论与方法偏好来从事各自的研究,而所谓最优研究范式的问题不再成为一个争论的焦点。这样,单一路径的政治学研究范式就被舍弃了。

总之,这一时期,政治哲学与政治思想的研究又获得了某种程度的复兴,实证研究中质性分析与量化分析两种路径的平行发展,理性选择范式扩展到更多的研究领域,后现代主义的研究方法也开始兴起——有的学者甚至开始走向解构和诠释的路径。

\tsection{推荐阅读书目}

乔治·萨拜因著,托马斯·索尔森修订:《政治学说史》(第四版上下册),邓正来译,上海:上海人民出版社2008年版、2010年版。

列奥·斯特劳斯、约瑟夫·克罗波西主编:《政治哲学史》(第三版),李洪润等译,北京:法律出版社2009年版。

罗伯特·古丁、汉斯-迪特尔·克林格曼主编:《政治科学新手册》,钟开斌等译,北京:生活·读书·新知三联书店2006年版。
